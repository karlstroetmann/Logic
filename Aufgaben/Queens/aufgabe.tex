\documentclass{article}
\usepackage{german}
\usepackage[latin1]{inputenc}
\usepackage{a4wide}
\usepackage{amssymb}
\usepackage{fancyvrb}
\usepackage{alltt}


\begin{document}
\noindent
{\Large Das 8-Damen-Problem}
\vspace{0.5cm}

\noindent
Das 8-Damen-Problem besteht darin, 8 Damen so auf dem Schach-Brett aufzustellen,
dass keine Dame eine andere Dame schlagen kann.
Unter \\[0.1cm]
\hspace*{1.3cm} 
\texttt{http://www.ba-stuttgart.de/\symbol{126}stroetma/SETL2/queens-frame.stl}
\\[0.1cm]
finden Sie ein Programm-Ger�st, in dem Sie die L�cken f�llen sollen.
\vspace{0.3cm}

\noindent
\textbf{Hinweis}:  Immer, wenn Sie eine der in den
Aufgaben beschriebenen Prozeduren implementiert haben, sollten Sie diese
testen.  Dazu kann sinnvoll sein, zun�chst den Fall von drei Damen
auf einem Brett der Gr��e 3x3 zu betrachten, denn in diesem Fall bleibt die
Gr��e der berechneten Klauseln �berschaubar.  Hierzu ist Zeile 3 in der Datei
\texttt{queens-frame.stl} geeignet abzu�ndern.  

\noindent
\textbf{Aufgabe 1}: Gegeben ist eine Menge $S$ von aussagenlogischen Variablen.
Implementieren Sie (in Zeile 63) eine
Prozedur \texttt{atMostOne}(), so dass der Aufruf \\[0.1cm]
\hspace*{1.3cm} $\mathtt{atMostOne}(S)$ \\[0.1cm]
eine Menge $K$ von Klauseln ausrechnet.  Die Konjunktion der Klauseln aus $K$
soll genau dann wahr sein, wenn \textbf{h�chstens} eine der Variablen aus $S$
den Wert \texttt{true} hat.  \vspace{0.3cm}

\noindent
\textbf{Aufgabe 2}: 
Es seien die folgenden Objekte gegeben:
\begin{enumerate}
\item \textsl{board} repr�sentiert ein Schach-Brett.  
\item $\textsl{column} \in \{1,\cdots,8\}$ spezifiziert eine Spalte auf dem Schach-Brett.
\end{enumerate}
Implementieren sie (in Zeile 70) eine Prozedur \texttt{atMostOneInColumn},
so dass der Aufruf \\[0.1cm]
\hspace*{1.3cm} \texttt{atMostOneInColumn}(\textsl{board}, \textsl{column}) \\[0.1cm]
eine Menge von Klauseln $K$ berechnet.  Die Konjunktion der Klauseln aus $K$
soll genau dann wahr sein, wenn in der mit \textsl{column} spezifizierten Reihe
\textbf{h�chstens} eine Dame steht.  \vspace{0.3cm}

\noindent
\textbf{Aufgabe 3}: 
Es seien die folgenden Objekte gegeben:
\begin{enumerate}
\item \textsl{board} repr�sentiert ein Schach-Brett.  
\item $\textsl{k} \in \{3,\cdots,15\}$  spezifiziert �ber die Gleichung \\[0.1cm]
      \hspace*{1.3cm} $\textsl{row} + \textsl{column} = \textsl{k}$
      \\[0.1cm]
      eine aufsteigende Diagonale auf dem Schach-Brett.
\end{enumerate}
Implementieren sie (in Zeile 80) eine Prozedur
\texttt{atMostOneInUpperDiagonal}, so dass der Aufruf \\[0.1cm]
\hspace*{1.3cm} \texttt{atMostOneInUpperDiagonal}(\textsl{board}, \textsl{k}) \\[0.1cm]
eine Menge von Klauseln $K$ berechnet.  Die Konjunktion der Klauseln aus $K$
soll genau dann wahr sein, wenn in der durch \textsl{k} spezifizierten
aufsteigenden Diagonalen \textbf{h�chstens} eine Dame steht.  \vspace{0.3cm}
\pagebreak

\noindent
\textbf{Aufgabe 4}: 
Es seien die folgenden Objekte gegeben:
\begin{enumerate}
\item \textsl{board} repr�sentiert ein Schach-Brett.  
\item $\textsl{k} \in \{-7,\cdots,7\}$  spezifiziert �ber die Gleichung\\[0.1cm]
      \hspace*{1.3cm} $\textsl{row} - \textsl{column} = \textsl{k}$
      \\[0.1cm]
      eine absteigende Diagonale auf dem Schach-Brett.
\end{enumerate}
Implementieren sie (in Zeile 90) eine Prozedur
\texttt{atMostOneInLowerDiagonal}, so dass
der Aufruf \\[0.1cm]
\hspace*{1.3cm} \texttt{atMostOneInLowerDiagonal}(\textsl{board}, \textsl{k}) \\[0.1cm]
eine Menge von Klauseln $K$ berechnet.  Die Konjunktion der Klauseln aus $K$
soll genau dann wahr sein, wenn in der durch \textsl{k} spezifizierten
absteigenden Diagonalen \textbf{h�chstens} eine Dame steht.  \vspace{0.3cm}

\noindent
\textbf{Aufgabe 5}: 
Es seien die folgenden Objekte gegeben:
\begin{enumerate}
\item \textsl{board} repr�sentiert ein Schach-Brett.  
\item $\textsl{row} \in \{1,\cdots,8\}$  spezifiziert eine Reihe auf dem Schach-Brett.
\end{enumerate}
Implementieren sie (in Zeile 99) eine Prozedur \texttt{oneInRow}, so dass
der Aufruf \\[0.1cm]
\hspace*{1.3cm} \texttt{oneInRow}(\textsl{board}, \textsl{row}) \\[0.1cm]
eine Menge von Klauseln $K$ berechnet.  Die Konjunktion der Klauseln aus $K$
soll genau dann wahr sein, wenn in der durch \textsl{row} spezifizierten Reihe
\textbf{\textsl{mindestens}} eine Dame steht.  \vspace{0.3cm}

\noindent
\textbf{Aufgabe 6}: 
Wieder repr�sentiert \textsl{board} ein Schach-Brett.
Implementieren sie (in Zeile 125) eine Prozedur \texttt{allClauses}, so dass
der Aufruf \\[0.1cm]
\hspace*{1.3cm} \texttt{allClauses}(\textsl{board}) \\[0.1cm]
eine Menge von Klauseln $K$ berechnet.  Die Konjunktion der Klauseln aus $K$ 
soll genau dann wahr sein, wenn das durch \textsl{board} repr�sentierte
Schach-Brett das 8-Damen-Problem l�st.
\vspace{0.6cm}


\end{document}

%%% Local Variables: 
%%% mode: latex
%%% TeX-master: t
%%% End: 
