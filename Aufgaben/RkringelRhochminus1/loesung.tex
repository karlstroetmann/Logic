\documentclass{article}
\usepackage{fleqn}
\usepackage{german}
\usepackage[latin1]{inputenc}
\usepackage{a4wide}
\usepackage{amssymb}
\usepackage{fancyvrb}
\usepackage{alltt}

%\pagestyle{empty}

\newcommand{\qed}{\hspace*{\fill} $\Box$}
\def\pair(#1,#2){\langle #1, #2 \rangle}

\begin{document}
\noindent
\textbf{Aufgabe}: Es sei $R \subseteq M \times N$.  Unter welchen Bedingungen gilt 
\\[0.1cm]
\hspace*{1.3cm}
$R \circ R^{-1} = \mathrm{id}_M$? 
\vspace{0.3cm}

\noindent
\textbf{L�sung}:
Die Gleichung $R \circ R^{-1} = \mathrm{id}_M$ gilt genau dann, wenn $R$ links-total und
links-eindeutig ist.  Wir wiederholen zun�chst die Definitionen dieses Begriffe.
Eine Relation $R \subseteq M \times M$ ist \emph{links-total} auf $M$ genau dann, wenn
\begin{equation}
  \label{eq:1}
  \forall x \in M: \exists y \in M: \pair(x,y) \in R
\end{equation}
gilt.  Eine Relation $R \subseteq M \times M$ ist \emph{links-eindeutig} genau dann, wenn
\begin{equation}
  \label{eq:2}
  \forall x,y,z \in M: \bigl(\pair(x,y) \in R \wedge \pair(z,y) \in R \rightarrow x = z\bigr)
\end{equation}
gilt.  Wir behaupten nun, dass Folgendes gilt:
\\[0.2cm]
\hspace*{1.3cm}
$R \circ R^{-1} = \mathrm{id}_M$ \quad g.d.w. \quad
$R$ ist links-eindeutig und links-total auf $M$.
\\[0.2cm]
\textbf{Beweis}:
Der Beweis einer Aussage``$A$ g.d.w. $B$'' zerf�llt oft in zwei Teile: Zun�chst nehmen wir
an, dass $A$ gilt und zeigen, dass daraus $B$ folgt.  Anschlie�end nehmen wir an, dass $B$
gilt und zeigen, dass daraus $A$ folgt.  Wir wenden dieses Beweis-Prinzip an und zerlegen
den Beweis damit in zwei Teile:
\begin{enumerate}
\item[``$\Rightarrow$'']
Wir zeigen zun�chst, dass aus der G�ltigkeit der Gleichung 
\\[0.2cm]
\hspace*{1.3cm}
$R \circ R^{-1} = \mathrm{id}_M$
\\[0.2cm]
folgt, dass $R$ sowohl links-total als auch links-eindeutig ist.
\begin{enumerate}
\item Als erstes zeigen wir, dass $R$ auf $M$ links-total ist.  Sei also eine beliebiges
      $x \in M$ gegeben.  Wir m�ssen zeigen, dass es ein $y \in M$ gibt, so dass $\pair(x,y) \in R$ ist. 
      Wir wissen, dass
      \\[0.2cm]
      \hspace*{1.3cm}
      $\pair(x,x) \in \mathrm{id}_M$
      \\[0.2cm]
      gilt.  Aus der vorausgesetzten Gleichung $R \circ R^{-1} = \mathrm{id}_M$ folgt dann
      \\[0.2cm]
      \hspace*{1.3cm}
      $\pair(x,x) \in R \circ R^{-1}$
      \\[0.2cm]
      Setzen wir hier die Definition des relationalen Produkts ein, so sehen wir, dass
      \\[0.2cm]
      \hspace*{1.3cm}
      $\pair(x,x) \in \bigl\{ \pair(x,z) \mid \exists y: (\pair(x,y) \in R \wedge \pair(y,z) \in R^{-1}) \bigr\}$
      \\[0.2cm]
      gilt.  Daraus folgt aber sofort die Formel
      \\[0.2cm]
      \hspace*{1.3cm}
      $\exists y: (\pair(x,y) \in R \wedge \pair(y,x) \in R^{-1})$
      \\[0.2cm]
      Ber�cksichtigen wir hier wieder, dass $\pair(y,x) \in R^{-1}$ g.d.w.~$\pair(x,y) \in R$ gilt,
      so haben wir
      \\[0.2cm]
      \hspace*{1.3cm}
      $\exists y: (\pair(x,y) \in R \wedge \pair(x,y) \in R$),
      \\[0.2cm]
      was �quivalent ist zu
      \\[0.2cm]
      \hspace*{1.3cm}
      $\exists y: \pair(x,y) \in R$.
      \\[0.2cm]
      Da es sich bei dem $x$, das in dieser Formel auftritt, um ein beliebiges Element
      der Menge $M$ handelt, haben wir damit gezeigt, dass $R$ auf $M$ links-total ist.
\item Wir zeigen nun, dass $R$ links-eindeutig ist.  Es seien nun Elemente $x$, $y$ und $z$ der Menge 
      gegeben, so dass
      \\[0.2cm]
      \hspace*{1.3cm}
      $\pair(x,y) \in R$ \quad und \quad $\pair(z,y) \in R$
      \\[0.2cm]
      gilt.  Wir m�ssen zeigen, dass dann $x = z$ gelten muss.  
      Dazu beobachten wir, dass 
      \\[0.2cm]
      \hspace*{1.3cm}
      $\pair(z,y) \in R$ \quad g.d.w. \quad $\pair(y,z) \in R^{-1}$
      \\[0.2cm]
      gilt.  Nun folgt aus
      $\pair(x,y) \in R$ und $\pair(y,z) \in R^{-1}$ nach Definition des relationalen
      Produkts, dass 
      \\[-0.1cm]
      \hspace*{1.3cm}
      $\pair(x,z) \in R \circ R^{-1}$
      \\[0.2cm]
      gilt.   Aus der vorausgesetzten Gleichung $R \circ R^{-1} = \mathrm{id}_M$ folgt dann
      \\[0.2cm]
      \hspace*{1.3cm}
      $\pair(x,z) \in \mathrm{id}_M$
      \\[0.2cm]
      und nach Definition der Relation $\mathrm{id}_M$ folgt daraus $x = z$.  
\end{enumerate}
Damit ist der Beweis der Richtung ``$\Rightarrow$'' abgeschlossen.

\item[``$\Leftarrow$'']
Wir nehmen nun an, dass $R$ links-eindeutig und links-total auf $M$ ist und zeigen,
dass unter diesen Bedingungen die Gleichung 
\begin{equation}
  \label{eq:3}
  R \circ R^{-1} = \mathrm{id}_M 
\end{equation}
g�ltig ist.  Da es sich hier um eine Gleichung zwischen Mengen handelt, zerlegen wir diesen
Beweis in zwei Teile indem wir zeigen, dass sowohl
\\[0.2cm]
\hspace*{1.3cm}
$R \circ R^{-1} \subseteq \mathrm{id}_M$ \quad als auch \quad $R \circ R^{-1} \supseteq \mathrm{id}_M$ 
\\[0.2cm]
gilt.
\begin{enumerate}
\item[``$\subseteq$'']: Sei $\pair(x,z) \in R \circ R^{-1}$.  Nach Definitionen
  des relationalen Produkts hei�t dies, dass 
  \\[0.2cm]
  \hspace*{1.3cm}
  $\pair(x,z) \in \bigl\{ \pair(x,z) \mid \exists y: (\pair(x,y) \in R \wedge \pair(y,z) \in R^{-1}) \bigr\}$
  \\[0.2cm]
  gilt. Daraus folgt sofort
  \\[0.2cm]
  \hspace*{1.3cm}
  $\exists y: \bigl(\pair(x,y) \in R \wedge \pair(y,z) \in R^{-1}\bigr)$
  \\[0.2cm]
  Wegen $\pair(y,z) \in R^{-1}$ g.d.w.~$\pair(z,y) \in R$ ist dies �quivalent zu
  \\[0.2cm]
  \hspace*{1.3cm}
  $\exists y: \pair(x,y) \in R \wedge \pair(z,y) \in R$
  \\[0.2cm]
  Mit der Links-Eindeutigkeit (\ref{eq:2}) der Relation $R$ folgt daraus 
  \\[0.2cm]
  \hspace*{1.3cm}
  $x = z$.
  \\[0.2cm]
  Also haben wir
  \\[0.2cm]
  \hspace*{1.3cm}
  $\pair(x,z) = \pair(x,x) \in \mathrm{id}_M$.
\item[``$\supseteq$'']: Sei nun $\pair(x,x) \in \mathrm{id}_M$.  Wir  m�ssen $\pair(x,x)
  \in R \circ R^{-1}$ zeigen.  Nach Definition des relationalen Produkts 
  ist das �quivalent zu
  \\[0.2cm]
  \hspace*{1.3cm}
  $\pair(x,x) \in \bigl\{ \pair(x,z) \mid \exists y: (\pair(x,y) \in R \wedge \pair(y,z) \in R^{-1}) \bigr\}$
  \\[0.2cm] 
  zeigen und dies ist �quivalent zu der Formel
  \\[0.2cm]
  \hspace*{1.3cm}
  $\exists y: \bigl(\pair(x,y) \in R \wedge \pair(y,x) \in R^{-1}\bigr)$.
  \\[0.2cm]
  Mit der Definition von $R^{-1}$ kann dies umgeformt werden zu
  \\[0.2cm]
  \hspace*{1.3cm}
  $\exists y: \bigl(\pair(x,y) \in R \wedge \pair(x,y) \in R\bigr)$.
  \\[0.2cm]
  Diese Formel ist offenbar �quivalent zu der Formel
  \\[0.2cm]
  \hspace*{1.3cm}
  $\exists y: \pair(x,y) \in R$
  \\[0.2cm]
  und diese Formel folgt sofort aus der Tatsache, dass die Relation $R$ links-total ist.
\end{enumerate}
Damit ist nun auch der Beweis der Richtung ``$\Leftarrow$'' abgeschlossen und wir haben
die Behauptung vollst�ndig bewiesen.
\qed
\end{enumerate}

\end{document}

%%% Local Variables: 
%%% mode: latex
%%% TeX-master: t
%%% End: 
