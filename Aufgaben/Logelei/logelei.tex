\documentclass{article}
\usepackage{ngerman}
\usepackage[latin1]{inputenc}
\usepackage{a4wide}
\usepackage{amssymb}
\usepackage{fancyvrb}
\usepackage{alltt}

\usepackage{hyperref}
\usepackage[all]{hypcap}
\hypersetup{
	colorlinks = true, % comment this to make xdvi work
	linkcolor  = blue,
	citecolor  = red,
        filecolor  = Gold,
        urlcolor   = [rgb]{0.5, 0.1, 0.0},
	pdfborder  = {0 0 0} 
}

\def\pair(#1,#2){\langle #1, #2 \rangle}
\renewcommand{\labelenumi}{(\alph{enumi})}
%\renewcommand{\labelenumi}{\arabic{enumi}.}

\newcounter{aufgabe}

\newcommand{\qed}{\hspace*{\fill} $\Box$}
\newcommand{\next}{\vspace*{0.1cm}

\noindent}

\newcommand{\exercise}{\vspace*{0.3cm}
\stepcounter{aufgabe}

\noindent
\textbf{Aufgabe \arabic{aufgabe}}: }

\newcommand{\solution}{\vspace*{0.3cm}

\noindent
\textbf{L�sung}: }

\pagestyle{empty}

\begin{document}
\noindent
\textbf{\large Hausaufgabe: Logelei}
\vspace*{0.3cm}

\noindent
Ein Teil der Familie Meier will die Familie M�ller besuchen kommen.
Herr Meier hat drei Kinder: Walter, Katrin und Franziska.
Herr Meier teilt Herrn M�ller folgendes mit:
\begin{enumerate}
\item Wenn Herr Meier kommt, bringt er auch Frau Meier mit.
\item Mindestens eines der beiden Kinder Walter und Katrin wird kommen.
\item Entweder kommt Frau Meier oder Franziska, aber nicht beide.
\item Entweder kommen Fransizka und Katrin zusammen oder beide kommen nicht.
\item Wenn Walter kommt, dann kommen auch Katrin und Herr Meier.
\end{enumerate}
Formalisieren Sie dieses Problem mit Hilfe der Aussagenlogik und schreiben Sie ein
Programm, das die Aufgabe l�st.  Orientieren Sie sich dabei an dem in der Vorlesung
besprochenen Programm zur Aufkl�rung des Einbruchs in einem Juwelier-Gesch�ft.  Sie finden
dieses Programm im Netz unter der folgenden Adresse:
\\[0.2cm]
\hspace*{1.3cm}
\href{http://www.dhbw-stuttgart.de/stroetmann/Logic/SetlX/watson.stlx}{\texttt{http://www.dhbw-stuttgart.de/stroetmann/Logic/SetlX/watson.stl}}
\\[0.2cm]
Passen Sie dieses Programm so an, dass es die oben gestellte Aufgabe l�st.

\end{document}

%%% Local Variables: 
%%% mode: latex
%%% TeX-master: t
%%% End: 
