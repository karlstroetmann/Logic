\documentclass{article}
\usepackage{german}
\usepackage[latin1]{inputenc}
\usepackage{a4wide}
\usepackage{amssymb}
\usepackage{fancyvrb}
\usepackage{alltt}

\pagestyle{empty}

\begin{document}
\noindent
{\large \textbf{Aufgaben-Blatt}: Berechnung des Weges mit den wenigsten Zwischen-Stops}
\vspace{0.5cm}

\noindent
Gegeben sei eine bin�re Relation $R$.  Ist eine Paar 
$\langle x, y \rangle$ ein Element in $R$, so interpretieren wir dies
als eine direkte Verbindung von $x$ nach $y$.
Eine \emph{Weg-Relation} $W$ definieren wir als  eine Menge von Paaren der Form 
 $\bigl\langle \langle x, y \rangle, p\bigr\rangle$ so dass gilt:
\begin{enumerate}
\item $\langle x, y \rangle$ ist ein Paar von Punkten.
\item $p$ ist eine Liste von Punkten.  Der erste Punkt der Liste ist $x$, der letzte Punkt
      ist $y$, in \textsc{Setl2}-Notation gilt also: \\[0.1cm]
      \hspace*{1.3cm}  $x = p(1)$ \quad und \quad $y = p(\#p)$. \\[0.1cm]
      Die Liste $p$ wird interpretiert als ein \emph{Pfad}, der von $x$ nach $y$ f�hrt.
\end{enumerate}
\vspace{0.3cm}

\noindent
\textbf{Aufgabe 1}:
Definieren sie die \emph{Komposition} einer Relation $R$  mit einer Weg-Relation $W$ 
so, dass $W \circ R$ wieder eine Weg-Relation ist.  Dabei soll gelten:  
Ist $\bigl\langle \langle x, y \rangle, p \bigr\rangle \in W$ und ist
$\langle y, z \rangle \in R$, so soll die Komposition $W \circ R$ das Element 
$\bigl\langle \langle x, z \rangle, p + [z] \bigr\rangle \in W$ enthalten.
\vspace{0.3cm}

\noindent 
\textbf{Aufgabe 2}:  
Implementieren Sie eine Prozedur \texttt{compose}, so dass der Aufruf  $\mathtt{compose}(W, R)$ 
f�r eine Weg-Relation $W$ und  eine Relation $R$ die Komposition 
$W \circ R$ berechnet.
\vspace{0.3cm}

\noindent
\textbf{Aufgabe 3}: Implementieren sie eine Funktion \texttt{closure}, so
dass der Aufruf \\[0.1cm]
\hspace*{1.3cm} $\mathtt{closure}(R)$ \\[0.1cm]
zu einer gegebenen bin�ren Relation $R$ eine Weg-Relation erzeugt, die alle zyklen-freien m�glichen Verbindungen
zwischen zwei Punkten enth�lt.  
\vspace{0.3cm}

%\noindent
%\textbf{Hinweis}:
%Orientieren Sie sich dabei an dem letzten Aufgaben-Blatt.
%\vspace{0.3cm}

\noindent
\textbf{Aufgabe 4}: Entwickeln Sie eine Prozedur \texttt{minimize}, so dass der Aufruf 
 $\mathtt{minimize}(\textsl{W})$ 
aus einer gegebenen Weg-Relation $W$ alle die Paare  $\bigl\langle \langle x, y \rangle, p \bigr\rangle$ 
entfernt, f�r die die Anzahl $\symbol{35}p$ nicht minimal ist.
\vspace{0.3cm}

%\noindent
%Schicken Sie Ihre L�sung bis zum 7.~2.~18:00 an \texttt{stroetmann@ba-stuttgart.de}.


\end{document}



%%% Local Variables: 
%%% mode: latex
%%% TeX-master: t
%%% End: 
