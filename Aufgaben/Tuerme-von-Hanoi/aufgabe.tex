\documentclass{article}
\usepackage{german}
\usepackage[latin1]{inputenc}
\usepackage{hyperref}
\usepackage{a4wide}
\usepackage{amssymb}
\usepackage{fancyvrb}
\usepackage{alltt}
\hypersetup{
%		colorlinks = true, % comment this to make xdvi work
		linkcolor  = blue,
		citecolor  = red,
        filecolor  = Gold,
        urlcolor   = [rgb]{0.5, 0.1, 0.0},
		pdfborder  = {0 0 0} 
}

\newcommand{\el}{\!\in\!}


\begin{document}
\noindent
{\Large \textbf{Aufgaben-Blatt}: Die T�rme von Hanoi}
\vspace{0.5cm}

\noindent
Stellen Sie sich drei T�rme vor, einen links, einen in der Mitte und
einen rechts.  Anfangs liegen auf dem linken Turm drei Scheiben der Gr��e 1, 2 und 3,
wobei die gr��te Scheibe unten liegt, Scheibe 2 liegt in der Mitte und Scheibe 1 liegt
oben.  Die anderen beiden T�rme sind leer.
\begin{verbatim}
                                                                
                |                    |                   |           
               111                   |                   |           
              22222                  |                   |           
             3333333                 |                   |           
      ===============================================================
                                                                
\end{verbatim}
Ihre Aufgabe besteht darin, die Scheiben so zu verschieben, dass am Ende die erste Scheibe
auf dem zweiten Turm liegt, die zweite Scheibe auf dem ersten Turm und die dritte Scheibe
auf dem dritten Turm liegt.  Dabei d�rfen Sie in jedem Zug eine Scheibe
von einem Turm nehmen und auf einen anderen Turm legen, wobei Sie aber niemals
eine Scheibe $x$ auf eine andere Scheibe $y$ legen d�rfen, wenn die Scheibe $x$
gr��er als $y$ ist.
\vspace{0.3cm}

\noindent
Sie sollen die Aufgabe mit Hilfe des Ger�stes, dass Sie unter \\[0.2cm]
\hspace*{0.8cm}      
\href{http://www.dhbw-stuttgart.de/stroetmann/Logic/SetlX/hanoi-frame.stlx}{\texttt{http://www.dhbw-stuttgart.de/stroetmann/Logic/SetlX/hanoi-frame.stlx}}
\\[0.2cm]
finden, l�sen.  Bearbeiten Sie dazu die folgenden Teilaufgaben:
\begin{enumerate}
\item Definieren Sie in Zeile 100 eine Prozedur $\texttt{partition}(s_1, s_2, s_3, a)$ die genau dann 
      \texttt{true} zur�ck liefert, wenn die Menge $\{ s_1, s_2, s_3 \}$
      eine Partition der Menge $a$ ist.  
   
      \textbf{Hinweis:}
      Der Begriff der Partition ist im Skript wie folgt definiert:
      
      \textbf{Definition}:
      Ist ${\cal P} \subseteq 2^M$ eine Menge von Teilmengen von $M$, so sagen wir, dass ${\cal P}$ eine 
      \emph{Partition} von $M$ ist, falls folgende Eigenschaften gelten:
      \begin{enumerate}
      \item $\forall x \el M : \exists K \el {\cal P} : x \el K$,
        
            jedes Element aus $M$ findet sich also in einer Menge aus ${\cal P}$ wieder.
      \item $\forall K \el {\cal P} : \forall L \el {\cal P} : K \cap L =\emptyset \vee K = L$,
        
            zwei \underline{verschiedene} Mengen aus ${\cal P}$ sind also disjunkt.
      \end{enumerate}
      \textbf{Beispiele}:  Es gilt also
      \begin{enumerate}
      \item $\texttt{partition}\bigl(\{1,3\}, \{\},\{2\},\{1,2,3\}\bigr) = \texttt{true}$,

            denn jedes Element aus der Menge $\{1,2,3\}$ kommt in einer der drei Mengen,
            $S_1= \{1,3\}$, $S_2= \{\}$, $S_3= \{2\}$ genau einmal vor.
      \item $\texttt{partition}\bigl(\{1,3\}, \{1\},\{2\},\{1,2,3\}\bigr) = \texttt{false}$,

            denn das Element 1 aus der Menge $\{1,2,3\}$ kommt sowohl in der Menge
            $S_1= \{1,3\}$ als auch in der Menge $S_2= \{1\}$ vor.
      \item $\texttt{partition}\bigl(\{1\}, \{\},\{2\},\{1,2,3\}\bigr) = \texttt{false}$,

            denn das Element 3 aus der Menge $\{1,2,3\}$ kommt in keiner der Mengen
            $S_1= \{1\}$, $S_2= \{\}$, $S_3 = \{2\}$ vor.
      \end{enumerate}

\pagebreak
\item Definieren Sie in Zeile 147 eine Menge $p$ von Punkten. Jeder Punkt aus $p$ soll einem
      Zustand der drei T�rme entsprechen.  Zweckm��ig ist eine Darstellung
      der verschiedenen Zust�nde durch ein Tripel $\langle s_1, s_2, s_3 \rangle$,
      wobei die Komponenten $s_1$, $s_2$ und $s_3$ Mengen sind, die die Zust�nde
      der einzelnen T�rme repr�sentieren. 

\item Die Prozedur \texttt{movePossible}(\textsl{source}, \textsl{target}) in Zeile 114
      nimmt als Argumente zwei Mengen, die jeweils den Zustand eines Turms repr�sentieren.
      Die Funktion liefert das Ergebnis \texttt{true}, wenn die oberste Scheibe von dem 
      Turm \textsl{source} auf den Turm \textsl{target} gelegt werden darf.

\item Definieren Sie jetzt eine Relation $r_{12}$ auf der Menge $p$.
      Ein Paar $\langle a, b \rangle$ soll genau dann in $R_{12}$ liegen, wenn
      der Zustand $b$ aus dem Zustand $a$ dadurch hervorgeht, dass im Zustand $a$ 
      die oberste Scheibe von dem Turm 1 auf den Turm 2 gelegt wird.  Dabei soll nat�rlich
      darauf geachtet werden, dass keine gro�e Scheibe auf eine kleinere Scheibe gelegt
      wird.

\item Definieren Sie analog zu der letzten Teilaufgabe die f�nf Relationen $r_{21}$,
      $r_{31}$, $r_{13}$, $r_{23}$, $r_{32}$ so, dass die Relation \\[0.2cm]
      \hspace*{1.3cm} 
      $r :=  r_{12} \cup r_{21} \cup r_{31} \cup r_{13} \cup r_{23} \cup r_{32}$ 
      \\[0.2cm]
      in Zeile 157 alle erlaubten Zustands-�bergange beschreibt. 

      Wenn Sie statt einer Copy-und-Paste-L�sung eine besonders elegante L�sung
      implementieren wollen, dan implementieren Sie in Zeile 82 die Prozedur
      $\texttt{relation}(p, i, j)$, die aus der Menge $p$ aller Zust�nde f�r zwei Indizes
      $i,j \in \{1,2,3\}$ die Relation $r_{ij}$ berechnet.  In diesen Fall m�ssen Sie
      Zeile einkommentieren und Zeile 159 auskommentieren.
\item Definieren Sie nun die Variablen \texttt{start} und \texttt{goal} so, dass
      \texttt{start} die Situation beschreibt, bei der alle Scheiben auf dem linken Turm
      liegen, w�hrend \texttt{goal} die Situation beschreibt, bei die erste Scheibe
      auf dem zweiten Turm liegt, die zweite Scheibe auf dem ersten Turm und die dritte Scheibe
      auf dem rechten Turm liegt.
\end{enumerate}

\end{document}

%%% Local Variables: 
%%% mode: latex
%%% TeX-master: t
%%% End: 
