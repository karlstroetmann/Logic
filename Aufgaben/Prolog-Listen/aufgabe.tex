\documentclass{article}
\usepackage{german}
\usepackage[latin1]{inputenc}
\usepackage{a4wide}
\usepackage{times}
\usepackage{oldgerm}
\pagestyle{empty}
\usepackage{amssymb}


\begin{document}
\noindent
{\Large \textsl{Prolog}: Pr�dikate auf Listen}


\noindent
\rule{15cm}{0.5pt}
      \vspace{0.3cm}

\noindent
Bearbeiten Sie die folgenden Aufgaben:
\begin{enumerate}
\item Schreiben sie ein Pr�dikat \texttt{myMember}, das der Typ-Spezifikation \\[0.1cm]
      \hspace*{1.3cm} \texttt{myMember(+\textsl{Number}, +\textsl{List}(\textsl{Number}))} \\[0.1cm]
      entspricht.  Der Aufruf  \\[0.1cm]
      \hspace*{1.3cm} $\texttt{myMember}(x, l)$ \\[0.1cm]
      soll genau dann erfolgreich sein, wenn die Zahl $x$ in der Liste $l$ auftritt.

      Bei allen weiteren Aufgaben sollen Sie zun�chst Gleichungen aufstellen,
      die das Verhalten der zu implementierenden Funktionen beschreiben.

\item Schreiben Sie ein Pr�dikat \texttt{intersect} das der Typ-Spezifikation \\[0.1cm]
      \hspace*{1.3cm} 
      \texttt{intersect(+\textsl{List}(\textsl{Number}), +\textsl{List}(\textsl{Number}), -\textsl{List}(\textsl{Number}))} 
      \\[0.1cm]
      entspricht.  Der Aufruf  \\[0.1cm]
      \hspace*{1.3cm}  $\mathtt{intersect}(l_1, l_2, \mathtt{L})$ \\[0.1cm]
      soll f�r zwei Listen $l_1$ und $l_2$ eine Liste $l$ berechnen, die alle die Elemente
      enth�lt, die sowohl in $l_1$ als auch in $l_2$ auftreten.
\item Schreiben Sie ein Pr�dikat \texttt{small}, das mit der Typ-Spezifikation \\[0.1cm]
      \hspace*{1.3cm} 
      \texttt{small(+\textsl{Number}), +\textsl{List}(\textsl{Number}), -\textsl{List}(\textsl{Number}))} 
      \\[0.1cm]
      vertr�glich ist.  Der Aufruf  \\[0.1cm]
      \hspace*{1.3cm}  $\mathtt{small}(x, l, \mathtt{S})$ \\[0.1cm]
      soll f�r eine Zahl $x$  und eine Liste von Zahlen $l$ die Liste aller der Zahlen
      aus $l$ berechnen, die kleiner oder gleich $x$ sind.
\item Schreiben Sie ein Pr�dikat \texttt{big}, das mit der Typ-Spezifikation \\[0.1cm]
      \hspace*{1.3cm} 
      \texttt{big(+\textsl{Number}), +\textsl{List}(\textsl{Number}), -\textsl{List}(\textsl{Number}))} 
      \\[0.1cm]
      vertr�glich ist.  Der Aufruf  \\[0.1cm]
      \hspace*{1.3cm}  $\mathtt{big}(x, l, \mathtt{S})$ \\[0.1cm]
      soll f�r eine Zahl $x$  und eine Liste von Zahlen $l$ die Liste aller der Zahlen
      aus $l$ berechnen, die gr��er als $x$ sind.
\item Schreiben Sie ein Pr�dikat \texttt{quick\_sort}, das mit der Typ-Spezifikation \\[0.1cm]
      \hspace*{1.3cm} 
      \texttt{quick\_sort(+\textsl{List}(\textsl{Number}), -\textsl{List}(\textsl{Number}))} 
      \\[0.1cm]
      vertr�glich ist.  Der Aufruf  \\[0.1cm]
      \hspace*{1.3cm}  $\mathtt{quick\_sort}(l, \mathtt{L})$ \\[0.1cm]
      soll die Liste $l$ sortieren.

      Das Pr�dikat \texttt{quick\_sort} soll nach der \emph{divide-and-conquer}-Methode
      arbeiten:
      \begin{enumerate}
      \item Teilen Sie die zu sortierende Liste $l$ zun�chst mit den Pr�dikaten
            \texttt{small/3} und \texttt{big/3} in zwei Listen $s$ und $b$ auf.
            Die  Liste $s$ soll dabei alle Elemente enthalten, die kleiner als
            das erste Element der Liste $l$ sind, w�hrend die Liste $b$ die Elemente
            aus $l$ enth�lt, die gr��er als das erste Element von $l$ sind.
      \item Sortieren Sie die Listen $s$ und $b$.
      \item Fassen Sie die beiden sortierten Listen zu einer sortierten Liste
            zusammen.  Benutzen Sie dazu das Pr�dikat \texttt{concat} aus der Vorlesung.
      \end{enumerate}
\end{enumerate}

\end{document}

%%% Local Variables: 
%%% mode: latex
%%% TeX-master: uebung-2
%%% End: 
