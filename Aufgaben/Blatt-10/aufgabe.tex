\documentclass{article}
\usepackage{german}
\usepackage[latin1]{inputenc}
\usepackage{a4wide}
\usepackage{amssymb}
\usepackage{fancyvrb}
\usepackage{alltt}
\usepackage{epsfig}
\usepackage{hyperref}
\usepackage{fancyhdr}
\usepackage{lastpage} 
\usepackage{color}
\hypersetup{
	colorlinks = true, % comment this to make xdvi work
	linkcolor  = blue,
	citecolor  = red,
        filecolor  = blue,
        urlcolor   = [rgb]{0.5, 0.4, 0.0},
	pdfborder  = {0 0 0} 
}

\renewcommand*{\familydefault}{\sfdefault}

\pagestyle{fancy}

\fancyfoot[C]{--- \thepage/\pageref{LastPage}\ ---}
 
\def\pair(#1,#2){\langle #1, #2 \rangle}
%\renewcommand{\labelenumi}{(\alph{enumi})}
\renewcommand{\labelenumi}{\arabic{enumi}.}


\begin{document}
\noindent
\textbf{\large Aufgabe: \quad \emph{Wem geh�rt das Zebra?}}
\vspace*{0.3cm}

\noindent
Schreiben Sie ein \textsc{SetlX}-Programm, welches das folgende R�tsel l�st.
\begin{enumerate}
\item Es gibt 5 H�user. Jedes der H�user hat eine andere Farbe.
\item In jedem Haus wohnt ein Bewohner einer anderen Nationalit�t.
\item Jeder Hausbewohner bevorzugt ein anderes Getr�nk, raucht eine andere
      Marke Zigaretten und h�lt genau ein Haustier.
\item \underline{Keine} der Personen trinkt das Gleiche, raucht das Gleiche 
      oder h�lt die gleiche Art Haustier. 
\end{enumerate}
Au�erdem wissen wir folgendes:
\begin{enumerate}
\item Der Brite wohnt im roten Haus.
\item Der Schwede h�lt einen Hund.
\item Der Amerikaner trinkt Whisky. 
\item Das gr�ne Haus steht links vom wei�en Haus.
\item Der Besitzer vom gr�nen Haus trinkt Kaffee.
\item Die Person, die Camel raucht, h�lt einen Papagei.
\item Der Mann, der im mittleren Haus wohnt, trinkt Milch.
\item Der Besitzer vom gelben Haus raucht Dunhill.
\item Der Norweger wohnt im ersten Haus.
\item Der Marlboro-Raucher wohnt neben dem, der eine Katze h�lt.
\item Der Mann, der ein Schwein h�lt, wohnt neben dem, der Dunhill raucht.
\item Der Winfieldraucher trinkt gerne Bier.
\item Der Norweger wohnt neben dem blauen Haus.
\item Der Deutsche raucht Rothmanns.
\item Der Marlboro-Raucher hat einen Nachbarn, der Wasser trinkt.
\end{enumerate}
Die Frage lautet nun: Wem geh�rt das Zebra?
\vspace{0.5cm}


\noindent
 Verwenden Sie zur L�sung des Problems die Vorlage, die Sie im Netz unter der Adresse
\\[0.2cm]
\hspace*{0.3cm}
\href{https://github.com/karlstroetmann/Logik/blob/master/Aufgaben/Blatt-10/zebra-frame.stlx}{\texttt{github.com/karlstroetmann/Logik/blob/master/Aufgaben/Blatt-10/zebra-frame.stlx}} 
\\[0.2cm]
finden.  Dieser Rahmen enth�lt bereits eine Implementierung der Funktion
\\[0.2cm]
\hspace*{1.3cm}
$\texttt{parseKNF}(s)$,
\\[0.2cm]
die eine Formel, die als String $s$ gegeben ist, in eine Menge von Klauseln umwandelt.  Weiter
enth�lt der Rahmen die Implementierung der Funktion
$\texttt{davisPutnam}$,
die f�r eine gegebene Menge von Klauseln eine L�sung berechnet.  Diese Funktion wird von
der bereits vorgegebenen Funktion \texttt{solve} verwendet.

\pagebreak
\noindent
Bei der L�sung dieser Aufgabe sollten Sie die folgenden Terme als aussagenlogische Variablen
verwenden:
\begin{enumerate}
\item $\texttt{Briton}(i)$ mit $i \in \{1,\cdots,5\}$.  Diese Variable dr�ckt aus, dass der Brite
      im Haus mit der Nummer $i$ wohnt.  Die anderen Nationalit�ten werden mit Hilfe der Variablen
      $\texttt{German}(i)$, $\texttt{Swede}(i)$, $\texttt{American}(i)$ und $\texttt{Norwegian}(i)$ kodiert.
\item $\mathtt{Red}(i)$ dr�ckt aus, dass das Haus mit der Nummer $i$ rot ist.
      F�r die anderen Farben benutzen Sie die Variablen
      $\texttt{Green}(i)$, $\texttt{White}(i)$, $\texttt{Blue}(i)$ und $\texttt{Yellow}(i)$.
\item $\texttt{Dunhill}(i)$ dr�ckt aus, dasss der Bewohner des $i$-ten Hauses Dunhill raucht.
      F�r die anderen Zigaretten-Marken benutzen Sie die Variablen $\texttt{Camel}(i)$, 
      $\texttt{Marlboro}(i)$, $\texttt{Winfield}(i)$ und $\texttt{Rothmanns}(i)$.
\item $\texttt{Dog}(i)$ dr�ckt aus, dass der Bewohner des $i$-ten Hauses einen Hund hat.
      F�r die anderen Haustiere benutzen Sie die Variablen
      $\texttt{Parrot}(i)$, $\texttt{Cat}(i)$, $\texttt{Pig}(i)$ und $\texttt{Zebra}(i)$.
\item $\texttt{Whiskey}(i)$ sagt aus, dass der Bewohner des $i$-ten Hauses Whiskey trinkt.
      F�r die anderen Getr�nke benutzen Sie die Variablen
      $\texttt{Coffee}(i)$, $\texttt{Beer}(i)$, $\texttt{Milk}(i)$ und $\texttt{Water}(i)$.
\end{enumerate}

\noindent
Zur L�sung des Problems sind die folgenden Teilaufgaben zu bearbeiten:
\begin{enumerate}
\item Implementieren Sie eine Funktion \texttt{onePerHouse} die beispielsweise in der Form
      \\[0.2cm]
      \hspace*{1.3cm}
      $\texttt{onePerHouse}(\{\symbol{34}\mathtt{German}\symbol{34},
       \symbol{34}\mathtt{Briton}\symbol{34}, 
       \symbol{34}\mathtt{Swede}\symbol{34}, \symbol{34}\mathtt{American}\symbol{34}, 
       \symbol{34}\mathtt{Norwegian}\symbol{34}\})$
      \\[0.2cm]
      aufgerufen wird und die ausdr�ckt, dass es ein Haus gibt, in dem der Deutsche wohnt,
      ein Haus, in dem der Britte wohnt, etc.   Au�erdem soll diese Formel noch ausdr�cken, 
      dass in jedem Haus nur eine Person wohnt.  Wenn also beispielsweise der Deutsche im Haus
      Nummer 4 wohnt, dann kann dort sonst niemand mehr wohnen.

      Es ist sinnvoll, wenn Sie sich geeignete Hilfsprozeduren definieren, mit denen Sie diese 
      Prozedur implementieren k�nnen.   In dem vorgegebenen Rahmen ist bereits die Prozedur \texttt{somewhere}
      vordefiniert, die ausdr�ckt, dass jeder irgendwo wohnen muss.  
      Sie sollten noch die Prozeduren \texttt{someone}  und \texttt{atMostOneAt} definieren.
      Die Prozedur \texttt{someone} dr�ckt aus, dass in jedem Haus jemand wohnt, w�hrend \texttt{atMostOneAt} 
      ausdr�ckt, dass in jedem Haus h�chstens einer wohnt.

      Beachten Sie hier, dass die Prozeduren \texttt{atMostOne} und \texttt{propVar} in dem 
      vorgegebenen Rahmen bereits definiert sind.
\item Die Prozedur \texttt{sameHouse}, mit der Sie ausdr�cken k�nnen, dass beispielsweise der Brite 
      in dem roten Haus wohnt, ist bereits vorgegeben und zeigt, wie Sie mit Hilfe von String-Interpolation 
      Formeln berechnen k�nnen.
\item \texttt{nextTo} kann beispielsweise in der Form \texttt{nextTo(\symbol{34}Marlboro\symbol{34}, \symbol{34}Cat\symbol{34})}
      aufgerufen werden um auszudr�cken, dass derjenige, der Marlboro raucht, neben dem Haus mit der Katze wohnt.
      Die Prozedur \texttt{nextTo} muss von Ihnen implementiert werden.
\item \texttt{allClauses} berechnet eine Menge von Klauseln, die zusammen genau der
      Problem-Beschreibung entsprechen.  Wenn Sie diese Prozedur korrekt implementieren,
      wird das Problem durch den Aufruf der Funktion \texttt{solve} gel�st.  Die
      Berechnung der L�sung dauert auf meinem Rechner weniger als drei Sekunden.
\end{enumerate}


\end{document}

%%% Local Variables: 
%%% mode: latex
%%% TeX-master: t
%%% End: 
