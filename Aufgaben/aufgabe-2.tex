\documentclass{article}
\usepackage{german}
\usepackage[latin1]{inputenc}
\usepackage{a4wide}
\usepackage{amssymb}
\usepackage{fancyvrb}
\usepackage{alltt}

\pagestyle{empty}

\begin{document}
\noindent
{\Large \textbf{Aufgaben-Blatt}: Berechnung von erreichbaren Punkten}
\vspace{0.5cm}

\noindent
\textbf{Hinweis}: Versuchen Sie bei der L�sung der nachfolgenden Aufgaben
m�glichst mit Mengen-Konstruktionen  und nicht mit Kontroll-Strukturen wie
 \texttt{for} oder \texttt{while} zu arbeiten.
\vspace{0.3cm}

\noindent
\textbf{Aufgabe 1}:  Gegeben sei eine bin�re Relation $R$ auf einer Menge $M$.
Falls $\langle a, b \rangle \in R$ ist, sagen wir, dass es eine direkte Verbindung von dem
Punkt $a$ zu dem Punkt $b$ gibt.  Implementieren Sie eine Prozedur \texttt{oneStep},
so dass der Aufruf \\[0.1cm]
\hspace*{1.3cm} $\mathtt{oneStep}(x,R)$ \\[0.1cm]
f�r einen Punkt $x$ und eine bin�re Relation $R$ alle die Punkte berechnet,
die von $x$ in einem Schritt erreicht werden k�nnen.

Implementieren Sie nun eine Prozedur \texttt{oneStepSet},
so dass der Aufruf \\[0.1cm]
\hspace*{1.3cm} $\mathtt{oneStepSet}(M,R)$ \\[0.1cm]
f�r eine Menge von  Punkten $M$ und eine bin�re Relation $R$ alle die Punkte berechnet,
die von einem Punkt $x\in M$  in einem Schritt erreicht werden k�nnen.

\vspace{0.3cm}

\noindent
\textbf{Aufgabe 2}:  Gegeben sei wieder eine bin�re Relation $R$ auf einer Menge $M$.
 Implementieren Sie unter \underline{Benutzun}g\underline{ von Auf}g\underline{abe 1} eine Prozedur \texttt{reachable},
so dass der Aufruf \\[0.1cm]
\hspace*{1.3cm} $\mathtt{reachable}(x,R)$ \\[0.1cm]
f�r einen Punkt $x$ und eine bin�re Relation $R$ alle die Punkte berechnet,
die von $x$ nach beliebig vielen Schritten erreicht werden k�nnen. 

\noindent
\textbf{Hinweis}: Verwenden Sie f�r die L�sung einen Fixpunkt-Algorithmus, der
die folgende Gleichung l�st: \\[0.1cm]
\hspace*{1.3cm}  $M = M + \mathtt{oneStepSet}(M,R)$.
\vspace{0.3cm}

\noindent
\textbf{Aufgabe 3}:  Gegeben ist nun  eine  \emph{Abstandsfunktion} $D$.  
Eine \emph{Abstandsfunktion} ist, wie in der Vorlesung, eine Menge von Elementen der Form
$\bigl\langle \langle a, b \rangle, l \bigr\rangle$, wobei $x$ und $y$ Punkte sind und $l$ den
Abstand zwischen diesen Punkten bezeichnet.  Implementieren Sie in Analogie zu Aufgabe
1 eine Funktion \\[0.1cm]
\hspace*{1.3cm} $\mathtt{oneStepSet}(M, R)$, \\[0.1cm]
die als Ergebnis eine Menge $E$ von Paaren der Form $\langle y, l\rangle$ berechnet.
Das Paar $\langle y, l\rangle$ soll genau dann in $E$ liegen, wenn es eine direkte
Verbindung von einem Punkt $x\in M$ zu dem Punkt $y$ gibt, so dass diese Verbindung die L�nge $l$ hat.
\vspace{0.3cm}

\noindent
\textbf{Aufgabe 4}: Gegeben ist wieder eine  \emph{Abstandsfunktion} $D$.  
 Implementieren Sie unter Benutzung von Aufgabe 3 eine Prozedur \texttt{reachable},
so dass der Aufruf \\[0.1cm]
\hspace*{1.3cm} $\mathtt{reachable}(x,D)$ \\[0.1cm]
f�r einen Punkt $x$ und eine Abstandsfunktion $D$ eine Menge $E$ von Paaren der Form $\langle y, l\rangle$ berechnet.
Das Paar $\langle y, l\rangle$ soll genau dann in $E$ liegen, wenn es eine 
Verbindung von $x$ nach $y$ mit beliebig vielen Zwischenschritten gibt,  die insgesamt
die L�nge $l$ hat und wenn es au�erdem keine k�rzere Verbindung gibt.
\vspace{0.3cm}

\noindent
Die folgende Aufgabe ist als Hausaufgabe f�r alle diejenigen gedacht,
die etwas sportlichen Ehrgeiz besitzen.
\vspace{0.1cm}

\noindent
\textbf{Aufgabe 5:} 
�ndern Sie das \underline{in Auf}g\underline{abe 4 erstellte Pro}g\underline{ramm} so ab, dass neben der Entfernung
auch noch ein k�rzester Weg berechnet wird.  Schicken Sie das erstellte Programm dann an meine
Email-Adresse: \texttt{stroetmann\symbol{64}ba-stuttgart.de}.  Die beste Einsendung 
wird mit einer Flasche Sekt pr�miert.
\vspace{0.1cm}

\noindent
\textbf{Hinweis}: Nat�rlich k�nnen Sie sich bei der L�sung an unserem Vorgehen bei der
Berechnung des transitiven Abschlusses orientieren und �hnlich wie dort eine Pfad-Relation
definieren.  Sie sollen aber \textbf{nicht} den transitiven Abschluss der Relation
berechnen, denn das ist zu aufwendig.
\end{document}

%%% Local Variables: 
%%% mode: latex
%%% TeX-master: t
%%% End: 
