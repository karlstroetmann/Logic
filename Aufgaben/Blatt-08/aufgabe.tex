\documentclass{article}
\usepackage{german}
\usepackage[latin1]{inputenc}
\usepackage{a4wide}
\usepackage{amssymb}
\usepackage{fancyvrb}
\usepackage{alltt}
\usepackage{epsfig}
\usepackage{hyperref}
\usepackage{fancyhdr}
\usepackage{lastpage} 
\usepackage{color}
\usepackage{enumerate}
\hypersetup{
	colorlinks = true, % comment this to make xdvi work
	linkcolor  = blue,
	citecolor  = red,
        filecolor  = blue,
        urlcolor   = [rgb]{1.0, 0.0, 0.0},
	pdfborder  = {0 0 0} 
}

\renewcommand*{\familydefault}{\sfdefault}

\pagestyle{fancy}

\fancyfoot[C]{--- \thepage/\pageref{LastPage}\ ---}
 
\def\pair(#1,#2){\langle #1, #2 \rangle}
%\renewcommand{\labelenumi}{(\alph{enumi})}
\renewcommand{\labelenumi}{\arabic{enumi}.}

\newcommand{\blue}[1]{{\color{blue}#1}}

\begin{document}
\noindent
\textbf{\large Aufgabe: \quad \emph{Eine Logelei}}
\vspace*{0.3cm}

\noindent
Die folgende Aufgabe ist dem Buch 
\\[0.2cm]
\hspace*{1.3cm}
\href{https://www.amazon.de/Logeleien-Zweistein-ihren-Antworten-Wegner/dp/B006YF0VUE}{99 Logeleien von Zweistein}
\\[0.2cm]
entnommen, das 1968 von 
\href{http://de.wikipedia.org/wiki/Thomas_von_Randow}{\textsl{Thomas von Randow}} unter dem
Pseudonym ``Zweistein'' im Verlag  
\textsl{Christian Wegner} ver�ffentlicht worden
ist.  
\vspace*{0.3cm}

\begin{minipage}[c]{0.9\linewidth}
Die Herren Amann, Bemann, Cemann und Demann heissen --- nicht unbedingt      
in derselben Reihenfolge --- mit Vornamen Erich, Fritz, Gustav und Heiner. 
Sie sind alle mit genau einer Frau verheiratet.  Au�erdem wissen wir �ber sie und ihre Ehefrauen 
noch dies: 
\begin{enumerate}
\item Entweder ist Amanns Vorname Heiner, oder Bemanns Frau heisst Inge. 
\item Wenn Cemann mit Josefa verheiratet ist, dann --- und nur in diesem Falle --- heisst Klaras
      Mann nicht Fritz.                              
\item Wenn Josefas Mann nicht Erich heisst, dann ist Inge mit Fritz verheiratet. 
\item Wenn Luises Mann Fritz heisst, dann ist der Vorname von Klaras Mann nicht Gustav. 
\item Wenn die Frau von Fritz Inge heisst, dann ist Erich nicht mit Josefa verheiratet. 
\item Wenn Fritz nicht mit Luise verheiratet ist, dann heisst Gustavs Frau Klara. 
\item Entweder ist Demann mit Luise verheiratet, oder Cemann heisst Gustav. 
\end{enumerate}
Wie heissen die Herren mit vollem Namen, wie ihre Ehefrauen mit Vornamen?  
\end{minipage}
\vspace*{0.3cm}

\noindent
 Verwenden Sie zur L�sung des Problems die Vorlage, die Sie im Netz unter der Adresse
\noindent
\\[0.2cm]
\href{https://github.com/karlstroetmann/Logik/blob/master/Aufgaben/Blatt-08/logelei-frame.stlx}{\texttt{github.com/karlstroetmann/Logik/blob/master/Aufgaben/Blatt-08/logelei-frame.stlx}} 
\\[0.2cm]
finden.  Dieser Rahmen enth�lt (�ber geeignete \texttt{load}-Befehle) die Implementierung der Funktion
\\[0.2cm]
\hspace*{1.3cm}
$\texttt{davisPutnam}(\textsl{Clauses}, \textsl{Literals})$,
\\[0.2cm]
die f�r eine gegebene Menge von Klauseln eine \blue{L�sung} berechnet.
Eine L�sung ist dabei eine
aussagenlogische Belegung, die allerdings als Menge sogenannter \blue{Unit-Literale} dargestellt wird.
Beispielsweise wird die Belegung
\\[0.2cm]
\hspace*{1.3cm}
$\bigl\{ \pair(p,\mathtt{true}), \pair(q, \mathtt{false}), \pair(r, \mathtt{true}) \bigr\}$
\\[0.2cm]
als die Menge
\\[0.2cm]
\hspace*{1.3cm}
$\bigl\{ \{p\}, \{\neg q\}, \{r\} \bigr\}$
\\[0.2cm]
dargestellt.  Beim Aufruf der Funktion
$\texttt{davisPutnam}$ gilt f�r den Parameter $\textsl{Literals}$ immer $\textsl{Literals} = \{\}$.  Der
Wert des Parameter $\textsl{Literals}$ ist nur bei rekursiven Aufrufen von $\{\}$ verschieden.
Au�erdem steht eine Funktion
\\[0.2cm]
\hspace*{1.3cm}
$\texttt{normalize}(F)$
\\[0.2cm]
zur Verf�gung, die eine aussagenlogische Formel $F$ in konjunktive Normalform �berf�hrt.
\pagebreak

\noindent
Bei der L�sung dieser Aufgabe sollten Sie die folgenden Terme als aussagenlogische Variablen
verwenden:
\begin{enumerate}
\item $\texttt{@Name}(x, y)$ ist genau dann war, wenn  $x$ der Vorname und $y$ der Nachname einer
      der Herren ist.  Beispielsweise ist
      \\[0.2cm]
      \hspace*{1.3cm}
      $\texttt{@Name(\symbol{34}Heiner\symbol{34}, \symbol{34}Amann\symbol{34})}$
      \\[0.2cm]
      genau dann wahr, wenn Herr Amann mit Vornamen \textsl{Heiner} heisst.
\item $\texttt{@Ehe}(x, y)$ ist genau dann wahr, wenn der Mann, der mit Vornamen $x$ heisst, mit der
      Frau, die mit Vornamen $y$ heisst, verheiratet ist.  Beispielsweise w�re
      \\[0.2cm]
      \hspace*{1.3cm}
      $\texttt{@Ehe(\symbol{34}Heiner\symbol{34}, \symbol{34}Luise\symbol{34})}$
      \\[0.2cm]
      genau dann wahr, wenn Heiner mit Luise verheiratet ist. 
\end{enumerate}


\noindent
Zur L�sung des Problems sollten Sie die folgenden Teilaufgaben bearbeiten:
\begin{enumerate}[(a)]
\item Implementieren Sie eine Funktion \texttt{atMostOne}.  F�r eine Menge $S$ von aussagenlogischen Variablen
      soll der Aufruf $\texttt{atMostOne}(S)$ eine Menge von Klauseln berechnen, die ausdr�ckt, dass h�chstens eine der Variablen
      in $S$ wahr ist.
\item Implementieren Sie eine Funktion \texttt{atLeastOne}.  F�r eine Menge $S$ von aussagenlogischen Variablen
      soll der Aufruf $\texttt{atLeastOne}(S)$ eine Menge von Klauseln berechnen, die ausdr�ckt, dass mindestens eine
      der Variablen  in $S$ wahr ist.
\item Implementieren Sie eine Funktion \texttt{exactlyOne}.  F�r eine Menge $S$ von aussagenlogischen Variablen
      soll der Aufruf $\texttt{exactlyOne}(S)$ eine Menge von Klauseln berechnen, die ausdr�ckt, dass genau eine
      der Variablen  in $S$ wahr ist.
\item Implementieren Sie eine Funktion \texttt{bijective}.
      Die Funktion \texttt{bijective} wird in der Form
      \\[0.2cm]
      \hspace*{1.3cm}
      $\texttt{bijective}(A, B, \textsl{fct})$
      \\[0.2cm]
      aufgerufen.  Hierbei sind $A$ und $B$ Mengen von Strings.  Beispielsweise
      k�nnte $A$ die Menge der m�nnlichen Vornamen sein und $B$ k�nnte die Menge der weiblichen Vornamen sein.
      $f$ ist der Name eines Funktors.  Beispielsweise k�nnte $f$ den Wert \texttt{\symbol{34}Ehe\symbol{34}}
      sein.  Die Funktion \texttt{bijective} berechnet eine Menge von Klauseln, die zu der folgenden Formel �quivalent ist:
      \\[0.2cm]
      \hspace*{1.3cm}
      $\bigl(\forall x \in A: \exists! y \in B: f(x,y)\bigr) \wedge
       \bigl(\forall y \in B: \exists! x \in A: f(x,y)\bigr)
      $
\item Beachten Sie, dass bereits eine Funktion \texttt{isWifeOf} vorhanden ist.   Der Aufruf
      \\[0.2cm]
      \hspace*{1.3cm}
      $\texttt{isWifeOf}(y, z)$
      \\[0.2cm]
      berechnet eine Formel, die ausdr�ckt, dass $y$ die Frau von $z$ ist.  Hierbei ist $z$ der
      Nachname eines Mannes.  Diese Formel wird als String dargestellt.  
\item Vervollst�ndigen Sie nun die Implementierung der Funktion \texttt{computeClauses}.  Wenn Sie
      alles richtig gemacht haben, berechnet der Aufruf der Funktion \texttt{solve} die L�sung des
      Problems.  Diese L\"osung wird dann mit Hilfe der vordefinierten Funktion
      \texttt{displaySolution} in lesbarer Form ausgegeben.
\end{enumerate}



\end{document}

%%% Local Variables: 
%%% mode: latex
%%% TeX-master: t
%%% ispell-local-dictionary: "deutsch8"
%%% End: 
