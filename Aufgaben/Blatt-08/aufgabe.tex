\documentclass{article}
\usepackage{german}
\usepackage[latin1]{inputenc}
\usepackage{a4wide}
\usepackage{amssymb}
\usepackage{fancyvrb}
\usepackage{alltt}
\usepackage{epsfig}
\usepackage{hyperref}
\usepackage{fancyhdr}
\usepackage{lastpage} 
\usepackage{color}
\usepackage{enumerate}
\hypersetup{
	colorlinks = true, % comment this to make xdvi work
	linkcolor  = blue,
	citecolor  = red,
        filecolor  = blue,
        urlcolor   = [rgb]{0.5, 0.4, 0.0},
	pdfborder  = {0 0 0} 
}

\renewcommand*{\familydefault}{\sfdefault}

\pagestyle{fancy}

\fancyfoot[C]{--- \thepage/\pageref{LastPage}\ ---}
 
\def\pair(#1,#2){\langle #1, #2 \rangle}
%\renewcommand{\labelenumi}{(\alph{enumi})}
\renewcommand{\labelenumi}{\arabic{enumi}.}


\begin{document}
\noindent
\textbf{\large Aufgabe: \quad \emph{Eine Logelei}}
\vspace*{0.3cm}

\noindent
Die folgende Aufgabe ist dem Buch 
\\[0.2cm]
\hspace*{1.3cm}
\emph{99 Logeleien von Zweistein}
\\[0.2cm]
entnommen, das 1968 von 
\href{http://de.wikipedia.org/wiki/Thomas_von_Randow}{\textsl{Thomas von Randow}} unter dem
Pseudonym ``Zweistein'' im Verlag  
\textsl{Christian Wegner} ver�ffentlicht worden
ist.  
\vspace*{0.3cm}

\begin{minipage}[c]{0.9\linewidth}
Die Herren Amann, Bemann, Cemann und Demann heissen --- nicht unbedingt      
in derselben Reihenfolge --- mit Vornamen Erich, Fritz, Gustav und Heiner. 
Sie sind alle verheiratet. Ausserdem weiss 
man �ber sie und ihre Ehefrauen 
noch dies: 
\begin{enumerate}
\item Entweder ist Amanns Vorname Heiner, oder Bemanns Frau heisst Inge. 
\item Wenn Cemann mit Josefa verheiratet ist, dann --- und nur in diesem Falle --- heisst Klaras
      Mann nicht Fritz.                              
\item Wenn Josefas Mann nicht Erich heisst, dann ist Inge mit Fritz verheiratet. 
\item Wenn Luises Mann Fritz heisst, dann ist der Vorname von Klaras Mann nicht Gustav. 
\item Wenn die Frau von Fritz Inge heisst, dann ist Erich nicht mit Josefa verheiratet. 
\item Wenn Fritz nicht mit Luise verheiratet ist, dann heisst Gustavs Frau Klara. 
\item Entweder ist Demann mit Luise verheiratet, oder Cemann heisst Gustav. 
\end{enumerate}
Wie heissen die Herren mit vollem Namen, wie ihre Ehefrauen mit Vornamen?  
\end{minipage}
\vspace*{0.3cm}

\noindent
 Verwenden Sie zur L�sung des Problems die Vorlage, die Sie im Netz unter der Adresse
\noindent
\\[0.2cm]
\hspace*{0.3cm}
\href{https://github.com/karlstroetmann/Logik/blob/master/Aufgaben/Blatt-08/logelei-frame.stlx}{\texttt{github.com/karlstroetmann/Logik/blob/master/Aufgaben/Blatt-08/logelei-frame.stlx}} 
\\[0.2cm]
finden.  Dieser Rahmen enth�lt bereits eine Implementierung der Funktion
\\[0.2cm]
\hspace*{1.3cm}
$\texttt{parseKNF}(s)$,
\\[0.2cm]
die eine Formel, die als String $s$ gegeben ist, in eine Menge von Klauseln umwandelt.  Weiter
enth�lt der Rahmen die Implementierung der Funktion
\\[0.2cm]
\hspace*{1.3cm}
$\texttt{davisputnam}(\textsl{clauses}, \textsl{literals})$,
\\[0.2cm]
die f�r eine gegebene Menge von Klauseln eine L�sung berechnet.  Wenn Sie diese Funktion verwenden,
sollten Sie f�r den Parameter $\textsl{literals}$ die leere Menge einsetzen.

Bei der L�sung dieser Aufgabe sollten Sie die folgenden Terme als aussagenlogische Variablen
verwenden:
\begin{enumerate}
\item $\texttt{@Name}(x, y)$ ist genau dann war, wenn der $x$ der Vorname und $y$ der Nachname einer
      der Herren ist.  Beispielsweise ist
      \\[0.2cm]
      \hspace*{1.3cm}
      $\texttt{@Name(\symbol{34}Heiner\symbol{34}, \symbol{34}Amann\symbol{34})}$
      \\[0.2cm]
      genau dann wahr, wenn Herr Amann mit Vornamen \textsl{Heiner} heisst.
\item $\texttt{@Ehe}(x, y)$ ist genau dann wahr, wenn der Mann, der mit Vornamen $x$ heisst, mit der
      Frau, die mit Vornamen $y$ heisst, verheiratet ist.  Beispielsweise w�re
      \\[0.2cm]
      \hspace*{1.3cm}
      $\texttt{@Ehe(\symbol{34}Heiner\symbol{34}, \symbol{34}Luise\symbol{34})}$
      \\[0.2cm]
      genau dann wahr, wenn Heiner mit Luise verheiratet ist. 
\end{enumerate}
\pagebreak

\noindent
Zur L�sung des Problems sollten Sie die folgenden Teilaufgaben bearbeiten:
\begin{enumerate}[(a)]
\item Implementieren Sie eine Funktion \texttt{exactlyOne} mit der Sie ausdr�cken k�nnen,
      dass zu jedem Vornamen genau ein Nachname geh�rt und dass jeder Mann mit genau einer
      Frau verheiratet ist. Die Funktion \texttt{exactlyOne} wird in der Form
      \\[0.2cm]
      \hspace*{1.3cm}
      $\texttt{exactlyOne}(a, b, \textsl{fct})$
      \\[0.2cm]
      aufgerufen.  Hierbei sind $a$ und $b$ Mengen von Strings.  Beispielsweise k�nnte
      $a$ die Menge der m�nnlichen Vornamen und $b$ die Menge der weiblichen Vornamen sein.
      In diesem Fall w�re \textsl{fct} der String \texttt{\symbol{34}Ehe\symbol{34}}.
      Die Funktion \texttt{exactlyOne} w�rde dann eine Menge von Klauseln erzeugen, die
      ausdr�ckt, dass jeder Mann aus der Menge $a$ mit genau einer Frau aus der Menge $b$
      verheiratet ist. 

      \textbf{Hinweis}: Mit Hilfe der \textsc{SetlX}-Funktion \texttt{makeTerm} k�nnen Sie
      die Terme erzeugen, mit denen wir die aussagenlogischen Variablen darstellen.
      Beispielsweise erzeugt der Aufruf
      \\[0.2cm]
      \hspace*{1.3cm}
      \texttt{makeTerm(\symbol{34}Ehe\symbol{34}, [\symbol{34}Heiner\symbol{34},\symbol{34}Luise\symbol{34}])}
      \\[0.2cm]
      den Term
      \\[0.2cm]
      \hspace*{1.3cm}
      \texttt{@Ehe(\symbol{34}Heiner\symbol{34},\symbol{34}Luise\symbol{34})},
      \\[0.2cm]
      der ausdr�ckt, dass Heiner mit Luise verheiratet ist.
\item Implementieren sie eine Funktion  \texttt{isWifeOf}, so dass der Aufruf
      \\[0.2cm]
      \hspace*{1.3cm}
      $\mathtt{isWifeOf}(y, z)$
      \\[0.2cm]
      eine Formel berechnet, die ausdr�ckt, dass $y$ die Frau von $z$ ist.  Hierbei ist $z$ der
      Nachname eines Mannes.
      Diese Formel soll als String dargestellt werden.
\item Implementieren eine Funktion \texttt{exclusiveOr}, so dass der Aufruf
      \\[0.2cm]
      \hspace*{1.3cm}
      $\texttt{exclusiveOr}(a, b)$
      \\[0.2cm]
      f�r zwei aussagenlogische Formeln $a$ und $b$, die als Strings vorliegen, eine Formel 
      berechnet, die genau dann wahr ist, wenn \underline{entweder} $a$ oder $b$ wahr ist.  
      Die Formel soll also falsch werden, wenn sowohl $a$ als auch $b$ wahr sind. 
      Die resultierende Formel soll in konjunktiver Normalform zur�ck gegeben werden.
\item Vervollst�ndigen Sie nun die Implementierung der Funktion \texttt{computeClauses}.  Wenn Sie
      alles richtig gemacht haben, berechnet der Aufruf der Funktion \texttt{solve} die L�sung des
      Problems.  Diese L\"osung wird dann mit Hilfe der vordefinierten Funktion
      \texttt{displaySolution} in lesbarer Form ausgegeben.
\end{enumerate}


\end{document}

%%% Local Variables: 
%%% mode: latex
%%% TeX-master: t
%%% End: 
