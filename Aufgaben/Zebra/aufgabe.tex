\documentclass{article}
\usepackage{german}
\usepackage[latin1]{inputenc}
\usepackage{a4wide}
\usepackage{amssymb}
\usepackage{fancyvrb}
\usepackage{alltt}
\pagestyle{empty}

\begin{document}
\noindent
{\Large Wem geh�rt das Zebra?}
\vspace{0.5cm}

\noindent
Schreiben Sie ein \textsl{Prolog}-Programm, das das folgende R�tsel l�st.
\begin{enumerate}
\item Es gibt 5 H�user. Jedes der H�user hat eine andere Farbe.
\item In jedem Haus wohnt ein Bewohner einer anderen Nationalit�t.
\item Jeder Hausbewohner bevorzugt ein anderes Getr�nk, raucht eine andere
      Marke Zigaretten und h�lt genau ein Haustier.
\item \underline{Keine} der Personen trinkt das Gleiche, raucht das Gleiche 
      oder h�lt die gleiche Art Haustier. 
\end{enumerate}
Au�erdem wissen wir folgendes:
\begin{enumerate}
\item Der Brite wohnt im roten Haus.
\item Der Schwede h�lt einen Hund.
\item Der Amerikaner trinkt Whisky. 
\item Das gr�ne Haus steht links vom wei�en Haus.
\item Der Besitzer vom gr�nen Haus trinkt Kaffee.
\item Die Person, die PallMall raucht, h�lt einen Vogel.
\item Der Mann, der im mittleren Haus wohnt, trinkt Milch.
\item Der Besitzer vom gelben Haus raucht Dunhill.
\item Der Norweger wohnt im ersten Haus.
\item Der Marlbororaucher wohnt neben dem, der eine Katze h�lt.
\item Der Mann, der ein Schwein h�lt, wohnt neben dem, der Dunhill raucht.
\item Der Winfildraucher trinkt gerne Bier.
\item Der Norweger wohnt neben dem blauen Haus.
\item Der Deutsche raucht Rothmanns.
\item Der Marlbororaucher hat einen Nachbarn, der Wasser trinkt.
\end{enumerate}
Die Frage lautet nun: Wem geh�rt das Zebra?


\end{document}

%%% Local Variables: 
%%% mode: latex
%%% TeX-master: t
%%% End: 
