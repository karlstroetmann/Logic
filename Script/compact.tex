\section{Der Kompaktheits-Satz}

\begin{Satz}[Kompaktheits-Satz]
{\em
  Es sei $M$ eine unendliche Menge von aussagenlogischen Formeln.  Falls 
  \[ M \models \falsum \]
  gilt, dann gibt es auch eine endliche Teilmenge $N \subseteq M$, so dass gilt:
  \[ N \models \falsum. \]
}  
\end{Satz}

\noindent
\textbf{Bemerkung}:  Auf den ersten Blick mag Ihnen die Voraussetzung dieses Satzes
wirklichkeitsfremd erscheinen.  Wieso sollte ein Informatiker sich mit einer Menge $M$
beschäftigen, die unendlich viele Formeln enthält?  Wir werden die Antwort auf diese Frage
finden, wenn wir die Prädikatenlogik behandeln.  Unser Ziel ist es, auch für die
Prädikatenlogik einen 
Kalkül zu konstruieren, der widerlegungs-vollständig ist.  Auf dem Weg zu diesem Ziel 
werden wir prädikatenlogische Klauseln in aussagenlogische Klauseln transformieren.
Dazu müssen wir All-Quantoren eliminieren und bei dieser Elimination kann aus einer
einzigen prädikatenlogischen Klausel eine unendliche Menge von aussagenlogischen
Klauseln werden.  Um dann den oben bewiesenen Satz über die
Widerlegungs-Vollständigkeit der Aussagen-Logik benutzen zu können, benötigen wir den
Kompaktheits-Satz.
\vspace*{0.1cm}

\noindent
\textbf{Beweis des Kompaktheits-Satzes}: 
Zunächst überlegen wir, wie die Voraussetzung $M \models \falsum$ zu interpretieren
ist.  Nach unserer Definition der Relation $\models$ bedeutet $M \models \falsum$, dass jede
aussagenlogische Belegung $\I$, die alle Formeln aus $M$ wahr macht, auch die
Formel $\falsum$ wahr macht.  Da es aber keine Belegung gibt, welche die Formel $\falsum$ wahr
macht, heißt dies, dass es auch keine Belegung gibt, die alle Formeln aus $M$ wahr
macht.  Mit anderen Worten: Die Menge $M$ ist unerfüllbar.  Wir müssen zeigen, dass
es dann bereits eine endliche Teilmenge $N$ von $M$ gibt, die unerfüllbar ist.
Wir werden den Beweis indirekt führen und werden annehmen, dass alle endlichen
Teilmengen von $M$ erfüllbar sind.  Das Ziel des indirekten Beweises wird es dann
sein eine aussagenlogische Belegung $\I$ zu konstruieren, die alle Formeln aus $M$
wahr macht.  Dies steht dann im Widerspruch zu der Voraussetzung $M \models \falsum$.

Wir betrachten hier nur den Fall, dass $M$ \emph{abzählbar} ist.  Es gibt also eine
Folge $(f_n)_{n\in\mathbb{N}}$ von Formeln, so dass 
\[ M = \bigl\{ f_n \mid n \el \mathbb{N} \bigr\} \]
gilt.  Der Fall, dass $M$ überabzählbar ist, hat für unsere Zwecke keine Bedeutung.
Wir definieren dann für alle $n \el \mathbb{N}$ die Mengen 
\[ M_n := \bigl\{ f_i \mid i \el\mathbb{N} \wedge i \leq n \bigr\} 
          = \bigl\{ f_0, f_1,\cdots, f_n \bigr\}. 
\] 
Die Menge $M_n$ enthält $n+1$ Elemente und ist daher endlich.  
Nach unserer Voraussetzung gibt es also für jede
dieser Mengen eine aussagenlogische Belegung $\I_n$, so dass $\I_n$ alle Formeln aus
$M_n$ wahr macht:
\[ \I_n(f_i) = \mathtt{true} \quad \mbox{für alle}\; i\leq n. \]
Unser Ziel ist es, aus den Belegungen $\I_n$ eine Belegung $\I$ zu konstruieren, die
alle Formeln aus $M$ wahr macht.  Wir nehmen an, dass die Menge $\mathcal{P}$ der
aussagenlogischen Variablen abzählbar ist und in der Form 
\[ \mathcal{P} = \bigl\{ p_i \mid i \el \mathbb{N} \bigr\} \]
geschrieben werden kann.  Genau wie oben ist auch diese Annahme eine Einschränkung, aber
für unsere Zwecke ausreichend.  
Bei der Konstruktion der Belegung $\I$ haben wir als Ausgangsmaterial nur die Belegungen
$\I_n$ zur Verfügung.  Wir werden die Belegung $\I$ durch
Induktion über $i \el \mathbb{N}$ für alle aussagenlogischen Variablen $p_i$ definieren.
Gleichzeitig definieren wir eine Menge von Indizes $B_i$ so, dass $B_i$ 
gerade die natürlichen Zahlen $n \el \mathbb{N}$ enthält, für die Interpretationen $\I$ und $\I_n$ für alle
aussagenlogischen Variablen $p_j$ mit $j \leq i$ übereinstimmt, es gilt also
\[ B_i = \bigl\{ n \el \mathbb{N} \mid 
                \forall j \el \mathbb{N} : j \leq i \rightarrow \I_n(p_j) = \I(p_j) \bigr\}.
\]
Definieren wir die Mengen $B_i$ auf diese Weise, so ist leicht zu sehen,
dass die Mengen $B_i$ und $B_{i+1}$ in der folgenden Beziehung stehen:
\[
  B_{i+1} = \bigl\{ n \el B_i :  \I_n(p_{i+1}) = \I(p_{i+1}) \}
\]
Also ist $B_{i+1}$  eine Teilmenge von $B_i$.
Die zentrale Idee bei der induktiven Definition von $\I(p_i)$ besteht nun darin,
dass wir sicherstellen, dass die Menge $B_i$ immer unendlich viele Elemente besitzt,
es soll also für alle $i\el \mathbb{N}$ die Identität $\textsl{card}(B_i) = \infty$ gelten.
\begin{enumerate}
\item[I.A.:] $i \mapsto 0$.

     Wir gehen demokratisch vor: Falls es unendlich viele Indizes $n \el \mathbb{N}$ gibt, so dass
     $\I_n(p_0) = \mathtt{true}$ ist, setzen wir $\I(p_0)$ auf \texttt{true},
     sonst auf \texttt{false}: 
      \\[0.1cm]
      \hspace*{1.3cm} 
     $
       \I(p_0) := \left\{
       \begin{array}[c]{ll}
         \mathtt{true}  & \mbox{falls}\;\; \textsl{card}\Bigl(\bigl\{ n \el \mathbb{N} \mid \I_n(p_0) = \mathtt{true}\bigr\}\Bigr) = \infty, \\
         \mathtt{false} & \mbox{sonst}.
       \end{array}
       \right.
     $
      \\[0.1cm]
      Falls die Menge  $D:=\{ n \el \mathbb{N} \mid \I_n(p_0) = \mathtt{true}\}$       unendlich
      viele Elemente hat, gilt wegen $\I(p_0) = \mathtt{true}$
    \[ B_0 = \bigl\{ n \el \mathbb{N} \mid \I_n(p_0) = \mathtt{true} \bigr\} = D \]
     und also auch $\textsl{card}(B_0) = \infty$.
     Ist die obige Menge $D$ endlich, so muss die Menge 
     \[ E:= \{ n \el \mathbb{N} \mid \I_n(p_0) = \mathtt{false}\}\]
     unendlich viele Elemente haben.   Da dann wegen $\I(p_0) = \mathtt{false}$
     \[ B_0 =\bigl\{ n \el \mathbb{N} \mid \I_n(p_0) = \mathtt{false} \bigr\} = E \]
     ist, folgt auch wieder $\textsl{card}(B_0) = \infty$.
\item[I.S.:] $i \mapsto i + 1$.

     Nach Induktions-Voraussetzung haben wir die aussagenlogische Belegung $\I$ schon für
     die Variablen $p_0$, $p_1$, $\cdots$, $p_i$ definiert und wir wissen, dass
     $\textsl{card}(B_i) = \infty$ gilt.
     Wir setzen daher
      \\[0.1cm]
      \hspace*{1.3cm}      
     $
       \I(p_{i+1}) := \left\{
       \begin{array}[c]{ll}
         \mathtt{true}  & \mbox{falls}\; \textsl{card}\Bigl(\bigl\{ n \el B_i \mid \I_n(p_{i+1}) = \mathtt{true}\bigr\}\Bigr) = \infty, \\
         \mathtt{false} & \mbox{sonst}.
       \end{array}
       \right.
     $
      \\[0.1cm]
      Da die Menge $B_i$ nach Induktions-Voraussetzung unendlich viele Elemente hat, ist
      mindestens eine der beiden Mengen 
     \[ F:=\textsl{card}\bigl(\{ n \el B_i \mid \I_n(p_{i+1}) = \mathtt{true}\}\bigr),\]
     \[G:=\textsl{card}\bigl(\{ n \el B_i \mid \I_n(p_{i+1}) = \mathtt{false}\}\bigr)\]
      unendlich groß, denn es gilt $F \cup G = B_i$.  Falls die  Menge $F$ unendlich ist,
      setzen wir $\I(p_{i+1}) := \mathtt{true}$ und damit gilt
      \[ B_{i+1} = \textsl{card}\bigl(\{ n \el B_i \mid \I_n(p_{i+1}) = \mathtt{true}\}\bigr) = F, \]
      so dass auch $B_{i+1}$ unendlich ist.  Falls $F$ endlich ist, muss $G$ unendlich
      sein.  Da wir in diesem Fall $\I(p_{i+1}) := \mathtt{false}$ setzen, gilt dann
       \[ B_{i+1} = \textsl{card}\bigl(\{ n \el B_i \mid \I_n(p_{i+1}) = \mathtt{false}\}\bigr) = G \]
      und wieder ist $B_{i+1}$ unendlich.
\end{enumerate}
Damit ist die Definition der aussagenlogischen Belegung $\I$ abgeschlossen.
Es bleibt zu zeigen, dass $\I$ tatsächlich alle Formeln aus $M$ wahr macht.
Wir betrachten also eine beliebige Formel $f\el M$ und zeigen, dass $\I(f) =
\mathtt{true}$ gilt.  Die gewählte Formel $f$ kann nur endlich viele aussagenlogische 
Variablen enthalten.  Wir nehmen an, dass für ein genügend großes $i$ 
die Menge der aussagenlogischen Variablen, die in $f$ vorkommen, in der Menge
$\{p_0,p_1,\cdots,p_i\}$ enthalten ist.
Da die Menge $B_i$ unendlich viele natürliche Zahlen enthält, muss sie auch beliebig große
Zahlen enthalten.  
Wir finden daher ein $k \el B_i$, so das $f \el M_k$ liegt.
Nach Definition von $B_i$ gilt 
\[ B_i = \bigl\{ n \el \mathbb{N} \mid 
                \forall j \el \mathbb{N} : j \leq i \rightarrow \I_n(p_j) = \I(p_j)  \bigr\}.
\]
Wegen $k \el B_i$ folgt also
\[ \forall j \el \mathbb{N} : j \leq i \rightarrow \I_k(p_j) = \I(p_j). \]
Damit stimmen die Interpretationen $\I$ und $\I_k$ aber für alle aussagenlogischen
Variablen aus der Menge 
$\{p_0,p_1,\cdots,p_i\}$ überein.  Da in der Formel $f$ keine anderen Variablen vorkommen,
gilt folglich $\I(f) = \I_k(f) = \mathtt{true}$, denn die Interpretation $I_k$ hatten wir
ja so gewählt, dass Sie alle Formeln aus $M_k$ wahr macht.
\hspace*{\fill} $\Box$
\vspace*{0.3cm}

Damit können wir nun den Satz über die Widerlegungs-Vollständigkeit der Aussagen-Logik auf
Formel-Mengen ausdehnen, die auch unendlich viele Formeln enthalten.

\begin{Satz}[Widerlegungs-Vollständigkeit der Aussagen-Logik]
  \label{widerlegungs-vollstaendig2}
 \hspace*{\fill} \\ 
{\em
  Ist $M$ eine eventuell abzählbar unendliche Menge von Klauseln, für die
  \[ M \models \falsum \quad \mbox{gilt, dann gilt auch} \quad M \vdash \{\}. \]
}
\end{Satz}

\noindent
\textbf{Beweis}:  Für endliche Mengen haben wir den Satz bereits bewiesen.
Gilt nun $\textsl{card}(M) = \infty$, dann gibt es nach dem Kompaktheits-Satzes eine
endliche Teilmenge $N \subseteq M$, so dass $N \models \falsum$
gilt.  Da $N$ endlich ist, können wir nach Satz \ref{widerlegungs-vollstaendig} schließen,
dass $N \vdash \{\}$
gilt.  Wenn die leere Klausel sich aber bereits aus den in $N$ enthaltenen Klauseln
herleiten lässt, dann  ganz sicher auch aus den in $M$ enthaltenen Klauseln, denn $M$ ist
ja eine Obermenge von $N$.  Also gilt $M \vdash \{\}$.
\hspace*{\fill} $\Box$
\vspace*{0.3cm}

\noindent
\textbf{Bemerkung}: Der Beweis des Kompaktheits-Satzes bereitet vielen Studenten
zunächst Probleme, weil er an einer Stelle der naiven Intuition widerspricht.
Die Folge der Mengen $(B_i)_{i\in\mathbb{N}}$ ist so aufgebaut, dass
\[ B_{i+1} \subseteq B_i \quad \mbox{für alle}\; i \el \mathbb{N} \]
gilt.  Dabei kann es sich bei den Inklusions-Beziehungen $B_{i+1} \subseteq B_i$ durchaus  
um echte Inklusionen handeln, ja es kann sogar passieren, dass die Differenz-Mengen
$B_i \backslash B_{i+1}$ unendlich viele Elemente enthalten.  Wie kann es nun sein, dass
wir aus einer Menge in jedem Schritt unendlich viele Elemente entfernen, aber die Menge
trotzdem nie leer wird?  Wir zeigen anhand eines konkreten Beispiels, wie dies möglich
ist.  Für dieses Beispiel benötigen wir die Tatsache, dass die Menge 
\[ \{ p \el\mathbb{N} \mid p\; \mbox{ist Primzahl} \} \]
unendlich ist.  Es sei $(p_n)_{n\in\mathbb{N}}$ eine Folge, die die Primzahlen der Größe
nach aufzählt, es gilt also $p_0 = 2$, $p_1 = 3$, $p_2 = 5$, $\cdots$. 
Dann definieren wir 
\[ B_i := \{ n \in \mathbb{N} \mid \forall k\el\mathbb{N} : k \leq i \rightarrow n\;\mathtt{mod}\; p_k \not= 0 \} \]
$B_i$ enthält also alle die Zahlen, die nicht durch die Primzahlen aus der Menge
$\{p_0,p_1, \cdots, p_i\}$ teilbar sind.  $B_i$ ist sicher unendlich, denn $B_i$ enthält
beispielsweise alle Zahlen aus der Menge
\[ \bigl\{ p_{i+1}^n \mid n \in \mathbb{N} \wedge n \geq 1 \bigr\}, \]
denn die Zahlen in dieser Menge sind nur durch die Primzahl $p_{i+1}$ teilbar.
Außerdem gilt $B_{i+1} \subseteq B_i$, denn  $B_{i+1}$ enthält genau die Elemente aus
$B_i$, die nicht durch $p_{i+1}$ teilbar sind: Genauer finden wir
für die Differenz der beiden Mengen 
\[ B_{i} \backslash B_{i+1} = \{ n \in \mathbb{N} \mid \forall k\el\mathbb{N} : n\;\mathtt{mod}\; p_{i+1} = 0 \}, \]
Zu dieser Menge gehören aber insbesondere alle
Zahlen aus der Menge
\[ \bigl\{ p_{i+1}^n \mid n \in \mathbb{N} \wedge n \geq 1 \bigr\}. \]
Diese Menge ist sicher unendlich.  Wir entfernen also in jedem Schritt beim Übergang von
$B_i$ nach $B_{i+1}$ unendlich viele Elemente.  Trotzdem sind alle Mengen $B_i$ unendlich
groß!



%%% Local Variables: 
%%% mode: latex
%%% TeX-master: "informatik-script"
%%% End: 
