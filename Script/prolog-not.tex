\section{Negation in \textsl{Prolog}}
In diesem Abschnitt besprechen wir die Implementierung des Negations-Operators in
\textsl{Prolog}.  Wir zeigen zun\"{a}chst anhand eines einfachen Beispiels die Verwendung
dieses Operators, besprechen dann seine Semantik und zeigen abschließend, in welchen
F\"{a}llen die Verwendung des Negations-Operators problematisch ist.

\subsection{Berechnung der Differenz zweier Listen}
In \textsl{Prolog} wird der Negations-Operator als ``\texttt{\symbol{92}+}'' geschrieben.
Wir erl\"{a}utern die Verwendung dieses 
Operators am Beispiel einer Funktion, die die Differenz zweier Mengen berechnen soll,
wobei die Mengen durch Listen dargestellt werden.  Wir werden  die Funktion \\[0.1cm]
\hspace*{1.3cm} 
$\texttt{difference}: \textsl{List}(\textsl{Number}) \times \textsl{List}(\textsl{Number}) \rightarrow \textsl{List}(\textsl{Number})$
\\[0.1cm]
durch bedingte Gleichungen spezifizieren.  Der Ausdruck \\[0.1cm]
\hspace*{1.3cm} $\mathtt{difference}(l_1,l_2)$ \\[0.1cm]
berechnet die Liste aller der Elemente aus $l_1$, die nicht Elemente der Liste $l_2$ sind.
In \textsc{SetlX} k\"{o}nnten wir diese Funktion wie in Abbildung \ref{fig:difference.stl}
gezeigt implementieren.
\begin{figure}[!ht]
\centering
\begin{Verbatim}[ frame         = lines, 
                  framesep      = 0.3cm, 
                  labelposition = bottomline,
                  numbers       = left,
                  numbersep     = -0.2cm,
                  xleftmargin   = 0.8cm,
                  xrightmargin  = 0.8cm,
                ]
    difference := procedure(l1, l2) {
        return [ x in l1 | !(x in l2) ];
    };
\end{Verbatim}
\vspace*{-0.3cm}
\caption{Implementierung der Prozedur \texttt{difference} in \textsc{SetlX}.}
\label{fig:difference.stl}
\end{figure}

\noindent
In \textsl{Prolog} erfolgt die Implementierung dieser Funktion durch Rekursion im ersten Argument.
Dazu stellen wir zun\"{a}chst bedingte Gleichungen auf:
\begin{enumerate}
\item $\textsl{difference}([], l) = []$.
\item $\neg \textsl{member}(h, l) \rightarrow \textsl{difference}([h|t], l) = [h |\textsl{difference}(t,l)]$,

      denn wenn das Element $h$ in der Liste $l$ nicht vorkommt, so bleibt dieses Element
      im Ergebnis erhalten.
\item $     \textsl{member}(h, l) \rightarrow \textsl{difference}([h|t], l) = \textsl{difference}(t,l)$.
\end{enumerate}

\begin{figure}[!h]
  \centering
\begin{Verbatim}[ frame         = lines, 
                  framesep      = 0.3cm, 
                  labelposition = bottomline,
                  numbers       = left,
                  numbersep     = -0.2cm,
                  xleftmargin   = 0.8cm,
                  xrightmargin  = 0.8cm
                ]
    % difference( +List(Number), +List(Number), -List(Number) ).
    difference( [], _L, [] ).
    
    difference( [ H | T ], L, [ H | R ] ) :-
    	\+ member( H, L ),
    	difference( T, L, R ).
    
    difference( [ H | T ], L, R ) :-
    	member( H, L ),
    	difference( T, L, R ).
\end{Verbatim}
\vspace*{-0.3cm}
  \caption{Berechnung der Differenz zweier Listen}
  \label{fig:difference}
\end{figure}

\subsection{Semantik des Negations-Operators in \textsc{Prolog}}
Es bleibt zu kl\"{a}ren, wie das \textsl{Prolog}-System eine Anfrage der Form
\\[0.1cm]
\hspace*{1.3cm} \texttt{\symbol{92}+} $A$ \\[0.1cm]
beantwortet, wie also der \texttt{not}-Operator implementiert ist.
\begin{enumerate}
\item Zun\"{a}chst versucht das System, die Anfrage ``$A$'' zu beantworten.
\item Falls die Beantwortung der Anfrage ``$A$'' scheitert, ist die Beantwortung
      der Anfrage ``\texttt{\symbol{92}+} $A$'' erfolgreich.  In diesem Fall werden keine
      Variablen instanziiert.
\item Falls die Beantwortung der Anfrage ``$A$'' erfolgreich ist, so scheitert die Beantwortung
      der Anfrage ``\texttt{\symbol{92}+} $A$''.
\end{enumerate}

Wichtig ist zu sehen, dass bei der Beantwortung einer negierten Anfrage in keinem Fall
Variablen instanziiert werden.  Eine negierte Anfrage
\\[0.1cm]
\hspace*{1.3cm} \texttt{\symbol{92}+} $A$ \\[0.1cm]
funktioniert daher nur dann wie erwartet, wenn die Anfrage $A$ keine Variablen mehr
enth\"{a}lt.  Zur Illustration betrachten wir das Programm in Abbildung \ref{fig:not-problem}.
Versuchen wir mit diesem Programm die Anfrage \\[0.1cm]
\hspace*{1.3cm} \texttt{smart1(X)} \\[0.1cm]
zu beantworten, so wird diese Anfrage reduziert zu der Anfrage \\[0.1cm]
\hspace*{1.3cm} \texttt{\symbol{92}+ roemer(X), gallier(X)}. \\[0.1cm]
Um die Anfrage ``\texttt{\symbol{92}+ roemer(X)}'' zu beantworten, 
versucht das \textsl{Prolog}-System rekursiv, die Anfrage ``\texttt{roemer(X)}''
zu beantworten.  Dies gelingt und die Variable \texttt{X} wird dabei an den Wert 
``\texttt{caesar}'' gebunden.  Da die Beantwortung der Anfrage ``\texttt{roemer(X)}''
gelingt, scheitert die Anfrage \\[0.1cm]
\hspace*{1.3cm} \texttt{\symbol{92}+ roemer(X)} \\[0.1cm]
und damit gibt es auch auf die urspr\"{u}ngliche Anfrage ``\texttt{smart1(X)}'' keine Antwort.

\begin{figure}[!ht]
  \centering
\begin{Verbatim}[ frame         = lines, 
                  framesep      = 0.3cm, 
                  labelposition = bottomline,
                  numbers       = left,
                  numbersep     = -0.2cm,
                  xleftmargin   = 0.8cm,
                  xrightmargin  = 0.8cm
                ]
    gallier(miraculix).
    
    roemer(caesar).
    
    smart1(X) :- \+ roemer(X), gallier(X).
    
    smart2(X) :- gallier(X), \+ roemer(X).
\end{Verbatim}
\vspace*{-0.3cm}
  \caption{Probleme mit der Negation}
  \label{fig:not-problem}
\end{figure}

Wenn wir voraussetzen, dass das Programm das Pr\"{a}dikate \texttt{roemer/1}
\underline{vollst\"{a}ndi}g beschreibt, dann ist dieses Verhalten rein logisch betrachtet nicht korrekt,
denn die Konjunktion \\[0.1cm]
\hspace*{1.3cm} 
$\neg \mathtt{roemer}(\mathtt{miraculix}) \wedge \mathtt{gallier}(\mathtt{miraculix})$
 \\[0.1cm]
folgt aus den im Programm gegebenen Fakten.  Wenn der dem \textsl{Prolog}-System zu Grunde liegende
automatische Beweiser anders implementiert w\"{a}re, dann k\"{o}nnte er dies auch erkennen.
Wir k\"{o}nnen uns in diesem Beispiel damit behelfen, dass wir die Reihenfolge der
Formeln im Rumpf umdrehen, so wie dies bei der Klausel in Zeile 7 der Abbildung
\ref{fig:not-problem} geschehen ist.  Die Anfrage \\[0.1cm]
\hspace*{1.3cm} \texttt{smart2(X)} \\[0.1cm]
liefert f\"{u}r \texttt{X} den Wert ``\texttt{miraculix}''.
Die zweite Anfrage funktioniert, weil zu dem Zeitpunkt, an dem die negierte Anfrage
``\texttt{\symbol{92}+ roemer(X)}'' aufgerufen wird, ist die Variable \texttt{X} bereits
an den Wert \texttt{miraculix} gebunden und die Anfrage ``\texttt{roemer}(\texttt{miraculix})''
scheitert.  Generell sollte in \textsl{Prolog}-Programmen der Negations-Operator
``\texttt{\symbol{92}+}'' nur auf solche Pr\"{a}dikate angewendet werden, die zum Zeitpunkt
des Aufrufs keine freien Variablen mehr enthalten.

\subsection{Extralogische Pr\"{a}dikate}
Es gibt bestimmte Pr\"{a}dikate, die nicht nach dem an fr\"{u}herer Stelle beschriebenen Algorithmus
ausgewertet werden, weil sie nicht als Fakten und Regeln gespeichert sind.  Hier handelt es sich um
die sogenannten \emph{vordefinierten} Pr\"{a}dikate.  Ein einfaches Beispiel ist das Pr\"{a}dikat
\texttt{writeln/1}, das sein Argument gefolgt von einem Zeilenumbruch ausgibt.  Ein interessanteres
Beispiel ist das Pr\"{a}dikat \texttt{is/2}: Dieses Pr\"{a}dikat dient der Auswertung arithmetischer
Ausdr\"{u}cke.  Dabei ist das \underline{zweite} Argument ein arithmetischer Ausdruck, der ausgewertet
werden soll.  Dieses Ergebnis wird dann an die Variable, die als erstes Argument \"{u}bergeben wird,
gebunden.  Beispielsweise liefert die Anfrage
\\[0.2cm]
\hspace*{1.3cm}
\texttt{?- is(X, 2 + 3).}
\\[0.2cm]
das Ergebnis
\\[0.2cm]
\hspace*{1.3cm}
\texttt{X = 5.}
\\[0.2cm]
Das Pr\"{a}dikat \texttt{is/2} kann auch als Infix-Operator verwendet werden.  Beispielsweise h\"{a}tten wir
die obige Anfrage auch als
\\[0.2cm]
\hspace*{1.3cm}
\texttt{X is 2 + 3.}
\\[0.2cm]
schreiben k\"{o}nnen.  Es ist wichtig zu wissen, dass der arithmetische Ausdruck, der dem Pr\"{a}dikat
\texttt{is/2} zur Auswertung \"{u}bergeben wird, keine ungebundenen Variablen mehr enthalten darf.
Beispielsweise liefert die Anfrage
\\[0.2cm]
\hspace*{1.3cm}
\texttt{?- 3 is 2 + X.}
\\[0.2cm]
nicht etwa das Ergebnis \texttt{X = 1} sondern statt dessen die Fehlermeldung
\\[0.2cm]
\hspace*{1.3cm}
\texttt{ERROR: is/2: Arguments are not sufficiently instantiated}.
\\[0.2cm]
Es w\"{a}re sch\"{o}n, wenn \textsl{Prolog} auch einfach Anfragen wie die obere korrekt beantworten k\"{o}nnte
und die obige Gleichung nach $X$ aufl\"{o}sen k\"{o}nnte.  Es gibt tats\"{a}chlich Erweiterungen von
\textsl{Prolog}, die dazu in der Lage sind.  Diese Disziplin wird als
\href{http://en.wikipedia.org/wiki/Constraint_logic_programming}{\emph{Constrained-Logic-Programming}} bezeichnet.
Neben dem Pr\"{a}dikat \texttt{is/2} haben auch die Pr\"{a}dikate zum Gr\"{o}ßenvergleich zweier Werte, also die
Pr\"{a}dikate ``\texttt{>/2}'', ``\texttt{</2}'', ``\texttt{>=/2}'', ``\texttt{=</2}'' die
Einschr\"{a}nkung, dass die beteiligten Argumente vollst\"{a}ndig instanziiert sein m\"{u}ssen.

\section{Die Tiefen-Suche in \textsl{Prolog}}
Wenn das \textsl{Prolog}-System eine Anfrage beantwortet, wird dabei als Such-Strategie
die sogenannte Tiefen-Suche (engl. \emph{depth first search}) angewendet.  Wir wollen
diese Strategie nun an einem weiteren Beispiel verdeutlichen.  Wir implementieren dazu ein
\textsl{Prolog}-Programm  
mit dessen Hilfe es m\"{o}glich ist, in einem Graphen eine Verbindung von einem gegebenen
Start-Knoten zu einem Ziel-Knoten zu finden.  Als Beispiel betrachten wir den Graphen in
Abbildung \ref{fig:graph}.  Die Kanten k\"{o}nnen durch ein \textsl{Prolog}-Pr\"{a}dikat \texttt{edge/2}
wie folgt dargestellt werden:

\begin{Verbatim}[ frame         = lines, 
                  framesep      = 0.3cm, 
                  labelposition = bottomline,
                  numbers       = left,
                  numbersep     = -0.2cm,
                  xleftmargin   = 0.8cm,
                  xrightmargin  = 0.8cm
                ]
    edge(a, b).
    edge(a, c).
    edge(b, e).
    edge(e, f).
    edge(c, f).
\end{Verbatim}

\begin{figure}[!h]
  \centering
  \epsfig{file=Figures/graph,scale=0.5}
  \caption{Ein einfacher Graph ohne Zykeln}
  \label{fig:graph}
\end{figure}

Wir wollen nun ein \textsl{Prolog}-Programm entwickeln, mit dem es m\"{o}glich ist, f\"{u}r zwei
vorgegebene Knoten $x$ und $y$ zu entscheiden, ob es einen Weg von $x$ nach $y$ gibt.
Außerdem soll dieser Weg dann als Liste von Knoten berechnet werden.
Unser erster Ansatz besteht aus dem Programm, das in Abbildung \ref{fig:connect} gezeigt
ist.  Die Idee ist, dass der Aufruf \\[0.1cm]
\hspace*{1.3cm} \texttt{find\_path(\textsl{Start}, \textsl{Goal}, \textsl{Path})} \\[0.1cm]
einen Pfad \textsl{Path} berechnet, der von \textsl{Start} nach \textsl{Goal} f\"{u}hrt.  Wir diskutieren
die Implementierung.

\begin{figure}[!h]
  \centering
\begin{Verbatim}[ frame         = lines, 
                  framesep      = 0.3cm, 
                  labelposition = bottomline,
                  numbers       = left,
                  numbersep     = -0.2cm,
                  xleftmargin   = 0.8cm,
                  xrightmargin  = 0.8cm
                ]
    % find_path( +Point, +Point, -List(Point) ).
    find_path( X, X, [ X ] ).
    
    find_path( X, Z, [ X | Path ] ) :-
        edge( X, Y ),
        find_path( Y, Z, Path ).
\end{Verbatim}
\vspace*{-0.3cm}
  \caption{Berechnung von Pfaden in einem Graphen}
  \label{fig:connect}
\end{figure}

\begin{enumerate}
\item Die erste Klausel sagt aus, dass es trivialerweise einen Pfad von \texttt{X} nach
      \texttt{X} gibt.  Dieser Pfad enth\"{a}lt genau den Knoten \texttt{X}.
\item Die zweite Klausel sagt aus, dass es einen Weg von \texttt{X} nach \texttt{Z}
      gibt, wenn es zun\"{a}chst eine direkte Verbindung von \texttt{X} zu einem Knoten
      \texttt{Y} gibt und wenn es dann von diesem Knoten \texttt{Y} eine Verbindung
      zu dem Knoten \texttt{Z} gibt.  Wir erhalten den Pfad, der von \texttt{X} nach
      \texttt{Z} f\"{u}hrt, dadurch, dass wir vorne an den Pfad, der von \texttt{Y} nach \texttt{Z}
      f\"{u}hrt, den Knoten \texttt{X} anf\"{u}gen.
\end{enumerate}
Stellen wir an das \textsl{Prolog}-System die Anfrage \texttt{find\_path(a,f,P)}, so
erhalten wir die Antwort
\begin{Verbatim}[ frame         = lines, 
                  framesep      = 0.3cm, 
                  labelposition = bottomline,
                  numbers       = left,
                  numbersep     = -0.2cm,
                  xleftmargin   = 0.8cm,
                  xrightmargin  = 0.8cm
                ]
    ?- find_path(a,f,P).
 
    P = [a, b, e, f] ;   
    P = [a, c, f] ;
    No
\end{Verbatim}
Durch Backtracking werden also alle m\"{o}glichen Wege von \texttt{a} nach \texttt{b} gefunden.
Als n\"{a}chstes testen wir das Programm mit dem in Abbildung \ref{fig:graph2} gezeigten
Graphen.  Diesen Graphen stellen wir wie folgt in \textsl{Prolog} dar:
\begin{Verbatim}[ frame         = lines, 
                  framesep      = 0.3cm, 
                  labelposition = bottomline,
                  numbers       = left,
                  numbersep     = -0.2cm,
                  xleftmargin   = 0.8cm,
                  xrightmargin  = 0.8cm
                ]
    edge(a, b).
    edge(a, c).
    edge(b, e).
    edge(e, a).
    edge(e, f).
    edge(c, f).
\end{Verbatim}

\begin{figure}[!h]
  \centering
  \epsfig{file=Figures/graph2,scale=0.5}
  \caption{Ein Graph mit einem Zykel}
  \label{fig:graph2}
\end{figure}

Jetzt erhalten wir auf die Anfrage \texttt{find\_path(a,f,P)} die Antwort
\begin{Verbatim}[ frame         = lines, 
                  framesep      = 0.3cm, 
                  labelposition = bottomline,
                  numbers       = left,
                  numbersep     = -0.2cm,
                  xleftmargin   = 0.8cm,
                  xrightmargin  = 0.8cm
                ]
    ?- find_path(a,f,P).
    ERROR: Out of local stack
\end{Verbatim}
Die Ursache ist schnell gefunden.
\begin{enumerate}
\item Wir starten mit der Anfrage \\[0.1cm]
      \hspace*{1.3cm} \texttt{find\_path(a,f,P)}.
\item Nach Unifikation mit der zweiten Klausel haben wir die Anfrage reduziert auf \\[0.1cm]
      \hspace*{1.3cm} 
      \texttt{edge( a, Y1 ), find\_path( Y1, f, P1 )}.
\item Nach Unifikation mit dem Fakt \texttt{edge(a,b)} haben wir die neue Anfrage \\[0.1cm]
      \hspace*{1.3cm} 
      \texttt{find\_path( b, f, P1 )}.
\item Nach Unifikation mit der zweiten Klausel haben wir die Anfrage reduziert auf \\[0.1cm]
      \hspace*{1.3cm} 
      \texttt{edge( b, Y2 ), find\_path( Y2, f, P2 )}.
\item Nach Unifikation mit dem Fakt \texttt{edge(b,e)} haben wir die neue Anfrage \\[0.1cm]
      \hspace*{1.3cm} 
      \texttt{find\_path( e, f, P2 )}.
\item Nach Unifikation mit der zweiten Klausel haben wir die Anfrage reduziert auf \\[0.1cm]
      \hspace*{1.3cm} 
      \texttt{edge( e, Y3 ), find\_path( Y3, f, P3 )}.
\item Nach Unifikation mit dem Fakt \texttt{edge(e,a)} haben wir die neue Anfrage \\[0.1cm]
      \hspace*{1.3cm} 
      \texttt{find\_path( a, f, P3 )}.
\end{enumerate}
Die Anfrage ``\texttt{find\_path(a, f, P3)}'' unterscheidet sich von der urspr\"{u}nglichen
Anfrage ``\texttt{find\_path(a,f,P)}'' nur durch den Namen der Variablen.  Wenn wir jetzt
weiterrechnen w\"{u}rden, w\"{u}rde sich die Rechnung nur wiederholen, ohne dass wir vorw\"{a}rts kommen.
Das Problem ist, das \textsl{Prolog} immer die erste
Klausel nimmt, die passt.  Wenn sp\"{a}ter die Reduktion der Anfrage scheitert, wird zwar nach
Backtracking die n\"{a}chste Klausel ausprobiert, aber wenn das Programm in eine
Endlos-Schleife l\"{a}uft, dann gibt es eben kein Backtracking, denn das Programm weiß ja
nicht, dass es in einer Endlos-Schleife ist.

Es ist leicht das Programm so umzuschreiben, dass keine Endlos-Schleife mehr
auftreten kann.  Die Idee ist, dass wir uns merken, welche Knoten wir bereits besucht
haben und diese nicht mehr ausw\"{a}hlen.  In diesem Sinne implementieren wir nun ein Pr\"{a}dikat \texttt{find\_path/4}.
Die Idee ist, dass der Aufruf \\[0.1cm]
\hspace*{1.3cm} \texttt{find\_path(\textsl{Start}, \textsl{Goal}, \textsl{Visited}, \textsl{Path})} \\[0.1cm]
einen Pfad berechnet, der von \textsl{Start} nach \textsl{Goal} f\"{u}hrt und der zus\"{a}tzlich
keine Knoten benutzt, die bereits in der Liste \textsl{Visited} aufgef\"{u}hrt sind.  Diese Liste
f\"{u}llen wir bei den rekursiven Aufrufen nach und nach mit den Knoten an, die wir bereits
besucht haben.  Mit Hilfe dieser Liste vermeiden wir es, einen Knoten zweimal zu besuchen.
Abbildung \ref{fig:connect2} zeigt die Implementierung.
\begin{figure}[!h]
  \centering
\begin{Verbatim}[ frame         = lines, 
                  framesep      = 0.3cm, 
                  labelposition = bottomline,
                  numbers       = left,
                  numbersep     = -0.2cm,
                  xleftmargin   = 0.8cm,
                  xrightmargin  = 0.8cm
                ]
    % find_path( +Point, +Point, +List(Point), -List(Point) )

    find_path( X, X, _Visited, [ X ] ).
    
    find_path( X, Z, Visited, [ X | Path ]) :-
        edge( X, Y ),
        \+ member( Y, Visited ),
        find_path( Y, Z, [ Y | Visited ], Path ).
    \end{Verbatim}
\vspace*{-0.3cm}
  \caption{Berechnung von Pfaden in zyklischen Graphen}
  \label{fig:connect2}
\end{figure}
\begin{enumerate}
\item In der ersten Klausel spielt das zus\"{a}tzliche Argument noch keine Rolle,
      denn wenn wir das Ziel erreicht haben, ist es uns egal, welche Knoten wir schon
      besucht haben.
\item In der zweiten Klausel \"{u}berpr\"{u}fen wir in Zeile 7, ob der Knoten \texttt{Y}
      in der Liste \textsl{Visited}, die die Knoten enth\"{a}lt, die bereits besucht wurden,
      auftritt.  Nur wenn dies nicht der Fall ist, versuchen wir rekursiv von \texttt{Y}
      einen Pfad nach \texttt{Z} zu finden.  Bei dem rekursiven Aufruf erweitern wir die Liste
      \texttt{Visited} um den Knoten \texttt{Y}, denn diesen Knoten wollen wir in Zukunft
      ebenfalls vermeiden.
\end{enumerate}
Mit dieser Implementierung ist es jetzt m\"{o}glich, auch in dem zweiten Graphen einen Weg von
\texttt{a} nach \texttt{f} zu finden, wir erhalten folgendes Ergebnis:
\pagebreak

\begin{Verbatim}[ frame         = lines, 
                  framesep      = 0.3cm, 
                  labelposition = bottomline,
                  numbers       = left,
                  numbersep     = -0.2cm,
                  xleftmargin   = 0.8cm,
                  xrightmargin  = 0.8cm
                ]
    ?- find_path(a,f,[a],P).
    P = [a, b, e, f] ;
    P = [a, c, f] ;    
    No
\end{Verbatim}


\subsection{Die Bekehrung der Ungl\"{a}ubigen}
Als spielerische Anwendung zeigen wir nun, wie sich mit Hilfe des oben definierten Pr\"{a}dikats 
\texttt{find\_path/4} ein theologisches Problem l\"{o}sen l\"{a}sst.
\vspace*{0.3cm}

\begin{minipage}[c]{14cm}
{\sl Drei Missionare und drei Ungl\"{a}ubige wollen zusammen einen Fluss 
\"{u}berqueren. Sie haben nur ein Boot, indem maximal zwei Passagiere fahren k\"{o}nnen.  
Sowohl die Ungl\"{a}ubigen als auch die Missionare k\"{o}nnen rudern.
Weder die Gl\"{a}ubigen, noch die Ungl\"{a}ubigen k\"{o}nnen \"{u}ber das Wasser laufen.
Die Ungl\"{a}ubigen sind hungrig, wenn die Missionare an einem der Ufer in der Unterzahl sind, 
haben sie ein Problem.  Die Aufgabe besteht darin, einen Fahrplan zu 
erstellen, so dass hinterher alle das andere  Ufer erreichen und die
Missionare zwischendurch kein Problem haben.}
\end{minipage}
\vspace*{0.4cm}

\noindent
Die Idee ist, das R\"{a}tsel, durch einen Graphen zu modellieren.  Die Knoten dieses 
Graphen sind dann die Situationen, die w\"{a}hrend des \"{u}bersetzens auftreten.  Wir
repr\"{a}sentieren diese Situationen durch Terme der Form \\[0.1cm]
\hspace*{1.3cm} $\texttt{side}(M,\;K,\;B)$.
\\[0.1cm]
Ein solcher Term repr\"{a}sentiert eine Situation, bei der auf der linken Seite des Ufers $M$ Missionare, $K$
Ungl\"{a}ubige und $B$ Boote sind.  Unsere Aufgabe besteht nun darin, das Pr\"{a}dikat
\texttt{edge/2} so zu implementieren, dass \\[0.1cm]
\hspace*{1.3cm} $\texttt{edge}(\;\texttt{side}(M_1,\;K_1,\;B_1),\;\texttt{side}(M_2,\;K_2,\;B_2)\;)$
\\[0.1cm]
genau dann wahr ist, wenn die Situation $\texttt{side}(M_1,\;K_1,\;B_1)$
durch eine Boots-\"{u}berfahrt in die Situation $\texttt{side}(M_2,\;K_2,\;B_2)$ \"{u}berf\"{u}hrt
werden kann und wenn zus\"{a}tzlich die Missionare in der neuen Situation kein Problem bekommen.
Abbildung \ref{fig:missionare.pl} auf Seite \pageref{fig:missionare.pl}
zeigt ein \textsl{Prolog}-Programm, was das R\"{a}tsel l\"{o}st.  Den von diesem Programm
berechneten Fahrplan finden Sie in Abbildung \ref{fig:missionare-solution} 
auf Seite \pageref{fig:missionare-solution}.
Wir diskutieren dieses Programm nun Zeile f\"{u}r Zeile.

\begin{figure}[!h]
  \centering
\begin{Verbatim}[ frame         = lines, 
                  framesep      = 0.3cm, 
                  labelposition = bottomline,
                  numbers       = left,
                  numbersep     = -0.2cm,
                  xleftmargin   = 0.8cm,
                  xrightmargin  = 0.8cm
                ]
    MMM   KKK   B      |~~~~~|                   
                       >  KK >
    MMM   K            |~~~~~|      B    KK      
                       <  K  <
    MMM   KK    B      |~~~~~|            K      
                       >  KK >
    MMM                |~~~~~|      B   KKK      
                       <  K  <
    MMM   K     B      |~~~~~|           KK      
                       > MM  >
    M     K            |~~~~~|      B    KK    MM
                       < M K <
    MM    KK    B      |~~~~~|            K     M
                       > MM  >
          KK           |~~~~~|      B     K   MMM
                       <  K  <
          KKK   B      |~~~~~|                MMM
                       >  KK >
          K            |~~~~~|      B    KK   MMM
                       <  K  <
          KK    B      |~~~~~|            K   MMM
                       >  KK >
                       |~~~~~|      B   KKK   MMM
\end{Verbatim}
\vspace*{-0.3cm}
  \caption{Fahrplan f\"{u}r Missionare und Ungl\"{a}ubige}
  \label{fig:missionare-solution}
\end{figure}      

\begin{figure}[!h]
  \centering
\begin{Verbatim}[ frame         = lines, 
                  framesep      = 0.3cm, 
                  labelposition = bottomline,
                  numbers       = left,
                  numbersep     = -0.2cm,
                  xleftmargin   = 0.8cm,
                  xrightmargin  = 0.8cm
                ]
    solve :-
        find_path( side(3,3,1), side(0,0,0), [ side(3,3,1) ], Path ),
        nl, write('L\"{o}sung:' ), nl, nl,
        print_path(Path).
    
    % edge( +Point, -Point ).    
    % This clause describes rowing from the left side to the right side.
    edge( side( M, K, 1 ), side( MN, KN, 0 ) ) :-
        between( 0, M, MB ),    % MB missionaries in the boat
        between( 0, K, KB ),    % KB infidels in the boat
        MB + KB >= 1,           % boat must not be empty
        MB + KB =< 2,           % no more than two passengers
        MN is M - MB,           % missionaries left on the left side
        KN is K - KB,           % infidels left on the left side
        \+ problem( MN, KN ).   % no problem must occur
    
    % This clause describes rowing from the right side to the left side.
    edge( side( M, K, 0 ), side( MN, KN, 1 ) ) :-
        otherSide( M, K, MR, KR ),
        edge( side( MR, KR, 1 ), side( MRN, KRN, 0 ) ),
        otherSide( MRN, KRN, MN, KN ).
    
    % otherSide( +Number, +Number, -Number, -Number ).
    otherSide( M, K, M_Other, K_Other ) :-
        M_Other is 3 - M,
        K_Other is 3 - K.
    
    % problem( +Number, +Number).
    problem(M, K) :- 
            problemSide(M, K).
    
    problem(M, K) :-
        otherSide( M, K, M_Other, K_Other ),
        problemSide(M_Other, K_Other).
        
    % problemSide( +Number, +Number).
    problemSide(Missionare, Kannibalen) :- 
            Missionare > 0, 
            Missionare < Kannibalen.
    
    % find_path( +Point, +Point, +List(Point), -List(Point) )
    find_path( X, X, _Visited, [ X ] ).
    
    find_path( X, Z, Visited, [ X | Path ]) :-
            edge( X, Y ),
            \+ member( Y, Visited ),
            find_path( Y, Z, [ Y | Visited ], Path ).
\end{Verbatim}
\vspace*{-0.3cm}
  \caption{Die Bekehrung der Ungl\"{a}ubigen}
  \label{fig:missionare.pl}
\end{figure}      

\begin{enumerate}
\item Wir beginnen mit dem  Hilfs-Pr\"{a}dikat \texttt{otherSide/4}, das in den
      Zeilen 24 -- 26 implementiert ist.  F\"{u}r eine vorgegebene Situation
      $\texttt{side}(M,K,B)$ berechnet der Aufruf \\[0.1cm]
      \hspace*{1.3cm} $\texttt{otherSide}(\; \texttt{side}(M, K, B),\; \textsl{OtherSide} \;)$
      \\[0.1cm] 
      einen Term, der die Situation am gegen\"{u}berliegenden Ufer beschreibt.
      Wenn an einen Ufer $M$ Missionare sind, so sind am anderen Ufer die restlichen
      Missionare und da es insgesamt $3$ Missionare gibt, sind das $3 - M$.
      Die Anzahl der Ungl\"{a}ubige am gegen\"{u}berliegenden Ufer wird analog berechnet. 
\item Das Pr\"{a}dikat \texttt{problem/2} in den Zeilen 29 -- 34 \"{u}berpr\"{u}ft, ob es bei einer vorgegeben
      Anzahl von Missionaren und Ungl\"{a}ubige zu einem Problem kommt.
      Da das Problem entweder am linken oder am rechten Ufer auftreten kann,
      besteht die Implementierung aus zwei Klauseln.  Die erste Klausel pr\"{u}ft,
      ob es auf der Seite, an der $M$ Missionare und $K$ Ungl\"{a}ubige sind, zum Problem
      kommt.  Die zweite Klausel \"{u}berpr\"{u}ft, ob es auf dem gegen\"{u}berliegenden
      Ufer zu einem Problem kommt.  Als Hilfs-Pr\"{a}dikat verwenden wir hier das Pr\"{a}dikat
      \texttt{problemSide/2}.  Dieses Pr\"{a}dikat ist in Zeile 37 implementiert
      und \"{u}berpr\"{u}ft die Situation an einer Seite:  Falls sich auf einer Seite $M$ Missionare
      und $K$ Ungl\"{a}ubige befinden, so gibt es dann ein Problem, wenn die Zahl $M$ von 0
      verschieden ist und wenn zus\"{a}tzlich $M < K$ ist.
\item Bei der Implementierung des Pr\"{a}dikats \texttt{edge/2} verwenden wir in den Zeilen 9
      und 10 das Pr\"{a}dikat \texttt{between/3}, das in dem \textsl{SWI-Prolog}-System 
      vordefiniert ist.  Beim Aufruf \\[0.1cm]
      \hspace*{1.3cm} $\texttt{between}(\textsl{Low}, \textsl{High}, N)$ \\[0.1cm]
      sind \textsl{Low} und \textsl{High} ganze Zahlen mit $\textsl{Low} \leq \textsl{High}$.
      Der Aufruf instantiert die Variable $N$ nacheinander mit den Zahlen \\[0.1cm]
      \hspace*{1.3cm} $\textsl{Low},\; \textsl{Low}+1,\; \textsl{Low}+2, \cdots, \;\textsl{High}$. \\[0.1cm]
      Beispielsweise gibt die Anfrage \\[0.1cm]
      \hspace*{1.3cm} \texttt{between(1,3,N), write(N), nl, fail.}  \\[0.1cm]
      nacheinander die Zahlen 1, 2 und 3 am Bildschirm aus.
    \item Die Implementierung des Pr\"{a}dikats \texttt{edge/2} besteht aus zwei Klauseln.  In
      der ersten Klausel betrachten wir den Fall, dass das Boot am linken Ufer ist.  In
      der Zeilen 9 generieren wir die Zahl der Missionare $\texttt{MB}$, die im Boot \"{u}bersetzen
      sollen.  Diese Zahl $\texttt{MB}$ ist durch $\texttt{M}$ beschr\"{a}nkt, denn es k\"{o}nnen nur die Missionare
      \"{u}bersetzen, die sich am linken Ufer befinden.  Daher benutzen wir das Pr\"{a}dikat
      \texttt{between/3} um eine Zahl zwischen 0 und $\texttt{M}$ zu erzeugen.  Analog generieren
      wir in Zeile 10 die Zahl $\texttt{KB}$ der Ungl\"{a}ubige, die im Boot \"{u}bersetzen.  In Zeile 11
      testen wir, dass es mindestens einen Passagier gibt, der mit dem Boot \"{u}bersetzt und
      in Zeile 12 testen wir, dass es h\"{o}chstens zwei Passagiere sind.  In Zeile 13 und 14
      berechnen wir die Zahl $\texttt{MN}$ der Missionare und die Zahl $\texttt{KN}$ der Ungl\"{a}ubige, die
      nach der \"{u}berfahrt auf dem linken Ufer verbleiben und testen dann in Zeile 15, dass es
      f\"{u}r diese Zahlen kein Problem gibt.
      
      Die zweite Klausel befasst sich mit dem Fall, dass das Boot am rechten Ufer liegt.
      Wir h\"{a}tten diese Klausel mit \textsl{Copy \& Paste} aus der vorhergehenden Klausel
      erzeugen k\"{o}nnen, aber es ist eleganter, diesen Fall auf den vorhergehenden Fall
      zur\"{u}ck zu f\"{u}hren.  Da da Boot nun auf der rechten Seite liegt, berechnen wir daher
      in Zeile 19 die Zahl $\texttt{MR}$ der Missionare auf der rechten Seite und die Zahl
      $\texttt{KR}$ der Ungl\"{a}ubige auf der rechten Seite.  Dann untersuchen wir die
      Situation $\mathtt{side}(\mathtt{MR}, \mathtt{KR}, 1)$, bei der $\texttt{MR}$
      Missionare und $\texttt{KR}$ Ungl\"{a}ubige am linken Ufer stehen.  Wenn diese so
      \"{u}bersetzen k\"{o}nnen, dass nachher $\texttt{MRN}$ Missionare und $\texttt{KRN}$
      Ungl\"{a}ubige am linken Ufer stehen, dann k\"{o}nnen wir in Zeile 21 berechnen, wieviele
      Missionare und Ungl\"{a}ubige sich dann am gegen\"{u}berliegenden Ufer befinden.
\item In den Zeilen 1 -- 4 definieren wir nun das Pr\"{a}dikat \texttt{solve/0}, dessen Aufruf
      das Problem l\"{o}st.  Dazu wird zun\"{a}chst das Pr\"{a}dikat \texttt{find\_path/4} 
      mit dem Start-Knoten \texttt{side(3,3,1)} und dem Ziel-Knoten \texttt{side(0,0,0)}
      aufgerufen.   Der berechnete Pfad wird dann ausgegeben mit dem Pr\"{a}dikat
      \texttt{print\_path/1},
      dessen Implementierung hier aus Platzgr\"{u}nden nicht angegeben wird.
\end{enumerate}


%%% Local Variables: 
%%% mode: latex
%%% TeX-master: "logik"
%%% End: 
