\section{Der Cut-Operator}
Wir haben ein Pr\"{a}dikat als \emph{funktional} definiert,
wenn wir die einzelnen Argumente klar in Eingabe- und Ausgabe-Argumente
aufteilen k\"{o}nnen.  Wir nennen ein Pr\"{a}dikat \emph{deterministisch} wenn es funktional ist
und wenn außerdem zu jeder Eingabe h\"{o}chstens eine Ausgabe berechnet
wird.  Diese zweite Forderung ist durchaus nicht immer erf\"{u}llt.  Betrachten wir die ersten
beiden Fakten zur Definition des Pr\"{a}dikats \texttt{mix/3}:
\begin{Verbatim}[ frame         = lines, 
                  framesep      = 0.3cm, 
                  labelposition = bottomline,
                  numbers       = left,
                  numbersep     = -0.2cm,
                  xleftmargin   = 0.8cm,
                  xrightmargin  = 0.8cm
                ]
    mix( [], Xs, Xs ).    
    mix( Xs, [], Xs ).
\end{Verbatim}
F\"{u}r die Anfrage ``\texttt{mix([], [], L)}'' k\"{o}nnen beide Fakten verwendet werden.
Das Ergebnis ist zwar immer dasselbe, n\"{a}mlich \texttt{L = []}, es wird aber zweimal ausgegeben:
\begin{Verbatim}[ frame         = lines, 
                  framesep      = 0.3cm, 
                  labelposition = bottomline,
                  numbers       = left,
                  numbersep     = -0.2cm,
                  xleftmargin   = 0.8cm,
                  xrightmargin  = 0.8cm
                ]
    ?- mix([],[],L).
    
    L = [] ;
    
    L = [] 
\end{Verbatim}
Dies kann zu Ineffizienz f\"{u}hren.  Aus diesem Grunde gibt es in \textsl{Prolog} den
Cut-Operator ``\texttt{!}''.  Mit diesem Operator ist es m\"{o}glich, redundante L\"{o}sungen aus
dem Suchraum heraus zu schneiden.  Schreiben wir die ersten beiden Klauseln der
Implementierung von \texttt{mix/3} in der Form 
\begin{Verbatim}[ frame         = lines, 
                  framesep      = 0.3cm, 
                  labelposition = bottomline,
                  numbers       = left,
                  numbersep     = -0.2cm,
                  xleftmargin   = 0.8cm,
                  xrightmargin  = 0.8cm
                ]
    mix( [], Xs, Xs ) :- !.
    mix( Xs, [], Xs ) :- !.
\end{Verbatim}
so wird auf die Anfrage  ``\texttt{mix([], [], L)}'' die L\"{o}sung \texttt{L = []} nur noch
einmal generiert.  Ist allgemein eine Regel der Form \\[0.1cm]
\hspace*{1.3cm} $P \;\texttt{:-}\; Q_1, \cdots, Q_m, \texttt{!}, R_1, \cdots, R_k$ \\[0.1cm]
gegeben, und gibt es weiter eine Anfrage $A$, so dass $A$ und $P$  unifizierbar sind, so
wird die Anfrage $A$ zun\"{a}chst zu der Anfrage \\[0.1cm]
\hspace*{1.3cm} $Q_1\mu, \cdots, Q_m\mu, \texttt{!}, R_1\mu, \cdots, R_k\mu$ \\[0.1cm]
reduziert.  Außerdem wird ein Auswahl-Punkt gesetzt, wenn es noch weitere Klauseln gibt,
deren Kopf mit $A$ unifiziert werden k\"{o}nnt.
Bei der weiteren Abarbeitung dieser Anfrage gilt folgendes:
\begin{enumerate}
\item Falls bereits die Abarbeitung einer Anfrage der Form \\[0.1cm]
      \hspace*{1.3cm} 
      $Q_i\sigma, \cdots, Q_m\sigma, \texttt{!}, R_1\sigma, \cdots, R_k\sigma$ \\[0.1cm]
      f\"{u}r ein $i\in\{1,\cdots,m\}$ scheitert, so wird der Cut nicht erreicht und hat keine
      Wirkung.
\item Eine Anfrage der Form \\[0.1cm]
      \hspace*{1.3cm} 
      $\texttt{!}, R_1\sigma, \cdots, R_k\sigma$ \\[0.1cm]
      wird reduziert zu \\[0.1cm]
      \hspace*{1.3cm} 
      $R_1\sigma, \cdots, R_k\sigma$. \\[0.1cm]
      Dabei werden alle Auswahl-Punkte, die bei der Beantwortung der Teilanfragen
      $Q_1, \cdots, Q_m$ gesetzt worden sind, gel\"{o}scht.  Außerdem wird ein eventuell bei
      der Reduktion der Anfrage $A$ auf die Anfrage \\[0.1cm]
      \hspace*{1.3cm} $Q_1\mu, \cdots, Q_m\mu, \texttt{!}, R_1\mu, \cdots, R_k\mu$
      \\[0.1cm]
      gesetzter Auswahl-Punkte gel\"{o}scht.
\item Sollte sp\"{a}ter die Beantwortung der Anfrage \\[0.1cm]
      \hspace*{1.3cm} $R_1\sigma, \cdots, R_k\sigma$. \\[0.1cm]
      scheitern, so scheitert auch die Beantwortung der Anfrage $A$.
\end{enumerate}
Zur Veranschaulichung betrachten wir ein Beispiel.
\begin{Verbatim}[ frame         = lines, 
                  framesep      = 0.3cm, 
                  labelposition = bottomline,
                  numbers       = left,
                  numbersep     = -0.2cm,
                  xleftmargin   = 0.8cm,
                  xrightmargin  = 0.8cm
                ]
    q(Z) :- p(Z).
    q(1).

    p(X) :- a(X), b(X), !, c(X,Y), d(Y).    
    p(3).
    
    a(1).    a(2).    a(3).
    
    b(2).    b(3).
    
    c(2,2).  c(2,4).
    
    d(3).
\end{Verbatim}
Wir verfolgen die Beantwortung der Anfrage \ \texttt{q(U)}. 
\begin{enumerate}
\item Zun\"{a}chst wird versucht \texttt{q(U)} mit dem Kopf der ersten Klausel 
      des Pr\"{a}dikats \texttt{q/1}  zu unifizieren.  Dabei wird \texttt{Z} mit
      \texttt{U} instantiert und die Anfrage wird reduziert zu \\[0.1cm]
      \hspace*{1.3cm} \texttt{p(U)}. \\[0.1cm]
      Da es noch eine weiter Klausel f\"{u}r das Pr\"{a}dikat \texttt{q/1} gibt, die zur
      Beantwortung der Anfrage q(U) in Frage kommt, setzen wir Auswahl-Punkt Nr.~1.
\item Jetzt wird versucht \texttt{p(U)} mit \texttt{p(X)} zu unifizieren.  Dabei
      wird die Variable \texttt{X} an \texttt{U} gebunden und die urspr\"{u}ngliche Anfrage
      wird reduziert zu der Anfrage \\[0.1cm]
      \hspace*{1.3cm}  
      \texttt{a(U), b(U), !, c(U,Y), d(Y)}. \\[0.1cm]
      Außerdem wird an dieser Stelle Auswahl-Punkt Nr.~2 gesetzt, denn die zweite Klausel
      des Pr\"{a}dikats \texttt{p/1} kann ja ebenfalls mit der urspr\"{u}nglichen Anfrage unifiziert
      werden. 
\item Um die Teilanfrage \texttt{a(U)} zu beantworten, wird \texttt{a(U)} mit
      \texttt{a(1)} unifiziert.  Dabei wird \texttt{U} mit 1 instantiiert und die Anfrage wird reduziert zu \\[0.1cm]
      \hspace*{1.3cm} 
      \texttt{b(1), !, c(1,Y), d(Y)}. \\[0.1cm]
      Da es f\"{u}r das Pr\"{a}dikat \texttt{a/1} noch weitere Klauseln gibt, wird Auswahl-Punkt Nr.~3
      gesetzt.
\item Jetzt wird versucht, die Anfrage \\[0.1cm]
      \hspace*{1.3cm}    \texttt{b(1), !, c(1,Y), d(Y)}. \\[0.1cm]
      zu l\"{o}sen.  Dieser Versuch scheitert jedoch, da sich die f\"{u}r das Pr\"{a}dikat \texttt{b/1}
      vorliegenden Fakten nicht mit \texttt{b(1)} unifizieren lassen.
\item Also springen wir zur\"{u}ck zum letzten Auswahl-Punkt (das ist Auswahl-Punkt Nr.~3)
      und machen die Instantiierung 
      $\texttt{U} \mapsto 1$ r\"{u}ckg\"{a}ngig.  Wir haben jetzt also wieder das Ziel \\[0.1cm]
      \hspace*{1.3cm} \texttt{a(U), b(U), !, c(U,Y), d(Y)}. 
\item Diesmal w\"{a}hlen wir das Fakt \texttt{a(2)} um es mit \texttt{a(U)} zu unifizieren.  Dabei wird \texttt{U} mit
      2 instantiiert und wir haben die  Anfrage \\[0.1cm]
      \hspace*{1.3cm} \texttt{b(2), !, c(2,Y), d(Y)}. \\[0.1cm]
      Da es noch eine weiter Klausel f\"{u}r das Pr\"{a}dikat \texttt{a/1} gibt, setzen wir 
      Auswahl-Punkt Nr.~4 an dieser Stelle.
\item Jetzt unifizieren wir  die Teilanfrage \texttt{b(2)} mit der ersten Klausel f\"{u}r das Pr\"{a}dikat
      \texttt{b/1}.  Die verbleibende Anfrage ist \\[0.1cm]
      \hspace*{1.3cm} \texttt{!, c(2,Y), d(Y)}. 
\item Diese Anfrage wird reduziert zu \\[0.1cm]
      \hspace*{1.3cm}  \texttt{c(2,Y), d(Y)}. \\[0.1cm]
      Außerdem werden bei diesem Schritt die Auswahl-Punkte Nr.~2 und Nr.~4 gel\"{o}scht.
\item Um diese Anfrage zu beantworten, unifizieren wir \texttt{c(2,Y)} mit dem Kopf der
      ersten Klausel f\"{u}r das Pr\"{a}dikat \texttt{c/2}, also mit \texttt{c(2,2)}.
      Dabei erhalten wir die Instantiierung
      $\mathtt{Y} \mapsto 2$.  Die Anfrage ist damit reduziert zu \\[0.1cm]
      \hspace*{1.3cm}  \texttt{d(2)}. \\[0.1cm]
      Außerdem setzen wir an dieser Stelle Auswahl-Punkt Nr.~5, denn das Pr\"{a}dikat
      \texttt{c/2} hat ja noch eine weitere Klausel, die in Frage kommt.
\item Die Anfrage ``\texttt{d(2)}'' scheitert.  Also springen wir zur\"{u}ck zum Auswahl-Punkt
      Nr.~5 und machen die Instantiierung $\mathtt{Y} \mapsto 2$ r\"{u}ckg\"{a}ngig.  Wir haben
      also wieder die Anfrage \\[0.1cm]
      \hspace*{1.3cm}  \texttt{c(2,Y), d(Y)}. 
\item Zur Beantwortung dieser Anfrage nehmen wir nun die zweite Klausel der
      Implementierung von \texttt{c/2} und erhalten die Instantiierung 
      $\mathtt{Y} \mapsto 4$.  Die verbleibende Anfrage lautet dann \\[0.1cm]
      \hspace*{1.3cm} \texttt{d(4)}.
\item Da sich \texttt{d(4)} und \texttt{d(3)} nicht unifizieren lassen,
      scheitert diese Anfrage.  Wir springen jetzt zur\"{u}ck zum Auswahl-Punkt Nr.~1 und
      machen die Instantiierung $\mathtt{U} \mapsto \mathtt{Z}$ r\"{u}ckg\"{a}ngig.  Die Anfrage
      lautet also wieder \\[0.1cm]
      \hspace*{1.3cm} \texttt{p(U)}.
\item W\"{a}hlen wir nun die zweite Klausel der Implementierung von \texttt{q/1},
      so m\"{u}ssen wir \texttt{q(U)} und \texttt{q(1)} unifizieren.  Diese Unifikation ist
      erfolgreich und wir erhalten die Instantiierung $\mathtt{U} \mapsto 1$, die die Anfrage beantwortet.
\end{enumerate}

\subsection{Verbesserung der Effizienz von \textsl{Prolog}-Programmen durch den Cut-Operator}
In der Praxis wird der Cut-Operator eingesetzt, um \"{u}berfl\"{u}ssige Auswahl-Punkte zu
entfernen und dadurch die Effizienz eines Programms zu steigern.  Als Beispiel
betrachten wir eine Implementierung des Algorithmus 
``\emph{Sortieren durch Vertauschen}'' (engl.~\emph{bubble sort}).
Wir spezifizieren diesen Algorithmus zun\"{a}chst durch bedingte Gleichungen.  Dabei 
benutzen wir die Funktion \\[0.1cm]
\hspace*{1.3cm} 
$\texttt{append}: \textsl{List}(\textsl{Number}) \times \textsl{List}(\textsl{Number}) \rightarrow \textsl{List}(\textsl{Number})$
\\[0.1cm]
Der Aufruf $\texttt{append}(l_1, l_2)$ liefert eine Liste, die aus allen Elementen von
$l_1$ gefolgt von den Elementen aus $l_2$ besteht.  In dem \textsl{SWI-Prolog}-System ist
ein entsprechendes Pr\"{a}dikat \texttt{append/3} implementiert.  Die Implementierung dieses
Pr\"{a}dikats deckt sich mit der Implementierung des Pr\"{a}dikats \texttt{concat/3}, die wir in
einem fr\"{u}heren Abschnitt vorgestellt hatten.  

Außerdem benutzen wir noch das Pr\"{a}dikat \\[0.1cm]
\hspace*{1.3cm} 
$\texttt{ordered}: \textsl{List}(\textsl{Number}) \rightarrow \mathbb{B}$, \\[0.1cm]
das \"{u}berpr\"{u}ft, ob eine Liste geordnet ist.
Die bedingten Gleichungen
zur Spezifikation der Funktion \\[0.1cm]
\hspace*{1.3cm} 
$\texttt{bubble\_sort}: \textsl{List}(\textsl{Number}) \rightarrow \textsl{List}(\textsl{Number})$
\\[0.1cm]
lauten nun:
\begin{enumerate}
\item $\mathtt{append}(l_1, [x,y|l_2], l) \wedge x > y \rightarrow \mathtt{bubble\_sort}(l) = \mathtt{bubble\_sort}(\mathtt{append}(l_1, [y,x|l_2]))$

      Wenn die Liste $l$ in zwei Teile $l_1$ und $[x,y|l_2]$ zerlegt werden kann
      und wenn weiter $x>y$ ist, dann vertauschen wir die Elemente $x$ und $y$
      und sortieren die so entstandene Liste rekursiv.
\item $\mathtt{ordered}(l) \rightarrow \mathtt{bubble\_sort}(l) = l$

      Wenn die Liste $l$ bereits sortiert ist, dann kann die Funktion
      $\mathtt{bubble\_sort}$ diese Liste unver\"{a}ndert als Ergebnis zur\"{u}ck geben.
\end{enumerate}
Die Gleichungen um das Pr\"{a}dikat \texttt{ordered} zu spezifizieren lauten:
\begin{enumerate}
\item $\mathtt{ordered}([]) = \mathtt{true}$

      Die leere Liste ist offensichtlich sortiert.
\item $\mathtt{ordered}([x]) = \mathtt{true}$

      Eine Liste, die nur aus einem Element besteht, ist ebenfalls sortiert.
\item $x \leq y \rightarrow \mathtt{ordered}([x,y|r]) = \mathtt{ordered}([y|r])$.

      Eine Liste der Form $[x,y|r]$ ist sortiert, wenn $x \leq y$ ist und
      wenn außerdem die Liste $[y|r]$ sortiert ist.
\end{enumerate}

\begin{figure}[!h]
  \centering
\begin{Verbatim}[ frame         = lines, 
                  framesep      = 0.3cm, 
                  labelposition = bottomline,
                  numbers       = left,
                  numbersep     = -0.2cm,
                  xleftmargin   = 0.8cm,
                  xrightmargin  = 0.8cm
                ]
    bubble_sort( L, Sorted ) :-
        append( L1, [ X, Y | L2 ], L ),
        X > Y,
        append( L1, [ Y, X | L2 ], Cs ),
        bubble_sort( Cs, Sorted ).
    
    bubble_sort( Sorted, Sorted ) :-
        is_ordered( Sorted ).
    
    
    is_ordered( [] ).
    
    is_ordered( [ _ ] ).
    
    is_ordered( [ X, Y | Ys ] ) :-
        X < Y,
        is_ordered( [ Y | Ys ] ).
\end{Verbatim}
\vspace*{-0.3cm}
  \caption{Der Bubble-Sort Algorithmus}
  \label{fig:bubble_sort}
\end{figure}

Abbildung \ref{fig:bubble_sort} zeigt die Implementierung des Bubble-Sort Algorithmus in
\textsl{Prolog}.  In Zeile 2 wird die als Eingabe gegebene Liste \texttt{L} in die beiden
Liste \texttt{L1} und \texttt{[X, Y | L2]} zerlegt.  Da es im Allgemeinen mehrere
M\"{o}glichkeiten gibt,  eine Liste in zwei Teillisten zu zerlegen, wird hierbei ein
Auswahl-Punkt gesetzt.  Anschließend wird gepr\"{u}ft, ob \texttt{Y} kleiner als \texttt{X}
ist.  Wenn dies der Fall ist, wird mit \texttt{append/3} die neue Liste 
\\[0.1cm]
\hspace*{1.3cm} $\mathtt{append(L_1, [Y,X|L_2])}$ \\[0.1cm]
gebildet und diese Liste wird rekuriv sortiert.  Wenn es nicht m\"{o}glich ist,
die Liste \texttt{L} so in zwei Listen \texttt{L1} und \texttt{[X, Y | L2]} zu zerlegen,
dass \texttt{Y} kleiner als \texttt{X} ist, dann muss die Liste \texttt{L} schon sortiert
sein.  In diesem Fall greift die zweite Klausel, die allerdings noch \"{u}berpr\"{u}fen muss, ob
\texttt{L} tats\"{a}chlich sortiert ist, denn sonst k\"{o}nnte beim Backtracking eine falsche
L\"{o}sung berechnet werden.

Das Problem bei dem obigen Programm ist die Effizienz.  Aufgrund der vielen M\"{o}glichkeiten
eine Liste zu zerlegen, wird beim Backtracking immer wieder dieselbe L\"{o}sung generiert. 
Beispielsweise liefert die Anfrage \\[0.1cm]
\hspace*{1.3cm} \texttt{bubble\_sort( [ 4, 3, 2, 1 ], L ), write(L), nl, fail.} \\[0.1cm]
16 mal dieselbe L\"{o}sung.  Abbildung \ref{fig:bubble_sort_cut} zeigt eine Implementierung,
bei der nur eine L\"{o}sung berechnet wird.  Dies wird durch den Cut-Operator in Zeile 4
erreicht.   Ist einmal eine Zerlegung der Liste \texttt{L} der Form
\\[0.2cm]
\hspace*{1.3cm}
$\mathtt{L} = \texttt{L}_1 +[ \texttt{X}, \mathtt{Y} | \mathtt{L}_2]$
\\[0.2cm] 
gefunden, bei der \texttt{Y} kleiner als \texttt{X} ist, so bringt es
nichts mehr, nach anderen Zerlegungen zu suchen, denn die urspr\"{u}nglich gegebene Liste
\texttt{L} l\"{a}sst sich ja auf jeden Fall dadurch sortieren, dass rekursiv die Liste 
\\[0.2cm]
\hspace*{1.3cm}
 $\mathtt{L}_1 + [\mathtt{Y},\mathtt{X}|\mathtt{L}_2]$ 
\\[0.1cm]
sortiert wird.  Dann kann auch der Aufruf der Pr\"{a}dikats \texttt{ordered/1} im Rumpf der zweiten
Klausel des Pr\"{a}dikats \texttt/2{bubble\_sort} entfallen, denn diese wird beim Backtracking
ja nur dann erreicht, wenn es keine Zerlegung der Liste \texttt{L} in  \texttt{L1} und 
\texttt{[X, Y | L2]} gibt, bei der \texttt{Y} kleiner als \texttt{X} ist.  Dann muss aber
die Liste \texttt{L} schon sortiert sein.

\begin{figure}[!h]
  \centering
\begin{Verbatim}[ frame         = lines, 
                  framesep      = 0.3cm, 
                  labelposition = bottomline,
                  numbers       = left,
                  numbersep     = -0.2cm,
                  xleftmargin   = 0.8cm,
                  xrightmargin  = 0.8cm
                ]
    bubble_sort( List, Sorted ) :-
        append( L1, [ X, Y | L2 ], List ),
        X > Y,
        !,
        append( L1, [ Y, X | L2 ], Cs ),
        bubble_sort( Cs, Sorted ).
    
    bubble_sort( Sorted, Sorted ).
\end{Verbatim}
\vspace*{-0.3cm}
  \caption{Effiziente Implementierung des  Bubble-Sort Algorithmus}
  \label{fig:bubble_sort_cut}
\end{figure}

Wenn wir bei der Entwicklung eines \textsl{Prolog}-Programms von bedingten Gleichungen
ausgehen, dann gibt es ein einfaches Verfahren, um das entstandene
\textsl{Prolog}-Programm durch die Einf\"{u}hrung von Cut-Operator effizienter zu machen:
Der Cut-Operator sollte nach den Tests, die vor dem Junktor ``$\rightarrow$'' stehen,
gesetzt werden.  Wir erl\"{a}utern dies durch ein Beispiel:  Unten sind noch einmal die
Gleichungen zur Spezifikation des Algorithmus ``\emph{Sortieren durch Mischen}'' wiedergegeben.
\begin{enumerate}
\item $\mathtt{odd}([]) = []$.
\item $\mathtt{odd}([h|t]) = [h|\mathtt{even}(t)]$.
\item $\mathtt{even}([]) = []$.
\item $\mathtt{even}([h|t]) = \mathtt{odd}(t)$.
\item $\mathtt{merge}([], l) = l$.
\item $\mathtt{merge}(l, []) = l$.
\item $x \leq y \rightarrow \mathtt{merge}([x|s], [y|t]) = [x|\mathtt{merge}(s, [y|t])]$.
\item $x  >   y \rightarrow \mathtt{merge}([x|s], [y|t]) = [y|\mathtt{merge}([x|s], t)]$.
\item $\mathtt{sort}([]) = []$.
\item $\mathtt{sort}([x]) = [x]$.
\item $\mathtt{sort}([x,y|t]) = \mathtt{merge}( \mathtt{sort}(\mathtt{odd}([x,y|t])), \mathtt{sort}(\mathtt{even}([x,y|t])))$.
\end{enumerate}
Das \textsl{Prolog}-Programm mit Cut-Operatoren sieht dann so aus wie in Abbildung
\ref{fig:merge-sort-cut} gezeigt.  Nur die Gleichungen 7.~und 8.~haben Bedingungen, bei
allen anderen Gleichungen gibt es keine Bedingungen.  Bei den Gleichungen 7.~und 8.~wird
der Cut-Operator daher nach dem Test der Bedingungen gesetzt, bei allen anderen Klauseln
wird der Cut-Operator dann am Anfang des Rumpfes gesetzt.

\begin{figure}[!h]
  \centering
\begin{Verbatim}[ frame         = lines, 
                  framesep      = 0.3cm, 
                  labelposition = bottomline,
                  numbers       = left,
                  numbersep     = -0.2cm,
                  xleftmargin   = 0.8cm,
                  xrightmargin  = 0.8cm
                ]
    % odd( +List(Number), -List(Number) ).    
    odd( [], [] ) :- !.    
    odd( [ X | Xs ], [ X | L ] ) :-
        !,
        even( Xs, L ).

    % even( +List(Number), -List(Number) ).
    even( [], [] ) :- !.
    even( [ _X | Xs ], L ) :-
        !,
        odd( Xs, L ).
    
    % merge( +List(Number), +List(Number), -List(Number) ).
    mix( [], Xs, Xs ) :- !.    
    mix( Xs, [], Xs ) :- !.
    mix( [ X | Xs ], [ Y | Ys ], [ X | Rest ] ) :-
        X =< Y,
        !,
        mix( Xs, [ Y | Ys ], Rest ).
    mix( [ X | Xs ], [ Y | Ys ], [ Y | Rest ] ) :-
        X > Y,
        !,
        mix( [ X | Xs ], Ys, Rest ).
    
    % merge_sort( +List(Number), -List(Number) ).
    merge_sort( [], [] ) :- !.
    merge_sort( [ X ], [ X] ) :- !.
    merge_sort( [ X, Y | Rest ], Sorted ) :-
        !,
        odd(  [ X, Y | Rest ], Odd  ),
        even( [ X, Y | Rest ], Even ),
        merge_sort( Odd,  Odd_Sorted  ),
        merge_sort( Even, Even_Sorted ),
        mix( Odd_Sorted, Even_Sorted, Sorted ).
\end{Verbatim}
\vspace*{-0.3cm}
  \caption{Sortieren durch Mischen mit Cut-Operatoren.}
  \label{fig:merge-sort-cut}
\end{figure}

Analysieren wir das obige Programm genauer, so stellen wir fest, dass viele der
Cut-Operatoren im Grunde \"{u}berfl\"{u}ssig sind.  Beispielsweise kann immer nur eine der beiden
Klauseln, die das Pr\"{a}dikat \texttt{odd/2} implementieren, greifen, denn entweder ist die
eingegebene Liste leer oder nicht.  Also sind die Cut-Operatoren in den Zeilen 2 und 4
redundant.  Andererseits st\"{o}ren sie auch nicht, so dass es f\"{u}r die Praxis das einfachste
sein d\"{u}rfte Cut-Operatoren stur nach dem oben angegebenen Rezept zu setzen.
\pagebreak

\vspace*{\fill}

\section{Literaturhinweise}
F\"{u}r eine umfangreiche und dem Thema angemessene Darstellung der Sprache \textsl{Prolog} 
fehlt in der einf\"{u}hrenden Vorlesung leider die Zeit.  Daher wird dieses Thema in einer
sp\"{a}teren Vorlesung auch wieder aufgegriffen.  Den Lesern, die ihre Kenntnisse jetzt schon
vertiefen wollen, m\"{o}chte ich die folgende Hinweise auf die Literatur geben:
\begin{enumerate}
\item \emph{The Art of Prolog} von Leon Sterling und Ehud Shapiro \cite{sterling:94}.
      Dieses Werk ist ein ausgezeichnete Lehrbuch, das auch f\"{u}r den Anf\"{a}nger gut lesbar ist.
\item \emph{Prolog Programming for Artificial Intelligence} von Ivan Bratko
      \cite{bratko:90}.  Neben der Sprache \textsl{Prolog} f\"{u}hrt dieses Buch auch in die
      k\"{u}nstliche Intelligenz ein.
\item \emph{Foundations of Logic Programnming}   von J.~W.~Lloyd \cite{lloyd:87}
      beschreibt die theoretischen Grundlagen der Sprache \textsl{Prolog}.
\item \emph{Prolog: The Standard} von Pierre Deransart, Abdel Ali Ed-Dbali und Laurent
      Cervoni \cite{deransart:96} gibt den ISO-Standard f\"{u}r die Sprache
      \textsl{Prolog} wieder.
\item \emph{SWI-Prolog 6.2.6 Reference Manual} von Jan Wielemaker \cite{wielemaker:2013}
      beschreibt das SWI-Prolog-System.  Dieses Dokument ist im Internet unter der Adresse 
      \\[0.1cm]
      \hspace*{1.3cm}      
      \href{http://www.swi-prolog.org/download/stable/doc/SWI-Prolog-6.2.6.pdf}{\texttt{http://www.swi-prolog.org/download/stable/doc/SWI-Prolog-6.2.6.pdf}}
      \\[0.1cm]
      verf\"{u}gbar.
\end{enumerate}

%%% Local Variables: 
%%% mode: latex
%%% TeX-master: "logik"
%%% End: 
