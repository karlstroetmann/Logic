\documentclass{article}
\usepackage{german}
\usepackage[latin1]{inputenc}
\usepackage{a4wide}
\usepackage{amssymb}
\usepackage{fancyvrb}
\usepackage{alltt}
\pagestyle{empty}

\begin{document}
\noindent
{\Large Exercise: Who Owns the Zebra?}
\vspace{0.5cm}

\noindent
Implement a \textsc{Prolog} program that solves the following puzzle.
\begin{enumerate}
\item There is a street that has exactly 5 houses on one side.
      Each of these houses has a different color.
\item Each of these houses is inhabited by a man with a different nationality.
\item These inhabitants each have a different pet, a different kind of drink, and a 
      different brand of cigarettes.
\item None of them has more than one drink, more than one pet, or more than one 
      brand of cigarettes.
\end{enumerate}
Furthermore, the following is known:
\begin{enumerate}
\item The Briton lives in the red house.
\item The Swede has a dog.
\item The American drinks whiskey. 
\item The green house is to the left of the white house.
\item The person living in the green house drinks coffee.
\item The person smoking  PallMall has a bird.
\item The man living in the middle house drinks milk.
\item The man living in the yellow house smokes Dunhill.
\item The Norwegian lives in the first house.
\item The Marlboro smoker lives next to someone who has a cat.
\item The person having a pig as pet lives next to the Dunhill smoker.
\item The Winfield smoker likes to drink beer.
\item The Norwegian lives next to the blue house.
\item The German smokes Rothmanns.
\item The Marlboro smoker lives next to someone who drinks water.
\end{enumerate}
The question is: Who owns the Zebra?
\vspace{0.5cm}

\noindent 
Try to solve this problem using an approach that is similar to the
approach that I have used in my lecture notes to solve the puzzle concerning the
programming contest.  The solution for the programming contest can be found in my website
at
\\[0.2cm]
\hspace*{1.3cm}
\texttt{http://www.dhbw-stuttgart.de/stroetmann/Prolog/contest.pl}

\end{document}

%%% Local Variables: 
%%% mode: latex
%%% TeX-master: t
%%% End: 
