\chapter{Grenzen der Berechenbarkeit}
\section{Das Halte-Problem}
In diesem Abschnitt werden wir eine konkrete Funktion angeben, die
nicht durch ein \texttt{C}-Programm implementiert werden kann.  Vorweg geben wir eine Definition, die
wir ben"otigen, um das Halte-Problem konkret formulieren zu k"onnen.
\begin{Definition}[Test-Funktion] {\em Ein String $t$ ist eine \emph{Test-Funktion} mit Namen $n$ wenn $t$ ein
String der Form \\[0.3cm]
\hspace*{1.3cm} {\tt int $n$(char* x) \{ $\cdots$ \}} \\[0.3cm]
ist, der sich als Definition einer $C$-Funktion parsen l"a\3t.  Die Menge der
Test-Funktionen bezeichnen wir mit $T\!F$.  Ist $t \in T\!F$ und hat den Namen $n$, so
schreiben wir $\mathtt{name}(t) = n$.} \hspace*{\fill} $\Box$
\end{Definition}

\noindent
\textbf{Beispiele}:  
\begin{enumerate}
\item $s_1$ = ``{\tt int simple(char* x) \{ return 0; \}}''

      $s_1$ ist eine (sehr einfache) Test-Funktion mit dem Namen \texttt{simple}.
\item $s_2$ = ``{\tt int loop(char* x) \{ while (1) ++x; \}}''

      $s_2$ ist eine Test-Funktion mit dem Namen \texttt{loop}. 
\item $s_3$ = ``{\tt int loop(char* x);}''

      $s_3$ ist keine Test-Funktion, denn es ist lediglich eine Deklaration einer
      \texttt{C}-Funktion und keine Definition.
\item $s_4$ = ``{\tt int hugo(char* x) begin i := 1; end;}''

      $s_4$ ist keine Test-Funktion, denn es l"a\3t sich mit einem
      \texttt{C}-Compiler nicht fehlerfrei parsen.
\item $s_5$ = ``{\tt int hugo(int x) \{ return i*i; \}}''

      $s_5$ ist auch keine Test-Funktion, denn der Typ des Arguments ist \texttt{int}
      und nicht \texttt{char*}.
\end{enumerate}
Um das Halte-Problem "ubersichtlicher formulieren zu k"onnen, f"uhren wir noch drei
zus"atzliche Notationen ein.
\begin{Notation}[$\leadsto$, $\downarrow$, $\uparrow$]
{\em
Ist $n$ der Name einer \texttt{C}-Funktion und sind $a_1$, $\cdots$, $a_n$ Argumente, die
vom Typ her der Deklaration von $n$ entsprechen, so schreiben wir \\[0.3cm]
\hspace*{1.3cm} $n(a_1, \cdots, a_n) \leadsto r$ \\[0.3cm]
wenn der Aufruf $n(a_1, \cdots, a_n)$ das Ergebnis $r$ liefert.  Sind wir an dem Ergebnis
selbst nicht interessiert, sondern wollen nur angeben, da\3 ein Ergebnis existiert, so
schreiben wir \\[0.3cm]
\hspace*{1.3cm} $n(a_1, \cdots, a_n) \,\downarrow$ \\[0.3cm]
Terminiert der Aufruf $n(a_1, \cdots, a_n)$ nicht, so schreiben wir \\[0.3cm]
\hspace*{1.3cm} $n(a_1, \cdots, a_n) \,\uparrow$ \hspace*{\fill} $\Box$
}
\end{Notation}

\noindent
\textbf{Beispiele}: Legen wir die Funktions-Definitionen zugrunde, die wir im Anschlu\3 an
die Definition des Begriffs der Test-Funktion gegeben haben, so gilt:
\begin{enumerate}
\item {\tt simple(\symbol{34}emil\symbol{34}) $\leadsto 0$}
\item {\tt simple(\symbol{34}emil\symbol{34}) $\downarrow$}
\item {\tt loop(\symbol{34}hugo\symbol{34}) $\uparrow$}
\end{enumerate}

\noindent
Wir betrachten jetzt ein \texttt{C}-Funktion \texttt{stops}, die folgende Deklaration
hat:
\begin{verbatim}
      int stops(char* t, char* a);
\end{verbatim}
Wir nehmen einmal an, dass es eine Implementierung dieser Funktion g"abe, die das Halte-Problem l"ost.  Genauer sagen wir, dass
f"ur den Aufruf \texttt{stops($t$, $a$)} folgendes gelten soll:
\begin{enumerate}
\item $t \not\in T\!F \quad\rightarrow\quad \mathtt{stops}(t, a) \leadsto 2$.

      In Worten: Wenn $t$ keine Test-Funktion ist, dann liefert der Aufruf \\
      \texttt{stops($t$, $a$)} den Wert 2 zur"uck.
\item $t \in T\!F \,\wedge\, \mathtt{name}(t) = n \,\wedge\, n(a)\downarrow \quad\rightarrow\quad
       \mathtt{stops}(t, a) \leadsto 1$.

      In Worten: Wenn $t$ eine Test-Funktion mit Namen $n$ ist und der Aufruf $n(a)$ terminiert,
      dann liefert der Aufruf \texttt{stops($t$, $a$)} den Wert 1 zur"uck.
\item $t \in T\!F \,\wedge\, \mathtt{name}(t) = n \,\wedge\, n(a)\uparrow \quad\rightarrow\quad
       \mathtt{stops}(t, a) \leadsto 0$.

      In Worten: Wenn $t$ eine Test-Funktion mit Namen $n$ ist und der Aufruf $n(a)$ \underline{nicht} terminiert,
      dann liefert der Aufruf \texttt{stops($t$, $a$)} den Wert 0 zur"uck. 
\end{enumerate}

Als n"achstes betrachten wir eine etwas seltsame Test-Funktion, die den Namen
\texttt{strange} tr"agt. Diese Funktion ist in Abbildung \ref{fig:strange} zu sehen.
Die Zeilen in der Abbildung sind numeriert.  Diese Zeilennummern sind aber nicht Teil der
\texttt{C}-Funktion und dienen nur der folgenden Erl"auterung.
\begin{figure}[!h]
\begin{verbatim}
 01   int strange(char* x) {
 02       int result;
 03       result = stops(x, x);
 04       if (result == 1) 
 05           while (1) 
 06               ++result;
 07       return result;
 08   }
\end{verbatim}
  \caption{Die Test-Funktion \texttt{strange}.}
  \label{fig:strange}
\end{figure}

Zun"achst "uberzeugen wir uns davon, dass die abgebildete Funktion sich als $C$-Funktion parsen
l"a\3t.  Dies k"onnen wir beispielsweise dadurch tun, dass wir die Funktion mit einem
\texttt{C}-Compiler "ubersetzen.  Als n"achstes bezeichnen wir den String in Abbildung 
\ref{fig:strange} mit $Strange$.  Wir definieren also \\[0.3cm]
\hspace*{1.3cm} $Strange$ = ``{\tt int strange(char* x) \{} \\
\hspace*{3.8cm}  {\tt       int result;} \\
\hspace*{3.8cm}  {\tt       result = stops(x, x);} \\
\hspace*{3.8cm}  {\tt       if (result == 1) } \\
\hspace*{4.3cm}  {\tt           while (1) } \\
\hspace*{4.8cm}  {\tt               ++result;} \\
\hspace*{3.8cm}  {\tt       return result;} \\
\hspace*{3.3cm}  {\tt   \}}'' \\[0.3cm]
und stellen fest, dass gilt: \\[0.3cm]
\hspace*{1.3cm} $Strange \in T\!F \;\wedge\; \mathtt{name}(Strange) = \mathtt{strange}$. \\[0.3cm]
Damit sind wir in der Lage, den String $Strange$ als Eingabe der Funktion \texttt{stops}
zu verwenden.  Wir betrachten nun den folgenden Aufruf: \\[0.3cm]
\hspace*{1.3cm} \texttt{stops($Strange$, $Strange$)}. \\[0.3cm]
Es k"onnen drei F"alle auftreten:
\begin{enumerate}
\item $\mathtt{stops}(Strange, Strange) \leadsto 0$. 

      Nach der Spezifikation von \texttt{stops} bedeutet dies \\[0.3cm]
      \hspace*{1.3cm} $\mathtt{strange}(Strange) \uparrow$ \\[0.3cm]
      Schauen wir nun, was wirklich beim Aufruf \texttt{strange($Strange$)} passiert:
      In Zeile 3 erh"alt die Variable \texttt{result} den Wert 0 zugewiesen.  In Zeile 4
      wird dann getestet, ob \texttt{result} den Wert 1 hat.  Dieser Test schl"agt fehl.
      Daher wird der Block der \texttt{if}-Anweisung nicht ausgef"uhrt und die Funktion liefert als
      n"achstes in Zeile 7 den Wert 0 zur"uck.  Insbesondere terminiert der Aufruf also, im
      Widerspruch zu dem, was die Funktion \texttt{stops} behauptet hat.
\item  $\mathtt{stops}(Strange, Strange) \leadsto 1$. 

      Aus der Spezifikation der Funktion \texttt{stops} folgt: \\[0.3cm]
      \hspace*{1.3cm} $\mathtt{strange}(Strange) \downarrow$ \\[0.3cm]
      Schauen wir nun, was wirklich beim Aufruf \texttt{strange($Strange$)} passiert:
      In Zeile 3 erh"alt die Variable \texttt{result} den Wert 1 zugewiesen.  In Zeile 4
      wird dann getestet, ob \texttt{result} den Wert 1 hat.  Diesmal gelingt der Test.
      Daher wird der Block der \texttt{if}-Anweisung ausgef"uhrt.  Dieser Block
      besteht aber nur aus einer Endlos-Schleife, aus der wir nie wieder zur"uck kommen.
      Das steht im Widerspruch zu dem, was die Funktion \texttt{stops} behauptet hat.
\item  $\mathtt{stops}(Strange, Strange) \leadsto 2$. 

      Dann m"u\3te aber der Programmtext $Strange$ einen Syntax-Fehler enthalten, was nicht
      der Fall ist.
\end{enumerate}
Insgesamt haben wir also in jedem Fall einen Widerspruch erhalten.  
Wir haben diesen Widerspruch aus der Annahme erhalten, dass die \texttt{C}-Funktion \texttt{stops}
das Halte-Problem l"ost.  Daher haben wir gezeigt, dass es keine \texttt{C}-Funktion
geben kann, die das Halte-Problem l"ost.

An dieser Stelle k"onnen wir uns fragen, ob es vielleicht eine andere Programmier-Sprache
gibt, in der wir das Halte-Problem dann vielleicht doch l"osen k"onnten.  
Wenn es in dieses Programmier-Sprache Unterprogramme gibt, und wenn wir dort
Programm-Texte als Argumente von Funktionen "ubergeben k"onnen, dann ist leicht zu sehen,
dass der obige Beweis der 
Unl"osbarkeit des Halte-Problems sich durch geeignete syntaktische Modifikationen auch auf
die andere Programmier-Sprache "ubertragen l"a\3t.

\section{Die Church'sche These}
In diesem Zusammenhang sollten wir noch die \emph{Church'sche These}\footnote{
Diese These wurde 1936 von Alonzo Church (1903-1995) aufgestellt.} erw"ahnen.  F"ur unsere Zwecke k"onnen wir diese wie folgt formulieren:
\begin{center}
\begin{tabular}{|l|}
 \hline
 \hline
    \rule{0pt}{14pt}
    Ist $f$ eine Funktion die wir mit einer beliebigen, geeigneten \\
    Programmier-Sprache berechnen k"onnen, so k"onnen wir $f$ \\
        auch durch ein geeignetes  \texttt{C}-Programm 
    berechnen.     
    \rule[-6pt]{0pt}{14pt}
    \\
 \hline
 \hline
\end{tabular}
\end{center}
Es kann nat"urlich durchaus sein, dass eine bestimmte Programmier-Sprache wesentlich
geeigneter ist, um ein gegebenes Problem zu l"osen.  F"ur ein gegebenes Problem ist es
leicht m"oglich, dass das \texttt{C}-Programm deutlich l"anger und komplizierter
ist, als ein Programm, das in einer geeigneteren Sprache entwickelt wird.  Nur gibt es
eben kein l"osbares Problem, das wir nicht schon in \texttt{C} l"osen k"onnen, auch wenn das
vielleicht viel m"uhsamer ist als in einer anderen Sprache.

Die Church'sche These ist ihrer Natur nach nicht formal beweisbar, weil jeder Beweis eine
irgendwie geartete Formalisierung des Berechenbarkeits-Begriffs voraussetzen w"urde.  
In der Literatur wird die Church'sche These in der Regel nicht durch Bezug auf die
Programmier-Sprache \texttt{C} formuliert. Stattdessen greift man dort auf den Begriff der
\emph{Turing-Berechenbarkeit}  zur"uck.  Diesem Begriff liegt das Konzept eines sehr
primitiven Computers zugrunde, der sogenannten \emph{Turing-Maschine}\footnote{
Dieses Modell eines Computers geht auf Alan Turing (1912-1954) zur"uck.}.
  Die Tatsache, dass es nie gelungen ist Programmier-Sprachen
zu entwickeln, die mehr berechnen k"onnen als eine Turing-Maschine berechnen kann, hat
zur allgemeinen Akzeptanz der Church'schen These gef"uhrt.

\section{Andere nicht berechenbare Funktionen}
Es gibt noch eine ganze Reihe anderer Funktionen, die nicht berechenbar sind.  In der
Regel werden wir den Nachweis, dass eine bestimmt Funktion nicht berechenbar ist dadurch f"uhren, dass
wir zun"achst annehmen, dass die gesuchte Funktion doch implementierbar ist.  Unter dieser Annahme
konstruieren wir dann eine Funktion, die das Halte-Problem l"ost, was im Widerspruch zu dem am Anfang dieses Abschnitts
bewiesen Sachverhalts steht.
Dieser Widerspruch zwingt uns zu der Folgerung, dass die gesuchte Funktion nicht implementierbar ist.
Wir werden dieses Verfahren an einem Beispiel demonstrieren. Vorweg ben"otigen wir aber
noch eine Definition.

\begin{Definition}[$\simeq$] 
{\em 
Es seien $n_1$ und $n_2$ zwei \texttt{C}-Funktionen und
  $a_1$, $\cdots$, $a_n$ seien Argumente, mit denen wir diese Funktionen f"uttern k"onnen. Wir definieren \\[0.3cm]
\hspace*{1.3cm} $n_1(a_1,\cdots,a_n) \simeq n_2(a_1,\cdots,a_n)$ \\[0.3cm]
g.d.w. einer der beiden folgen F"alle auftritt:
\begin{enumerate}
\item $n_1(a_1,\cdots,a_n)\uparrow \quad\wedge\quad n_2(a_1,\cdots,a_n)\uparrow$
\item $\exists r: \Bigl(n_1(a_1,\cdots,a_n) \leadsto r \quad\wedge\quad n_2(a_1,\cdots,a_n) \leadsto r\Bigr)$
  \hspace*{\fill} $\Box$
\end{enumerate}}
\end{Definition}

\noindent
Wir kommen jetzt zum \emph{"Aquivalenz-Problem}.  Die Funktion
\begin{verbatim}
       int equal(char* p1, char* p2, char* a)
\end{verbatim}
m"oge folgender Spezifikation gen"ugen:
\begin{enumerate}
\item $p_1 \not\in T\!F \;\vee\; p_2 \not\in T\!F \quad\rightarrow\quad \mathtt{equal}(p_1, p_2, a) \leadsto 2$.
\item Falls 
  \begin{enumerate}
  \item $p_1 \in T\!F \;\wedge\; \mathtt{name}(p_1) = n_1$,
  \item $p_2 \in T\!F \;\wedge\; \mathtt{name}(p_2) = n_2$ \quad und
  \item $n_1(a) \simeq n_2(a)$
  \end{enumerate}
    gilt, dann mu\3 gelten: \\[0.3cm]
   \hspace*{1.3cm}  $\mathtt{equal}(p_1, p_2, a) \leadsto 0$.
\item Ansonsten gilt \\[0.1cm]
      \hspace*{1.3cm} $\mathtt{equal}(p_1, p_2, a) \leadsto 1$.
\end{enumerate}
Wir sagen, dass eine Funktion, die der eben angegebenen Spezifikation gen"ugt, das
\emph{"Aquivalenz-Problem} l"ost.

Machen wir uns also an die Arbeit und zeigen, dass es keine Funktion geben kann, die das
"Aquivalenz-Problem l"ost.  Wie oben skizziert f"uhren wir den Beweis indirekt und nehmen
an, dass es doch eine Implementierung der Funktion \texttt{equal} gibt, die das
"Aquivalenz-Problem l"ost.  Wir betrachten die in Abbildung 
\ref{fig:stops} angegeben Implementierung der Funktion \texttt{stops}.
\begin{figure}[!h]
\begin{verbatim}
01 int stops(char* p, char* a) {
02    return equal("int loop(char* x) { while (1) ++x; }", p, a);
03 }
\end{verbatim}
  \caption{Eine Implementierung der Funktion \texttt{stops}.}
  \label{fig:stops}
\end{figure}

Zu beachten ist, dass in Zeile 2 die Funktion \texttt{equal} mit einem String aufgerufen
wird, der eine Test-Funktion ist, und zwar mit dem Namen \texttt{loop}.  Diese hat die
folgende Form:
\begin{verbatim}
      int loop(char* x) { while (1) ++x; }
\end{verbatim}
Es ist offensichtlich, dass die Funktion \texttt{loop} f"ur kein Ergebnis terminiert.
Ist also das Argument $p$ eine Test-Funktion mit Namen $n$, so liefert die Funktion \texttt{equal} immer dann den Wert 0,
wenn $n(a)$ nicht terminiert, andernfalls mu\3 sie den Wert 1 zur"uck geben.
Damit w"urde aber die Funktion \texttt{stops} das Halte-Problem l"osen und die Annahme,
dass \texttt{equal} das "Aquivalenz-Problem l"ost, ist widerlegt.
\pagebreak

%%% Local Variables: 
%%% mode: latex
%%% TeX-master: "informatik-script.tex"
%%% End: 
