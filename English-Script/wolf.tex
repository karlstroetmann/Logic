\subsection{Puzzle: Crossing the River}
We next show how to solve a puzzle using the theory developed so far.
\vspace*{0.3cm}

\begin{minipage}[c]{14cm}
{\sl
A farmer needs to bring a wolf, a goat, and a cabbage across the river to reach the market
place.  The boat is small and the farmer can only transport one passenger at a time.
If he leaves the wolf and the goat alone together, the wolf will eat the goat.
If he leaves the goat and the cabbage alone together, the goat will eat the cabbage.
Of course, neither the wolf nor the goat or the cabbage can steer the boat alone.
}
\end{minipage}
\vspace*{0.3cm}

\noindent
Our goal is to develop a schedule that can be used by the farmer to transport all trade goods 
safely to the market place at the other shore of the river.
In order to do so we represent the problem using a binary relation
$R \subseteq P \times P$.
The elements of the set $P$ will represent the different situations that can occur.
We will model these situations as subsets of the set 
\\[0.2cm]
\hspace*{1.3cm} 
$\texttt{All} := \{ \textsl{farmer}, \textsl{wolf}, \textsl{goat},\textsl{cabbage} \}$.
\\[0.2cm]
A subset $s \subseteq \mathtt{All}$ is interpreted as the set of those objects 
that are on the left shore of the river. For example, the set 
\\[0.2cm]
\hspace*{1.3cm}
$s = \{ \textsl{farmer}, \textsl{goat} \}$
\\[0.2cm]
describes the situation where the farmer and the goat is on the left shore of the river, while the
wolf and the cabbage wait on the right shore.  Therefore, the set $P$ is just the power set of \texttt{All}:
\\[0.2cm]
\hspace*{1.3cm}
$P := 2^\mathtt{All}$.
\\[0.2cm]
In order to define the relation  $R$ we define a procedure 
$\textsl{problem}(s)$. This procedure gets a set $s \el P$ representing the objects on the left shore
of the river after the farmer has crossed over to the right shore.  The procedure checks
whether there is a problem on the left shore.
Now there is a problem if either the goat is together with the cabbage or the wolf is
together with the goat.  Therefore, this procedure is implemented as follows:
\begin{verbatim}
  procedure problem(S);
      return ("goat" in S and "cabbage"  in S) or ("wolf" in S and "goat" in S);
  end problem;
\end{verbatim}
Using this procedure, we can define a relation $R_1$ that describes all safe passages from the left
shore of the river to the right shore as follows:
\begin{verbatim}
    R1  := { [ S, S - B ] : S in P, B in pow S |
             "farmer" in B and #B <= 2 and not problem(S - B) 
           };
\end{verbatim}
This relation can be interpreted as follows.
\begin{enumerate}
\item $S$ is the set of objects on the left shore.  In principle,
      every subset of \texttt{All} can be on the left shore of the river.  Therefore, 
      we have the iterator ``\texttt{S in P}''.
\item $B$ is the set of objects crossing over to the right shore using the boat.
      Of course, for an object to cross from the left shore to the right shore, it has to be
      on the left shore first.  Therefore, $B$ has to be a subset of $S$.  To enforce this,
      we use the iterator ``\texttt{B in pow S}''.
\item $S - B$ is the set of objects that remain on the left shore.
\item Furthermore, we have three conditions concerning the crossing:
      \begin{enumerate}
      \item The farmer has to be a passenger of the boat.  Therefore, we must have
            \\[0.2cm]
            \hspace*{1.3cm}
            \texttt{"farmer" in B}.
      \item The boat can only contain two passengers.  Therefore, 
            \\[0.2cm]
            \hspace*{1.3cm} \texttt{\#B <= 2}.
      \item There must not be a problem on the left shore after the crossing:
            \\[0.2cm]
            \hspace*{1.3cm}
            \texttt{not problem(S - B)}.
      \end{enumerate}
\end{enumerate}
In a similar way, we can describe the crossings form the right shore to the left shore.
\begin{verbatim}
    R2  := { [ S, S + B ] : S in P, B in pow (All - S) |
             "farmer" in B and #B <= 2 and not problem(All - (S + B)) };
\end{verbatim}
The interpretation of \texttt{R2} is as follows:
\begin{enumerate}
\item Again, $S$ is the set of objects on then left shore.   Therefore, 
      we have the iterator ``\texttt{S in P}''.
\item $B$ is the set of objects crossing over from the right shore to the left shore using the boat.
      Of course, for an object to cross from the right shore to the left shore, it has to be
      on the right shore first.  Now, if $S$ is the set of objects on the left shore, the
      set  of objects on the right shore is given as $\texttt{All} - S$.    
      Therefore, $B$ has to be a subset of $\texttt{All} - S$.  To enforce this,
      we use the iterator ``\texttt{B in pow (All - S)}''.
\item $S + B$ is the set of objects that remain on the left shore.
\item Furthermore, we have the following  conditions:
      \begin{enumerate}
      \item The farmer has to be a passenger of the boat.  Therefore, we must have
            \\[0.2cm]
            \hspace*{1.3cm}
            \texttt{"farmer" in B}.
      \item The boat can only contain two passengers.  Therefore, 
            \\[0.2cm]
            \hspace*{1.3cm} \texttt{\#B <= 2}.
      \item There must not be a problem on the right shore after the crossing.
            After the crossing, the set of objects on the right shore is given as
            $\mathtt{All} - (S + B)$.  Therefore, we need the condition
            \\[0.2cm]
            \hspace*{1.3cm}
            \texttt{not problem(All - (S + B))}.
      \end{enumerate}
\end{enumerate}
 The set of all possible crossings is the union of
both \texttt{R1} and \texttt{R2}.
Finally, we have to model the start state and the goal state we want to reach.  
In the beginning, everything is on the left shore, so we define
\\[0.2cm]
\hspace*{1.3cm}
\texttt{start := All;}
\\[0.2cm]
The goal is to bring everything to the right shore, so that nothing is left on the left
shore.  Therefore, we define
\\[0.2cm]
\hspace*{1.3cm}
\texttt{goal  := \{\};}
\\[0.2cm]
This gives rise to the program shown in figure  \ref{fig:wolf-ziege-kohl-assym.stl}.
Note that the procedure \texttt{reachable} is the same as shown in the previous section.
The solution computed by the program is shown in figure  \ref{fig:wolf-ziege-solution}.
In order to be more comprehensible, this output has been formated.

\begin{figure}[!ht]
  \centering
\begin{Verbatim}[ frame         = lines, 
                  framesep      = 0.3cm, 
                  labelposition = bottomline,
                  numbers       = left,
                  numbersep     = -0.2cm,
                  xleftmargin   = 0.0cm,
                  xrightmargin  = 0.0cm,
                ]
    program main;
        All := { "farmer", "wolf", "goat", "cabbage" };
        P   := pow All;
        R1  := { [ S, S - B ] : S in P, B in pow S |
                 "farmer" in B and #B <= 2 and not problem(S - B) };
        R2  := { [ S, S + B ] : S in P, B in pow (All - S) |
                 "farmer" in B and #B <= 2 and not problem(All - (S + B)) };
        R   := R1 + R2;
    
        start := All;
        goal  := {};
    
        path := reachable(start, goal, R);
        print(path);
    
        procedure problem(S);
            return ("goat" in S and "cabbage"  in S) or
                   ("wolf"  in S and "goat" in S);
        end problem;
    
        procedure reachable(x, y, R);
            P := { [x] };
            loop
                Old_P := P;
                P     := P + path_product(P, R);
                Found := { p in P | p(#p) = y };
                if Found /= {} then
                    return arb Found;
                end if;
                if P = Old_P then
                    return;
                end if;
            end loop;
        end reachable;
    
        procedure path_product(P, Q);
            return { add(p,q) : p in P, q in Q | p(#p) = q(1) and not cyclic(add(p,q)) };
        end path_product;    
    
        procedure cyclic(p);
            return #{ x : x in p } < #p;
        end cyclic;
    
        procedure add(p, q);
            return p + q(2..);
        end add;    
    
    end main;
\end{Verbatim} 
\vspace*{-0.3cm}
\caption{How does the farmer reach the other shore?}  
\label{fig:wolf-ziege-kohl-assym.stl}
\end{figure} %\$


\noindent

\begin{figure}[!ht]
  \centering
\begin{Verbatim}[ frame         = lines, 
                  framesep      = 0.3cm, 
                  labelposition = bottomline,
                  numbers       = left,
                  numbersep     = -0.2cm,
                  xleftmargin   = 0.0cm,
                  xrightmargin  = 0.0cm,
                ]
    {"goat", "cabbage", "wolf", "farmer"}                                         {}
                             >>>> {"goat", "farmer"} >>>> 
    {"cabbage", "wolf"}                                           {"goat", "farmer"}
                             <<<< {"farmer"} <<<< 
    {"cabbage", "wolf", "farmer"}                                           {"goat"}
                             >>>> {"cabbage", "farmer"} >>>> 
    {"wolf"}                                           {"goat", "cabbage", "farmer"}
                             <<<< {"goat", "farmer"} <<<< 
    {"goat", "wolf", "farmer"}                                           {"cabbage"}
                             >>>> {"wolf", "farmer"} >>>> 
    {"goat"}                                           {"cabbage", "wolf", "farmer"}
                             <<<< {"farmer"} <<<< 
    {"goat", "farmer"}                                           {"cabbage", "wolf"}
                             >>>> {"goat", "farmer"} >>>> 
    {}                                         {"goat", "cabbage", "wolf", "farmer"}
\end{Verbatim} 
\vspace*{-0.3cm}
\caption{A schedule for the farmer.}  
\label{fig:wolf-ziege-solution}
\end{figure} %\$
\noindent
\pagebreak

\subsection{Concluding Remarks}
We have discussed only a small set of the features of  \textsc{Setl}.
In particular we haven't discussed the following topics:
\begin{enumerate}
\item string processing,
\item reading and writing of files,
\item interaction with the operation system,
\item object orientation,
\item packages,
\item the interface to the programming language  \textsl{C},
\item the interface to \textsl{Tk} which enables a graphical user interface.
\end{enumerate}
A detailed discussion of all these aspects is given in the book of  Jack
Schwartz \cite{schwarz03} that can be found at
 \\[0.2cm]
\hspace*{1.3cm} \texttt{http://www.settheory.com}. \\[0.2cm]

\noindent
\textbf{Note}:
Most of the algorithms developed in this chapter are not at all efficient.
They have rather been chosen to clarify the notions of set theory.  
We will discuss more efficient algorithms in the second term.

%%% Local Variables: 
%%% mode: latex
%%% TeX-master: "logic"
%%% End: 
