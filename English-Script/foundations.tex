\chapter{Mathematical Foundations} 
\emph{Formal logic} and \emph{set theory} are at the foundation of computer science.
Therefore, these two topics are the central themes of this introductory computer science lecture. 
Both set theory and formal logic are rather abstract and quite difficult.  
According to my experience, a lot of students  have a  hard time understanding these
topics.  Therefore it is necessary to motivate the use of set theory and formal logic.
To this end, the following section will convince you that computer science needs a solid
scientific foundation.   As this lecture progresses you will see that set theory and logic
can indeed provide this scientific foundation.

\section{Motivation}
Modern software and hardware system are the most complex systems ever designed by humanity.
There are IT projects  that take several thousand developers and several years to complete.
It is obvious that the failure of such a project is connected with enormous financial losses.
Let me give a few examples of software projects that illustrate this point.
\begin{enumerate}
\item On June 9th of 1996 the Ariane 5 rocket disintegrated on its maiden voyage.
      This was due to a chain of software errors:
      \begin{itemize}
      \item A sensor of the navigation system measured the horizontal inclination and stored
            this number as a 64 bit floating point number.
      \item This number was converted into a 16 bit fixed point number.
            During this conversion there was an overflow because the 64 bit number could not be
            represented with 16 bits.
      \item As a consequence, the navigation system produced an error message, which was sent
            to the controlling unit.
      \item The controlling unit wrongly interpreted the error message as flight data and
            tried to correct the flight path accordingly.  
      \item The resulting accelerations caused the rocket to disintegrate.  
            Four satellites were lost, the financial loss
            was estimated to be a few hundred million dollars.

      \item You can find a complete report on the Ariane 5 disaster at \\[0.1cm]
      \hspace*{1.3cm} \texttt{http://www.ima.umn.edu/\symbol{126}arnold/disasters/ariane5rep.html}
     
      \end{itemize}
\item The Therac 25 is a medical device for X-ray treatment of cancer patients.
      Due to a software error in 1985 at least 6 patients got a severe overdose of radiation.
      Three of these did not survive the overdose.
      A detailed description of these accidents is available at \\[0.1cm]
      \hspace*{1.3cm} \texttt{http://courses.cs.vt.edu/\symbol{126}cs3604/lib/Therac\_25/Therac\_1.html}      
\item During the first golf war, an Iraqi  \textsl{Scud} missile could not be intercepted
      by the  \textsl{Patriot} anti-aircraft system.  This was due to a programming error
      in the \textsl{Patriot} control software.
      As a result, 28 soldiers lost their lives, and a hundred more were severely injured.
\item You can find a listing of more software related accidents at \\[0.1cm]
      \hspace*{1.3cm} \texttt{http://www.cs.tau.ac.il/\symbol{126}nachumd/horror.html}.
\end{enumerate}
The examples given above show that the construction of IT systems requires both diligence
and precision.  Therefore, the development of IT systems needs a solid scientific foundation.
This foundation consists of both formal logic and set theory.
Besides being used as foundation, both set theory and formal logic have immediate
applications in computer science. 
\begin{enumerate}
\item Set theory and the theory of relations (which is a part of set theory)
      is the foundation of the theory of relational databases.
\item The programming languages \textsl{Prolog}, which is used primarily in 
      AI\footnote{AI is short for \emph{artificial intelligence}.} applications,
      is based on predicate logic.
\end{enumerate}
Besides the immediate applications of formal logic and set theory, our engagement with these topics
has another very important reason:  Complex systems can only be controlled by suitable abstractions.
Nobody is able to understand every detail of a program consisting of several hundred thousand lines
of codes.  The only way to manage a software system of that size is to introduce suitable
abstractions.  Therefore, a higher-than-average ability to abstract is the most important ability of
a computer scientist.  Our treatment of logic and set theory aims at improving this ability.

Of course, there is another very important reason that should motivate you to study both formal
logic and set theory sincerely and I don't want to conceal this reason:  It is the written
examination at the end of this term.

\paragraph{Overview} 
This lecture will cover the following topics:
\begin{enumerate}
\item mathematical {formul\ae}

      Mathematical {formul\ae} can express complex statements of affairs
      much more concise than descriptions given in natural language.
\item set theory

      Set theory is the foundation of both modern mathematics and computer science.
\item \textsc{Setl}
  
      The programming language \textsc{Setl} is directly based on set theory.
\item propositional logic

      Propositional logic analyzes the meaning of connectors like \emph{and}, \emph{or}, 
      and \emph{if $\cdots$ then}.
\item predicate logic      

      Predicate logic adds the \emph{quantifiers} ``\emph{forall}'' and ``\emph{exists}''
      to propositional logic.
\item \textsl{Prolog}

      \textsl{Prolog} is a programming language based on predicate logic.
\item limits of computability

      Certain problems cannot be solved by algorithms.  We will see that the question,
      whether a given program halts on a given input, is not decidable.
\item Hoare calculus

      The \emph{Hoare calculus} is a system that enables us to formally prove the correctness
      of algorithms.
\end{enumerate}

\section{Mathematical {Formul\ae}}
Mathematical {formul\ae} are used to represent facts unambiguously.  We will define
mathematical {formul\ae} as abbreviations.  Let us first motivate their use.

\subsection{Why Use Mathematical {Formul\ae}?}
Take a look at the following statement:
\begin{center}
\begin{minipage}{14cm}
{\em 
  If wee add two numbers and then square this sum,  we will get the same result that we get
  when we square the numbers separately, add these squares and finally add the product of
  both numbers twice.
}
\end{minipage}
\end{center}
This text given above is nothing else than the well known formula 
\\[0.2cm]
\hspace*{1.3cm} $(a+b)^2 = a^2 + b^2 + 2\cdot a\cdot b$. 
\\[0.2cm]
Obviously, this formula is much easier to understand than the text given above.
But besides being easier to read, mathematical {formul\ae} offer two more benefits.
\begin{enumerate}
\item Mathematical {formul\ae} can be processed and analyzed by a computer program.
      At the present time, it is not possible for a computer to ``\emph{understand}''
      natural language.
\item The meaning of a mathematical formula can be defined unambiguously.
      In contrast, natural language is often ambiguous:  Different people may interpret
      the same sentence differently.
\end{enumerate}

\subsection{Mathematical {Formul\ae} As Abbreviations}
Let us first introduce the ingredients that are used to build mathematical {formul\ae}.
\begin{enumerate}
\item \emph{variables}

      Variables are used as names for arbitrary objects.  In the example given above we
      used the variables $a$ and $b$ to denote arbitrary numbers.
\item \emph{constants}

      A constant is used to denote a fixed object.  In mathematics, numbers like $1$ or
      $\pi$ are used as constants.  If we were to devise a theory describing biblical
      genealogy, we would use constants like \texttt{Adam} and \texttt{Eve}.

      Both variables and constants are known as \emph{atomic terms}.  The attribute
      \emph{atomic} relates to the fact that neither variables nor constants can be
      subdivided into further components.  

\item \emph{function symbols}

      Function symbols are used to construct complex expressions.  In the example above, we
      have used ``$+$'', ``$*$'', and ``$^2$'' as function symbols.  Expressions that are
      made up from function symbols, variables and constants are called ``\emph{terms}''.

      Every function symbol has an \emph{arity}.  The arity specifies the number of
      arguments that the function symbol needs.  To give an example, the function symbol
      ``+'' expects two arguments, so its arity is two.  The square root function
      $\sqrt{\;\;}$ is an example of a function symbol with arity 1.
      
      Instead of saying that the function symbol $f$ has arity $n$ we will call $f$ an
      $n$-ary function symbol.  If a function symbol is 2-ary, it is also called a
      \emph{binary} function symbol, and a 1-ary function symbol is also known as a
      \emph{unary} function symbol.  
\item \emph{terms}

      Terms denote complex objects.  The notion of a term is defined inductively:
      \begin{enumerate}
      \item Every variable is a term.
      \item Every constant is a term.
      \item If $f$ is an $n$-ary function symbol and we already know that $t_1$, $t_2$,
            $\cdots$, $t_n$ are terms, then $f(t_1, \cdots, t_n)$ is also a term.
      \end{enumerate}
      The notation $f(t_1, \cdots, t_n)$ is called the \emph{prefix notation}.  In
      mathematical practice we often use an infix notation for binary function symbols.
      If $o$ is a binary function symbol and $t_1$ and $t_2$ are terms, then we often write
      \\[0.2cm]
      \hspace*{1.3cm}
      $t_1\;o\;t_2$ \quad instead of \quad $o(t_1, t_2)$.
      \\[0.2cm]
      A well known  example is the binary function symbol ``+'':  Of course we write
      $x + y$ instead of $+(x,y)$.   However, when several different binary function
      symbols are used, we have to define a \emph{precedence} of the different function
      symbols,  or otherwise the interpretation of a term would be ambiguous.
      For example, in school mathematics the term
      \\[0.2cm]
      \hspace*{1.3cm}
      $x + y * z$ \quad is interpreted as \quad $+(x, *(y, z))$
      \\[0.2cm]
      as the binary function symbol ``\texttt{*}'' has a higher precedence than the binary
      function symbol ``$+$''.
\item \emph{predicate symbols}

      Predicate symbols are used to make a statement about objects.  Like a function symbol,
      a predicate symbol has an arity which is a natural number.  If $p$ is an $n$-ary
      predicate symbol and if $t_1$, $t_2$, $\cdots$, $t_n$ are terms, then
      \\[0.2cm]
      \hspace*{1.3cm}
      $p(t_1, t_2, \cdots, t_n)$
      \\[0.2cm]
      is an \emph{atomic formula}.  It states that the terms $t_1$, $\cdots$, $t_n$ are in
      the relation denoted by the predicate symbol $p$.  To give an example, consider the
      two terms $x$ and $+(x,1)$ and the predicate symbol $<$.
      Then
      \\[0.2cm]
      \hspace*{1.3cm}
      $<(x, +(x,1))$
      \\[0.2cm]
      is an \emph{atomic formula}. Of course, this is easier to read if presented in infix notation as
      \\[0.2cm]
      \hspace*{1.3cm}
      $x < x + 1$.
      \\[0.2cm]
      While terms denote objects, atomic {formul\ae} have a \emph{truth value}:  They are
      either true or false.
\item \emph{sentential connectives}

      \emph{Sentential connectives} are also know as \emph{logical operators}.
      They are used to build more complex {formul\ae} from atomic {formul\ae}.
      The simplest sentential connective is ``\emph{and}''.  We will write the logical
      operator denoting ``\emph{and}'' as ``$\wedge$''.  Given two {formul\ae} $F_1$ and
      $F_2$, we can connect these  {formul\ae} with the operator ``$\wedge$'' to build the
      new formula
      \\[0.2cm]
      \hspace*{1.3cm}
      $F_1 \wedge F_2$.
      \\[0.2cm]
      We read this formula as ``\emph{$F_1$ and $F_2$}''.  It states that both $F_1$ and
      $F_2$ hold true.
      
      The following tables lists all the logical operators that we are going to use:
      \\[0.2cm]
      \hspace*{1.3cm}
      \hspace*{1.3cm} 
      \begin{tabular}{|l|l|}
      \hline
      operator & read as \\
      \hline
      \hline
        $\neg F$ & not $F$ \\
      \hline
        $F_1 \wedge F_2$ & $F_1$ and $F_2$ \\
      \hline
        $F_1 \vee F_2$ & $F_1$ or $F_2$ \\
      \hline
        $F_1 \rightarrow F_2$ & if $F_1$ holds, then $F_2$ holds \\
      \hline
        $F_1 \leftrightarrow F_2$ &  $F_1$ holds if and only if $F_2$ holds \\
      \hline
      \end{tabular}
      \\[0.2cm]
      In a complex formula we will have to use parenthesis to eliminate ambiguities.
      However, as excessive use of parenthesis renders a complex formula unreadable,
      we agree to assign precedences to these logical operators.  
      The operator  ``$\neg$'' has the highest precedence.  Both  ``$\wedge$''
      and ``$\vee$'' have the same precedence and this precedence is higher than the
      precedence of ``$\rightarrow$''.  Finally, the precedence of  ``$\leftrightarrow$'' 
      is lower than any other precedence.  Therefore, the formula
       \\[0.2cm]
      \hspace*{1.3cm} 
      $P \wedge Q \rightarrow R \leftrightarrow \neg R \rightarrow \neg P \vee \neg Q$ \\[0.2cm]
      is interpreted as  \\[0.2cm]
      \hspace*{1.3cm}  
      $\bigl((P \wedge Q) \rightarrow R\bigr) \leftrightarrow \bigl(\,(\neg R) \rightarrow ((\neg P) \vee (\neg Q))\,\bigr)$. 

      In formal logic, the logical operators given above are known under the following
      name:
      \begin{enumerate}
      \item $\neg$: negation
      \item $\wedge$: conjunction
      \item $\vee$: disjunction
      \item $\rightarrow$: conditional
      \item $\leftrightarrow$: biconditional
      \end{enumerate}
      In natural language, the usage of the operator ``\emph{or}'' is ambiguous, because
      it can be used \emph{exclusively} as in ``\emph{either $A$ or $B$ but not both}'' or
      \emph{inclusively} as in ``\emph{$A$ or $B$ or both}''.  In mathematics, the
      agreement is to use the operator ``\emph{or}'' inclusively, so the formula
      $A \vee B$ is also true if both $A$ and $B$ are true.
      We will defer a more formal treatment of the interpretation of the logical operators to a
      later chapter.
\item \emph{quantifiers}

      A quantifier specifies how a variable is used in a formula.  We will use two
      quantifiers:
      \begin{enumerate}
      \item $\forall$ (read: \emph{for all}) is known as the \emph{universal quantifier}.  If $x$ is a variable
            and $F$ is a formula presumably containing $x$ then
            \\[0.2cm]
            \hspace*{1.3cm}
            $\forall x: F$
            \\[0.2cm]
            is a formula that is read as 
            \\[0.2cm]
            \hspace*{1.3cm}
            ``\emph{for all $x$, $F$ is true}''
            \\[0.2cm]
            To give an example, if $F$ is the formula $x \leq x \cdot x$, then
            \\[0.2cm]
            \hspace*{1.3cm}
            $\forall x: x \leq x \cdot x$
            \\[0.2cm]
            is a formula stating that for all $x$ the square of $x$ is at least as big as
            $x$.  Whether the formula is actually true depends on the 
            \emph{universe of interpretation}:  If this universe is made up of all the natural numbers,
            the formula is true.  However, if the universe also contains the rational
            numbers, the formula is false as 
            $\frac{1}{2} \cdot \frac{1}{2} = \frac{1}{4} < \frac{1}{2}$.
      \item $\exists$ (read: \emph{exists}) is known as the \emph{existential quantifier}.  If $x$ is a variable
            and $F$ is a formula presumably containing $x$ then
            \\[0.2cm]
            \hspace*{1.3cm}
            $\exists x: F$
            \\[0.2cm]
            is a formula that is read as 
            \\[0.2cm]
            \hspace*{1.3cm}
            ``\emph{there exists a value for $x$ such that $F$ is true}''
            \\[0.2cm]
            To give an example, if $F$ is the formula $x \cdot x = -1$, then
            \\[0.2cm]
            \hspace*{1.3cm}
            $\exists x: x \cdot x = -1$
            \\[0.2cm]
            is a formula stating that there is a number $x$ such that the square of $x$ 
            is $-1$.  Again, the truth of this depends on the 
            universe of interpretation:  Using the real numbers, this formula is obviously
            false.  However, in the theory of complex numbers the formula is true. 
      \end{enumerate}
      Sometimes you will find so called ``\emph{qualified quantifiers}''.  A universal
      qualified quantifier has the form
      \\[0.2cm]
      \hspace*{1.3cm} $\forall x \in M: F$. 
      \\[0.2cm]
      This is read as 
      \\[0.2cm]
      \hspace*{1.3cm}
      ``\emph{for all $x$ from $M$ we have $F$}''
      \\[0.2cm]
      Here $M$ denotes some set.  In fact, the qualified quantifier can be understood as an
      abbreviation: 
      \\[0.2cm]
      \hspace*{1.3cm} 
      $\forall x \in M: F$ \quad is the same as \quad
      $\forall x\colon (x\in M \rightarrow F)$.
      \\[0.2cm]
      In a similar way we have \\[0.2cm]
      \hspace*{1.3cm}  
      $\exists x \in M: F$ \quad is an abbreviation for \quad
      $\exists x\colon (x\in M \wedge F)$. 
      \\[0.2cm]
      It is time for an example.
      The formula \\[0.2cm]
      \hspace*{1.3cm} 
      $\forall x \in \mathbb{R}: \exists n \in \mathbb{N} : n > x$ 
      \\[0.2cm]
      is read as follows:
      \begin{center}
        {\em
        \begin{minipage}{12cm}
          For all  $x$ from $\mathbb{R}$ there is an $n$ from $\mathbb{N}$
          such that $n$ is bigger than $x$.
        \end{minipage}
        }
      \end{center}
      If we have a formula containing both quantifiers and logical operators we have to
      assign a precedence to the quantifiers.  
      Our agreement will be that quantifiers have a higher precedence than all logical
      operators.
      Therefore \\[0.2cm]
      \hspace*{1.3cm} $\forall x \colon p(x) \wedge q(y)$ \\[0.2cm]
      is the same as: \\[0.2cm]
      \hspace*{1.3cm} $\bigl(\forall x \colon p(x)\bigr) \wedge q(y)$. 
\end{enumerate}

\subsection{Examples}
To clarify the notions of \emph{terms} and \emph{formul\ae} we proceed to present some
examples.  Rather than choosing examples from mathematics we choose {formul\ae} that
describe kindred-ship.  We start by specifying the constants, variables, function and
predicate symbols that we are going to use.
\begin{enumerate}
\item The following words are used as  \emph{constants}: \\[0.2cm]
      \hspace*{1.3cm} ``\texttt{adam}'', ``\texttt{eve}'', ``\texttt{cain}'' and ``\texttt{abel}''.
\item The \emph{variables} are the letters \\[0.2cm]
      \hspace*{1.3cm} ``$x$'', ``$y$'' and ``$z$''.
\item The following words are  \emph{function symbols}: \\[0.2cm]
      \hspace*{1.3cm} 
      ``\texttt{father}'' and ``\texttt{mother}''. 
      \\[0.2cm]
      Both of these function symbols are unary.
\item The following words are \emph{predicate symbols}: \\[0.2cm]
      \hspace*{1.3cm} 
      ``\texttt{brother}'', ``\texttt{sister}'', ``\texttt{uncle}'',
      ``\texttt{male}'', and ``\texttt{female}''. \\[0.2cm]
      In addition, we use the symbol ``$=$'' as a predicate symbol.
      All of these predicate symbols are binary.
\end{enumerate}
A collection of constants, variables, function symbols, and predicate symbols together
with their arity is known as a \emph{signature}.  First, we give some terms that can be
build with the given signature.
\begin{enumerate}
\item ``\texttt{cain}'' is a term, as ``\texttt{cain}'' is a constant.
\item ``$\mathtt{father}(\mathtt{cain})$'' is also a term, since  ``\texttt{cain}''
      is a  term and ``\texttt{father}'' is a unary function symbol.
\item ``$\mathtt{mother}\bigl(\mathtt{father}(\mathtt{cain})\bigr)$'' is a term 
      as  ``$\mathtt{father}(\mathtt{cain})$'' is a term and  ``\texttt{mother}'' is
      a unary function symbol,
\item ``$\texttt{male}(\mathtt{cain})$'' is a formula, as
      ``\texttt{cain}'' is a term and
      ``\texttt{male}'' is a unary predicate symbol.
\item ``$\texttt{male}(\mathtt{lisa})$'' is a formula, as
      ``\texttt{lisa}'' is a term. 
\item ``$\forall x: \forall y: \bigl(\mathtt{father}(x) = \mathtt{father}(y) \wedge 
          \mathtt{mother}(x) = \mathtt{mother}(y)
         \rightarrow       \mathtt{brother}(x,y) \vee \mathtt{sister}(x,y)\bigr)$''
      
      is a formula.
\item ``$\forall x\colon \forall y\colon\bigl( \mathtt{brother}(x,y) \vee  \mathtt{sister}(x,y)\bigr)$'' 

      is a formula.  Of course, this formula is wrong if we choose the obvious universe of
      interpretation.
\end{enumerate}
For now, we treat {formul\ae} as abbreviations.  In order to be able to define the \emph{truth}
of a formula  we need to  make the notion of the \emph{universe of interpretation} 
precise.   In order to do this, we have to introduce set theory first.  This is done
 in the remaining part of this chapter.

\section{Sets and Relations}
Set theory has  been developed at the end of the 19th century in order to put mathematics
on a solid foundation.  This was deemed necessary as the notion of infinity gave rise to a
number of paradoxes that troubled the mathematicians of that time,

Set theory was devised by Georg Cantor (1845 -- 1918).  The first definition of a set that
was given by Cantor reads as follows \cite{cantor:1895}:
\begin{center}
A \emph{set} is a well-defined collection of \emph{elements}.  
\end{center}
The attribute  ``\emph{well-defined}'' expresses that, given a set $M$ and an object $x$
we must be able to decide whether $x$ is an element of the set $M$.
In this case we will write \\[0.2cm]
\hspace*{1.3cm} $x \in M$ \\[0.2cm]
and read this formula as  ``\emph{$x$ is an element of $M$}''.
As you can see, we use the symbol ``$\in$'' as a binary predicate symbol that is used with
infix notation.

In order to make the notion of a  \emph{well-defined collection} precise,
Cantor used the so called  \emph{comprehension axiom}.
This works as follows: If  $p(x)$ is a property that an object $x$ either has or does not
have, then we can build the set of all those objects, that have this property.   This set
is written as 
\\[0.2cm]
\hspace*{1.3cm} $M = \bigl\{ x \;|\; p(x) \bigr\}$. 
\\[0.2cm]
We read this formula as
\\[0.2cm]
\hspace*{1.3cm}
 ``$M$ is the set of all $x$ for which $p(x)$ is true''.
\\[0.2cm]
Here, the property $p(x)$ is a formula that contains the variable $x$.
Let us give an example:
Let $\mathbb{N}$ denote the set of all natural numbers.  Starting from $\mathbb{N}$ 
we can then define the set of all \emph{even} natural numbers.  In order to do this
we have to define the property of \emph{evenness} via a mathematical formula.
Now a natural number $x$ is even if and only if there is a natural number $y$, such that $x$
is two times $y$.
Therefore, the property $p(x)$ can be defined as follows: \\[0.2cm]
\hspace*{1.3cm} $p(x) \;:=\; (\exists y\in \mathbb{N}: x = 2 \cdot y)$. \\[0.2cm]
Now we can define the set of all even natural numbers as \\[0.2cm]
\hspace*{1.3cm} $\{ x \;|\; \exists y\in \mathbb{N}: x = 2 \cdot y \}$. 
\vspace{0.2cm}

Unfortunately, the unrestricted use of the comprehension axion readily leads into difficulties.
To see the problem, let us define the following set using the comprehension axiom:
 \\[0.2cm]
\hspace*{1.3cm} 
$R := \{ x \;|\; \neg\; x \in x \}$.  
\\[0.2cm]
This set contains all those sets that do not contain themselves.
Intuitively, you might think that no set can possibly contain itself.
But let us check formally whether $R$ is a member of itself or not.
Basically, there can only be two cases:
\begin{enumerate}
\item $\neg\; R \in R$. 

      So $R$ does not contain itself.
      Now $R$ is defined as the set of all those sets that do not contain itself.
      Therefore, $R$ would be a member of itself.  This contradicts our starting assumption
      $\neg\; R \in R$.
      
      Thus we are forced to investigate the next case.
\item case: $R \in R$. 

      Substituting the definition of $R$ for the second $R$ in the formula we conclude \\[0.2cm]
      \hspace*{1.3cm}  
      $R \in \{ x \;|\; \neg\; x \in x \}$. 
      \\[0.2cm]
      But this would entail that the formula $\neg\; x \in x$, which is the defining
      formula for $R$, holds true for $R$ itself.  Thus we conclude
      \\[0.2cm]
      \hspace*{1.3cm}
      $\neg\; R \in R$.
      \\[0.2cm]
      Unfortunately, this contradicts the starting assumption of the second case.
\end{enumerate}
As we get a contradiction in either case, something must be wrong.  The only way out is to
realize that the object 
\\[0.2cm]
\hspace*{1.3cm} $\{ x \mid \neg\; x \in x \}$ \\[0.2cm]
is not a set, the comprehension axiom is way too general and has to be restricted.
It was the British logician Bertrand Russell (1872 -- 1970) who discovered the paradox
demonstrated above.  
In order to avoid this and other paradoxes it is necessary to be more careful when
constructing sets.  Therefore we will now provide methods to construct sets that are
weaker than the comprehension axiom, but that have not yet led to inconsistencies.


\subsection{Enumerations}
The simplest method to construct a set is to list all its elements.  These elements
will be bordered by the curly braces ``\texttt{\{}'' and ``\texttt{\}}''.  They are
separated by commas ``\texttt{,}''.
An example is \\[0.2cm]
\hspace*{1.3cm} $M := \{ 1, 2, 3 \}$, \\[0.2cm]
which contains the elements  $1$, $2$ and $3$.
As a second example, the set of all lower case letters is given as: 
\\[0.2cm]
\hspace*{1.3cm} 
$\{\mathtt{a}, \mathtt{b}, \mathtt{c}, \mathtt{d}, \mathtt{e},
 \mathtt{f}, \mathtt{g}, \mathtt{h}, \mathtt{i}, \mathtt{j}, \mathtt{k}, \mathtt{l},
 \mathtt{m}, \mathtt{n}, \mathtt{o}, \mathtt{p}, \mathtt{q}, \mathtt{r}, \mathtt{s},
 \mathtt{t}, \mathtt{u}, \mathtt{v}, \mathtt{w}, \mathtt{x}, \mathtt{y}, \mathtt{z\}}$.

\subsection{The Set of all Natural Numbers}
If we list the elements of a set explicitly, we can never get an infinite set.
However, we know that there are constructs of mathematical imagination that consist
of an unbounded number of elements.  The simplest such construct is the set of natural
numbers.  We postulate that the collection $\mathbb{N}$ of all natural numbers is a well
defined set: \\[0.2cm]
\hspace*{1.3cm} $\mathbb{N} := \{ 0, 1, 2, 3, \cdots \}$. \\[0.2cm]
Besides the set $\mathbb{N}$ of all natural numbers we will use the following sets of
numbers:
\begin{enumerate}
\item $\mathbb{Z}$ is the set of all integers.
\item $\mathbb{Q}$ is the set of all rational numbers.
\item $\mathbb{R}$ is the set of all real numbers.
\end{enumerate}
In mathematics, it is shown that these sets can all be derived from the set of natural numbers.

\subsection{The Axiom of Separation}
The axiom of \emph{separation}  is a weak form of the axiom of comprehension.
The idea is to select those elements from an already given set that possess some property $p(x)$.
We write this as: \\[0.2cm]
\hspace*{1.3cm} 
$N = \{ x\in M \;|\; p(x) \}$ 
\\[0.2cm]
Then $N$ is the set of all those elements of $M$ such that $p(x)$ is true.
\vspace{0.2cm}

\noindent
\textbf{Example}:
The set of even numbers can be defined as: \\[0.2cm]
\hspace*{1.3cm} $\{ x \in \mathbb{N} \;|\; \exists y\in \mathbb{N}: x = 2 \cdot y \}$. 

\subsection{The Axiom of Power  Sets}
In order to introduce the axiom of power sets we have to define subsets first.
Given two sets $M$ and $N$,  $M$ is a 
\emph{subset} of $N$ if and only if every element from $M$ is also an element of $N$.
This is written as $M \subseteq N$.  Formally this can be defined as: \\[0.2cm]
\hspace*{1.3cm} $M \subseteq N \;\stackrel{de\!f}{\Longleftrightarrow}\; \forall x: (x \in M \rightarrow x \in N)$ \\[0.2cm]
Now the  \emph{power set} of a given set $M$ is the set of all subsets of $M$.
It is written as $2^M$.  Formally we have \\[0.2cm]
\hspace*{1.3cm} $2^M = \{ x \;|\; x \subseteq M \}$.
\vspace{0.2cm}

\noindent
\textbf{Example}: 
We compute the power set of the set $\{1,2,3\}$ and find: \\[0.2cm]
\hspace*{1.3cm} $2^{\{1,2,3\}} = \big\{ \{\},\, \{1\}, \, \{2\},\, \{3\},\, \{1,2\}, \, \{1,3\}, \, \{2,3\},\, \{1,2,3\}\big\}$. \\[0.2cm]
This set has  $8 = 2^3$ elements.  It can be shown that the power set of a set containing
$n$ elements will itself contain $2^n$ elements.  If we denote the number of elements of a
finite set with 
$\textsl{card}(M)$, we therefore have
\\[0.2cm]
\hspace*{1.3cm}
$\textsl{card}\left(2^M\right) = 2^{\textsl{card}(M)}$.
\\[0.2cm]
This is the motivation for the denoting the power set as  $2^M$.

\subsection{Union}
Given two sets  $M$ and $N$, the union of  $M$ and $N$ contains all elements that are
either in $M$ or in $N$.  We will write  $M \cup N$ to denote the union of $M$ and $N$.
Then we have: \\[0.2cm]
\hspace*{1.3cm} $M \cup N := \{ x \;|\; x \in M \vee x \in N \}$. 
\vspace{0.2cm}

\noindent
\textbf{Example}: 
If  $M = \{1,2,3\}$ and $N = \{2,5\}$, then we have: \\[0.2cm]
\hspace*{1.3cm} $\{1,2,3\} \cup \{2,5\} = \{1,2,3,5\}$. 
\vspace{0.2cm}

The notion of the union of two sets can be extended to the notion of a union of a set of
sets.  So imagine a set $X$ all of whose elements are itself sets.
We can then construct the union of all those sets contained in $X$.
This union is written as $\bigcup X$.  Formally, we have: \\[0.2cm]
\hspace*{1.3cm} $\bigcup X = \{ y \;|\; \exists x \in X: y \in x \}$.
\vspace{0.2cm}

\noindent
\textbf{Example}: 
Assume that the set $X$ is defined as follows: \\[0.2cm]
\hspace*{1.3cm} $X = \big\{ \{\},\, \{1,2\}, \, \{1,3,5\}, \, \{7,4\}\,\big\}$. \\[0.2cm]
Then we have \\[0.2cm]
\hspace*{1.3cm} $\bigcup X = \{ 1, 2, 3, 4, 5, 7 \}$.

\subsection{Intersection}
The \emph{intersection} of two sets  $M$ and $N$ is defined as the set that contains all those
objects that are elements of both $M$ and $N$.  The intersection of $M$ and $N$ is denoted
as $M \cap N$.  More formally,  $M \cap N$ is defined as follows: \\[0.2cm]
\hspace*{1.3cm} $M \cap N := \{ x \mid x \in M \wedge x \in N \}$. 
\vspace{0.2cm}

\noindent
\textbf{Example}: 
We compute the intersection of the sets $M = \{ 1, 3, 5 \}$ and $N = \{ 2, 3, 5, 6 \}$.
We find
\\[0.2cm]
\hspace*{1.3cm} $M \cap N = \{ 3, 5 \}$

\subsection{Set Difference}
Given two sets $M$ and $N$, the  \emph{set difference} of 
 $M$ without $N$ is the set of all those elements of  $M$ that are not elements of $N$.  
The set difference is written as $M \backslash N$.  This is read as $M$ \emph{without} $N$.
The formal definition is: \\[0.2cm]
\hspace*{1.3cm} $M \backslash N := \{ x \in M \wedge x \not\in N \}$. 
\vspace{0.2cm}

\noindent
\textbf{Example}: 
We compute the set difference of the sets $M = \{ 1, 3, 5, 7 \}$ and $N = \{ 2, 3, 5, 6
\}$.  We find
\\[0.2cm]
\hspace*{1.3cm} $M \backslash N = \{ 1, 7 \}$.

\subsection{Images}
If  $M$ is a set and $f$ is a function that is defined for all $x$ from $M$, then for all
$x \in M$ we call $f(x)$ the \emph{image} of $x$ under $f$.  The set of all images
from elements from $M$ is called the \emph{image} of $M$ under $f$ and is denoted as $f(M)$.
Formally, the image $f(M)$ is defined as follows:
 \[ f(M) := \{ y \;|\; \exists x \in M: y = f(x) \}. \]
Equivalently, we can write 
\[ f(M) = \bigl\{ f(x) \;|\; x \in M \}. \]
\vspace{0.2cm}

\noindent
\textbf{Example}: 
The set of all integer squares is defined as 
\[ Q := \bigl\{ y \mid \exists x \in \mathbb{N}: y = x^2 \bigr\} = 
        \bigl\{ x^2 \mid x \in \mathbb{N} \bigr\}. 
\]


\subsection{Cartesian Products}
In order to introduce the notion of the Cartesian product of two sets we first need the
notion of an \emph{ordered pair} of two objects $x$ and $y$.  The ordered pair of $x$
and $y$ is written as  
\\[0.2cm]
\hspace*{1.3cm} $\langle x, y \rangle$.
\\[0.2cm]
The object $x$ is the \emph{first component} of the pair $\langle x, y \rangle$, 
while $y$ is the  \emph{second component}.  Two ordered pairs $\langle x_1, y_1 \rangle$ and $\langle x_2, y_2 \rangle$
are equal if and only if the respective components are equal.  Therefore we have 
\\[0.2cm]
\hspace*{1.3cm} 
$\langle x_1, y_1 \rangle \,=\,\langle x_2, y_2 \rangle$ 
\quad iff\footnote{
In computer science and mathematics, 
it is customary to use ``\emph{iff}'' as an abbreviation for ``\emph{if and only if}''.
} \quad
$x_1 = x_2 \wedge y_1 = y_2$.
 \\[0.2cm]
The \emph{Cartesian product} of two sets $M$ and $N$ is defined as the set of all ordered
pairs $p$ such that the first component of $p$ is an element of $M$ and the second
component is an element of $N$.  The Cartesian product is written as $M \times N$.
Formally, we have 
\[ M \times N := \big\{ z \mid \exists x\colon \exists y\colon z = \langle x,y\rangle \wedge x\in M \wedge y \in N \}. \]
This can also be written as 
\[ M \times N = \big\{ \langle x,y\rangle \mid  x\in M \wedge y \in N \}. \]
The Cartesian product is also known as the \emph{product set}.
\vspace{0.2cm}

\noindent
\textbf{Example}:  Let $M = \{ 1, 2, 3 \}$ and $N = \{ 5, 7 \}$. Then \\[0.2cm]
\hspace*{1.3cm} 
$M \times N = \bigl\{ \pair(1,5),\pair(2,5),\pair(3,5),\pair(1,7),\pair(2,7),\pair(3,7)\bigr\}$.
\vspace{0.2cm}

\noindent
The notion of an ordered pair is easily extended to the notion of a finite sequence:
A finite sequence of length $n$ has the form \\[0.2cm]
\hspace*{1.3cm} $\langle x_1, x_2, \cdots, x_n \rangle$. \\[0.2cm]
Finite sequences are also known as \emph{lists}.  Finite sequences enable us to extend the
notion of product sets to more than two sets.  We define the product set of the sets
$M_1$, $\cdots$, $M_n$ as 
\\[0.2cm]
\hspace*{1.3cm}
$M_1 \times \cdots \times M_n =\big\{ \langle x_1,x_2,\cdots,x_n \rangle  \mid
x_1\in M_1 \wedge \cdots \wedge x_n \in M_n \big\}$. 
\\[0.2cm]
If  $f$ is a function defined on $M_1 \times \cdots \times M_n$, then we agree to simplify
our notation as follows: We write 
\\[0.2cm]
\hspace*{1.3cm} 
$f(x_1, \cdots, x_n)$ \quad instead of \quad $f(\langle x_1, \cdots,
x_n\rangle)$. 

\subsection{Equality of Sets}
Two sets are considered equal if and only if they have the same elements.
Therefore we have: \\[0.2cm]
\hspace*{1.3cm} $M = N \;\leftrightarrow\; \forall x: (x \in M \leftrightarrow x \in N)$ 
\\[0.2cm]
Therefore the order in which the elements of a set are listed, is unimportant.  
For example, we have
 \\[-0.2cm]
\hspace*{1.3cm} 
$\{1,2,3\} = \{3,2,1\}$ \\[0.2cm]
since both sets contain the same elements.

If sets are defined by explicitly listing their elements, then the question, whether two
sets are equal, is straightforward to answer.  However, in general it may be arbitrarily
difficult to decide whether two sets are equal.  For example, it can be shown that
\\[0.2cm]
\hspace*{1.3cm} 
$\{ n \in \mathbb{N} \mid \exists x, y, z\in\mathbb{N}: x > 0 \wedge y > 0 \wedge x^n + y^n = z^n \} 
= \{1,2\}$. 
\\[0.2cm]
However the proof of this claim is equivalent to  \emph{Fermat's Last Theorem}, which was
postulated in 1637 by {\sl Pierre de Fermat}.  This proof could only be given in  1995 by
Andrew Wiles.  In general, it can be shown that the question, whether two sets are equal,
is algorithmically undecidable.


\subsection{Set Algebra}
The set operations ``$\cup$'', ``$\cap$'', and ``$\backslash$'' satisfy a number of equations. 
 These are given below.  As is customary,  the empty set  $\{\}$ is denoted as $\emptyset$. 
\\[0.2cm]
$\begin{array}{rlcl}
\quad 1. & M \cup \emptyset = M         & \hspace*{0.1cm} & M \cap \emptyset = \emptyset \\[0.1cm]
2. & M \cup M = M         & \hspace*{0.1cm} & M \cap M = M          \\[0.1cm]
3. & M \cup N = N \cup M  &  & M \cap N = N \cap M  \\[0.1cm]
4. & (K \cup M) \cup N = K \cup (M \cup N) &  & (K \cap M) \cap N = K \cap (M \cap N) \\[0.1cm]
5. & (K \cup M) \cap N = (K \cap N) \cup (M \cap N) &  & (K \cap M) \cup N = (K \cup N) \cap (M \cup N)  \\[0.1cm]
6. & M \backslash \emptyset = M & & M \backslash M = \emptyset \\[0.1cm]
7. & K \backslash (M \cup N) = (K \backslash M) \cap (K \backslash N) &&
     K \backslash (M \cap N) = (K \backslash M) \cup (K \backslash N) \\[0.1cm]
8. & (K \cup M) \backslash N = (K \backslash N) \cup (M \backslash N) &&
     (K \cap M) \backslash N = (K \backslash N) \cap (M \backslash N) \\[0.1cm]
9. & K \backslash (M \backslash N) = (K \backslash M) \cup (K \cap N) &&
     (K \backslash M) \backslash N = K \backslash (M \cup N) \\[0.1cm]
10. & M \cup (N \backslash M) = M \cup N &&
      M \cap (N \backslash M) = \emptyset  \\[0.1cm]
11. & M \cup (M \cap N) = M  &&
      M \cap (M \cup N) = M 

\end{array}$
\\[0.3cm]
In order to show how these equations can be proved,  we demonstrate the validity of the
equation 
\\[0.2cm]
\hspace*{1.3cm}
 $K \backslash (M \cup N) = (K \backslash M) \cap (K \backslash N)$.
\\[0.2cm]
In order to prove the equality of two sets $M$ and $N$ we have to show that $M$ and $N$
contain the same elements.
We have the following chain of equivalences: \\[0.3cm]
\hspace*{1.3cm} $
\begin{array}{ll}
                & x \in K \backslash (M \cup N)        \\[0.1cm]
\leftrightarrow & x \in K \;\wedge\; \neg\; x \in M \cup N \\[0.1cm]
\leftrightarrow & x \in K \;\wedge\; \neg\; (x \in M \vee x \in N) \\[0.1cm]
\leftrightarrow & x \in K \;\wedge\;  (\neg\; x \in M) \wedge (\neg\; x \in N) \\[0.1cm]
\leftrightarrow & (x \in K \wedge \neg\;x \in M) \;\wedge\; (x \in K \wedge \neg\;x \in N) \\[0.1cm]
\leftrightarrow & (x \in K \backslash M) \;\wedge\; (x \in K \backslash N) \\[0.1cm]
\leftrightarrow & x \in (K \backslash M) \cap (K \backslash N). \\[0.1cm]
\end{array}$ \\[0.3cm]
The remaining equations can be verified in the same way.


\noindent
In order to simplify our proofs we introduce the following notation:
If $M$ is a set and $x$ is an object, we write $x \notin M$  instead of
 $\neg\; x \in M$: \\[0.2cm]
\hspace*{1.3cm} $x \notin M \;\stackrel{de\!f}{\Longleftrightarrow}\; \neg\; x \in M$.
\\[0.2cm]
We use an analogous notation for equality:
\\[0.2cm]
\hspace*{1.3cm} 
$x \not= y \;\stackrel{de\!f}{\Longleftrightarrow}\; \neg\; (x = y)$.
\vspace{0.2cm}

\noindent
\textbf{Exercise}:  Prove the equation $(K \cup M) \cap N = (K \cap N) \cup (M \cap N)$.
\pagebreak

\section{Relations}
Relations abound in computer science.  A very important application of 
relations is found in the theory of relational databases.  We start our investigation of
relations with the special case of \emph{binary relations}.  We will investigate the
connection
between binary relations and functions.  We will see, that functions are 
a special case of binary relations.  Turned around, this means that
the notion of a binary relation is a generalization of the notion of a function.

Furthermore, we will discuss order relations and equivalence relations.

\subsection{Binary Relations and Functions}
If a set $R$ is a subset of the Cartesian product $M \times N$ of two sets $M$ and $N$,
that is  if we have \\[0.2cm]
\hspace*{1.3cm} $R \subseteq M \times N$, \\[0.2cm]
then $R$ is called a \emph{binary relation}.  In this case, the \emph{domain} of  $R$ is
defined as 
\\[0.2cm]
\hspace*{1.3cm} 
$\dom(R) := \{ x \mid \exists y \in N \colon \langle x, y \rangle \in R \}$,
\\[0.2cm]
while the \emph{range} of $R$ is defined as 
\\[0.2cm]
\hspace*{1.3cm} 
$\rng(R) := \{ y \mid \exists x \in M \colon \langle x, y \rangle \in R\}$.


\example
Define $R = \{ \pair(1,1), \pair(2,4), \pair(3,9) \}$.  Then we have \\[0.2cm]
\hspace*{1.3cm} $\dom(R) = \{1,2,3\}$ \quad and \quad $\rng(R) = \{1,4,9\}$. \qed
\next


\example
The database of a car dealer stores facts concerning customers and the cars they bought.
Let us assume that the sets \textsl{Car} and \textsl{Customer} are defined as follows:
\\[0.2cm]
\hspace*{1.3cm}
$\textsl{Customer} = \{ \mathrm{Bauer}, \mathrm{Maier}, \mathrm{Schmidt} \}$
\quad and \quad
$\textsl{Car} = \{ \mathrm{Polo}, \mathrm{Fox}, \mathrm{Golf} \}$.
\\[0.2cm]
Then the binary relation 
\\[0.2cm]
\hspace*{1.3cm}
$\textsl{Sales} \subseteq \textsl{Customer} \times \textsl{Car}$
\\[0.2cm]
could be given as follows:
\\[0.2cm]
\hspace*{1.3cm}
$\{ \pair(\mathrm{Bauer}, \mathrm{Golf}), \pair(\mathrm{Bauer}, \mathrm{Fox}), \pair(\mathrm{Schmidt}, \mathrm{Polo})\}$.
\\[0.2cm]
This relation is interpreted as follows:
\begin{itemize}
\item Mr.~Bauer bought both a Golf and a Fox.
\item Mr.~Schmidt has bought a Polo.
\item Mr. Maier hasn't bought a car so far.
\end{itemize}
In the theory of databases, this information is expressed by a table as follows:
\begin{center}
  \begin{tabular}[c]{|l|l|}
\hline
\textsl{Customer} & \textsl{Car} \\
\hline
\hline
  Bauer   & Golf \\
\hline
  Bauer   & Fox  \\
\hline
  Schmidt & Polo \\
\hline
  \end{tabular}
\end{center}
The header row containing the column captions \textsl{Customer} and \textsl{Car}
is not part of the relation but is the  \emph{relation schema}.
The relation together with its schema is called a  \emph{table}.  


\paragraph{Left-Unique and Right-Unique Relations}
A relation $R \subseteq M \times N$ is called
\emph{right-unique} iff the following holds: \\[0.2cm]
\hspace*{1.3cm} 
$\forall x \el M \colon \forall y_1, y_2 \el N \colon \bigl(\langle x, y_1 \rangle \in R \wedge \langle x, y_2 \rangle \in R \rightarrow y_1 = y_2\bigr)$.
\\[0.2cm]
If the relation $R \subseteq M \times N$ is right-unique, then for every $x\in M$ there is
at most one  $y \in N$ such that $\langle x, y \rangle \in R$ holds.  
A right-unique relation is also known as a \emph{functional relation}.

Analogously a relation $R \subseteq M \times N$ \emph{left-unique} iff we have: \\[0.2cm]
\hspace*{1.3cm} 
$\forall y \el N \colon \forall x_1, x_2 \el M \colon \bigl(\langle x_1, y \rangle \in R \wedge \langle x_2, y \rangle \in R \rightarrow x_1 = x_2\bigr)$.
\\[0.2cm]
Therefore, if $R \subseteq M \times N$ is left-unique, then for every $y\in N$ there
is at most one $x \in M$ such that $\langle x, y \rangle \in R$ gilt.
A left-unique relation is also known as an \emph{injective relation}.

\example
Let $M = \{1,2,3\}$ and $N = \{4,5,6\}$.
\begin{enumerate}
\item Assume that the relation $R_1$ is defined as \\[0.2cm]
      \hspace*{1.3cm} $R_1 = \{ \pair(1,4), \pair(1,6) \}$. \\[0.2cm]
      This relation is  \underline{not} right-unique, since  $4 \not= 6$.
      However, this relation is left-unique as the right hand sides of all tuples
      in  $R_1$ are different.
\item Assume that the relation  $R_2$ is defined as \\[0.2cm]
      \hspace*{1.3cm} $R_2 = \{ \pair(1,4), \pair(2,6) \}$. \\[0.2cm]
      This relation is both left-unique and right-unique.
\item Assume the relation  $R_3$ is defined as \\[0.2cm]
      \hspace*{1.3cm} $R_3 = \{ \pair(1,4), \pair(2,6), \pair(3,6) \}$. \\[0.2cm]
      This relation is right-unique, but it is not left-unique, because we have
      $\pair(2,6) \el R$ and $\pair(3,6) \el R$, but $2 \not= 3$.
\end{enumerate}

\paragraph{Left-Total and Right-Total Relations}
A binary relation $R \subseteq M \times N$ is \emph{left-total on $M$} iff \\[0.2cm]
\hspace*{1.3cm} $\forall x \in M \colon \exists y \in N \colon \pair(x,y) \in R$. \\[0.2cm]
That means for every $x$ from $M$ there is a  $y$ from $N$ such that
$\pair(x,y)$ is a member of the relation $R$.  In the example above, the relation $R_3$ is
left-total.


A binary relation $R \subseteq M \times N$ is  \emph{right-total on $N$} iff \\[0.2cm]
\hspace*{1.3cm} $\forall y \in N \colon \exists x \in M \colon \pair(x,y) \in R$.
 \\[0.2cm]
That means for every  $y$ from $N$ there is an  $x$ from $M$ such that the pair
$\pair(x,y)$ is a member of the relation  $R$.  



\paragraph{Functions}
If a  relation $R \subseteq M \times N$ is both left-total on $M$ and right-unique, then
$R$ is a \emph{function} on $M$.  The reason for this definition is that in this case
we can define a function $f_R\colon M \rightarrow N$ as follows: \\[0.2cm]
\hspace*{1.3cm} $f_R(x) := y \;\stackrel{de\!f}{\Longleftrightarrow}\; \pair(x,y) \in R$. 
\\[0.2cm]
This definition works, because the fact that  $R$ is left-total ensures that for every 
 $x\in M$ there is an  $y \in N$ such that  $\pair(x,y) \in R$.  Therefore, $f_R$ is defined
 for all $x$ from the set $M$.   The fact that $R$ is right-unique ensures that we find at
 most one $y$ such that $\pair(x,y) \in R$, so $f_R(x)$ is indeed well-defined.

On the other hand, if a function  \mbox{$f:M \rightarrow N$} is given, we can define a relation
 $\textsl{graph}(f) \subseteq M \times N$ as follows.
\[ \textsl{graph}(f) := \bigl\{ \langle x,f(x) \rangle \mid  x\in M \bigr\}. \]
The  relation $\textsl{graph}(f)$ is left-total because the function $f$ computes a result $f(x)$
for every $x \el M$  and  the relation is right-unique because
for a given argument $x \in M$ there is at most one value $f(x)$.

Therefore functions are just a special case of relations.
The set of all relations on $M \times N$ that are functions on M is written as $N^M$, 
we have
\[ N^M := \{ R \subseteq M \times N \mid \mbox{$R$ is a function on $M$}\, \}. \]
This notation is explained as follows: If  $M$ and $N$ are finite sets with  $m$
and $n$ elements, then there are  $n^m$ different functions from  $M$ to $N$, we have
\[ \textsl{card}\left(N^M\right) = \textsl{card}(N)^{\textsl{card}(M)}. \]
In the following, we make no distinction between a relation that is both left-total on $M$
 and right-unique and the corresponding function.  If  $R \subseteq M \times N$ is
 left-total on $M$ and right-unique and $x \in M$, then we will write $R(x)$ to denote the
uniquely determined value  $y \in N$ that satisfies $\pair(x,y) \in R$.

\noindent
\textbf{Example}: 
\begin{enumerate}
\item Let $M = \{1,2,3\}$, $N = \{1,2,3,4,5,6,7,8,9\}$ and define\\[0.2cm]
      \hspace*{1.3cm} $R := \{ \pair(1,1),\pair(2,4),\pair(3,9) \}$. \\[0.2cm]
      Then $R$ is a function on  $M$.  This function computes the squares on the set $M$.
\item This time we have  $M = \{1,2,3,4,5,6,7,8,9\}$ and $N = \{1,2,3\}$ and define \\[0.2cm]
      \hspace*{1.3cm} $R := \{ \pair(1,1),\pair(4,2),\pair(9,3) \}$. \\[0.2cm]
      $R$ is not a function on  $M$, as  $R$ is not left-total on $M$.  
      For example, the number  $2$ is not mapped by $R$ to an element of  
      $N$.
\item Let  $M = \{1,2,3\}$, $N = \{1,2,3,4,5,6,7,8,9\}$ and define \\[0.2cm]
      \hspace*{1.3cm} $R := \{ \pair(1,1),\pair(2,3),\pair(2,4),\pair(3,9) \}$ \\[0.2cm]
       $R$ is not a function on $M$, as $R$ is not right-unique.
       For example, the element $2$ is mapped both to  $3$ and also to $4$.
\end{enumerate}
If $R \subseteq M \times N$ is a binary  relation and  $X \subseteq M$, then we define the
 \emph{image of  $X$ under $R$} as \\[0.2cm]
\hspace*{1.3cm} $R(X) := \{ y \mid \exists x \in X \colon \pair(x,y) \in R \}$. 

\paragraph{Inverse Relation}
Given a relation $R \subseteq M \times N$, we define the \emph{inverse} relation \\
$R^{-1} \subseteq N \times M$ as follows: \\[0.2cm]
\hspace*{1.3cm} $R^{-1} := \bigl\{ \pair(y,x) \mid \pair(x,y) \in R  \bigr\}$. \\[0.2cm]
Therefore $R^{-1}$ is right-unique iff $R$ is left-unique.  Furthermore, $R^{-1}$ is
left-total on $N$ iff $R$ 
is right-total on $N$.  
If a relation $R$ is both  left-unique and right-unique and furthermore
both left-total on $M$ and right-total on $N$, then $R$ is called  \emph{bijective}.
In this case, we can define two functions: 
\[ f_R: M \rightarrow N \quad \mbox{and} \quad f_{R^{-1}}:N \rightarrow M \]
The definition of $f_{R^{-1}}$ reads:
 \\[0.2cm]
\hspace*{1.3cm}
 $f_{R^{-1}}(y) := x \;\stackrel{de\!f}{\Longleftrightarrow}\; \pair(y,x) \in R^{-1}
 \Longleftrightarrow \pair(x,y) \in R$. 
\\[0.2cm]
The function $f_{R^{-1}}$ is the \emph{inverse function} of  $f_R$, as we have \\[0.2cm]
\hspace*{1.3cm}
 $\forall y \in N \colon f_R\bigl(f_{R^{-1}}(y)\bigr) = y$ \quad and \quad
 $\forall x \in M \colon f_{R^{-1}}\bigl(f_R(x)\bigr) = x$. \\[0.2cm]
This justifies the notation $R^{-1}$ for the inverse relation.

\paragraph{Composition of Relations}
As functions can be composed, so can relations be composed.
Assume $L$, $M$ and $N$ are sets.
If we have two relations  $R \subseteq L \times M$ and $Q \subseteq M \times N$, then the
 \emph{relational product} $R \circ Q$ is defined as follows: \\[0.2cm]
\hspace*{1.3cm}
$R \circ Q := \bigl\{ \pair(x,z) \mid \exists y \in M \colon(\pair(x,y) \in R \wedge \pair(y,z) \in Q) \bigr\}$ 
\\[0.2cm]
The relational product $R \circ Q$ is also known as the \emph{composition} of
$Q$ and $R$.  In the theory of relational databases, you will meet the
composition of relations again in the  more general form of a \emph{join}.
\vspace{0.2cm}

\noindent
\textbf{Example}: 
Let $L = \{1,2,3\}$, $M = \{4,5,6\}$ and $N = \{7,8,9\}$.  Furthermore, the relations
$Q$ and $R$ are given as follows: \\[0.2cm]
\hspace*{1.3cm} $R = \{ \pair(1,4), \pair(1,6), \pair(3,5) \}$ \quad and \quad
                $Q = \{ \pair(4,7), \pair(6,8), \pair(6,9) \}$. \\[0.2cm]
Then we have \\[0.2cm]
\hspace*{1.3cm} $R \circ Q = \{ \pair(1,7), \pair(1,8), \pair(1,9) \}$.
\vspace{0.2cm}


\noindent
If $R \subseteq L \times M$ is a function on $L$ and  $Q \subseteq M \times N$ is a
function on $M$, then
$R \circ Q$ is a function on $L$ and the function $f_{R \circ Q}$ can be computed as
follows from $f_R$ and $f_Q$:
\\[0.2cm]
\hspace*{1.3cm}
$f_{R \circ Q}(x) = f_Q\bigl(f_R(x)\bigr)$.
\\[0.2cm]
Observe that the order, in which $R$ and $Q$ appear on the left hand side of this equation,
is different than the order of $R$ and $Q$ on the right hand side.


\noindent
\textbf{Remark}:  You should be aware that some text books define the relational product
$R \circ Q$ as $Q \circ R$.
In that case, the definition of $R \circ Q$ reads as follows:
If  $R \subseteq M \times N$ and $Q \subseteq L \times M$, then
\\[0.2cm]
\hspace*{1.3cm}
$R \circ Q := \bigl\{ \pair(x,z) \mid \exists y \in M \colon(\pair(x,y) \in Q \wedge
\pair(y,z) \in R) \bigr\}$.
\\[0.2cm]
The definition used in this lecture is more intuitive when we compute the transitive
closure of a binary relation.  \qed



\example
The next example demonstrates the use of the composition of two relations in the context
of a database.  Let us assume that the database of a car dealer contains the following two tables.
\begin{center}
\textsl{Purchase}:  \begin{tabular}[t]{|l|l|}
\hline
\textsl{Customer} & \textsl{Car} \\
\hline
\hline
  Bauer   & Golf \\
\hline
  Bauer   & Fox  \\
\hline
  Schmidt & Polo \\
\hline
  \end{tabular}
\qquad \textsl{Price}:
  \begin{tabular}[t]{|l|l|}
\hline
\textsl{Car} & \textsl{Amount} \\
\hline
\hline
  Golf    & $20\,000$ \\
\hline
  Fox     & $10\,000$ \\
\hline
  Polo    & $13\,000$ \\
\hline
  \end{tabular}
\end{center}
Then the  relational product of $\textsl{Purchase} \circ \textsl{Price}$ is given as follows:

\begin{center}
  \begin{tabular}[t]{|l|l|}
\hline
\textsl{Car} & \textsl{Amount} \\
\hline
\hline
  Bauer   & $20\,000$ \\
\hline
  Bauer   & $10\,000$ \\
\hline
  Schmidt & $13\,000$ \\
\hline
  \end{tabular} 
\end{center}
This relation could be used for invoicing practice. \qed
\vspace{0.2cm}


\paragraph{Properties of  the Relational Product}
The composition of relations is \emph{associative}:  If \\[0.2cm]
\hspace*{1.3cm} $R \subseteq K \times L$, \quad $Q \subseteq L \times M$ \quad and \quad 
                $P \subseteq M \times N$ \\[0.2cm]
are binary relations, then we have \\[0.2cm]
\hspace*{1.3cm} $(R \circ Q) \circ P = R \circ (Q \circ P)$. \\[0.2cm]
\textbf{Proof}:  We show
\begin{equation}
  \label{eq:ass0}
   \pair(x,u) \in (R \circ Q) \circ P \leftrightarrow \pair(x,u) \in R \circ (Q \circ P)   
\end{equation}
First, we manipulate the left hand side  $\pair(x,u) \in (R \circ Q) \circ P$ of
(\ref{eq:ass0}). We have
\[
\begin{array}{cll}
                  & \pair(x,u) \in (R \circ Q) \circ P \\[0.2cm]
  \leftrightarrow & \exists z: \bigl(\pair(x,z) \in R \circ Q \wedge \pair(z,u) \in P\bigr) &
                    \mbox{definition of}\; (R \circ Q) \circ P \\[0.2cm]
  \leftrightarrow & \exists z: \bigl(\bigl(\exists y: \pair(x,y) \in R \wedge \pair(y,z) \in Q\bigr) \wedge \pair(z,u) \in P\bigr) &
                    \mbox{definition of}\; R \circ Q \\
\end{array}
\]
As the variable $y$ does not occur in the formula $\pair(z,u) \in P$, the existential
quantifier  $\exists y$ can be lifted as follows:
\begin{equation}
  \label{eq:ass1}
  \exists z: \exists y: \bigl(\pair(x,y) \in R \wedge \pair(y,z) \in Q \wedge \pair(z,u) \in P\bigr)
\end{equation}
Now we manipulate the right hand side of (\ref{eq:ass0}):
\[
\begin{array}{cll}
                & \pair(x,u) \in R \circ (Q \circ P) \\[0.2cm] 
\leftrightarrow & \exists y: \bigl(\pair(x,y) \in R \wedge \pair (y,u) \in Q \circ P\bigr) &
                  \mbox{definition of}\; R \circ (Q \circ P) \\[0.2cm]
\leftrightarrow & \exists y: \bigl(\pair(x,y) \in R \wedge 
                  \exists z: \bigl(\pair(y,z) \in Q \wedge \pair(z,u) \in P\bigr)\bigr) &
                  \mbox{definition of}\; Q \circ P \\
\end{array}
\]
As the variable  $z$ does not occur in the formula $\pair(x,y) \in R$, the existential
quantifier with respect to $z$ can be lifted and we get
\begin{equation}
  \exists z: \exists y: \bigl(\pair(x,y) \in R \wedge \pair(y,z) \in Q \wedge \pair(z,u) \in P\bigr)
\end{equation}
Interchanging the order of the existential quantifiers we arrive at
\begin{equation}
  \label{eq:ass2}
  \exists z: \exists y: \bigl(\pair(x,y) \in R \wedge \pair(y,z) \in Q \wedge \pair(z,u) \in P\bigr)
\end{equation}
Now the  {formul\ae} (\ref{eq:ass1}) and (\ref{eq:ass2}) are identical.  Therefore we have
shown the equivalence
(\ref{eq:ass0}) and the law of associativity is proved.  \hspace*{\fill} $\Box$
\vspace{0.2cm}

Another important property of the relational product is the following: If
 $R \subseteq L \times M$ and $Q \subseteq M \times N$ are two relations, then we have \\[0.2cm]
\hspace*{1.3cm} $(R \circ Q)^{-1} = Q^{-1} \circ R^{-1}$. \\[0.2cm]
You should note that the sequence of  $Q$ and $R$ is different on the left side of this
equation
from the order in which $Q$ and $R$ appear on the right side.  To prove this equation we
have to show that for all pairs $\pair(z,x) \in N \times L$ the equivalence \\[0.2cm]
\hspace*{1.3cm} $\pair(z,x) \in (Q \circ R)^{-1} \leftrightarrow \pair(z,x) \in R^{-1} \circ Q^{-1}$ \\[0.2cm]
is valid.  The proof is given as follows: 
\[ 
\begin{array}{cl}
                & \pair(z,x) \in (R \circ Q)^{-1}                                             \\[0.2cm]
\leftrightarrow & \pair(x,z) \in R \circ Q                                                    \\[0.2cm]
\leftrightarrow & \exists y \in M \colon \bigl(\pair(x,y) \in R \wedge \pair(y,z) \in Q\bigr) \\[0.2cm]
\leftrightarrow & \exists y \in M \colon \bigl(\pair(y,z) \in Q \wedge \pair(x,y) \in R\bigr) \\[0.2cm]
\leftrightarrow & \exists y \in M \colon \bigl(\pair(z,y) \in Q^{-1} \wedge \pair(y,x) \in R^{-1}\bigr) 
                  \\[0.2cm]
\leftrightarrow & \pair(z,x) \in Q^{-1} \circ R^{-1}   \hspace*{\fill} \Box                                       
\end{array}
\]
\vspace{0.2cm}

\noindent
We note the following law of distributivity:  If $R_1$ and $R_2$
are relations on $L \times M$ and  $Q$ is a  relation on $M \times N$, then we have: \\[0.2cm]
\hspace*{1.3cm}  $(R_1 \cup R_2) \circ Q = (R_1 \circ Q) \cup (R_2 \circ Q)$. \\[0.2cm]
In a similar way we have \\[0.2cm]
\hspace*{1.3cm}  $R \circ (Q_1 \cup Q_2) = (R \circ Q_1) \cup (R \circ Q_2)$, \\[0.2cm]
provided $R$ is a  relation on $L \times M$ and $Q_1$ and $Q_2$ are relations on $M \times
N$.  
In order to be able to give a concise proof of the first equation we agree that the
composition operator $\circ$ has a higher precedence than both $\cup$ and $\cap$.  We prove
the first law of distributivity by showing  
\begin{equation}
  \label{eq:dis0}
\pair(x,z) \in (R_1 \cup R_2) \circ Q \;\leftrightarrow\; \pair(x,z) \in R_1 \circ Q \cup R_2 \circ Q.
\end{equation}
To this end, the formula  $\pair(x,z) \in (R_1 \cup R_2) \circ Q$ is manipulated as follows:
\[
\begin{array}{cll}
                  & \pair(x,z) \in (R_1 \cup R_2) \circ Q  \\[0.2cm]
  \leftrightarrow & \exists y: \bigl(\pair(x,y) \in R_1 \cup R_2 \wedge \pair(y,z) \in Q\bigr) 
                  & \mbox{definition of}\; (R_1 \cup R_2) \circ Q \\[0.2cm]
  \leftrightarrow & \exists y: \bigl(\bigl(\pair(x,y) \in R_1 \vee \pair(x,y) \in R_2\bigr) \wedge \pair(y,z) \in Q\bigr) 
                  & \mbox{definition of}\; R_1 \cup R_2 \\[0.2cm]
\end{array}
\]
This formula is manipulated using the law of distributivity from propositional logic.
Later, in propositional logic we will show that for arbitrary {formul\ae}
$F_1$, $F_2$ and $G$ the equivalence 
\[ (F_1 \vee F_2) \wedge G \;\leftrightarrow\; (F_1 \wedge G) \vee (F_2 \wedge G) \]
holds.  Using this law we get:
\begin{eqnarray}
  \nonumber
  & & \exists y: \bigl(\bigl(\underbrace{\pair(x,y) \in R_1}_{F_1} \vee \underbrace{\pair(x,y) \in R_2}_{F_2}\bigr) \wedge \underbrace{\pair(y,z) \in Q}_G\bigr) 
\\[0.2cm] 
  \label{eq:dis1}
  & \leftrightarrow &
    \exists y: \bigl(\bigl(\underbrace{\pair(x,y) \in R_1}_{F_1} \wedge \underbrace{\pair(y,z) \in Q}_G \bigr) \vee 
               \bigl(\underbrace{\pair(x,y) \in R_2}_{F_2} \wedge \underbrace{\pair(y,z) \in Q}_G \bigr) \bigr)   
\end{eqnarray}
Next, we manipulate $\pair(x,z) \in R_1 \circ Q \cup R_2 \circ Q$:
\[
\begin{array}{cll}
                & \pair(x,z) \in R_1 \circ Q \cup R_2 \circ Q \\[0.2cm]
\leftrightarrow & \pair(x,z) \in R_1 \circ Q \;\vee\; \pair(x,z) \in R_2 \circ Q &
                  \mbox{definition of}\;\cup \\[0.2cm]
\leftrightarrow & \bigl(\exists y: (\pair(x,y) \in R_1 \wedge \pair(y,z) \in Q)\bigr) \;\vee\; 
                  \bigl(\exists y: (\pair(x,y) \in R_2 \wedge \pair(y,z) \in Q)\bigr) \\[0.2cm]
                & \mbox{definition of}\; R_1 \circ Q \; \mbox{and}\; R_2 \circ Q \\[0.2cm]
\end{array}
\]
This formula is now manipulated using a law of distributivity for predicate logic.
In the chapter on predicate logic we will see that for arbitrary {formul\ae}
 $F_1$ and $F_2$ the equivalence
\[ \exists y: \bigl(F_1 \vee F_2\bigr) \;\leftrightarrow\; \bigl(\exists y: F_1\bigr) \vee \bigl(\exists y: F_2\bigr) \]
holds.  Then we conclude 
\begin{eqnarray}
\nonumber
 & & \exists y: \bigl(\underbrace{\pair(x,y) \in R_1 \wedge \pair(y,z) \in Q}_{F_1}\bigr) \;\vee\; 
     \exists y: \bigl(\underbrace{\pair(x,y) \in R_2 \wedge \pair(y,z) \in Q}_{F_2}\bigr) \\[0.2cm]
  \label{eq:dis2}
 & \leftrightarrow &
     \exists y: \Bigl(\bigl(\underbrace{\pair(x,y) \in R_1 \wedge \pair(y,z) \in Q}_{F_1}\bigr) \;\vee\; 
                \bigl(\underbrace{\pair(x,y) \in R_2 \wedge \pair(y,z) \in Q}_{F_2}\bigr)\Bigr) 
\end{eqnarray}
As (\ref{eq:dis1}) and (\ref{eq:dis2}) are identical, the proof is complete.
\hspace*{\fill} $\Box$
\vspace{0.2cm}

\noindent
\textbf{Remark:}
It is interesting to note that for the intersection and the composition operator
there is no law of distributivity, the equation
\\[0.2cm]
\hspace*{1.3cm}
$(R_1 \cap R_2) \circ Q = R_1 \circ Q \cap R_2 \circ Q$
\\[0.2cm]
does not hold in general.  In order to validate this claim we provide a counter example.
Let us define the relations $R_1$, $R_2$ and $Q$ as follows: \\[0.2cm]
\hspace*{1.3cm} $R_1 := \{ \pair(1,2) \}$, \quad $R_2 := \{ \pair(1,3) \}$ \quad and \quad
                $Q = \{ \pair(2,4), \pair(3,4) \}$. \\[0.2cm]
Then we have \\[0.2cm]
\hspace*{1.3cm} $R_1 \circ Q = \{ \pair(1,4) \}$, \quad $R_2 \circ Q = \{ \pair(1,4) \}$,
\quad therefore 
\\[0.2cm]
\hspace*{1.3cm}
                $R_1 \circ Q \cap R_2 \circ Q = \{ \pair(1,4) \}$. \\[0.2cm]
But on the other hand, we have  \\[0.2cm]
\hspace*{1.3cm} 
$(R_1 \cap R_2) \circ Q  = \emptyset \circ Q = \emptyset \not=
\{\pair(1,4) \} = R_1 \circ Q  \cap R_2 \circ Q$. 
\qed

\paragraph{Identical Relation} If  $M$ is a set, we define the \emph{identical relation} $\id_M \subseteq M \times M$
as follows: \\[0.2cm]
\hspace*{1.3cm} $\id_M := \bigl\{ \pair(x,x) \mid x \in M \bigr\}$. 
\vspace*{0.2cm}

\noindent
\textbf{Example}: Let $M = \{1,2,3\}$.  Then \\[0.2cm]
\hspace*{1.3cm}  $\id_M := \bigl\{ \pair(1,1),  \pair(2,2),  \pair(3,3) \bigr\}$.
\vspace*{0.2cm}

\noindent
An immediate consequence of this definition is \\[0.2cm]
\hspace*{1.3cm} $\id_M^{-1} = \id_M$. \\[0.2cm]
If  $R \subseteq M \times N$ is a binary relation we have 
\[ R \circ \id_N = R \quad \mbox{and} \quad \id_M \circ R = R. \] 
We prove the second equation.  According to the definition of the composition we have
\[ \id_M \circ R = \bigl\{ \pair(x,z) \mid \exists y: \pair(x,y) \in \id_M \wedge \pair(y,z) \in R \bigr\}. \]
Now we have $\pair(x,y) \in \id_M$ if and only if  $x = y$, therefore
\[ \id_M \circ R = \bigl\{ \pair(x,z) \mid \exists y: x = y \wedge \pair(y,z) \in R \bigr\}. \]
Furthermore, we have
\[ \exists y: \bigl( x = y \wedge \pair(y,z) \in R \bigr) \leftrightarrow \pair(x,z) \in R. \]
This is easy to see:  If we have $\exists y: x = y \wedge \pair(y,z) \in R$,
the $y$ must be equal to  $x$ and then we have  $\pair(x,z) \in R$.  If on the other hand
 $\pair(x,z) \in R$, define $y := x$.  For this $y$ we obviously have 
$x = y \wedge \pair(y,z) \in R$.  Using the equivalence given above we conclude 
\[ \id_M \circ R = \bigl\{ \pair(x,z) \mid \pair(x,z) \in R \bigr\}. \]
As  $R \subseteq M \times N$ we know that $R$ is a set of ordered pairs and therefore
\[ R = \bigl\{ \pair(x,z) \mid \pair(x,z) \in R \bigr\}. \]
This proves $\id_M \circ R = R$. \hspace*{\fill} $\Box$
\vspace*{0.2cm}

\exercise
Assume $R \subseteq M \times N$.  State conditions for the relation $R$ such that
\\[0.2cm]
\hspace*{1.3cm}
$R \circ R^{-1} = \id_M$  
\\[0.2cm]
holds.  State conditions such that 
\\[0.2cm]
\hspace*{1.3cm}
$R^{-1} \circ R = \id_M$
\\[0.2cm]
holds.  \qed
% The first equation holds iff $R$ is left-total and left-unique.
\pagebreak


\subsection{Binary Relations on a Set}
In the following, we discuss the special case of those relations $R \subseteq M \times N$
that satisfy  $M = N$.  We define: A relation $R \subseteq M \times M$ is a  relation
\emph{on} the set $M$. In the rest of this section we restrict ourself to discuss
relations of this kind.  Instead of $R \subseteq M \times M$ we sometimes write $R \subseteq M^2$.
Furthermore, if $R$ is a relations on  $M$ and if $x, y \in M$, we sometimes write $x R y$ instead of 
$\pair(x,y) \in R$ and thereby treat $R$ as an infix operator.  For example, the relation $\leq$ 
on  $\mathbb{N}$ is defined as follows: \\[0.2cm]
\hspace*{1.3cm}
 $\leq\; := \{ \pair(x,y) \in \mathbb{N} \times \mathbb{N} \mid \exists z \in \mathbb{N} \colon x + z = y \}$.
\\[0.2cm]
Instead of  $\pair(x,y) \in\; \leq$ it is customary to write $x \leq y$.
  
\begin{Definition} 
The relation $R \subseteq M \times M$ is called \emph{reflexive} on  $M$ iff \\[0.2cm]
\hspace*{1.3cm} $\forall x\in M \colon \pair(x,x) \in R$. 
\end{Definition}

\begin{Proposition}
The relation $R \subseteq M \times M$ is  reflexive on $M$ iff $\mbox{\rm id}_M \subseteq R$.
\end{Proposition}

\noindent
\textbf{Proof}: We have
\\[0.2cm]
\hspace*{1.3cm}
$
\begin{array}[b]{cl}
              & \id_M \subseteq R \\[0.2cm]
\mbox{iff} & \bigl\{ \pair(x,x) \mid x \in M \bigr\} \subseteq R \\[0.2cm] 
\mbox{iff} & \forall x \in M: \pair(x,x) \in R \\[0.2cm] 
\mbox{iff} & \mbox{$R$ is reflexive.}  
\end{array}
$
\hspace*{\fill} $\Box$
\vspace*{0.2cm}

\begin{Definition}
The relation $R \subseteq M \times M$ is \emph{symmetric} iff \\[0.2cm]
\hspace*{1.3cm} 
$\forall x,y\el M \colon \bigl(\pair(x,y) \in R \rightarrow\pair(y,x)\in R\bigr)$. 
\end{Definition}

\begin{Proposition}
The relation $R \subseteq M \times M$ is symmetric iff $R^{-1} \subseteq R$.
\end{Proposition}

\noindent
\textbf{Proof}:
Let us rewrite the formula $R^{-1} \subseteq R$ by substituting
\[ R^{-1} = \bigl\{ \pair(y,x) \mid \pair(x,y) \in R \bigr\}. \]
Then $R^{-1} \subseteq R$ is equivalent to
\[ \bigl\{ \pair(y,x) \mid \pair(x,y) \in R \bigr\} \subseteq R. \]
According to the definition of $\subseteq$ this is the same as
\[ \forall x, y \in M \colon \bigl(\pair(x,y) \in R \rightarrow\pair(y,x) \in R\bigr) \]
and this is the condition for $R$ being symmetric. \qed
\vspace*{0.2cm}

\begin{Definition}
The relation $R \subseteq M \times M$  is \emph{anti-symmetric} iff \\[0.2cm]
\hspace*{1.3cm} 
$\forall x, y \in M \colon \bigl(\pair(x,y) \in R \wedge \pair(y,x) \in R \rightarrow x = y\bigr)$.
\end{Definition}

\begin{Proposition}
The relation $R \subseteq M \times M$  is anti-symmetric iff
$R \cap R^{-1} \subseteq \mbox{\rm id}_{M}$.
\end{Proposition}

\noindent
\textbf{Proof}:
We split the proof into two parts.  Let us first assume that $R$ is anti-symmetric and therefore
\[ \forall x, y \in M \colon \bigl(\pair(x,y) \in R \wedge \pair(y,x) \in R \rightarrow x = y\bigr) \]
holds.  We show that $R \cap R^{-1} \subseteq \id_{M}$ is a consequence of this
assumption.  Assume $\pair(x,y) \in R \cap R^{-1}$.
Then $\pair(x,y) \in R$ and from $\pair(x,y) \in R^{-1}$ we know  $\pair(y,x) \in R$.
From our assumption we can then conclude  $x=y$.
This implies $\pair(x,y) \in \id_M$, showing  $R \cap R^{-1} \subseteq \id_{M}$.

To prove the remaining part, assume $R \cap R^{-1} \subseteq \id_{M}$.  We have to show
that this implies 
\[ \forall x, y \in M \colon \bigl(\pair(x,y) \in R \wedge \pair(y,x) \in R \rightarrow x = y\bigr). \]
Therefore, take $x,y \in M$ such that $\pair(x,y) \in R$ and $\pair(y,x) \in R$.  We have
to show  $x=y$.  From $\pair(y,x) \in R$ we have
$\pair(x,y) \in R^{-1}$.  Therefore $\pair(x,y) \in R \cap R^{-1}$.
As we have assumed $R \cap R^{-1} \subseteq \id_{M}$, we can conclude
$\pair(x,y) \in \id_M$ and that implies $x = y$. \hspace*{\fill} $\Box$
\vspace*{0.2cm}

\begin{Definition}
The relation $R \subseteq M \times M$  is \emph{asymmetric} iff \\[0.2cm]
\hspace*{1.3cm} 
$\forall x, y \in M \colon \neg \bigl(\pair(x,y) \in R \wedge \pair(y,x) \in R)$.
\end{Definition}

Although the names are similar, the concept of  an asymmetric relation is quite different
from the concept  of an anti-symmetric relation.
Therefore, it is important not to confound these notions.

\begin{Definition}
The  relation $R \subseteq M \times M$  is \emph{transitive} iff \\[0.2cm]
\hspace*{1.3cm} 
$\forall x, y, z \in M \colon 
 \bigl(\pair(x,y) \in R \wedge \pair(y,z) \in R \rightarrow \pair(x,z) \in R\bigr)$.
\end{Definition}

\begin{Proposition}
The relation $R \subseteq M \times M$  is transitive iff $R \circ R \subseteq R$.
\end{Proposition}

\noindent
\textbf{Proof}:  Let us first assume that $R$ is transitive.  We have to show that this
implies $R \circ R \subseteq R$.  
In order to show $R \circ R \subseteq R$ assume $\pair(x,z) \in R \circ R$.  We have
to show $\pair(x,z) \in R$.  According to the definition of the composition $R \circ R$
the assumption $\pair(x,z) \in R \circ R$ entails that there exists a $y$ such that we
have 
\\[0.2cm]
\hspace*{1.3cm}
$\pair(x,y) \in R$ \quad and \quad $\pair(y,z) \in R$.
\\[0.2cm]
As we have assumed $R$ transitive this implies  $\pair(x,z) \in R$.

To prove the second part, let us assume $R \circ R \subseteq R$.  We have to show that this
implies
\[ \forall x, y, z \in M \colon 
   \bigl(\pair(x,y) \in R \wedge \pair(y,z) \in R \rightarrow \pair(x,z) \in R\bigr). 
\]
In order to show this, assume  $\pair(x,y) \in R$ and $\pair(y,z) \in R$.
According to the definition of the relational product,
this implies  $\pair(x,z) \in R \circ R$ and using the assumption $R \circ R \subseteq R$
we  conclude  $\pair(x,z) \in R$. \qed
\vspace*{0.2cm}



\noindent
\textbf{Examples}:  For the first two examples, define $M = \{1,2,3\}$.
\begin{enumerate}
\item $R_1 = \{ \pair(1,1), \pair(2,2), \pair(3,3) \}$.

      $R_1$ is reflexive on $M$, symmetric, anti-symmetric, and transitive.
      Note that $R_1$ is not asymmetric as we can set $x := 1$ and $y := 1$ 
      and then have both $\pair(x,y) \in R$ and $\pair(y,x) \in R$.
\item $R_2 = \{ \pair(1,2), \pair(2,1), \pair(3,3) \}$.

      $R_2$ is not  reflexive on $M$, as $\pair(1,1) \not\in R_2$.
      $R_2$ is symmetric. 
      $R_2$ isn't anti-symmetric, because we have both $\pair(1,2) \in R_2$ and 
      $\pair(2,1) \in R_2$, but $2 \not=1$.
      As $\pair(3, 3) \in R$, $R$ isn't asymmetric either.
      Finally,  $R_2$ isn't transitive, because if it were, then the facts
      $\pair(1,2) \in R_2$ and 
      $\pair(2,1) \in R_2$ would imply $\pair(1,1) \in R_2$, which is wrong.

      In the following examples we have $M = \mathbb{N}$.
\item $R_3 := \{ \pair(n,m) \in \mathbb{N}^2 \mid n \leq m \}$.

      As we have $n \leq n$ for all $n \in \mathbb{N}$, $R_3$ is reflexive on $\mathbb{N}$.  
      $R_3$ is not symmetric: For example, we have $1 \leq 2$ but we don't have $2 \leq
      1$.  
      $R_3$ is  anti-symmetric: If we have both $n \leq m$ and $m \leq n$, then we can
      conclude $m = n$.
      $R_3$ is not asymmetric, as we have $1 \leq 1$.
      Finally, $R_3$ is transitive: If we have $k \leq m$ and $m \leq n$, then we can conclude
      $k \leq n$.
\item $R_4 := \{ \pair(m,n) \in \mathbb{N}^2 \mid \exists k\in \mathbb{N}: m\cdot k = n \}$

      For two positive integers $m$ and $n$ we have $\pair(m,n) \in R_4$ if $m$ divides
      $n$.  Therefore $R_4$ is reflexive on $\mathbb{N}$, because every positive 
      integer divides itself.  $R_4$ is not  symmetric: For example, $1$ divides  $2$ 
      but $2$ does not divide $1$.  However, $R_4$ is anti-symmetric: If 
      $m$ divides $n$  and $n$ divides $m$, then we must have $m = n$.  
      As $R_4$ is reflexive, it cannot be asymmetric.
      Finally,  $R_4$ is transitive: If  $m$ divides $n$ and
      $n$ divides  $o$,  we must have natural numbers  $a$ and $b$ such that
      \\[0.2cm]
      \hspace*{1.3cm}
      $n = a \cdot m$ \quad and \quad $o = b \cdot n$.
      \\[0.2cm]
      This implies $o = b \cdot a \cdot m$ and therefore $m$ divides $o$.
\end{enumerate}
Sometimes, it is necessary to convert a relation $R$ that is not transitive into a
transitive relation.  In order to do so, we define the powers $R^n$ for all
 $n \in \mathbb{N}$.  This definition is given by induction on $n$.
\begin{enumerate}
\item Base Case: $n= 0$.  Define \\[0.2cm]
      \hspace*{1.3cm} $R^0 := \id_M$
\item Induction Step: $n \rightarrow n + 1$. 

      According to the inductive hypothesis,  $R^n$ is already defined.
      We define $R^{n+1}$ as \\[0.2cm]
      \hspace*{1.3cm} $R^{n+1} = R \circ R^n$.
\end{enumerate}
Later, we need the following law of powers: For any  $k,l \in \mathbb{N}$ we have
\[ R^k \circ R^l = R^{k+l}. \]
\textbf{Proof}:  We proof this law by induction on  $k$.
\begin{enumerate}
\item[B.C.:] $k = 0$.  We have
             \[ R^0 \circ R^l = \textsl{id}_M \circ R^l = R^l = R^{0+l}. \]
\item[I.S.:] $k \mapsto k+1$.  We have
             \[
             \begin{array}{lcll}
               R^{k+1} \circ R^l & = & (R \circ R^k) \circ R^l &
                                       \mbox{definition of}\; R^{k+1} \\
                                 & = & R \circ (R^k \circ R^l) &
                                       \mbox{associativity of $\circ$} \\
                                 & = & R \circ R^{k+l} &
                                       \mbox{induction hypothesis} \\
                                 & = & R^{(k+l)+1} &
                                       \mbox{definition}\; R^{(k+l)+1} \\
                                 & = & R^{(k+1)+l}. & \hspace*{\fill} \Box
             \end{array}
             \]
\end{enumerate}
\vspace*{0.3cm}

\noindent
We define the  \emph{transitive closure} of a binary relation $R$ on a set $M$ as \\[0.2cm]
\hspace*{1.3cm} $R^+ := \bigcup\limits_{n=1}^{\infty} R^n$. \\
Here, the expression  $\bigcup\limits_{i=1}^{\infty} R^n$ 
is to be interpreted as  \\[0.2cm]
\hspace*{1.3cm} 
$\bigcup\limits_{i=1}^{\infty} R^n = R^1 \cup R^2 \cup R^3 \cup \cdots $. 

\begin{Proposition}
Assume $M$ is a set  and $R \subseteq M \times M$ is a binary relation on $M$.
Then we have the following:
\begin{enumerate}
\item $R^+$ is transitive.
\item With respect to the inclusion ordering $\subseteq$, the relation $R^+$ 
      is the smallest relation $T$ on  $M$ that is both transitive and extends $R$.
      To put it differently: If $T$ is a transitive relation on $M$ such that $R \subseteq
      T$, then 
      we must have  $R^+ \subseteq T$, as $R^+$ is the smallest relation with these properties.
\end{enumerate}
\end{Proposition}

\noindent
\textbf{Proof}:
\begin{enumerate}
\item Let us first prove that $R^+$ is transitive.  We have to prove
\[ \forall x, y, z: 
   \bigl(\pair(x,y) \in R^+ \wedge \pair(y,z) \in R^+ \rightarrow \pair(x,z) \in   R^+\bigr). 
\]
Therefore, assume  $\pair(x,y) \in R^+$ and $\pair(y,z) \in R^+$.  We have to show
that this implies  $\pair(x,z) \in R^+$.  According to the definition of  $R^+$ we have
\[ \pair(x,y) \in \bigcup\limits_{n=1}^{\infty} R^n \quad \mbox{and} \quad
   \pair(y,z) \in \bigcup\limits_{n=1}^{\infty} R^n.
\]
According to the definition of  $\bigcup\limits_{n=1}^{\infty} R^n$ there are natural
numbers  $k,l\in\mathbb{N}$ such that we have
\[ \pair(x,y) \in R^k \quad \mbox{and} \quad \pair(y,z) \in R^l. \]
Using the definition of the composition $R^k \circ R^l$ we conclude
\[  \pair(x,z) \in R^k \circ R^l. \]
Using the law of powers we have 
\[ R^k \circ R^l = R^{k+l}. \]
Combining these {formul\ae}, we conclude $\pair(x,z) \in R^{k+l}$ and this implies
\[  \pair(x,z) \in \bigcup\limits_{n=1}^{\infty} R^n. \]
Therefore,  we have $\pair(x,z) \in R^+$ and thus have shown $R^+$ to be transitive. 

\item Next, we prove that $R^+$ is indeed the smallest  relation that is both transitive
      and contains $R$.  Therefore, assume  $T$ is a transitive relation such that
      $R \subseteq T$.  We have to show  $R^+ \subseteq T$.

      In order to do this, we first prove the following claim by induction on $n$
      \[ \forall n \in \mathbb{N}: \bigl(n \geq 1 \rightarrow R^n \subseteq T\bigr). \]

      \textbf{B.C.:} $n=1$.  We have to show  $R^1 \subseteq T$.  We have
      \[ R^1 = R \circ \id_M = R. \]
      Therefore, $R^1 \subseteq T$ is an immediate consequence of our assumption $R \subseteq T$.

      \textbf{I.S.:} $n \mapsto n+1$.  According to the induction hypothesis, we have
      \[ R^n \subseteq T. \]
      We  multiply this inclusion with $R$ and arrive at
      \[ R^{n+1} = R \circ R^n \subseteq R \circ T. \]
      If we multiply both sides of the assumption  $R \subseteq T$ with $T$ we get
      \[ R \circ T \subseteq T \circ T. \]
      As $T$ is transitive, we have 
      \[ T \circ T \subseteq T. \]
      Taken together, we have the following chain of inclusions
      \[ R^{n+1} \subseteq R \circ T \subseteq T \circ T \subseteq T. \]
      Therefore, we have shown $R^{n+1} \subseteq T$ and the claim is proven.

      From this, $R^+ \subseteq T$ is an immediate consequence of the definition of
      $R^+$.  \qed
\end{enumerate}

\example
Define  \textsl{Human} as the set of all human beings that have ever lived.
We define the \textsl{parent} on \textsl{Human} as follows:
\\[0.2cm]
\hspace*{1.3cm}
$\textsl{parent} := \{ \pair(x,y) \in \textsl{Human}^2 \mid \mbox{$x$ is father of $y$ or
                                                                  $x$ is mother of $y$} \}.$
\\[0.2cm]
The transitive closure of  \textsl{parent} is the set of all those pairs
$\pair(x,y)$ such that $x$ is an ancestor of $y$:
\\[0.2cm]
\hspace*{1.3cm}
$\textsl{parent}^+ = \{ \pair(x,y) \in \textsl{Human}^2 \mid \mbox{$x$ is an ancestor of $y$} \}$.
\next

\example
Define \textsl{F} as the set of all airports.  Let us define a 
relation \textsl{D} on $F$ as follows:
\\[0.2cm]
\hspace*{1.3cm}
$\textsl{D} := \{ \pair(x,y) \in \textsl{F} \times \textsl{F} \mid
                  \mbox{there is a direct flight from $x$ to $y$} \}$.
\\[0.2cm]
So $D$ is the set of all direct connections.  The relation $D^2$ is defined as
\\[0.2cm]
\hspace*{1.3cm}
$D^2 = \{ \pair(x,z) \in \textsl{F} \times \textsl{F} \mid 
          \exists y \in \textsl{F}: \bigl(\pair(x,y) \in D \wedge \pair(y,z) \in D\bigr) \}$.
\\[0.2cm]
These are all pairs $\pair(x,y)$ such that there is a connection from $x$ to $y$ with
exactly one stop.
Similarly, $D^3$ is the set of all pairs $\pair(x,y)$ such that you can get from $x$ to
$y$ with two stops.  In general, $D^k$ is the set of those pairs $\pair(x,z)$, 
such that you need  $k-1$ stops to go from $x$ to $y$.
The transitive closure $D^+$ contains all those pairs of airports $\pair(x,y)$ such that 
there is a connection from $x$ to $y$ that can contain any number of stops in between.


\exercise
Let us define the relation $R$ on the set $\mathbb{N}$ as follows:
\\[0.2cm]
\hspace*{1.3cm}
$R = \{ \pair(k, k + 1) \mid k \in \mathbb{N} \}$.
\\[0.2cm]
Compute the following relations:
\begin{enumerate}
\item $R^2$,
\item $R^3$,
\item $R^n$ for any $n \in \mathbb{N}$ such that $n \geq 1$,
\item $R^+$.
\end{enumerate}


\begin{Definition}
The relation $R \subseteq M \times M$  is an
\emph{equivalence relation} on $M$ iff
\begin{enumerate}
\item $R$ is reflexive on $M$,
\item $R$ is symmetric and
\item $R$ is transitive.
\end{enumerate}
\end{Definition}

The notion of an equivalence relation is a generalization of the notion of equality, as
the identical relation $\textsl{id}_M$ is a trivial example of an equivalence relation on $M$.
To present a non-trivial example, given a positive natural number $n$, we define the relation 
 \\[0.2cm]
\hspace*{1.3cm}
 $\approx_n \;:=\; \{ \pair(x,y) \in \mathbb{Z}^2 \mid \exists k \in \mathbb{Z} \colon k \cdot n = x - y \}$
\\[0.2cm]
We have   $x \approx_n y$ iff $x$ and $y$ yield the same remainder when divided by $n$.
We show that for  $n \not=0$ the relation $\approx_n$ is an equivalence relation on $\mathbb{Z}$.
\begin{enumerate}
\item In order to prove that $\approx_n$ is reflexive we have to show that
      \\[0.2cm]
      \hspace*{1.3cm}
      $\pair(x,x) \in\; \approx_n$ \quad holds for all $x \in \mathbb{Z}$.
      \\[0.2cm]
      According to the definition of $\approx_n$
      we therefore have to show \\[0.2cm]
      \hspace*{1.3cm}
      $\pair(x,x) \in \bigl\{ \pair(x,y) \in \mathbb{Z}^2 \mid \exists k \in \mathbb{Z}: k
      \cdot n = x - y \bigr\}$ \quad for all $x \in \mathbb{Z}$.
      \\[0.2cm]
      This is equivalent to \\[0.2cm]
      \hspace*{1.3cm}
      $\exists k \in \mathbb{Z}: k \cdot n = x - x$.
      \\[0.2cm]
      Obviously, choosing $k=0$ satisfies this equation.
\item Next, we prove that $\approx_n$ is symmetric.  Assume 
      $\pair(x,y) \in\; \approx_n$.  Then there is a  $k \in \mathbb{Z}$ such that
      \\[0.2cm]
      \hspace*{1.3cm}      
      $k\cdot n = x - y$.
      \\[0.2cm] 
      Therefore, we also have
      \\[0.2cm]
      \hspace*{1.3cm}      
      $(-k)\cdot n = y - x$.
      \\[0.2cm]
      Hence $\pair(y,x) \in\; \approx_n$.
\item In order to prove $\approx$ transitive we assume that both
       $\pair(x,y) \in\; \approx_n$ and $\pair(y,z) \in\; \approx_n$ holds.  Then there
       are $k_1,k_2 \in \mathbb{Z}$ such that
      \\[0.2cm]
      \hspace*{1.3cm}      
      $k_1 \cdot n = x - y$ \quad and \quad $k_2 \cdot n = y - z$.
      \\[0.2cm]
      Adding these equation we see
      \\[0.2cm]
      \hspace*{1.3cm}      
      $(k_1 + k_2) \cdot n = x - z$.
      \\[0.2cm]
      Defining  $k_3 := k_1 + k_2$ we have $k_3\cdot n = x - z$ and that shows
      $\pair(x,z) \in\; \approx_n$.   Therefore, the relation $\approx_n$ is transitive.  \qed
\end{enumerate}

\begin{Proposition} Assume $M$ and $N$ are sets and
\\[0.2cm]
\hspace*{1.3cm}
$f : M \rightarrow N$
\\[0.2cm]
is a function mapping $M$ to $N$.  If we define the  relation $R_f \subseteq M \times M$ as
\\[0.2cm]
\hspace*{1.3cm}
$R_f := \bigl\{ \pair(x,y) \in M \times M \mid f(x) = f(y) \bigr\}$,
\\[0.2cm]
then $R_f$ is an equivalence relation.  We call $R_f$ the equivalence relation \emph{induced}
by $f$.
\end{Proposition}

\noindent
\textbf{Proof}: We have to prove that $R_f$ is reflexive on $M$, symmetric, and transitive.
\begin{enumerate}
\item We have
      \\[0.2cm]
      \hspace*{1.3cm}
      $\forall x \in M: f(x) = f(x)$.
      \\[0.2cm]
      Therefore
      \\[0.2cm]
      \hspace*{1.3cm}
      $\forall x \in M: \pair(x,x) \in R_f$,
      \\[0.2cm]
      showing that $R_f$ is reflexive.
\item In order to establish $R_f$ as symmetric, we have to prove
      \\[0.2cm]
      \hspace*{1.3cm}
      $\forall x,y \in M: (\pair(x,y) \in R_f \rightarrow \pair(y,x) \in R_f)$.
      \\[0.2cm]
      Let us  therefore assume $\pair(x,y) \in R_f$.  According to the definition of $R_f$ we conclude
      \\[0.2cm]
      \hspace*{1.3cm}
      $f(x) = f(y)$.
      \\[0.2cm]
      From this 
      \\[0.2cm]
      \hspace*{1.3cm}
      $f(y) = f(x)$
      \\[0.2cm]
      is immediate and according to the definition of $R_f$ this implies 
      \\[0.2cm]
      \hspace*{1.3cm}
      $\pair(y,x) \in R_f$.
\item To establish $R_f$ as transitive, we have to prove
      \\[0.2cm]
      \hspace*{1.3cm}
      $\forall x,y,z \in M: \bigl(\pair(x,y) \in R_f \wedge \pair(y,z) \in R_f \rightarrow \pair(x,z)\bigr)$.
      \\[0.2cm]
      Therefore, assume 
      \\[0.2cm]
      \hspace*{1.3cm}
      $\pair(x,y) \in R_f \wedge \pair(y,z) \in R_f$.
      \\[0.2cm]
      According to the definition of $R_f$ we have 
      \\[0.2cm]
      \hspace*{1.3cm}
      $f(x) = f(y) \wedge f(y) = f(z)$.
      \\[0.2cm]
      But then 
      \\[0.2cm]
      \hspace*{1.3cm}
      $f(x) = f(z)$.
      \\[0.2cm]
      This is the same as 
      \\[0.2cm]
      \hspace*{1.3cm}
      $\pair(x,z) \in R_f$.  \qed
\end{enumerate}

\example
Take  $H$ as the set of all human beings and take $N$ as the set of all nations.
To simplify the discussion, let us assume that every human being has exactly one citizenship.
Then we can define a function
\\[0.2cm]
\hspace*{1.3cm}
$\textsl{cs}: H \rightarrow N$
\\[0.2cm]
such that for every human $x$,  $\textsl{cs}(x)$ is the citizenship of $x$.  
In the equivalence relation $R_{\textsl{cs}}$ induced by the function $\textsl{cs}$, all those human
beings are equivalent that have the same citizenship.


\begin{Definition}[Equivalence Class]
If $R$ is an  equivalence relation on  $M$, we define 
the set $[x]_R$ for all  $x \in M$ as follows: 
\\[0.2cm]
\hspace*{1.3cm}
$[x]_R \;:=\; \bigl\{ y \in M \mid x \mathop{R} y \bigr\}$ \qquad
(Here,  $x R y$ is used as abbreviation for $\pair(x, y) \in R$.) 
\\[0.2cm]
The set  $[x]_R$ is called the \emph{equivalence class generated by $x$}.
\end{Definition}

\begin{Proposition} \label{prop:14}
If $R \subseteq M \times M$ is an equivalence relation, then we have: 
\begin{enumerate}
\item $\forall x \el M \colon x \el [x]_R$
\item $\forall x, y \el M \colon \bigl(x \mathop{R} y \rightarrow [x]_R = [y]_R\bigr)$
\item $\forall x, y \el M \colon \bigl(\neg x \mathop{R} y \rightarrow [x]_R \cap [y]_R = \emptyset\bigr)$
\end{enumerate}
\end{Proposition}

\noindent
\textbf{Remark}: For any $x,y\el M$ we either have $x \mathop{R} y$ or
 $\neg (x \mathop{R} y)$.  
Therefore, the equivalence classes generated by two elements $x$ and $y$ are either
identical or they share no elements:
\\[0.2cm]
\hspace*{1.3cm}
$\forall x, y \el M: \bigl([x]_R = [y]_R \vee [x]_R \cap [y]_R = \emptyset\bigr)$.
\pagebreak

\noindent
\textbf{Proof} of proposition \ref{prop:14}:  
\begin{enumerate}
\item We have
      \\[0.2cm]
      \hspace*{1.3cm}
      $   x \el [x]_R \;\leftrightarrow\;
          x \in \bigl\{ y \el M \mid x \mathop{R} y \bigr\} \;\leftrightarrow\;
          x \mathop{R} x
      $.
      \\[0.2cm]
      Now $x \mathop{R} x$ is always true since $R$ is an equivalence
      relation and therefore has to be reflexive.  This shows
      \\[0.2cm]
      \hspace*{1.3cm}
      $\forall x \el M \colon x \el [x]_R$.
\item Assume $x \mathop{R} y$.  In  order to prove $[x]_R = [y]_R$ we have to show both
      $[x]_R \subseteq [y]_R$ and $[y]_R \subseteq [x]_R$.

      We start with the proof of  $[x]_R \subseteq [y]_R$.  Assume 
      $u \el [x]_R$.  According to the definition of $[x]_R$ this implies 
      $x \mathop{R} u$.  From $x \mathop{R} y$, since $R$ is symmetric, we have
      $y \mathop{R} x$.  Now since $R$ is transitive, from $y \mathop{R} x$ and $x
      \mathop{R} u$ 
      we conclude  $y \mathop{R} u$ gilt.  According to the definition of  $[y]_R$ we
      therefore have  $u \el [y]_R$.   This shows $[x]_R \subseteq [y]_R$.

      We show $[y]_R \subseteq [x]_R$ next and assume $u \el [y]_R$.  Then we have
      $y \mathop{R} u$.  From  $x \mathop{R} y$ and $y \mathop{R} u$ we conclude 
      $x \mathop{R} u$, as $R$ is transitive.  This implies $u \el [x]_R$ and therefore we
      have proven $[y]_R \subseteq [x]_R$.
\item Assume $\neg (x \mathop{R} y)$.  In order to prove  $[x]_R \cap [y]_R = \emptyset$
      we assume that there exists a $z$ such that $z \el [x]_R \cap [y]_R$.  
      We will show that this assumption must lead to a contradiction.
  
      So we start with  $z \el[x]_R$ and $z \el [y]_R$.  According to the definition of
       $[x]_R$ and $[y]_R$ we must have
      \\[0.2cm]
      \hspace*{1.3cm}      
      $x \mathop{R} z$ \quad and \quad $y \mathop{R} z$.
      \\[0.2cm]
      Since $R$ is symmetric we can turn  $y \mathop{R} z$ around and have
      \\[0.2cm]
      \hspace*{1.3cm}      
      $x \mathop{R} z$ \quad and \quad $z \mathop{R} y$.
      \\[0.2cm]
      As $R$ is transitive we conclude $x R y$, contradiction the assumption $\neg (x \mathop{R} y)$.
      Therefore, there can be no $z$ such that $z \el [x]_R \cap [y]_R$ holds.
      As a consequence, the set  $[x]_R \cap [y]_R$ must be empty.
      \hspace*{\fill} $\Box$
\end{enumerate}

\begin{Definition}[Partition] 
Let  ${\cal P} \subseteq 2^M$ be a set of subsets of $M$.  We call  ${\cal P}$ a
\emph{partition} of  $M$ iff the following is true:
\begin{enumerate}
\item $\forall x \el M : \exists K \el {\cal P} : x \el K$, \hspace*{\fill}  (\emph{completeness property})

      for every element from $M$ there is a set in ${\cal P}$ containing this element.
\item $\forall K, L \el {\cal P} : \bigl(K \cap L =\emptyset \vee K = L\bigr)$, \hspace*{\fill}
      (\emph{separation property})
      
      two sets from ${\cal P}$ are either equal or they do not share any elements.
\end{enumerate}
\end{Definition}

\noindent
\textbf{Remark}:
The last  proposition shows that for every equivalence relation $R$ on a set $M$, 
the set
\\[0.2cm]
\hspace*{1.3cm}
$\bigl\{ [x]_R \mid x \in M \bigr\}$
\\[0.2cm]
of equivalence classes generated by $R$ is a partition of $M$.  
Next, we will show that this can be turned around: If we have a partition $\mathcal{P}$ on
$M$, then this partition gives rise to an equivalence relation on $M$.

\begin{Proposition} 
Assume  $M$ is a set and $\mathcal{P} \subseteq 2^M$ is a partition of $M$.  Define the
relation $R$ as follows:
\\[0.2cm]
\hspace*{1.3cm}
$R := \bigl\{ \pair(x,y) \in M \times M \mid \exists K \in \mathcal{P}: \bigl(x \in K \wedge y \in K\bigr) \bigr\}$.
\\[0.2cm]
Then $R$ is an equivalence relation on $M$.
\end{Proposition}

\noindent
\textbf{Proof}: We have to show that $R$ is reflexive on $M$, symmetric and
transitive.
\begin{enumerate}
\item In order to show that $R$ is reflexive on $M$ we have to show
      \\[0.2cm]
      \hspace*{1.3cm}
      $\forall x \in M: x \mathop{R} x$.
      \\[0.2cm]
      According to the definition of $R$, this is the same as 
      \\[0.2cm]
      \hspace*{1.3cm}
      $\forall x \in M: \exists K \in \mathcal{P}: \bigl(x \in K \wedge x \in K\bigr)$
      \\[0.2cm]
      However, the latter is an immediate consequence of the completeness property of $\mathcal{P}$
      \\[0.2cm]
      \hspace*{1.3cm}
      $\forall x \in M: \exists K \in \mathcal{P}: x \in K$.
\item In order to show that $R$ is symmetric we have to show
      \\[0.2cm]
      \hspace*{1.3cm}
      $\forall x, y \in M:\bigl( x \mathop{R} y \rightarrow y \mathop{R} x\bigr)$.
      \\[0.2cm]
      Let us assume that 
      \\[0.2cm]
      \hspace*{1.3cm}
      $x \mathop{R} y$ 
      \\[0.2cm]
      holds.  From the definition of $R$ this is the same as
      \\[0.2cm]
      \hspace*{1.3cm}
      $\exists K \in \mathcal{P}: \bigl(x \in K \wedge y \in K\bigr)$.
      \\[0.2cm]
      Now this formula can be rewritten to
      \\[0.2cm]
      \hspace*{1.3cm}
      $\exists K \in \mathcal{P}: \bigl(y \in K \wedge x \in K\bigr)$
      \\[0.2cm]
      and then the definition of $R$ shows
      \\[0.2cm]
      \hspace*{1.3cm}
      $y \mathop{R} x$.
\item In order to show that $R$ is transitive we have to prove 
      \\[0.2cm]
      \hspace*{1.3cm}
      $\forall x,y,z \in M:\bigl( x \mathop{R} y \wedge y \mathop{R} z \rightarrow x \mathop{R} z\bigr)$.
      \\[0.2cm]
      Let us assume
      \\[0.2cm]
      \hspace*{1.3cm}
      $x \mathop{R} y \wedge y \mathop{R} z$.
      \\[0.2cm]
      According to the definition of $R$ this is the same as
      \\[0.2cm]
      \hspace*{1.3cm}
      $\exists K \in \mathcal{P}: \bigl(x \in K \wedge y \in K\bigr) \wedge 
       \exists L \in \mathcal{P}: \bigl(y \in L \wedge z \in L\bigr)$.
      \\[0.2cm]
      But then there are two sets $K,L\in\mathcal{P}$ such that
      \\[0.2cm]
      \hspace*{1.3cm}
      $x \in K \wedge y \in K \cap L \wedge z \in L$.
      \\[0.2cm]
      Therefore $K \cap L \not= \emptyset$ and the separation property of  $\mathcal{P}$
      shows that
      \\[0.2cm]
      \hspace*{1.3cm}
      $K = L$.
      \\[0.2cm]
      Then we have 
      \\[0.2cm]
      \hspace*{1.3cm}
      $\exists K \in \mathcal{P}: \bigl(x \in K \wedge z \in K\bigr)$
      \\[0.2cm]
      and by the definition of  $R$ this means
      \\[0.2cm]
      \hspace*{1.3cm}
      $x \mathop{R} z$. 
      \qed
\end{enumerate}
\pagebreak

\paragraph{Partial Order, Linear Order}
A  relation $R \subseteq M \times M$  is a 
\emph{partial order (in the sense of $\leq$) on $M$} iff $R$ is
\begin{enumerate}
\item reflexive on $M$,
\item anti-symmetric, and
\item transitive.
\end{enumerate}
The relation $R$ is a \emph{linear order on $M$} if it is a partial order and,
furthermore, we have
\\[0.2cm]
\hspace*{1.3cm} $\forall x \el M : \forall y \el M :\bigl( x\mathop{R}y \vee y \mathop{R} x\bigr)$.

The obvious example of a linear order is the relation $\leq$ on integers.
However, as we will soon see, there are many more interesting examples.

\example 
The divisibility relation $\mathop{\mathtt{div}}$ on natural numbers is defined as follows:
\\[0.2cm]
\hspace*{1.3cm}
$ \mathop{\mathtt{div}} := 
   \bigl\{ \pair(x,y) \in \mathbb{N} \times \mathbb{N} \mid \exists k \in \mathbb{N}: k \cdot x = y\bigr\}$.
\\[0.2cm] 
We prove that $\mathtt{div}$ is a partial order on $\mathbb{N}$ in the sense of $\leq$.
To this end, we prove that $\mathtt{div}$ is reflexive on $\mathbb{N}$, symmetric, and 
\begin{enumerate}
\item $\mathtt{div}$ is reflexive on $\mathbb{N}$:  We have to prove 
      \\[0.2cm]
      \hspace*{1.3cm}
      $\forall x \in \mathbb{N}: x \mathop{\mathtt{div}} x$.
      \\[0.2cm]
      According to the definition of $\mathop{\mathtt{div}}$ this is the same as
      \\[0.2cm]
      \hspace*{1.3cm}
      $\forall x \in \mathbb{N}: \exists k \in \mathbb{N}: k \cdot x = x$. 
      \\[0.2cm]
      Choosing  $k=1$ we have $k \cdot x = x$ for all $x \in \mathbb{N}$ and therefore
      $\mathtt{div}$ is reflexive on $\mathbb{N}$.
\item $\mathtt{div}$ is symmetric:  We have to prove 
      \\[0.2cm]
      \hspace*{1.3cm}
      $\forall x, y \in \mathbb{N}:\bigl( x \mathop{\mathtt{div}} y \wedge y \mathop{\mathtt{div}} x \rightarrow x = y\bigr)$
      \\[0.2cm] 
      Therefore, assume
      \\[0.2cm]
      \hspace*{1.3cm}
      $x \mathop{\mathtt{div}} y \wedge y \mathop{\mathtt{div}} x$.
      \\[0.2cm]
      We have to show $x=y$.  According to the definition of $\mathop{\mathtt{div}}$, our
      assumption is equivalent to
      \\[0.2cm]
      \hspace*{1.3cm}
      $\bigl(\exists k_1 \in \mathbb{N}: k_1 \cdot x = y \bigr) \wedge
       \bigl(\exists k_2 \in \mathbb{N}: k_2 \cdot y = x \bigr)$ 
      \\[0.2cm]
      Therefore we have natural numbers  $k_1$ and $k_2$ such that
      \\[0.2cm]
      \hspace*{1.3cm}
      $k_1 \cdot x = y \wedge k_2 \cdot y = x$.
      \\[0.2cm]
      Substituting these equations into each other we arrive at
      \\[0.2cm]
      \hspace*{1.3cm}
      $k_1 \cdot k_2 \cdot y = y$ \quad and \quad
      $k_2 \cdot k_1 \cdot x = x$.
      \\[0.2cm] 
      But then we must have
      \\[0.2cm]
      \hspace*{1.3cm}
      $k_1 \cdot k_2 = 1 \;\vee\; (x = 0 \wedge y = 0)$.
      \\[0.2cm]
      From $k_1 \cdot k_2 = 1$ we immediately conclude  $k_1 = 1$ and $k_2 = 1$, as both
      $k_1$ and $k_2$ are natural numbers.  Then $x = y$ is immediate.
      If we have both $x = 0$ and $y = 0$, we again have  $x = y$.
      So in any case, $x = y$ and this had to be shown.
      \pagebreak
\item $\mathtt{div}$ is transitive: We have to show
      \\[0.2cm]
      \hspace*{1.3cm}
      $\forall x, y, z \in \mathbb{N}:\bigl( x \mathop{\mathtt{div}} y \wedge y \mathop{\mathtt{div}} z \rightarrow x \mathop{\mathtt{div}} z\bigr)$
      \\[0.2cm] 
      Let us assume 
      \\[0.2cm]
      \hspace*{1.3cm}
      $x \mathop{\mathtt{div}} y \wedge y \mathop{\mathtt{div}} z$.
      \\[0.2cm]
      We have to show $x \mathop{\mathtt{div}} z$.  According to the definition of $\mathop{\mathtt{div}}$
      this means
      \\[0.2cm]
      \hspace*{1.3cm}
      $\bigl(\exists k_1 \in \mathbb{N}: k_1 \cdot x = y \bigr) \wedge
       \bigl(\exists k_2 \in \mathbb{N}: k_2 \cdot y = z \bigr)$. 
      \\[0.2cm]
      Then there are natural numbers $k_1$ and $k_2$ such that
      \\[0.2cm]
      \hspace*{1.3cm}
      $k_1 \cdot x = y \wedge k_2 \cdot y = z$.
      \\[0.2cm]
      Substituting the first equation into the second we get
      \\[0.2cm]
      \hspace*{1.3cm}
      $k_2 \cdot k_1 \cdot x = z$.
      \\[0.2cm] 
      Setting  $k_3 := k_2 \cdot k_1$ we have  $k_3 \cdot x = z$ and therefore we conclude
      \\[0.2cm]
      \hspace*{1.3cm}
      $x \mathop{\mathtt{div}} z$.
\end{enumerate}
The relation  $\mathtt{div}$ is not a linear order on $\mathbb{N}$:  For example, we
neither have $2 \mathop{\mathtt{div}} 3$ nor $3 \mathop{\mathtt{div}} 2$.  \exend

\exercise
Define the relation  $\leq$ 
as follows: 
\\[0.2cm]
\hspace*{1.3cm}
$\leq := \bigl\{ \pair(x,y) \in \mathbb{N} \times \mathbb{N} \mid \exists k \in \mathbb{N}: x + k = y \bigr\}$.
\\[0.2cm]
Prove that $\leq$ is a linear order on  $\mathbb{N}$.
\exend

\exercise
The inclusions relation $\subseteq$ is defined on  the power set $2^\mathbb{N}$ as follows:
\\[0.2cm]
\hspace*{1.3cm}
$\subseteq := 
\bigl\{ \pair(A,B) \in 2^\mathbb{N} \times 2^\mathbb{N}\mid \exists C \in 2^\mathbb{N}: A \cup C = B \bigr\}$
\\[0.2cm]
Prove that  $\subseteq$ is a partial order on $2^\mathbb{N}$ and show that it is not a
linear order.
\exend
\next

\noindent
Our excursion into set theory ends here.  More details can be found in the literature.
A good starting point is the book ``Set Theory and Related Topics'' by Seymour Lipshutz
\cite{lipschutz98}.  This books is also notable for its large number of exercises, most of
them with solutions.

%%% Local Variables: 
%%% mode: latex
%%% TeX-master: "logic"
%%% End: 

% LocalWords:  Ariane arnold Therac Scud nachumd abel lisa Georg de Fermat
