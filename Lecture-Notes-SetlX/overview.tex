\documentclass{article}
\usepackage{ngerman}
\usepackage[latin1]{inputenc}
\usepackage{a4wide}
\usepackage{amssymb}
\usepackage{textcomp}
\usepackage{fancyvrb}
\usepackage{alltt}
\usepackage{fleqn}
\usepackage{epsfig}
\usepackage{theorem}

\newcommand{\next}{\vspace*{0.2cm}

\noindent}

\newcommand{\exercise}{\vspace*{0.1cm}
\stepcounter{aufgabe}

\noindent
\textbf{Aufgabe \arabic{aufgabe}}: }

\newcommand{\solution}{\vspace*{0.1cm}

\noindent
\textbf{L\"{o}sung}: }

\newcommand{\example}{\vspace*{0.2cm}

\noindent
\textbf{Beispiel}: \ }

\newcommand{\el}{\!\in\!}
\newcommand{\verum}{\top}
\newcommand{\falsum}{\bot}
\newcommand{\dx}{\,\textrm{d}}
\newcommand{\schluss}[2]{\frac{\displaystyle\quad \rule[-6pt]{0pt}{12pt}#1 \quad}{\displaystyle\quad \rule{0pt}{10pt}#2 \quad}}
\newcommand{\vschlus}[1]{{\displaystyle\rule[-6pt]{0pt}{12pt} \atop \rule{0pt}{10pt}#1}}
\newcommand{\bruch}[2]{\frac{\displaystyle#1}{\displaystyle #2}}
\newcommand{\folge}[1]{\bigl(#1\bigr)_{n\in\mathbb{N}}}
\def\pair(#1,#2){\langle #1, #2 \rangle}
\newcommand{\qed}{\hspace*{\fill} $\Box$}
\newcommand{\var}{\mathrm{Var}}

{\theorembodyfont{\rm}
\newtheorem{Definition}{Definition}
\newtheorem{Satz}[Definition]{Satz}
\newtheorem{Lemma}[Definition]{Lemma}
}

\newcounter{aufgabe}

\title{Informatik-I \\
       {\Large Logik und Mengenlehre} \\
{\large \"{u}bersicht: Formeln und Definitionen}}
\author{Karl Stroetmann}

\begin{document}
\maketitle
\section{Mengen}
Mengen werden definiert durch das Pr\"{a}dikats-Zeichen ``$\in$'' 
\\[0.1cm]
\hspace*{1.3cm}
Schreibweise: $x \in M$ \quad (gelesen $x$ Element der Menge $M$)
\\[0.1cm] 
\textbf{Extensionalit\"{a}ts-Axiom}: \quad $M=N \;\leftrightarrow\; \forall x: \bigl( x \in M \leftrightarrow x \in N)$
Mengen-Bildung
\begin{enumerate}
\item Explizite Aufz\"{a}hlung aller Elemente 
      \\[0.1cm]
      \hspace*{1.3cm}
      Beispiel:\quad  $M := \{1,2,3\}$.
\item Menge der nat\"{u}rlichen Zahlen 
      \\[0.1cm]
      \hspace*{1.3cm}
      $\mathbb{N} := \{0,1,2,3, \cdots \}$
\item Auswahl-Prinzip 
      \\[0.1cm]
      \hspace*{1.3cm}
      $M := \{ x \in N \;|\; p(x) \}$
\item Potenz-Menge 
      \\[0.1cm]
      \hspace*{1.3cm}
      $2^M: = \{ x \;|\; x \subseteq M \}$
\item Vereinigungs-Menge 
      \\[0.1cm]
      \hspace*{1.3cm}
      $M \cup N := \{ x | x \in M \vee x \in N \}$
      \\[0.1cm]
      Verallgemeinerung:  Sei $M$ Menge von Mengen
      \\[0.1cm]
      \hspace*{1.3cm}
      $\bigcup M :=  \{ y \;|\; \exists x \in M: y \in x \}$.
\item Schnitt-Menge 
      \\[0.1cm]
      \hspace*{1.3cm}
      $M \cap N := \{ x | x \in M \wedge x \in N \}$
      \\[0.1cm]
      Verallgemeinerung:  Sei $M$ Menge von Mengen
      \\[0.1cm]
      \hspace*{1.3cm}
      $\bigcap M :=  \{ y \;|\; \forall x \in M: y \in x \}$.
\item Differenz-Menge \\[0.1cm]
      \hspace*{1.3cm} $M \backslash N := \{ x \mid x \in M \wedge x \not\in N \}$. 
      \\[0.1cm]
      \hspace*{1.3cm}
      Schreibweise: $x \not\in N$ steht f\"{u}r $\neg\, x \in N$.
\item Bild-Mengen 

      Seien $M$, $N$ Mengen, $f:M \rightarrow N$ Funktion 
      \\[0.1cm]
      \hspace*{1.3cm}
      $f(M) := \{ y \;|\; \exists x \in M: y = f(x) \}$

      Alternative Schreibweise: 
      \\[0.1cm]
      \hspace*{1.3cm}
      $f(M) := \bigl\{ f(x) \;|\; x \in M \}$
\item Geordnetes Paar: $\pair(x,y)$ 
      \begin{enumerate}
      \item $x$ ist erste  Komponente von $\pair(x,y)$
      \item $x$ ist zweite Komponente von $\pair(x,y)$
      \end{enumerate}
      Paare sind genau dann gleich, wenn Sie dieselben Komponenten haben: 
      \\[0.1cm]
      \hspace*{1.3cm} 
      $\pair(x,y) = \pair(u,v) \;\leftrightarrow\; x = u \wedge y = v$
\item kartesisches Produkt zweier Mengen 
      \\[0.1cm]
      \hspace*{1.3cm}
      $M \times N := \bigl\{ \pair(x,y) \;|\; x \in M \wedge y \in N \}$
\item Verallgemeninerung des geordneten Paares: $n$-Tupel: \quad $\langle x_1, x_2, \cdots, x_n \rangle$

      Liste aus $n$ Elementen.  Alternative Schreibweise: $[x_1,x_2,\cdots,x_n]$
\item Allgemeines kartesisches Produkt 
      \\[0.1cm]
      \hspace*{1.3cm}
      $M_1 \times \cdots \times M_n := \bigl\{ \langle x_1, \cdots, x_n \rangle \;\big|\; x_1 \in M_1 \wedge \cdots \wedge x_n \in M_n \bigr\}$
\end{enumerate}

\noindent
Mengen-Algebra: Rechenregeln f\"{u}r das Arbeiten mit Mengen 
\\[0.3cm]
$\begin{array}{rlcl}
\quad 1. & M \cup \emptyset = M         & \hspace*{0.1cm} & M \cap \emptyset = \emptyset \\
2. & M \cup M = M         & \hspace*{0.1cm} & M \cap M = M          \\
3. & M \cup N = N \cup M  &  & M \cap N = N \cap M  \\
4. & (K \cup M) \cup N = K \cup (M \cup N) &  & (K \cap M) \cap N = K \cap (M \cap N) \\
5. & (K \cup M) \cap N = (K \cap N) \cup (M \cap N) &  & (K \cap M) \cup N = (K \cup N) \cap (M \cup N)  \\
6. & M \backslash \emptyset = M & & M \backslash M = \emptyset \\
7. & K \backslash (M \cup N) = (K \backslash M) \cap (K \backslash N) &&
     K \backslash (M \cap N) = (K \backslash M) \cup (K \backslash N) \\
8. & (K \cup M) \backslash N = (K \backslash N) \cup (M \backslash N) &&
     (K \cap M) \backslash N = (K \backslash N) \cap (M \backslash N) \\
9. & K \backslash (M \backslash N) = (K \backslash M) \cup (K \cap N) &&
     (K \backslash M) \backslash N = K \backslash (M \cup N) \\
10. & M \cup (N \backslash M) = M \cup N &&
      M \cap (N \backslash M) = \emptyset  \\
11. & M \cup (M \cap N) = M  &&
      M \cap (M \cup N) = M 

\end{array}$

\noindent
\begin{Definition}[Bin\"{a}re Relation] \hspace*{\fill} \\[-0.5cm]
  \begin{enumerate}
  \item $M$, $N$: Mengen
  \item $R \subseteq M \times N$
  \end{enumerate}
  Dann ist $R$ bin\"{a}re Relation.
  \begin{enumerate}
  \item Definitions-Bereich: \quad $\textsl{dom}(R) := \{ x \mid \exists y \in N \colon \langle x, y \rangle \in R \}$
  \item Werte-Bereich: \hspace*{1cm} $\textsl{rng}(R) := \{ y \mid \exists x \in M \colon \langle x, y \rangle \in R\}$ 
  \end{enumerate} 
\end{Definition}

\begin{Definition}[rechts-eindeutig,links-eindeutig] \hspace*{\fill} \\[-0.5cm]
  \begin{enumerate}
  \item $R \subseteq M \times N$ \emph{rechts-eindeutig} g.d.w. \\[0.1cm]
        \hspace*{1.3cm} 
        $\forall x \colon \forall y_1 \colon \forall y_2 \colon \bigl(\langle x, y_1 \rangle \in R \wedge \langle x, y_2 \rangle \in R \rightarrow y_1 = y_2\bigr)$
  \item $R \subseteq M \times N$ \emph{links-eindeutig} g.d.w.  \\[0.1cm]
        \hspace*{1.3cm} 
        $\forall y \colon \forall x_1 \colon \forall x_2 \colon \bigl(\langle x_1, y \rangle \in R \wedge \langle x_2, y \rangle \in R \rightarrow x_1 = x_2\bigr)$
  \end{enumerate}
\end{Definition}

\begin{Definition}[links-total, rechts-total] \hspace*{\fill} \\[-0.5cm]
  \begin{enumerate}
  \item $R \subseteq M \times N$ \emph{links-total auf $M$} g.d.w. \\[0.1cm]
        \hspace*{1.3cm} $\forall x \in M \colon \exists y \in N \colon \pair(x,y) \in R$ 
  \item $R \subseteq M \times N$  \emph{rechts-total auf $N$} g.d.w. \\[0.1cm]
        \hspace*{1.3cm} $\forall y \in N \colon \exists x \in M \colon \pair(x,y) \in R$
  \end{enumerate}
\end{Definition}

\begin{Definition}[Funktionale Relationen] \hspace*{\fill} \\[0.1cm]
 $R \subseteq M \times N$ funktional g.d.w. \\[0.1cm]
\hspace*{1.3cm} $R$ links-total auf $M$  und \\[0.1cm]
\hspace*{1.3cm} $R$ rechts-eindeutig \\[0.2cm]
Falls $R$ funktional definiere Funktion 
\\[0.1cm]
\hspace*{1.3cm} $f_R\colon M \rightarrow N$ \\[0.1cm]
\hspace*{1.3cm} $f_R(x) := y \;\stackrel{de\!f}{\Longleftrightarrow}\; \pair(x,y) \in R$
\end{Definition}


\begin{Definition}[$\textsl{graph}(f)$] \hspace*{\fill} \\[-0.5cm]
  \begin{tabbing}
    \textbf{Vor.:} \quad \=  $f:M \rightarrow N$ \quad Funktion \\[0.1cm]
    \textbf{Setze:} \>  $\textsl{graph}(f) := \bigl\{ \pair(x,f(x)) \mid  x\in M \bigr\}$   \\[0.1cm]
    \textbf{Dann:}  \>  $\textsl{graph}(f)$ funktionale Relation
  \end{tabbing}
\end{Definition}

\begin{Definition}[Bild von $X$ unter $R$] \hspace*{\fill} \\[-0.5cm]
  \begin{tabbing}
    \textbf{Geg.:}  \quad \= $R \subseteq M \times N$ bin\"{a}re Relation \\[0.1cm]
                    \>  $X \subseteq M$                               \\[0.1cm]
    \textbf{Setze:} \>  $R(X) := \{ y \mid \exists x \in X \colon \pair(x,y) \in R \}$. 
  \end{tabbing}
\end{Definition}

\begin{Definition}[Inverse Relation] \hspace*{\fill} \\[-0.5cm]
  \begin{tabbing}
    \textbf{Geg.:} \quad \=  $R \subseteq M \times N$ bin\"{a}re Relation \\[0.1cm]
    \textbf{Setze:} \> $R^{-1} := \bigl\{ \pair(y,x) \mid \pair(x,y) \in R  \bigr\}$. \\[0.1cm]
  \end{tabbing}
\end{Definition}

\begin{Definition}[Komposition von Relationen] \hspace*{\fill} \\[-0.5cm]
  \begin{tabbing}
  \textbf{Geg.:}  \quad \= $L$, $M$, $N$: Mengen \\[0.1cm]
                  \> $R \subseteq L \times M$  \\[0.1cm]
                  \> $Q \subseteq M \times N$  \\[0.1cm]
  \textbf{Setze:} \> $R \circ Q := \bigl\{ \pair(x,z) \mid \exists y \in M \colon \pair(x,y) \in R \wedge \pair(y,z) \in Q \bigr\}$ \\[0.1cm]
                  \>   $R \circ Q$: relationales Produkt von $R$ und $Q$, auch: Komposition von $R$ und $Q$
  \end{tabbing}
\end{Definition}
\begin{tabbing} 
Beispiel: \= $R$ Zuordnung \textsl{Name} zu \textsl{Telefon-Nr.}, \\[0.1cm]
          \> $Q$ Zuordnung \textsl{Telefon-Nr.} zu \textsl{Adresse}\\[0.1cm]
          \> $R \circ Q$ Zuordnung \textsl{Name} zu \textsl{Adresse}\\[0.1cm]
          \> $R = \bigl\{ \pair(\mathrm{Mayer}, \mathrm{1234}), \pair(\mathrm{Weber}, \mathrm{4567}), \cdots \bigr\}$  \\
          \> $Q = \bigl\{ \pair(\mathrm{1234}, \mathrm{Krumme\ Gasse}\;7), \pair(4567, \mathrm{Schlo\3allee}\; 3), \cdots \bigr\}$  \\
          \> $R \circ Q = \bigl\{ \pair(\textsl{Mayer}, \textsl{Krumme\ Gasse}\;7), \cdots \bigr\}$ 
\end{tabbing}


\begin{Satz}[Assoziativ-Gesetz f\"{u}r das relationale Produkt] \hspace*{\fill} \\[-0.5cm]
  \begin{tabbing}
    \textbf{Geg.:} \quad \= $K$, $L$, $M$, $N$: Mengen \\[0.1cm]
                  \> $R \subseteq K \times L$  \\[0.1cm]
                  \> $Q \subseteq L \times M$  \\[0.1cm]
                  \> $P \subseteq M \times N$  \\[0.1cm]
    \textbf{Beh.:} \> $\bigl(R \circ Q\bigr) \circ P = R \circ (Q \circ P)$
  \end{tabbing}
\end{Satz}


\begin{Satz}[Inverse eines relationalen Produkts] \hspace*{\fill} \\[-0.5cm]
  \begin{tabbing}
    \textbf{Geg.:} \quad \= $L$, $M$, $N$: Mengen \\[0.1cm]
                  \> $R \subseteq L \times M$  \\[0.1cm]
                  \> $Q \subseteq M \times N$  \\[0.1cm]
    \textbf{Beh.:} \> $\bigl(R \circ Q\bigr)^{-1}  = Q^{-1} \circ R^{-1}$
  \end{tabbing}
\end{Satz}

\begin{Satz}[Distributiv-Gesetz f\"{u}r relationale Produkte] \hspace*{\fill} \\[-0.5cm]
  \begin{tabbing}
    \textbf{Geg.:} \quad \= $L$, $M$, $N$: Mengen \\[0.1cm]
                   \> $R_1 \subseteq L \times M$  \\[0.1cm]
                   \> $R_2 \subseteq L \times M$  \\[0.1cm]
                   \> $Q \subseteq M \times N$  \\[0.1cm]
    \textbf{Beh.:} \> $\bigl(R_1 \cup R_2\bigr) \circ Q  = \bigl(R_1 \circ Q\bigr) \cup \bigl(R_2 \circ Q\bigr)$
  \end{tabbing}
\end{Satz}

\noindent
\textbf{Bemerkung}: Im allgemeinen gilt 
\\[0.1cm]
\hspace*{1.3cm}  $\bigl(R_1 \cap R_2\bigr) \circ Q \not= \bigl(R_1 \circ Q\bigr) \cap \bigl(R_2 \circ Q\bigr)$.

\section{Setl2}
Mengenbildung in \textsc{Setl2}:
\begin{enumerate}
\item Explizites Auflisten der Elemente 

      \texttt{\{ 1, 5, 7 \}}
\item Arithmetische Aufz\"{a}hlung

      \texttt{\{ 1 .. 100 \}}

      entspricht: \quad $\{ n \in \mathbb{N} : 1 \leq n \wedge n \leq 100 \}$
\item Arithmetische Aufz\"{a}hlung mit Angabe der Schrittweite

      \texttt{\{ 3, 8 .. 100 \}}

      entspricht: \quad $\{ 3 + n \cdot 5 : n \in \mathbb{N} \wedge 3 + n \cdot 5 \leq 100 \}$
\item Definition von Mengen durch Iteratoren (Bildmengen)
      
      Beispiel: Menge aller Produkte $n \cdot m$ mit $2 \leq n, m \leq 100$

      \texttt{\{ n * m : n in \{2..10\}, m in \{2..10\} \}}
\item Mengenbildung durch Auswahl

      Beispiel: Menge aller Teiler von $p$

      \texttt{\{ x : x in \{ 1 .. p \} | p mod x = 0 \}}

      Einfachere Schreibweise:

      \texttt{\{ x in \{1..p\} | p mod x = 0 \}}
\item Mengenbildung durch Iteratoren und zus\"{a}tzliche Auswahl

      Beispiel: Menge aller Produkte $m \cdot n$ f\"{u}r die gilt
      \\[0.1cm]
      \hspace*{1.3cm} $m + n = 16$

      \texttt{\{ n * m : n in \{ 0 .. 16 \}, m in \{ 0 .. 16 \} | m + n = 16 \}}

      allgemeine Form

      \texttt{M := \{ $\textsl{expr}(x_1,\cdots,x_n)$ : $x_1$ in $S_1$, $\cdots$, $x_n \in S_n$ | $\textsl{cond}(x_1,\cdots,x_n)$ \}}
      \begin{enumerate}
      \item $\textsl{expr}(x_1,\cdots,x_n)$: Ausdruck, in dem die Variablen $x_1$, $\cdots$ $x_n$ vorkommen.
      \item $S_1$, $\cdots$ $S_n$: Mengen
      \item $\textsl{cond}(x_1, \cdots, x_n)$: Ausdruck, dessen Auswertung ``\texttt{true}'' 
            oder ``\texttt{false}'' liefert.
      \item $M$: Menge, die alle Elemente  $\textsl{expr}(x_1,\cdots,x_n)$ 
            enth\"{a}lt, f\"{u}r die $x_i \in S_i$ gilt und $\textsl{cond}(x_1, \cdots, x_n)$
            den Wert ``\texttt{true}'' liefert.
      \end{enumerate}
\end{enumerate}

\subsection{Paare und Funktionen}
\begin{enumerate}
\item Paar: \quad \texttt{[ x, y ]} \quad statt \quad $\langle x, y \rangle$ 
\item Tupel: \quad \texttt{[ $x_1$, $x_2$, $\cdots$, $x_n$ ]} \quad statt \quad $\langle x_1, x_2, \cdots, x_n \rangle$ 
\item Definitions-Bereich einer Relation: \quad \texttt{domain(R)} \quad statt \quad $\textsl{dom}(R)$
\item Werte-Bereich einer Relation: \quad \texttt{range(R)} \quad statt \quad $\textsl{rng}(R)$
\item funktionale Relationen

      $R$ Relation, links-total und rechts-eindeutig: $R$ Funktion 

      dann: \quad $y = R(x)$ \quad g.d.w. \quad $[x,y] \in R$
\item Falls $\textsl{card}(\{ y | \pair(x,y) \in R \}\bigr) \not= 1$, dann 
      \\[0.1cm]
      \hspace*{1.3cm} \texttt{R(x) = <om>}

      \texttt{<om>}: gelesen $\Omega$, undefinierter Wert
\item Alternative: \texttt{R\{x\}}

      $R\{x\} := \{ y \mid \pair(x,y) \in R \}$
\end{enumerate}

\subsection{Tupel als ``\emph{geordnete Mengen}''}
\begin{enumerate}
\item Explizites Auflisten der Elemente 

      \texttt{[ 1, 5, 7 ]}
\item Arithmetische Aufz\"{a}hlung

      \texttt{[ 1 .. 100 ]}

      geordnete Liste mit den Elementen 1, 2, $\cdots$, bis $100$

\item Definition von Listen durch Iteratoren 
      
      Beispiel: Liste aller Produkte $n \cdot m$ mit $2 \leq n, m \leq 100$

      \texttt{[ [ n, m ] : n in [2..4], m in  [2..4] ]}

      liefert

      \texttt{[[2, 2], [2, 3], [2, 4], [3, 2], [3, 3], [3, 4], [4, 2], [4, 3], [4, 4]]}
      
      letzte Variable entspricht innerster Schleife
      
\item Listenbildung durch Auswahl

      Beispiel: Menge aller Teiler von $p$

      \texttt{[ x : x in [ 1 .. p ] | p mod x = 0 ]}

      Einfachere Schreibweise:

      \texttt{[ x in [1..p] | p mod x = 0 ]}
\end{enumerate}

\subsection{Mengen-Operatoren}
\begin{enumerate}
\item Summe aller Elemente einer Menge 
      \\[0.1cm]
      \hspace*{1.3cm}
      \texttt{+/ \{ 1, 2, 3 \} = 1 + 2 + 3}
\item Produkt aller Elemente einer Menge 
      \\[0.1cm]
      \hspace*{1.3cm}
      \texttt{*/ \{ 1, 2, 3 \} = 1 * 2 * 3}

      Berechnung der Fakult\"{a}t $n! = 1 * 2 * 3 * \cdots * (n-1) *n$

      \texttt{n\_fakultaet := */ \{1 .. n\}}
\item Allgemein: $\textsl{op}$ bin\"{a}rer Operator \quad $\mapsto$ \quad \textsl{op/} Mengen-Operator

\item Problem: \quad \texttt{+/{} = <om>}

      L\"{o}sung: \quad \texttt{x +/{} = x} \quad (meist $x = 0$)
\item \texttt{min/ S}: Minimum der Menge $S$
\item \texttt{max/ S}: Minimum der Menge $S$
\item \texttt{x from S;}

      Danach gilt:
      \begin{enumerate}
      \item \texttt{x} ist beliebiges Element von \texttt{S}
      \item \texttt{S} enth\"{a}lt \texttt{x} nicht mehr
      \end{enumerate}
\item \texttt{arb(S)}: beliebiges Element aus Menge \texttt{S}, Menge wird nicht ver\"{a}ndert
\item \texttt{x fromb L;} \quad (\texttt{L} List)

      Danach gilt:
      \begin{enumerate}
      \item \texttt{x} ist erstes Element von \texttt{L}
      \item erstes Element von \texttt{L} wird entfernt
      \end{enumerate}
\item \texttt{x frome L;} \quad (\texttt{L} List)

      Danach gilt:
      \begin{enumerate}
      \item \texttt{x} ist letztes Element von \texttt{L}
      \item erstes Element von \texttt{L} wird entfernt
      \end{enumerate}
\item Anzahl der Elemente einer Menge $S$ , L\"{a}nge einer Liste $L$

      \texttt{\#S = $\textsl{card}(S)$} \quad und \quad
      \texttt{\#[}$x_1,\cdots,x_n$\texttt{]} $ = n$
\item Indizierung von Elementen: 

      $L(k)$: \quad $k$-tes Element der Liste $L$
\end{enumerate}

\section{Aussagenlogik}    
\begin{itemize}
\item \textbf{Anwendung}:
  \begin{enumerate}
  \item Entwicklung digitaler Schaltungen
  \item Grundlage der Pr\"{a}dikaten-Logik
  \item kann zur L\"{o}sung kombinatorischer Probleme eingesetzt werden

        Beispiele: 8-Damen-Problem, Stundenplan-Erstellung
  \end{enumerate}
\item \textbf{Aufgaben}: 
  \begin{enumerate}
  \item Analyse der Verkn\"{u}pfung \emph{atomarer} Aussagen durch \emph{Junktoren}.  
  \item \textbf{Herleitungsbegriff}:
        Folgerung neuer Aussagen aus gegebenen Axiomen.
  \end{enumerate}
\item \textbf{Junktoren}:  ``\emph{und}'', ``\emph{oder}'', ``\emph{nicht}'', ``\emph{wenn $\cdots$, dann}'',
        ``\emph{genau dann, wenn}''.
\item \textbf{atomare} Aussagen
      \begin{enumerate}
      \item wahr oder falsch 
      \item enthalten keine Junktoren 
      \end{enumerate}
\end{itemize}
Beispiele f\"{u}r atomare Aussagen:
\begin{enumerate}
\item ``{\em Die Sonne scheint.}''
\item ``{\em Es regnet.}''
\item ``{\em Am Himmel ist ein Regenbogen.}''
\end{enumerate}
Beispiel f\"{u}r \textbf{zusammengesetzte} Aussage:
\\[0.3cm]
\hspace*{0.3cm} {\em Wenn die Sonne scheint und es regnet, dann ist ein Regenbogen am Himmel.} 
\\[0.3cm]
Beispiel f\"{u}r \textbf{Folgerung}: Aus den Axiomen
\begin{enumerate}
\item ``{\em Die Sonne scheint.}''
\item ``{\em Es regnet.}'' 
\item ``{\em Wenn die Sonne scheint und es regnet, dann ist ein Regenbogen am Himmel.} 
\end{enumerate}
folgt logisch die Behauptung \\[0.3cm]
\hspace*{1.3cm} ``{\em Am Himmel ist ein Regenbogen.}'' 
\\[0.3cm]
Verallgemeinerung von logischen Schl\"{u}ssen durch \textbf{Aussage-Variablen}
$$ \schluss{p \quad\quad q \quad\quad p \wedge q \rightarrow r}{r} $$
\textbf{Schluss-Regel} mit \textbf{Pr\"{a}missen}  ``$p$'', ``$q$'', ``$p \wedge q \rightarrow r$''
und \textbf{Konklusion} ``$r$''.
\vspace*{0.3cm}

\noindent
\textbf{Aufgabe:} Formalisieren Sie die folgende Schluss-Regel:
\begin{center}
\begin{minipage}[c]{8.2cm}
{\em  Wenn es regnet, ist die Stra\3e na\3.  Es regnet nicht.  Also ist die Stra\3e nicht na\3.}
\end{minipage}  
\end{center}

\noindent
\textbf{Herleitungsbegriff}: 
\\[0.3cm]
\hspace*{1.3cm}
 $M \vdash r$ \qquad wird gelesen als \qquad ``\emph{$M$ leitet $r$ her}''.
\\[0.3cm]
\textbf{Herleitung}: Aneinanderreihen von logischen Schl\"{u}ssen, Beispiel: 
\\[0.3cm]
\hspace*{1.3cm}
$\displaystyle \schluss{\schluss{p \quad\quad q \quad\quad p \wedge q \rightarrow r}{r} \qquad
  \schluss{\neg s \rightarrow \neg r}{r \rightarrow s}}{s}$
\\[0.3cm]
Herleitungs-Begriff \emph{syntaktisch}, Spielregeln zum Aufbau von Beweisen
\vspace*{0.3cm}

\emph{Folgerungs-Begriff}: inhaltliche Aussage
\\[0.3cm]
\hspace*{1.3cm} $M \models r$ \qquad ``\emph{$M$ folgt aus $r$}'' 
\\[0.3cm]
Interpretation: Immer wenn alle Formeln aus $M$ wahr sind, dann ist aus $r$ wahr.
\vspace*{0.3cm}

\noindent
\textbf{Ziel} der Aussagenlogik: Herleitungsbegriff mit folgenden Eigenschaften
\begin{enumerate}
\item Korrektheit: \\[0.3cm]
      \hspace*{1.3cm} Aus $M \vdash r$ folgt $M \models r$. 
\item Vollst\"{a}ndigkeit: \\[0.3cm]
      \hspace*{1.3cm} Aus $M \models r$ folgt $M \vdash r$. 
      \\[0.1cm]
      Wenn die Aussage $r$ aus $M$ folgt, dann soll $r$ auch aus der Menge $M$
      hergeleitet werden k\"{o}nnen.
\end{enumerate}
\vspace*{0.3cm}


\noindent
\textbf{Def.:} Aussagenlogische Formeln \\
\textbf{Geg.:} $\mathcal{P}$; Menge von  \emph{Aussage-Variablen} 
\\[0.3cm]
\hspace*{1.3cm} $\mathcal{P} = \{ p, q, r, \cdots \}$
\\[0.3cm]
\hspace*{1.3cm} Alphabet
\\[0.3cm]
\hspace*{1.3cm} $\mathcal{A} := \mathcal{P} \cup \bigl\{ \verum, \falsum, \neg, \vee, \wedge,
                   \rightarrow, \leftrightarrow, (, ) \bigr\}$
\\[0.3cm]
Induktive Definition der aussagenlogischen Formeln:
\begin{enumerate}
\item $\verum \in \mathcal{F}$ und $\mathtt{\falsum} \in \mathcal{F}$.

      Hier steht $\verum$ f\"{u}r die Formel, die immer wahr ist, w\"{a}hrend $\falsum$ f\"{u}r die 
      Formel steht, die immer falsch ist.  Die Formel $\verum$ tr\"{a}gt auch den Namen \emph{Verum},
      f\"{u}r $\falsum$ sagen wir auch \emph{Falsum}.
\item Ist $p \in \mathcal{P}$, so gilt auch $p \in \mathcal{F}$.
\item Ist $f \in \mathcal{F}$, so gilt auch $\neg f \in \mathcal{F}$.
\item Sind $f_1, f_2 \in \mathcal{F}$, so gilt auch
      \begin{enumerate}
      \item  $(f_1 \vee f_2) \in \mathcal{F}$,
      \item  $(f_1 \wedge f_2) \in \mathcal{F}$,
      \item  $(f_1 \rightarrow f_2) \in \mathcal{F}$,
      \item  $(f_1 \leftrightarrow f_2) \in \mathcal{F}$.
      \end{enumerate}
\end{enumerate}


\noindent
{\large Aussagenlogische \"{a}quivalenzen: (Boole'sche Algebra)} \\[0.3cm]
\hspace*{0.3cm} 
$\begin{array}{lll}
\models \neg \falsum \leftrightarrow \verum & \models \neg \verum \leftrightarrow \falsum &  \\[0.1cm]
 \models p \vee   \neg p \leftrightarrow \verum & \models p \wedge \neg p \leftrightarrow \falsum & \mbox{Tertium-non-Datur} \\[0.1cm]
 \models p \vee   \falsum \leftrightarrow p & \models p \wedge \verum  \leftrightarrow p & \mbox{Neutrales Element}\\[0.1cm]
 \models p \vee   \verum  \leftrightarrow \verum & \models p \wedge \falsum \leftrightarrow \falsum &  \\[0.1cm]
 \models p \wedge p \leftrightarrow p  & \models p \vee p \leftrightarrow p &  \mbox{Idempotenz} \\[0.1cm]
 \models p \wedge q \leftrightarrow q \wedge p & \models p \vee   q \leftrightarrow q \vee p & \mbox{Kommutativit\"{a}t} \\[0.1cm]
 \models (p \wedge q) \wedge r \leftrightarrow p \wedge (q \wedge r) & \models (p \vee   q) \vee r \leftrightarrow p \vee   (q \vee r)  &
 \mbox{Assoziativit\"{a}t} \\[0.1cm]
 \models \neg \neg p \leftrightarrow p & & \mbox{Elimination von $\neg \neg$} \\[0.1cm]
 \models p \wedge (p \vee q)   \leftrightarrow p & \models p \vee   (p \wedge q) \leftrightarrow p &  \mbox{Absorption} \\[0.1cm]
 \models p \wedge (q \vee r)   \leftrightarrow (p \wedge q) \vee   (p \wedge r) & 
 \models p \vee   (q \wedge r) \leftrightarrow (p \vee q)   \wedge (p \vee   r) & \mbox{Distributivit\"{a}t} \\[0.1cm]
 \models \neg (p \wedge q) \leftrightarrow  \neg p \vee   \neg q &  \models \neg (p \vee   q) \leftrightarrow  \neg p \wedge \neg q &
 \mbox{DeMorgan'sche Regeln}  \\[0.1cm]
 \models (p \rightarrow q) \leftrightarrow \neg p \vee q & &  \mbox{Elimination von $\rightarrow$} \\[0.1cm]
 \models (p \leftrightarrow q) \leftrightarrow (\neg p \vee q) \wedge (\neg q \vee p) & & \mbox{Elimination von $\leftrightarrow$}
\end{array}$ \\[0.3cm]

\begin{center}
{\Large Pr\"{a}dikaten-Logik}
\end{center}

\noindent
analysiert Aufbau \textsl{atomarer Formeln} aus \textsl{Pr\"{a}dikaten} und \textsl{Termen}

\begin{enumerate}
\item \emph{Terme}:  Bezeichnungen f\"{u}r Objekte.

      zusammengesetzte aus \emph{Variablen} und \emph{Funktions-Zeichen}: 
      \[ \textsl{vater}(x),\quad \textsl{mutter}(\textsl{issac}), \quad x+7, \quad \cdots \]
\item \emph{Pr\"{a}dikats-Zeichen} erm\"{o}glichen Aussagen \"{u}ber Objekte:
      \[ \textsl{istBruder}\bigl(\textsl{albert},
         \textsl{vater}(\textsl{bruno})\bigr),\quad x+7 < x*7,\quad n \in \mathbb{N},
         \quad \cdots \]
      keine Junktoren: \emph{atomare} Formeln.
\item Verkn\"{u}pfung atomare Formeln mit  aussagenlogische Junktoren liefern Formeln:
      \[ x > 1 \rightarrow x + 7 < x * 7 \]
\item \emph{Quantoren}: differenzieren Bedeutung von Variablen:
      \[ \forall x \in \mathbb{R}: \exists n \in \mathbb{N}: x < n \]
\end{enumerate}

\noindent
\textbf{Vorgehen}:
\begin{enumerate}
\item \textsl{Syntax}: Aufbau von Formeln
\item \textsl{Sematik}: Bedeutung von Formeln
\end{enumerate}

\begin{center}
{\Large  Syntax der Pr\"{a}dikaten-Logik}
\end{center}

\begin{Definition}[Signatur]

  Signatur: 4-Tupel \\[0.1cm]
  \hspace*{1.3cm} $\Sigma = \langle \mathcal{V}, \mathcal{F}, \mathcal{P}, \textsl{arity} \rangle$ 
  \begin{enumerate}
  \item $\mathcal{V}$:  Menge der Variablen
  \item $\mathcal{F}$:  Menge der Funktions-Zeichen
  \item $\mathcal{P}$:  Menge der Pr\"{a}dikats-Zeichen.
  \item $\textsl{arity}$: \emph{Stelligkeit} 
         
        $\textsl{arity}: \mathcal{F} \cup \mathcal{P} \rightarrow \mathbb{N}$. 

        $\textsl{arity}(f) = n$: \quad $f$ $n$-stellig
  \item  $\mathcal{V}$, $\mathcal{F}$, $\mathcal{P}$  paarweise disjunkt: \quad
         $\mathcal{V} \cap \mathcal{F} = \emptyset$, \quad $\mathcal{V} \cap \mathcal{P} = \emptyset$, \quad und \quad $\mathcal{F} \cap \mathcal{P} = \emptyset$.
  \end{enumerate}

\end{Definition}
\quad

\begin{Definition}[Terme] \hspace*{\fill} \\[-0.5cm]
  \begin{tabbing}
  \qquad \= \textbf{Geg.:} \quad \= $\Sigma = \langle \mathcal{V}, \mathcal{F}, \mathcal{P}, \textsl{arity}\rangle$: \quad \= Signatur \\[0.1cm]
         \> \textbf{Def.:}       \> $\mathcal{T}_\Sigma$:  \> Menge der $\Sigma$-Terme, induktive Definition    
  \end{tabbing}
  \begin{enumerate}
  \item $x \el \mathcal{T}_\Sigma$ falls $x \in \mathcal{V}$
  \item $f(t_1,\cdots,t_n) \el \mathcal{T}_\Sigma$ \quad falls
    \begin{enumerate}
    \item $f \in \mathcal{F}$ 
    \item $\textsl{arity}(f) = n$
    \item $t_1,\cdots,t_n \el \mathcal{T}_\Sigma$
    \end{enumerate}
  \item $c \el \mathcal{T}_\Sigma$ \quad falls $f \el \mathcal{F}$ und $\textsl{arity}(f) = 0$.
  \end{enumerate}
\end{Definition}

\noindent
\textbf{Beispiel}: Gegeben
\begin{enumerate}
\item $\mathcal{V} = \{ x, y, z \}$: \quad Menge der Variablen
\item $\mathcal{F} = \{ 0, 1, \mathtt{+}, \mathtt{-}, \mathtt{*} \}$: \quad Menge der Funktions-Zeichen
\item $\mathcal{P} = \{\mathtt{=}, \leq\}$: \quad  Menge der Pr\"{a}dikats-Zeichen    

\item   $\textsl{arity} = \bigl\{ \pair(0,0), \pair(1,0), \pair(\mathtt{+},2), \pair(\mathtt{-},2), \pair(\mathtt{*},2)
\bigr\}$: Stelligkeit
\item $\Sigma_\mathrm{arith} := \langle \mathcal{V}, \mathcal{F}, \mathcal{P}, \textsl{arity} \rangle$:
      \quad Signatur  
\end{enumerate}
Beispiele f\"{u}r $\Sigma_{\mathrm{arith}}$-Terme:
\begin{enumerate}
\item $x, y, z \in \mathcal{T}_{\Sigma_{\mathrm{arith}}}$
\item $0, 1 \in \mathcal{T}_{\Sigma_{\mathrm{arith}}}$
\item $\mathtt{+}(0,x) \in \mathcal{T}_{\Sigma_{\mathrm{arith}}}$
\item $\mathtt{*}(\mathtt{+}(0,x),1) \in \mathcal{T}_{\Sigma_{\mathrm{arith}}}$
\end{enumerate}

\begin{Definition}[Atomare Formeln] (Keine Junktoren, keine Quantoren)
  \begin{tabbing}
  \qquad \= \textbf{Geg.:} \quad \= $\Sigma = \langle \mathcal{V}, \mathcal{F}, \mathcal{P}, \textsl{arity} \rangle$ \quad \= Signatur \\[0.2cm]
         \> \textbf{Def.:}       \>  $p(t_1,\cdots,t_n) \in \mathcal{A}_\Sigma$ \quad \> atomaren $\Sigma$-Formel falls\\[0.1cm]
         \>                      \>             \> $p \in \mathcal{P}$, \\[0.1cm]
         \>                      \>             \> $\textsl{arity}(p) = n$, \\[0.1cm]
         \>                      \>             \> $t_i \in \mathcal{T}_{\Sigma_{\mathrm{arith}}}$ f\"{u}r $i=1,\cdots,n$. \\[0.1cm]
 \end{tabbing}
\end{Definition}

\end{document}

%%% Local Variables: 
%%% mode: latex
%%% TeX-master: "overview"
%%% End: 
