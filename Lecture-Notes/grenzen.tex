\chapter{Grenzen der Berechenbarkeit}
In jeder Disziplin der Wissenschaft wird die Frage gestellt, welche Grenzen die
verwendeten Methoden haben.   Wir wollen daher in diesem Kapitel beispielhaft ein Problem
untersuchen, bei dem die Informatik an ihre Grenzen st\"{o}\ss{}t.  Es handelt sich um das
\href{http://de.wikipedia.org/wiki/Halteproblem}{Halte-Problem}.  

\section{Das Halte-Problem}
Das Halte-Problem ist die Frage, ob eine gegebene Funktion f\"{u}r eine bestimmte Eingabe 
terminiert.  Bevor wir formal beweisen, dass das Halte-Problem im Allgemeinen unl\"{o}sbar ist, wollen 
wir versuchen, anschaulich zu verstehen, warum dieses Problem schwer sein muss.  Dieser informalen
Betrachtung des Halte-Problems ist der n\"{a}chste Abschnitt gewidmet.  Im Anschluss an diesen Abschluss
zeigen wir dann die Unl\"{o}sbarkeit des Halte-Problems. 

\subsection{Informale Betrachtungen zum Halte-Problem}


\begin{figure}[!ht]
\centering
\begin{Verbatim}[ frame         = lines, 
                  framesep      = 0.3cm, 
                  firstnumber   = 1,
                  labelposition = bottomline,
                  numbers       = left,
                  numbersep     = -0.2cm,
                  xleftmargin   = 0.8cm,
                  xrightmargin  = 0.8cm
                ]
    findCounterExample := procedure(n) {
        legendre := procedure(n) {
            k := n * n + 1;
            while (k < (n + 1) ** 2) {
                if (isPrime(k)) {
                    print("$n$**2 < $k$ < $n+1$**2");
                    return true;
                }
                k += 1;
            }
            return false;
        };
        while (true) {
           if (legendre(n)) {
               n := n + 1;
           } else {
               print("Legendre was wrong, no prime between $n**2$ and $(n+1)**2$!");
               return;
           }
        }
    };
\end{Verbatim} 
\vspace*{-0.3cm}
\caption{Eine Funktion zur \"{U}berpr\"{u}fung der Vermutung von Legendre.}
\label{fig:legendre.stlx}
\end{figure}

Um zu verstehen, warum das Halte-Problem schwer ist, betrachten wir  
das in Abbildung \ref{fig:legendre.stlx} gezeigte Programm. 
Dieses Programm ist dazu gedacht, die
\href{http://de.wikipedia.org/wiki/Legendresche_Vermutung}{\emph{Legendresche Vermutung}} zu
\"{u}berpr\"{u}fen.  Der franz\"{o}sische 
Mathematiker  \href{http://de.wikipedia.org/wiki/Adrien-Marie_Legendre}{Adrien-Marie Legendre} 
(1752 --- 1833) hatte vor etwa 200 Jahren die Vermutung 
ausgesprochen, dass zwischen zwei positiven Quadratzahlen immer eine Primzahl liegt.  Die Frage, ob diese
Vermutung richtig ist, ist auch Heute noch unbeantwortet.  Die in Abbildung \ref{fig:legendre.stlx}
definierte Funktion $\texttt{legendre}(n)$ \"{u}berpr\"{u}ft f\"{u}r eine gegebene positive nat\"{u}rliche Zahl $n$,
ob zwischen $n^2$ und $(n+1)^2$ eine Primzahl liegt.  Falls dies, wie von Legendre vorhergesagt, der
Fall ist, gibt die Funktion als Ergebnis \texttt{true} zur\"{u}ck, andernfalls wird \texttt{false}
zur\"{u}ck gegeben.

Abbildung \ref{fig:legendre.stlx} enth\"{a}lt dar\"{u}ber hinaus die Definition der Funktion
$\texttt{findCounterExample}(n)$, die versucht, f\"{u}r eine gegebene positive nat\"{u}rliche Zahl $n$ eine
Zahl $k \geq n$ zu finden, so dass zwischen $k^2$ und $(k+1)^2$ keine Primzahl liegt.  Die Idee bei
der Implementierung dieser Funktion ist einfach:  Zun\"{a}chst \"{u}berpr\"{u}fen wir durch den Aufruf
$\texttt{legendre}(n)$, ob zwischen $n^2$ und $(n+1)^2$
eine Primzahl ist.  Falls dies der Fall ist, untersuchen wir anschlie\ss{}end das Intervall von
$(n+1)^2$ bis $(n+2)^2$, dann das Intervall von 
$(n+2)^2$ bis $(n+3)^2$ und so weiter, bis wir schlie\ss{}lich eine Zahl $m$ finden, so dass zwischen
$m^2$ und $(m+1)^2$ keine Primzahl liegt.  Falls Legendre Recht hatte, werden wir nie ein solches
$k$ finden und in diesem Fall wird der Aufruf $\texttt{findCounterExample}(1)$ nicht terminieren. 

Nehmen wir nun an, wir h\"{a}tten ein schlaues Programm, nennen wir es \texttt{stops}, das als Eingabe
eine \textsc{SetlX} Funktion $f$ und ein Argument $a$ verarbeitet und dass uns die Frage, ob die Berechnung von $f(a)$
terminiert, beantworten kann.  Die Idee w\"{a}re, dass f\"ur die Funktion \texttt{stops} Folgendes gilt:
\\[0.2cm]
\hspace*{1.3cm}
$\texttt{stops}(f, a) = 1$ \quad g.d.w. \quad der Aufruf $f(a)$ terminiert.
\\[0.2cm]
Falls der Aufruf $f(a)$ nicht terminiert,  sollte statt dessen $\texttt{stops}(f,a) = 0$ gelten.
Wenn wir eine solche Funktion \texttt{stops} h\"{a}tten, dann k\"{o}nnten wir 
\\[0.2cm]
\hspace*{1.3cm}
\texttt{stops(findCounterExample, 1)}
\\[0.2cm]
aufrufen und w\"{u}ssten anschlie\ss{}end, ob die Vermutung von Legendre wahr ist oder nicht:  Wenn
\\[0.2cm]
\hspace*{1.3cm}
$\texttt{stops(findCounterExample, 1)} = 1$
 \\[0.2cm]
ist, dann w\"{u}rde das hei\ss{}en,
dass der Funktions-Aufruf $\texttt{findCounterExample}(1)$ terminiert.  Das passiert aber nur dann,
wenn ein Gegenbeispiel gefunden wird.  W\"{u}rde der Aufruf 
\\[0.2cm]
\hspace*{1.3cm}
\texttt{stops(findCounterExample, 1)}
\\[0.2cm]
statt dessen eine $0$ zur\"{u}ck liefern, so k\"{o}nnten wir schlie\ss{}en, dass der Aufruf \texttt{findCounterExample(1)}
nicht terminiert. Mithin w\"{u}rde die Funktion \texttt{findCounterExample} kein Gegenbeispiel finden und
das w\"{u}rde hei\ss{}en, dass die Vermutung von Legendre wahr ist.

Es gibt eine Reihe weiterer offener  mathematischer Probleme, die alle auf die Frage abgebildet
werden k\"{o}nnen, ob eine gegebene Funktion terminiert.  Daher zeigen die vorhergehenden \"{U}berlegungen,
dass es sehr n\"{u}tzlich w\"{a}re, eine Funktion wie \texttt{stops} zur Verf\"{u}gung zu haben.  Andererseits
k\"{o}nnen wir an dieser Stelle schon ahnen, dass die Implementierung der Funktion \texttt{stops}
nicht ganz einfach sein kann.  

 
\subsection{Formale Analyse des Halte-Problems}
Wir werden in diesem Abschnitt beweisen, dass das Halte-Problem unl\"{o}sbar ist.  Dazu f\"{u}hren
wir den Begriff einer Test-Funktion ein.  

\begin{Definition}[Test-Funktion] 
{\em Ein String $t$ ist genau dann eine \emph{Test-Funktion}, wenn $t$ 
 die Form \\[0.3cm]
\hspace*{1.3cm} {\tt procedure(x) \{ $\cdots$ \}} \\[0.3cm]
hat und sich als \textsc{SetlX}-Funktion parsen l\"{a}sst.  Die Menge der
Test-Funktionen bezeichnen wir mit $T\!F$.}  \eox
\end{Definition}

\noindent
\textbf{Beispiele}:  
\begin{enumerate}
\item $s_1$ = ``{\tt procedure(x) \{ return 0; \}}''

      $s_1$ ist eine (sehr einfache) Test-Funktion.
\item $s_2$ = ``{\tt procedure(x) \{ while (true) \{ x := x + 1; \} \}}''

      $s_2$ ist ebenfalls eine Test-Funktion.  Offenbar liefert diese Test-Funktion nie ein
      Ergebnis, aber f\"{u}r die Frage, ob $s_2$ eine Test-Funktion ist oder nicht, ist dies
      irrelevant.
\item $s_3$ = ``{\tt procedure(x) \{ return ++x; \}}''

      $s_3$ ist keine Test-Funktion, denn da \textsc{SetlX} den Pr\"{a}fix-Operator
      ``\texttt{++}'' nicht unterst\"{u}tzt, l\"{a}sst sich der String $s_3$  nicht fehlerfrei parsen.
\item $s_4$ = ``{\tt procedure(x, y) \{ return x + y; \}}''

      $s_4$ ist keine Test-Funktion, denn ein String ist nur dann eine Test-Funktion, wenn die Funktion
      mit genau einen Parameter aufgerufen wird.
\end{enumerate}
Um das Halte-Problem \"{u}bersichtlicher formulieren zu k\"{o}nnen, f\"{u}hren wir noch drei
zus\"{a}tzliche Notationen ein.
\begin{Notation}[$\leadsto$, $\downarrow$, $\uparrow$]
{\em
Ist $t$ eine \textsc{SetlX}-Funktion, die $k$ Argumente verarbeitet und sind $a_1$, $\cdots$, $a_k$ Argumente,
so schreiben wir \\[0.3cm]
\hspace*{1.3cm} $t(a_1, \cdots, a_k) \leadsto r$ \\[0.3cm]
wenn der Aufruf $t(a_1, \cdots, a_k)$ das Ergebnis $r$ liefert.  Sind wir an dem Ergebnis
selbst nicht interessiert, sondern wollen nur angeben, dass ein Ergebnis existiert, so
schreiben wir \\[0.3cm]
\hspace*{1.3cm} $t(a_1, \cdots, a_k) \,\downarrow$ \\[0.3cm]
und sagen, dass der Aufruf $t(a_1, \cdots, a_k)$ \colorbox{yellow}{\emph{terminiert}}.
Terminiert der Aufruf $t(a_1, \cdots, a_k)$ nicht, so schreiben wir \\[0.3cm]
\hspace*{1.3cm} $t(a_1, \cdots, a_k) \,\uparrow$ \\[0.3cm]
und sagen, dass der Aufruf $t(a_1, \cdots, a_k)$ \colorbox{yellow}{\emph{divergiert}}.
\hspace*{\fill} $\Box$
}
\end{Notation}

\noindent
\textbf{Beispiele}: Legen wir die Funktions-Definitionen zugrunde, die wir im Anschluss an
die Definition des Begriffs der Test-Funktion gegeben haben, so gilt:
\begin{enumerate}
\item {\tt procedure(x) \{ return 0; \}(\symbol{34}emil\symbol{34})} $\leadsto 0$
\item {\tt procedure(x) \{ return 0; \}(\symbol{34}emil\symbol{34})} $\downarrow$
\item {\tt procedure(x) \{ while (true) \{ x := x + 1; \} \}(\symbol{34}hugo\symbol{34})} $\uparrow$
\end{enumerate} 

\noindent
Das \emph{Halte-Problem} f\"{u}r
\textsc{SetlX}-Funktionen ist die Frage, ob es eine \texttt{SetlX}-Funktion \\[0.1cm] 
\hspace*{1.3cm} 
$\texttt{stops := procedure}(t,\;a)\; \{\;\cdots\;\}$ \\[0.1cm]
gibt, die als Eingabe eine Testfunktion $t$ und einen String $a$ erh\"{a}lt und die folgende
Eigenschaft hat:
\begin{enumerate}
\item $t \not\in T\!F \quad\Leftrightarrow\quad \mathtt{stops}(t, a) \leadsto 2$.

      Der Aufruf \texttt{stops($t$, $a$)} liefert genau dann den Wert 2 zur\"{u}ck, 
      wenn $t$ keine Test-Funktion ist.

\item $t \in T\!F \,\wedge\, t(a)\downarrow \quad\Leftrightarrow\quad
       \mathtt{stops}(t, a) \leadsto 1$.

      Der Aufruf \texttt{stops($t$, $a$)} liefert genau dann den Wert 1 zur\"{u}ck,
      wenn $t$ eine Test-Funktion ist und der Aufruf $t(a)$ terminiert.

\item $t \in T\!F \,\wedge\, t(a)\uparrow \quad\Leftrightarrow\quad
       \mathtt{stops}(t, a) \leadsto 0$.

      Der Aufruf \texttt{stops($t$, $a$)} liefert genau dann den Wert 0 zur\"{u}ck,
      wenn $t$ eine Test-Funktion ist und der Aufruf $t(a)$ \underline{nicht} terminiert.
\end{enumerate}
Falls eine \textsc{SetlX}-Funktion \texttt{stops} mit den obigen Eigenschaften existiert, dann
sagen wir, dass das Halte-Problem f\"{u}r \textsc{SetlX} entscheidbar ist.

\begin{Theorem}[\href{http://de.wikipedia.org/wiki/Alan_Turing}{Alan Turing}, 1936]
{\em
  Das Halte-Problem ist unentscheidbar.
} 
\end{Theorem}

\noindent
\textbf{Beweis}:  Zun\"{a}chst eine Vorbemerkung.  Um die Unentscheidbarkeit des
Halte-Problems nachzuweisen, m\"{u}ssen wir zeigen, dass etwas, n\"{a}mlich eine Funktion mit
gewissen Eigenschaften nicht existiert.  Wie kann so ein Beweis \"{u}berhaupt funktionieren?
Wie k\"{o}nnen wir \"{u}berhaupt zeigen, dass irgendetwas nicht existiert?
Die einzige M\"{o}glichkeit zu zeigen, dass etwas nicht existiert ist indirekt:
Wir nehmen also an, dass eine Funktion \texttt{stops} existiert, die das Halte-Problem l\"{o}st.
Aus dieser Annahme werden wir einen Widerspruch ableiten.  Dieser Widerspruch zeigt
uns dann, dass eine Funktion \texttt{stops} mit den gew\"{u}nschten Eigenschaften nicht
existieren kann.
Um zu einem Widerspruch zu kommen, definieren wir den String \texttt{turing} wie in Abbildung
\ref{fig:turing-string} gezeigt.

\begin{figure}[!h]
  \centering
\begin{Verbatim}[ frame         = lines, 
                  framesep      = 0.3cm, 
                  labelposition = bottomline,
                  numbers       = left,
                  numbersep     = -0.2cm,
                  xleftmargin   = 0.8cm,
                  xrightmargin  = 0.8cm,
                  commandchars  = \\\{\},
                ]
    \texttt{turing} := "procedure(x) \{
                   result := stops(x, x);
                   if (result == 1) \{
                       while (true) \{
                           print("... looping ...");
                       \}
                   \}
                   return result;
               \}"
\end{Verbatim}
  \vspace*{-0.3cm}
  \caption{Die Definition des Strings \texttt{turing}.}
  \label{fig:turing-string}
\end{figure}

Mit dieser Definition ist klar, dass \texttt{turing} eine Test-Funktion ist: \\[0.3cm]
\hspace*{1.3cm} $\texttt{turing} \in T\!F$. \\[0.3cm]
Damit sind wir in der Lage, den String \texttt{turing} als Eingabe der Funktion \texttt{stops}
zu verwenden.  Wir betrachten nun den folgenden Aufruf: \\[0.3cm]
\hspace*{1.3cm} \texttt{stops(\texttt{turing}, \texttt{turing});} \\[0.3cm]
Da \texttt{turing} eine Test-Funktion ist, k\"{o}nnen nur zwei F\"{a}lle auftreten:
\\[0.1cm]
\hspace*{1.3cm} 
$\texttt{stops(\texttt{turing}, \texttt{turing})} \leadsto 0 \quad \vee\quad
 \texttt{stops(\texttt{turing}, \texttt{turing})} \leadsto 1$. \\[0.1cm]
Diese beiden F\"{a}lle analysieren wir nun im Detail:
\begin{enumerate}
\item $\texttt{stops(\texttt{turing}, \texttt{turing})} \leadsto 0$. 

      Nach der Spezifikation von \texttt{stops} bedeutet dies \\[0.1cm]
      \hspace*{1.3cm} $\texttt{turing}(\texttt{turing}) \uparrow$ \\[0.1cm]
      Schauen wir nun, was wirklich beim Aufruf \texttt{turing(\texttt{turing})} passiert:
      In Zeile 2 erh\"{a}lt die Variable \texttt{result} den Wert 0 zugewiesen.  In Zeile 3
      wird dann getestet, ob \texttt{result} den Wert 1 hat.  Dieser Test schl\"{a}gt fehl.
      Daher wird der Block der \texttt{if}-Anweisung nicht ausgef\"{u}hrt und die Funktion liefert als
      n\"{a}chstes in Zeile 8 den Wert 0 zur\"{u}ck.  Insbesondere terminiert der Aufruf also, im
      Widerspruch zu dem, was die Funktion \texttt{stops} behauptet hat. $\lightning$

      Damit ist der erste Fall ausgeschlossen.
\item  $\mathtt{stops}(\texttt{turing}, \texttt{turing}) \leadsto 1$. 

      Aus der Spezifikation der Funktion \texttt{stops} folgt, dass der Aufruf
      $\texttt{turing}(\texttt{turing})$ terminiert: \\[0.1cm]
      \hspace*{1.3cm} $\mathtt{turing}(\texttt{turing}) \downarrow$ \\[0.1cm]
      Schauen wir nun, was wirklich beim Aufruf \texttt{turing(\texttt{turing})} passiert:
      In Zeile 2 erh\"{a}lt die Variable \texttt{result} den Wert 1 zugewiesen.  In Zeile 3
      wird dann getestet, ob \texttt{result} den Wert 1 hat.  Diesmal gelingt der Test.
      Daher wird der Block der \texttt{if}-Anweisung ausgef\"{u}hrt.  Dieser Block
      besteht aber nur aus einer Endlos-Schleife, aus der wir nie wieder zur\"{u}ck kommen.
      Das steht im Widerspruch zu dem, was die Funktion \texttt{stops} behauptet hat.
      $\lightning$

      Damit ist der zweite Fall ausgeschlossen.
\end{enumerate}
Insgesamt haben wir also in jedem Fall einen Widerspruch erhalten.  
Damit muss die Annahme, dass die \textsc{SetlX}-Funktion \texttt{stops}
das Halte-Problem l\"{o}st, falsch sein, denn diese Annahme ist die Ursache f\"{u}r die Widerspr\"{u}che, die
wir erhalten haben.  Insgesamt haben wir daher gezeigt, dass es keine \textsc{SetlX}-Funktion
geben kann, die das Halte-Problem l\"{o}st. \hspace*{\fill} $\Box$
\vspace*{0.3cm}

\noindent
\textbf{Bemerkung}:
Der Nachweis, dass das Halte-Problem unl\"{o}sbar ist, wurde 1936 von Alan Turing (1912 -- 1954)
\cite{turing:36} erbracht.  Turing hat das Problem damals nat\"{u}rlich nicht f\"{u}r die Sprache
\textsc{SetlX} gel\"{o}st, sondern f\"{u}r die heute nach ihm benannten \emph{Turing-Maschinen}.  
Eine Turing-Maschine ist abstrakt gesehen nichts anderes als eine Beschreibung eines
Algorithmus.  Turing hat also gezeigt, dass es keinen Algorithmus gibt, der entscheiden
kann, ob ein gegebener anderer Algorithmus terminiert.
\vspace*{0.3cm}

\noindent
\textbf{Bemerkung}:
An dieser Stelle k\"{o}nnen wir uns fragen, ob es vielleicht eine andere Programmier-Sprache
gibt, in der wir das Halte-Problem dann vielleicht doch l\"{o}sen k\"{o}nnten.  
Wenn es in dieses Programmier-Sprache Prozeduren, \texttt{if}-Verzweigungen und
\texttt{while}-Schleifen gibt, und wenn wir dort 
Programm-Texte als Argumente von Funktionen \"{u}bergeben k\"{o}nnen, dann ist leicht zu sehen,
dass der obige Beweis der 
Unl\"{o}sbarkeit des Halte-Problems sich durch geeignete syntaktische Modifikationen auch auf
die andere Programmier-Sprache \"{u}bertragen l\"{a}sst.


\section{Unl\"{o}sbarkeit des \"{A}quivalenz-Problems}
Es gibt noch eine ganze Reihe anderer Funktionen, die nicht berechenbar sind.  In der
Regel werden wir den Nachweis, dass eine bestimmt Funktion nicht berechenbar ist, indirekt f\"{u}hren
und annehmen, dass die gesuchte Funktion doch berechenbar ist.  Unter dieser Annahme
konstruieren wir dann eine Funktion, die das Halte-Problem l\"{o}st, was im Widerspruch zu der Unl\"{o}sbarkeit
des Halte-Problems steht.
Dieser Widerspruch zwingt uns zu der Folgerung, dass die gesuchte Funktion nicht berechenbar ist.
Wir werden dieses Verfahren an einem Beispiel demonstrieren.  Vorweg ben\"{o}tigen wir aber
noch eine Definition.

\begin{Definition}[$\simeq$] 
{\em 
Es seien $t_1$ und $t_2$ zwei \textsc{SetlX}-Funktionen und
  $a_1$, $\cdots$, $a_k$ seien Argumente, mit denen wir diese Funktionen f\"{u}ttern k\"{o}nnen. Wir definieren \\[0.1cm]
\hspace*{1.3cm} $t_1(a_1,\cdots,a_k) \simeq t_2(a_1,\cdots,a_k)$ \\[0.1cm]
g.d.w. einer der beiden folgen F\"{a}lle auftritt:
\begin{enumerate}
\item $t_1(a_1,\cdots,a_k)\uparrow \quad\wedge\quad t_2(a_1,\cdots,a_k)\uparrow$,

      beide Funktionen divergieren also f\"{u}r die gegebenen Argumente.
\item $\exists r: \Bigl(t_1(a_1,\cdots,a_k) \leadsto r \;\wedge\; t_2(a_1,\cdots,a_k) \leadsto r\Bigr)$,

      die Funktionen liefern also f\"{u}r die gegebenen Argumente das gleiche Ergebnis.
\end{enumerate}
      In diesem Fall sagen wir, dass die beiden Funktions-Aufrufe 
      $t_1(a_1,\cdots,a_k) \simeq t_2(a_1,\cdots,a_k)$ \emph{partiell \"{a}quivalent} sind.  
      \qed
}
\end{Definition}

\noindent
Wir kommen jetzt zum \emph{\"{A}quivalenz-Problem}.  Die Funktion $\texttt{equal}$, die die Form
\\[0.2cm]
\hspace*{1.3cm}
$\texttt{equal := procedure(p1, p2, a) \{ ... \}}$
\\[0.2cm]
hat, m\"{o}ge folgender Spezifikation gen\"{u}gen:
\begin{enumerate}
\item $p_1 \not\in T\!F \;\vee\; p_2 \not\in T\!F \quad\Leftrightarrow\quad \mathtt{equal}(p_1, p_2, a) \leadsto 2$.
\item Falls 
  \begin{enumerate}
  \item $p_1 \in T\!F$,
  \item $p_2 \in T\!F$ \quad und
  \item $p_1(a) \simeq p_2(a)$
  \end{enumerate}
    gilt, dann muss gelten: \\[0.1cm]
   \hspace*{1.3cm}  $\mathtt{equal}(p_1, p_2, a) \leadsto 1$.
\item Ansonsten gilt \\[0.1cm]
      \hspace*{1.3cm} $\mathtt{equal}(p_1, p_2, a) \leadsto 0$.
\end{enumerate}
Wir sagen, dass eine Funktion, die der eben angegebenen Spezifikation gen\"{u}gt, das
\emph{\"{A}quivalenz-Problem} l\"{o}st.

\begin{Theorem}[Rice, 1953]
Das \"{A}quivalenz-Problem ist unl\"{o}sbar.  
\end{Theorem}

\noindent
\textbf{Beweis}:
Wir f\"{u}hren den Beweis indirekt und nehmen
an, dass es doch eine Implementierung der Funktion \texttt{equal} gibt, die das
\"{A}quivalenz-Problem l\"{o}st.  Wir betrachten die in Abbildung 
\ref{fig:stops} angegeben Implementierung der Funktion \texttt{stops}.


\begin{figure}[!h]
  \centering
\begin{Verbatim}[ frame         = lines, 
                  framesep      = 0.3cm, 
                  labelposition = bottomline,
                  numbers       = left,
                  numbersep     = -0.2cm,
                  xleftmargin   = 0.3cm,
                  xrightmargin  = 0.3cm
                ]
     stops := procedure(p, a) {
         f := "procedure(x) { while (true) { x := x + x; } }"; 
         e := equal(f, p, a);
         if (e == 2) {
             return 2;
         } else {
             return 1 - e;
         }
     };
\end{Verbatim}
  \vspace*{-0.3cm}
  \caption{Eine Implementierung der Funktion \texttt{stops}.}
  \label{fig:stops}
\end{figure}

Zu beachten ist, dass in Zeile 3 die Funktion \texttt{equal} mit einem String aufgerufen
wird, der eine Test-Funktion ist.  Diese 
Test-Funktion hat die
folgende Form:
\begin{verbatim}
      procedure(x) { while (true) { x := x + x; } };
\end{verbatim}
Es ist offensichtlich, dass diese Funktion f\"{u}r kein Ergebnis terminiert.
Ist also das Argument $p$ eine Test-Funktion, so liefert die Funktion
\texttt{equal} immer dann den Wert 1, 
wenn $p(a)$ nicht terminiert, andernfalls muss sie den Wert 0 zur\"{u}ck geben.
Damit liefert die Funktion \texttt{stops} aber f\"{u}r eine Test-Funktion $p$ 
und ein Argument 
$a$ genau dann 1, wenn der Aufruf $p(a)$ terminiert und w\"{u}rde folglich das Halte-Problem
l\"{o}sen.  Das kann nicht sein, also kann es keine Funktion  \texttt{equal}
geben, die das \"{A}quivalenz-Problem l\"{o}st. 
\qed
\vspace*{0.3cm}

\noindent
Die Unl\"{o}sbarkeit des \"{A}quivalenz-Problems und vieler weiterer praktisch interessanter
Probleme folgen aus dem 1953 von Henry G.~Rice \cite{rice:53} bewiesenen
\href{http://de.wikipedia.org/wiki/Satz_von_Rice}{Satz von Rice}.


%%% Local Variables: 
%%% mode: latex
%%% TeX-master: "logic"
%%% End: 
