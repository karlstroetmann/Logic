\section{Der G\"{o}del'sche Vollst\"{a}ndigkeits-Satz}
In diesem Abschnitt wollen wir zeigen, dass der Kalk\"{u}l, der nur aus der Substitutions-Regel
\[ \schluss{k}{k\sigma} \qquad \mbox{und der Schnitt-Regel} \qquad
   \schluss{k_1 \cup \{ p \} \quad \{ \neg p \} \cup k_2}{k_1 \cup k_2} \]
besteht, f\"{u}r die Pr\"{a}dikatenlogik widerlegungs-vollst\"{a}ndig ist.  Wir nehmen also an, dass
$K$ eine widerspr\"{u}chliche Menge pr\"{a}dikatenlogischer Klauseln ist, es gilt also
\[ K \models \falsum. \]
Wir zeigen, dass dann mit den obigen beiden Regeln die leere Klausel abgeleitet werden kann, es gilt
also
\[ K \vdash \{\}. \]
Wir f\"{u}hren den Beweis dieser Behauptung durch Kontraposition und zeigen, dass
\[ K \not\,\vdash \{\} \quad \Rightarrow \quad K \not\,\models \falsum \]
gilt.  Wir gehen also von $K \not\,\vdash \{\}$. 
Wir definieren nun die Menge $\widehat{K}$, die aus allen geschlossenen Instanzen von Klauseln
aus $K$ besteht, es gilt also
\[ \widehat{K} := 
   \bigl\{ k\sigma \mid k \in K \wedge \sigma \in \textsl{Subst} \wedge \textsl{Var}(k\sigma) = \{\} \bigr\} 
\]
Nehmen wir einmal an,  wir k\"{o}nnten aus $\widehat{K}$ mit der Schnitt-Regel die leere Klausel
ableiten.  F\"{u}r diese Ableitung h\"{a}tten wir dann eine Menge von Klauseln der Form
\[ \{ k_1\sigma_1, \cdots, k_n\sigma_n \} \]
verwendet, wobei die Klauseln $k_i$ f\"{u}r $i=1,\cdots,n$ aus der Menge $K$ stammen.
Dann k\"{o}nnten wir aber aus den Klauseln
\[ \{ k_1, \cdots, k_n \} \]
mit Hilfe der Substitutions-Regel zun\"{a}chst die Klauseln 
\[ k_1\sigma, \cdots k_n\sigma  \]
herleiten und daraus k\"{o}nnten wir mit der Schnitt-Regel die leere Klausel ableiten, so dass wir
insgesamt die leere Klausel aus $K$ ableiten k\"{o}nnten.  Da wir aber eingangs angenommen haben, dass
wir aus $K$ nicht die leere Klausel ableiten k\"{o}nnen, folgt, dass wir aus $\widehat{K}$ ebenfalls
nicht die leere Klausel ableiten k\"{o}nnen.  

Unser Ziel ist es, ein pr\"{a}dikatenlogisches Modell f\"{u}r die Klausel-Menge $K$ zu konstruieren und
dadurch zu zeigen, dass $K \not\,\models \falsum$ gilt.
Wir fassen dazu die atomaren Formeln
\[ p(t_1,\cdots,t_n), \]
aus denen die Klauseln der Menge $\widehat{K}$ aufgebaut sind, als aussagenlogische Variablen auf.
Da wir aus den Klauseln von $\widehat{K}$ nicht die leere Klausel ableiten k\"{o}nnen, zeigt der
Vollst\"{a}ndigkeits-Satz der Aussagenlogik, dass die Menge $\widehat{K}$ aussagenlogisch erf\"{u}llbar 
sein muss.  Damit gibt es eine aussagenlogische Belegung $\mathcal{I}$, so dass mit dieser Belegung
alle Klauseln aus $K$ wahr werden.  Aus dieser Belegung werden wir eine pr\"{a}dikatenlogische Struktur
erzeugen, die ein Modell der Formeln der Menge $K$ ist.  Wir definieren zun\"{a}chst das
\emph{Herbrand-Universum} $\mathcal{U}$ als die Menge aller \emph{geschlossenen} Terme,
also als die Menge der Terme, in denen keine Variablen mehr auftreten:
\[ \mathcal{U} := \bigl\{ t \in \mathcal{T}_\Sigma \mid \textsl{Var}(t) = \{\} \bigr\}. \]
F\"{u}r die Funktions-Zeichen $f \in \mathcal{F}$ definieren wir nun die \emph{Herbrand-Interpretation}
$f^\mathcal{H}$ als
\[ f^\mathcal{H}(t_1, \cdots, t_n) := f(t_1,\cdots,t_n), \]
bei der Herbrand-Interpretation wird also jedes Funktions-Zeichen durch sich selbst interpretiert!
Die Interpretation $p^\mathcal{H}$ der Pr\"{a}dikats-Zeichen definieren wir durch R\"{u}ckgriff auf die
aussagenlogische Interpretation $\mathcal{I}$ wie folgt
\[ 
  p^\mathcal{H}(t_1,\cdots,t_n) := \mathcal{I}\bigl(p(t_1,\cdots,t_n)\bigr).
\]
F\"{u}r die so definierte Struktur kann jede Variablen-Belegung als eine Substitution
aufgefasst werden, denn eine Variablen-Belegung $\mathcal{L}$ ist eine Funktion
\[ \mathcal{L}: \mathcal{V} \rightarrow \mathcal{U} \]
von der Menge der Variablen $\mathcal{V}$ in das Universum $\mathcal{U}$ und da das
Universum nur aus geschlossenen Termen besteht, ist $\mathcal{L}$ damit auch eine
Substitution.  Ist nun $\mathcal{L}$ eine Variablen-Belegung, so k\"{o}nnen wir zeigen, dass f\"{u}r jeden
Term $t$ 
\[ \mathcal{S}^\mathcal{H}(\mathcal{L}, t) = \mathcal{S}^\mathcal{H}(\mathcal{L},t\mathcal{L})  \]
gilt, wobei wir $\mathcal{L}$ in dem Ausdruck $t\mathcal{L}$ als Substitution auffassen.  
Der Beweis dieser Behauptung kann durch Induktion nach dem Aufbau von $t$ gef\"{u}hrt werden.

%%% Local Variables: 
%%% mode: latex
%%% TeX-master: "logik"
%%% End: 
