\documentclass{slides}
\usepackage[latin1]{inputenc}
\usepackage{german}
\usepackage{epsfig}
\usepackage{amssymb}

\pagestyle{empty}
\setlength{\textwidth}{17cm}
\setlength{\textheight}{24cm}
\setlength{\topmargin}{0cm}
\setlength{\headheight}{0cm}
\setlength{\headsep}{0cm}
\setlength{\topskip}{0.2cm}
\setlength{\oddsidemargin}{0.5cm}
\setlength{\evensidemargin}{0.5cm}

\newcommand{\is}{\;|\;}
\newcommand{\schluss}[2]{\frac{\displaystyle\quad \rule[-8pt]{0pt}{18pt}#1 \quad}{\displaystyle\quad \rule{0pt}{14pt}#2 \quad}}
\newcommand{\vschlus}[1]{{\displaystyle\rule[-6pt]{0pt}{12pt} \atop \rule{0pt}{10pt}#1}}
\newcommand{\verum}{\top}
\newcommand{\falsum}{\bot}
\newcommand{\gentzen}{\vdash_G}
\newcommand{\komplement}[1]{\overline{#1}}

\newcounter{mypage}

\begin{document}
\begin{slide}
\begin{center}
Widerlegungs-Vollst"andigkeit des Gentzen-Kalk"uls
\end{center}
\vspace*{0.5cm}

\footnotesize
\textbf{Gegeben}: $M \subseteq \mathcal{F}$ und $f \in \mathcal{F}$

\textbf{Gesucht}: Verfahren, um $M \models f$ zu entscheiden.

Bekannte Verfahren: Sei $M = \{ g_1, \cdots, g_n \}$.  Dann gilt \\[0.3cm]
\hspace*{1.3cm} $M \models f$ \quad g.d.w. \quad $\models g_1 \wedge \cdots \wedge g_n \rightarrow f$ \\[0.3cm]
Entscheidung, ob Tautologie durch
\begin{enumerate}
\item Werte-Tabelle oder
\item Konjunktive Normalform
\end{enumerate}
Beide Verfahren sind ineffizient:
\begin{enumerate}
\item Werte-Tabelle: 

      Bei $n$ Aussage-Variablen hat Tabelle $2^n$ Zeilen.
\item KNF: Sei $g_i$ Klausel mit $m_i$ Literalen.

      Dann hat KNF von $g_1 \wedge \cdots \wedge g_n \rightarrow f$ mindestens

      $m_1 * \cdots * m_n$ Klauseln.
\end{enumerate}

Andere M"oglichkeit: \\[0.3cm]
\hspace*{1.3cm} $M \models f$ \quad g.d.w. \quad  $M \cup \{\neg f\} \models \falsum$ \\[0.3cm]
\hspace*{4.3cm} g.d.w.~\quad $M \cup \{\neg f\} \vdash_G \falsum$


\vspace*{\fill}
\tiny \addtocounter{mypage}{1} 
\rule{17cm}{1mm}
Widerlegungs-Vollst"andigkeit  \hspace*{\fill} Seite \arabic{mypage}
\end{slide}

%%%%%%%%%%%%%%%%%%%%%%%%%%%%%%%%%%%%%%%%%%%%%%%%%%%%%%%%%%%%%%%%%%%%%%%%

\begin{slide}{}
\normalsize
\begin{center}
\end{center}
\vspace{0.5cm}

\footnotesize
Sei $M \subseteq \mathcal{F}$ und $f \in \mathcal{F}$.

\textbf{Korrektheit}:  \\[0.3cm] 
\hspace*{1.3cm} $M \vdash_G f \quad\Rightarrow\quad  M \models f$

\textbf{Widerlegungs-Vollst"andigkeit}: \\[0.3cm]
\hspace*{1.3cm} $M \models \falsum \quad\Rightarrow\quad M \vdash_{G} \falsum$

Zusammen: $M \models \falsum$ \quad g.d.w. \quad $M \vdash_{G} \falsum$
\vspace{0.5cm}

\textbf{Definition}: Sei $k \in \mathcal{K}$, $M \subseteq \mathcal{K}$  und $l \in \mathcal{L}$. \\[0.5cm]
\hspace*{1.3cm} $\textsl{redukt}(k,l) := \left\{ 
     \begin{array}{ll}
        \verum                &  \mathrm{falls}\; l \in k; \\
        k \backslash \Bigl\{ \komplement{\,l\,} \Bigr\}  &
   \mathrm{falls}\; l\not\in k\; \mathrm{und}\; \komplement{\,l\,} \in k; \\
        k                     &  \mathrm{sonst}.
     \end{array}
     \right.$ \\[0.5cm]
  \hspace*{1.3cm} $\textsl{Redukt}(M, l) := \bigl\{\; \textsl{redukt}(k,l) \;|\; k \in M \;\bigr\}$.
\vspace{0.5cm}

\textbf{Beispiel}:\\[0.1cm]
\hspace*{1.3cm} 
$M = \Bigg\{ \{ \neg r, p, q \}, \{ \neg r, \neg p, \neg o \}, \{r, q\}, \{ \neg q, p \},$\\
\hspace*{3.3cm} 
$\{ \neg p, o \}, \{ \neg p, q \}, \{ \neg q, r, \neg o \} \Bigg\}$
    
$\textsl{Redukt}(M,p) = \Bigg\{ \verum, \{ \neg r, \neg o \}, \{r, q\}, \{ o \}, \{ q \}, \{ \neg q, r, \neg o \} \Bigg\}$

$\textsl{Redukt}(M, \neg p) = \Bigg\{ \{ \neg r, q \}, \verum, \{r, q\}, \{ \neg q \}, \{ \neg q, r, \neg o \} \Bigg\}$

\vspace*{\fill}
\tiny \addtocounter{mypage}{1}
\rule{17cm}{1mm}
Widerlegungs-Vollst"andigkeit  \hspace*{\fill} Seite \arabic{mypage}
\end{slide}


%%%%%%%%%%%%%%%%%%%%%%%%%%%%%%%%%%%%%%%%%%%%%%%%%%%%%%%%%%%%%%%%%%%%%%%%

\begin{slide}{}
\normalsize
\begin{center}
Eigenschaften von $\textsl{Redukt}(M, l)$
\end{center}
\vspace{0.5cm}

\footnotesize
\textbf{Beobachtung}: \\
$p$ tritt weder in $\textsl{Redukt}(M, p)$ noch in
$\textsl{Redukt}(M, \neg p)$ auf.
\vspace{0.5cm}

\textbf{Satz}: Ist $M \subseteq \mathcal{K}$ und $l \in \mathcal{L}$, so gilt \\[0.3cm]
\hspace*{1.3cm} $M \models \falsum \quad \Rightarrow \quad \textsl{Redukt}(M, l) \models \falsum$.
\vspace{0.5cm}

\textbf{Satz}: Ist $M \subseteq \mathcal{K}$, $f \in \mathcal{K}$ und $l \in \mathcal{L}$, so gilt: \\[0.3cm]
\hspace*{0.3cm}  $\textsl{Redukt}(M,l) \gentzen f$  $\;\Rightarrow\;$ 
                 $M \gentzen f$  oder  $M \gentzen f \cup \bigl\{\komplement{\,l\,}\bigr\}$.
\vspace{0.5cm}

\textbf{Aufgabe}: Zeigen Sie
\begin{enumerate}
\item $\Bigg\{ \{ \neg r, \neg o \}, \{r, q\}, \{ o \}, \{ q \}, \{ \neg q, r, \neg o \} \Bigg\} \vdash_G \falsum$
\item $\Bigg\{ \{ \neg r, q \}, \{r, q\}, \{ \neg q \}, \{ \neg q, r, \neg o \} \Bigg\} \vdash_G \falsum$
\item $\Bigg\{ \{ \neg r, p, q \}, \{ \neg r, \neg p, \neg o \}, \{r, q\}, \{ \neg q, p \},$\\
\hspace*{3.3cm} 
      $\{ \neg p, o \}, \{ \neg p, q \}, \{ \neg q, r, \neg o \} \Bigg\} \vdash_G \falsum$
\end{enumerate}


\vspace*{\fill}
\tiny \addtocounter{mypage}{1}
\rule{17cm}{1mm}
Widerlegungs-Vollst"andigkeit  \hspace*{\fill} Seite \arabic{mypage}
\end{slide}

%%%%%%%%%%%%%%%%%%%%%%%%%%%%%%%%%%%%%%%%%%%%%%%%%%%%%%%%%%%%%%%%%%%%%%%%

\begin{slide}{}
\normalsize
\begin{center}
Widerlegungs-Vollst"andigkeit des Gentzen-Kalk"uls
\end{center}
\vspace{0.5cm}

\footnotesize

\textbf{Theorem}: (Widerlegungs-Vollst"andigkeit von $G$) \\[0.3cm]
 Sei $M \subseteq \mathcal{K}$. Dann gilt\\[0.3cm]
\hspace*{1.3cm}  $M \models \falsum$ \quad$\Rightarrow$\quad $M \vdash_{G} \falsum$. \\[0.3cm]
\textbf{Beweis}: Induktion "uber Anzahl $n$ der Aussage-Variablen in $M$.
\vspace{0.5cm}

\textbf{Satz}:  Sei $M \subseteq \mathcal{K}$, $f \in \mathcal{K}$ und $p \in \mathcal{P}$ \\[0.1cm]
\hspace*{1.9cm} $M \models \falsum$ \quad \\[0.3cm]
g.d.w. \\[0.3cm]
\hspace*{1.9cm} $\textsl{Redukt}(M,p) \vdash_G \falsum$ \quad und\quad $\textsl{Redukt}(M,\neg p) \vdash_G \falsum$
\vspace{0.5cm}

\textbf{Aufgabe}: Zeigen Sie \\
\hspace*{0.8cm}
$\Bigg\{ \{t,\neg s,q \}, \{ \neg t,q,p \}, \{ t,s,\neg r\}, \{ \neg t,s,\neg r \}, \{s, p\},$ \\[0.1cm]
\hspace*{1.3cm} $ \{ t,p \}, \{ \neg t,\neg q,p \}, \{ \neg s,\neg p\}, \{u,r,\neg p\}, \{ \neg u,\neg p \} \Bigg\} \models \falsum$

\vspace*{\fill}
\tiny \addtocounter{mypage}{1}
\rule{17cm}{1mm}
Widerlegungs-Vollst"andigkeit  \hspace*{\fill} Seite \arabic{mypage}
\end{slide}

%%%%%%%%%%%%%%%%%%%%%%%%%%%%%%%%%%%%%%%%%%%%%%%%%%%%%%%%%%%%%%%%%%%%%%%%

\begin{slide}{}
\footnotesize
\hspace*{1.3cm} 
$M = \Bigg\{ \{ \neg r, p, q \}, \{ \neg r, \neg p, \neg o \}, \{r, q\}, \{ \neg q, p \},$\\
\hspace*{3.3cm} 
$\{ \neg p, o \}, \{ \neg p, q \}, \{ \neg q, r, \neg o \} \Bigg\}$
    
$\textsl{Redukt}(M,p) = $\\[0.1cm]
\hspace*{1.3cm} $\Bigg\{ \verum, \{ \neg r, \neg o, \underline{\neg p} \}, \{r, q\}, \{ o, \underline{\neg p} \}, \{ q, \underline{\neg p} \}, \{ \neg q, r, \neg o \} \Bigg\}$

$\textsl{Redukt}(M, \neg p) = \Bigg\{ \{ \neg r, q, \underline{p} \}, \verum, \{r, q\}, \{ \neg q, \underline{p} \}, \{ \neg q, r, \neg o \} \Bigg\}$


Es gilt $\textsl{Redukt}(M, p) \vdash_G \falsum$ wegen
\begin{enumerate}
\item $\{ o, \underline{\neg p}\}$, $\{ \neg r, \neg o, \underline{\neg p} \}$ $\gentzen$ $\{\neg r, \underline{\neg p} \}$
\item $\{\neg r, \underline{\neg p} \}$, $\{ \neg q, r, \neg o \}$ $\gentzen$ $\{\neg q, \neg o, \underline{\neg p} \}$
\item $\{o, \underline{\neg p} \}$, $\{\neg q, \neg o, \underline{\neg p} \}$, $\gentzen$ $\{ \neg q, \underline{\neg p} \}$
\item $\{ \neg q, \underline{\neg p} \}$, $\{q, \underline{\neg p} \}$,  $\gentzen$ $\{ \underline{\neg p} \}$
\end{enumerate}

Es gilt $\textsl{Redukt}(M, \neg p) \vdash_G \falsum$ wegen
\begin{enumerate}
\item $\{ \neg q, \underline{p} \}$, $\{ \neg r, q, \underline{p} \}$ $\gentzen$ $\{\neg r, \underline{p} \}$ 
\item $\{ \neg r, \underline{p} \}$, $\{ r, q \}$ $\gentzen$ $\{ q, \underline{p} \}$
\item $\{ q, \underline{p} \}$, $\{ \neg q, \underline{p} \}$ $\gentzen$ $\{ \underline{p} \}$
\end{enumerate}

Also gilt:
 $M \gentzen \{\neg p\}$ und  $M \gentzen \{p\}$

Mit Schnitt folgt: $M \gentzen \falsum$

\vspace*{\fill}
\tiny \addtocounter{mypage}{1}
\rule{17cm}{1mm}
Widerlegungs-Vollst"andigkeit  \hspace*{\fill} Seite \arabic{mypage}
\end{slide}

%%%%%%%%%%%%%%%%%%%%%%%%%%%%%%%%%%%%%%%%%%%%%%%%%%%%%%%%%%%%%%%%%%%%%%%%

\begin{slide}{}
\normalsize
\begin{center}
Davis-Putnam Verfahren
\end{center}
\vspace{0.5cm}

\footnotesize
\textbf{Geg.}:  $K$ Menge von Klauseln

\textbf{Ges.}:  $\mathcal{I}$ aussagenlogische Belegung mit 
                \[\forall k \in K:\mathcal{I}(k) = \mathtt{true} \]

Davis-Putnam Verfahren
\begin{enumerate}
\item F�hre alle Schnitte mit Unit-Klauseln durch:

      $\schluss{\{p\} \quad\quad \{\neg p\} \cup k}{k}$ 
      \qquad
      $\schluss{k \cup \{p\} \quad\quad \{\neg p\}}{k}$ 


\item Vereinfache mit Subsumption 

       $K \cup \biggl\{ \{l\}, \{l, l_1,\cdots,l_m\}, k_1,\cdots,k_n \biggr\} \leadsto 
        \biggl\{ \{l\}, k_1,\cdots,k_n \biggr\}$
\item W�hle aussagenlogische Variable $p$ aus $K$.
      \begin{enumerate}
      \item Suche rekursiv L�sung f�r $\mathcal{I}$ f�r $K \cup \biggl\{\{p\} \biggr\}$.

      \item Falls (a) erfolglos ist:
      
            Suche rekursiv L�sung f�r $\mathcal{I}$ f�r $K \cup \biggl\{\{\neg p\}\biggr\}$
      \end{enumerate}
\end{enumerate}

\setcounter{mypage}{0}
\vspace*{\fill}
\tiny \addtocounter{mypage}{1}
\rule{17cm}{1mm}
Davis-Putnam   \hspace*{\fill} Seite \arabic{mypage}
\end{slide}

%%%%%%%%%%%%%%%%%%%%%%%%%%%%%%%%%%%%%%%%%%%%%%%%%%%%%%%%%%%%%%%%%%%%%%%%

\begin{slide}{}
\normalsize
\begin{center}
Davis-Putnam Verfahren
\end{center}
\vspace{0.5cm}

\footnotesize
Berechnung von Unit-Schnitten:

$\texttt{unitCut}: 2^\mathcal{K} \times \mathcal{L} \rightarrow 2^\mathcal{K}$ 

$\mathtt{unitCut}(K,l)$: bilde alle Unit-Schnitte mit Klausel $\{l\}$.

$\mathtt{unitCut}(K,l) =$ \\[0.3cm]
$\bigl\{\, k - \{\neg l\} \;|\; k \in K \wedge (\neg l) \in k \,\bigr\} \,\cup\, \bigl\{\, k \;|\; k \in K \wedge (\neg l) \not\in k \}$

Subsumption:

$\mathtt{unitSubsumption}: 2^\mathcal{K} \times \mathcal{L} \rightarrow 2^\mathcal{K}$.

$\mathtt{unitSubsumption}(K,l)$: Entferne alle Klauseln aus $K$, die von der Klausel $\{l\}$ subsumiert werden.

$\mathtt{unitSubsumption}(K,l) = \bigl\{\, k \;|\; k \in K \wedge l \not\in k \,\bigr\}$.


\vspace*{\fill}
\tiny \addtocounter{mypage}{1}
\rule{17cm}{1mm}
Davis-Putnam  \hspace*{\fill} Seite \arabic{mypage}
\end{slide}

%%%%%%%%%%%%%%%%%%%%%%%%%%%%%%%%%%%%%%%%%%%%%%%%%%%%%%%%%%%%%%%%%%%%%%%%

\end{document}

%%% Local Variables: 
%%% mode: latex
%%% TeX-master: t
%%% End: 
