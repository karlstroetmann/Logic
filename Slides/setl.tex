\documentclass{slides}
\usepackage[latin1]{inputenc}
\usepackage{german}
\usepackage{epsfig}
\usepackage{amssymb}

\pagestyle{empty}
\setlength{\textwidth}{18cm}
\setlength{\textheight}{24cm}
\setlength{\topmargin}{0cm}
\setlength{\headheight}{0cm}
\setlength{\headsep}{0cm}
\setlength{\topskip}{0.2cm}
\setlength{\oddsidemargin}{0.5cm}
\setlength{\evensidemargin}{0.5cm}

\newcommand{\is}{\;|\;}

\newcounter{mypage}

\begin{document}

%%%%%%%%%%%%%%%%%%%%%%%%%%%%%%%%%%%%%%%%%%%%%%%%%%%%%%%%%%%%%%%%%%%%%%%%

\begin{slide}{}
\normalsize
\begin{center}
Mengen-Operatoren
\end{center}
\vspace{0.5cm}

\footnotesize
\begin{enumerate}
\item Vereinigung

      \texttt{c := a + b;}

      entspricht: $c = a \cup b$ 
\item Durchschnitt

      \texttt{c := a * b;}

      entspricht: $c = a \cap b$ 
\item Mengen-Differenz

      \texttt{c := a - b;}

      entspricht: $c = a \backslash b$
\item Potenz-Menge 
  
      \texttt{c := pow a;}

      entspricht: $c := 2^a$
\end{enumerate}

\setcounter{page}{1}
\vspace*{\fill}
\tiny \addtocounter{mypage}{1}
\rule{18cm}{1mm}
Setl2  \hspace*{\fill} Seite \arabic{mypage}
\end{slide}

%%%%%%%%%%%%%%%%%%%%%%%%%%%%%%%%%%%%%%%%%%%%%%%%%%%%%%%%%%%%%%%%%%%%%%%%

\begin{slide}{}
\normalsize
\begin{center}
Setl2 --- Definition von Mengen
\end{center}
\vspace{0.5cm}

\footnotesize
\begin{enumerate}
\item Explizites Auflisten aller Elemente 

      \texttt{\{ 1, 5, 7 \}}
\item Arithmetische Aufz�hlung

      \texttt{\{ 1 .. 100 \}}

      entspricht: \quad $\{ n \in \mathbb{N} \mid 1 \leq n \wedge n \leq 100 \}$
\item Arithmetische Aufz�hlung mit konstanter Schrittweite

      \texttt{\{ 3, 8 .. 100 \}}

      entspricht: \quad $\{ 3 + n \cdot 5 \mid n \in \mathbb{N} \;\wedge\; 3 + n \cdot 5 \leq 100 \}$
\item Definition von Mengen durch Iteratoren (Bildmengen)
      
      Beispiel: Menge aller Produkte $n \cdot m$ mit $2 \leq n, m \leq 100$

      \texttt{\{ n * m : n in \{2..10\}, m in \{2..10\} \}}
\item Mengenbildung durch Auswahl

      Beispiel: Menge aller Teiler von $p$

      \texttt{\{ x : x in \{ 1 .. p \} | p mod x = 0 \}}

      Einfachere Schreibweise:

      \texttt{\{ x in \{1..p\} | p mod x = 0 \}}
      
      entspricht: 
      $\Biggl\{ x \in \{ n \in \mathbb{N} \mid 1 \leq n \wedge n \leq p \} \;\mid\; 
               p \;\mathop{\mathtt{mod}}\; x = 0 \Biggr\}$
\end{enumerate}


\setcounter{page}{1}
\vspace*{\fill}
\tiny \addtocounter{mypage}{1}
\rule{18cm}{1mm}
Setl2  \hspace*{\fill} Seite \arabic{mypage}
\end{slide}

%%%%%%%%%%%%%%%%%%%%%%%%%%%%%%%%%%%%%%%%%%%%%%%%%%%%%%%%%%%%%%%%%%%%%%%%

\begin{slide}{}
\normalsize
\begin{center}
Setl2 --- Definition von Mengen
\end{center}
\vspace{0.5cm}

\footnotesize
\begin{enumerate}
\setcounter{enumi}{5}
\item Mengenbildung durch Iteratoren und zus�tzliche Auswahl

      Beispiel: Menge aller Produkte $m \cdot n$ f�r die gilt
      \\[0.1cm]
      \hspace*{1.3cm} $m + n = 9$

      \texttt{\{ n*m : n in \{ 0 .. 9 \}, m in \{ 0 .. 9 \} | m+n = 9 \}}

      allgemeine Form

      \texttt{M := \{ $\textsl{expr}(x_1,\cdots,x_n)$ : $x_1$ in $S_1$, $\cdots$, $x_n \in S_n$ \\
      \hspace*{6.6cm} | $\textsl{cond}(x_1,\cdots,x_n)$ \}}
      \begin{enumerate}
      \item $\textsl{expr}(x_1,\cdots,x_n)$: 

            Ausdruck, in dem Variablen $x_1$, $\cdots$ $x_n$ vorkommen.
      \item $S_1$, $\cdots$ $S_n$: Mengen
      \item $\textsl{cond}(x_1, \cdots, x_n)$: 

            Ausdruck, der  ``\texttt{true}'' oder ``\texttt{false}'' ergibt.
      \item $M$: Menge, die alle Elemente  $\textsl{expr}(x_1,\cdots,x_n)$ 
            enth�lt, f�r die $x_i \in S_i$ gilt und $\textsl{cond}(x_1, \cdots, x_n)$
            den Wert ``\texttt{true}'' liefert.
      \end{enumerate}
\end{enumerate}

\setcounter{page}{1}
\vspace*{\fill}
\tiny \addtocounter{mypage}{1}
\rule{18cm}{1mm}
Setl2  \hspace*{\fill} Seite \arabic{mypage}
\end{slide}

%%%%%%%%%%%%%%%%%%%%%%%%%%%%%%%%%%%%%%%%%%%%%%%%%%%%%%%%%%%%%%%%%%%%%%%%

\begin{slide}{}
\normalsize
\begin{center}
\end{center}
\vspace{0.5cm}

\footnotesize

\setcounter{page}{1}
\vspace*{\fill}
\tiny \addtocounter{mypage}{1}
\rule{18cm}{1mm}
Setl2  \hspace*{\fill} Seite \arabic{mypage}
\end{slide}

%%%%%%%%%%%%%%%%%%%%%%%%%%%%%%%%%%%%%%%%%%%%%%%%%%%%%%%%%%%%%%%%%%%%%%%%

\begin{slide}{}
\normalsize
\begin{center}
\end{center}
\vspace{0.5cm}

\footnotesize

\setcounter{page}{1}
\vspace*{\fill}
\tiny \addtocounter{mypage}{1}
\rule{18cm}{1mm}
Setl2  \hspace*{\fill} Seite \arabic{mypage}
\end{slide}

%%%%%%%%%%%%%%%%%%%%%%%%%%%%%%%%%%%%%%%%%%%%%%%%%%%%%%%%%%%%%%%%%%%%%%%%

\begin{slide}{}
\normalsize
\begin{center}
\end{center}
\vspace{0.5cm}

\footnotesize

\setcounter{page}{1}
\vspace*{\fill}
\tiny \addtocounter{mypage}{1}
\rule{18cm}{1mm}
Setl2  \hspace*{\fill} Seite \arabic{mypage}
\end{slide}

%%%%%%%%%%%%%%%%%%%%%%%%%%%%%%%%%%%%%%%%%%%%%%%%%%%%%%%%%%%%%%%%%%%%%%%%

\begin{slide}{}
\normalsize
\begin{center}
\end{center}
\vspace{0.5cm}

\footnotesize

\setcounter{page}{1}
\vspace*{\fill}
\tiny \addtocounter{mypage}{1}
\rule{18cm}{1mm}
Setl2  \hspace*{\fill} Seite \arabic{mypage}
\end{slide}

%%%%%%%%%%%%%%%%%%%%%%%%%%%%%%%%%%%%%%%%%%%%%%%%%%%%%%%%%%%%%%%%%%%%%%%%

\begin{slide}{}
\normalsize
\begin{center}
\end{center}
\vspace{0.5cm}

\footnotesize

\setcounter{page}{1}
\vspace*{\fill}
\tiny \addtocounter{mypage}{1}
\rule{18cm}{1mm}
Setl2  \hspace*{\fill} Seite \arabic{mypage}
\end{slide}

%%%%%%%%%%%%%%%%%%%%%%%%%%%%%%%%%%%%%%%%%%%%%%%%%%%%%%%%%%%%%%%%%%%%%%%%

\begin{slide}{}
\normalsize
\begin{center}
\end{center}
\vspace{0.5cm}

\footnotesize

\setcounter{page}{1}
\vspace*{\fill}
\tiny \addtocounter{mypage}{1}
\rule{18cm}{1mm}
Setl2  \hspace*{\fill} Seite \arabic{mypage}
\end{slide}

%%%%%%%%%%%%%%%%%%%%%%%%%%%%%%%%%%%%%%%%%%%%%%%%%%%%%%%%%%%%%%%%%%%%%%%%

\begin{slide}{}
\normalsize
\begin{center}
\end{center}
\vspace{0.5cm}

\footnotesize

\setcounter{page}{1}
\vspace*{\fill}
\tiny \addtocounter{mypage}{1}
\rule{18cm}{1mm}
Setl2  \hspace*{\fill} Seite \arabic{mypage}
\end{slide}

%%%%%%%%%%%%%%%%%%%%%%%%%%%%%%%%%%%%%%%%%%%%%%%%%%%%%%%%%%%%%%%%%%%%%%%%

\begin{slide}{}
\normalsize
\begin{center}
\end{center}
\vspace{0.5cm}

\footnotesize

\setcounter{page}{1}
\vspace*{\fill}
\tiny \addtocounter{mypage}{1}
\rule{18cm}{1mm}
Setl2  \hspace*{\fill} Seite \arabic{mypage}
\end{slide}

%%%%%%%%%%%%%%%%%%%%%%%%%%%%%%%%%%%%%%%%%%%%%%%%%%%%%%%%%%%%%%%%%%%%%%%%

\begin{slide}{}
\normalsize
\begin{center}
\end{center}
\vspace{0.5cm}

\footnotesize

\setcounter{page}{1}
\vspace*{\fill}
\tiny \addtocounter{mypage}{1}
\rule{18cm}{1mm}
Setl2  \hspace*{\fill} Seite \arabic{mypage}
\end{slide}

%%%%%%%%%%%%%%%%%%%%%%%%%%%%%%%%%%%%%%%%%%%%%%%%%%%%%%%%%%%%%%%%%%%%%%%%

\begin{slide}{}
\normalsize
\begin{center}
\end{center}
\vspace{0.5cm}

\footnotesize

\setcounter{page}{1}
\vspace*{\fill}
\tiny \addtocounter{mypage}{1}
\rule{18cm}{1mm}
Setl2  \hspace*{\fill} Seite \arabic{mypage}
\end{slide}



\end{document}

%%% Local Variables: 
%%% mode: latex
%%% TeX-master: t
%%% End: 
