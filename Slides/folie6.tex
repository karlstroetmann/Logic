\documentclass{slides}
\usepackage[latin1]{inputenc}
\usepackage{german}
\usepackage{epsfig}
\usepackage{amssymb}

\pagestyle{empty}
\setlength{\textwidth}{17cm}
\setlength{\textheight}{24cm}
\setlength{\topmargin}{0cm}
\setlength{\headheight}{0cm}
\setlength{\headsep}{0cm}
\setlength{\topskip}{0.2cm}
\setlength{\oddsidemargin}{0.5cm}
\setlength{\evensidemargin}{0.5cm}

\newcommand{\is}{\;|\;}
\newcommand{\schluss}[2]{\frac{\displaystyle\quad \rule[-8pt]{0pt}{18pt}#1 \quad}{\displaystyle\quad \rule{0pt}{14pt}#2 \quad}}
\newcommand{\vschlus}[1]{{\displaystyle\rule[-6pt]{0pt}{12pt} \atop \rule{0pt}{10pt}#1}}
\newcommand{\verum}{\top}
\newcommand{\falsum}{\bot}
\newcommand{\var}{\textsl{Var}}
\newcommand{\struct}{\mathcal{S}}
\newcommand{\el}{\!\in\!}
\newcommand{\FV}{\textsl{FV}}
\newcommand{\BV}{\textsl{BV}}
\newcommand{\gentzen}{\vdash_\mathcal{G}}
\newcommand{\komplement}[1]{\overline{#1}}

\newcounter{mypage}

\begin{document}
\begin{center}
Pr�dikatenlogik --- Normalformen
\end{center}
\vspace*{0.5cm}

\footnotesize
\textbf{Definition}: $\Sigma$-Formeln $f$ und $g$  \emph{�quivalent} g.d.w.  \\[0.3cm]
\hspace*{1.3cm} $\models f \leftrightarrow g$ 

\textbf{Beispiele}:
\begin{enumerate}
\item $\models \big(\neg\forall x: f\big) \leftrightarrow \big(\exists x: \neg f\big)$
\item $\models \big(\neg\exists x: f\big) \leftrightarrow \big(\forall x: \neg f\big)$
\item $\models \big(\forall x: f) \wedge \big(\forall x: g) \leftrightarrow \big(\forall x: f \wedge g\big)$
\item $\models \big(\exists x: f) \vee \big(\exists x: g) \leftrightarrow \big(\exists x: f \vee g\big)$
\item $\models \big(\forall x: \forall y:f \big) \leftrightarrow \big(\forall y: \forall x: f \big)$
\item $\models \big(\exists x: \exists y:f \big) \leftrightarrow \big(\exists y: \exists x: f \big)$
\item Falls  $x \not\in \FV(f)$, gilt 
  \begin{enumerate}
  \item $\models  \big(\forall x: f) \leftrightarrow f$
  \item $\models  \big(\exists x: f) \leftrightarrow f$.
  \item $\models \big(\forall x: g) \vee f \leftrightarrow \big(\forall x: g \vee f\big)$
  \item $\models f \vee \big(\forall x: g) \leftrightarrow \big(\forall x: f \vee g\big)$
  \item $\models \big(\exists x: g) \wedge f \leftrightarrow \big(\exists x: g \wedge f\big)$
  \item $\models f \wedge \big(\exists x: g) \leftrightarrow \big(\exists x: f \wedge g\big)$
  \end{enumerate}
\end{enumerate}


\vspace*{\fill}
\tiny \addtocounter{mypage}{1} 
\rule{17cm}{1mm}
Pr�dikatenlogik --- Normalformen \hspace*{\fill} Seite \arabic{mypage}

%%%%%%%%%%%%%%%%%%%%%%%%%%%%%%%%%%%%%%%%%%%%%%%%%%%%%%%%%%%%%%%%%%%%%%%%

\begin{slide}{}
\normalsize
\begin{center}
Pr�nexe Normalform
\end{center}
\vspace{0.5cm}

\footnotesize
\textbf{Definition}: $f$ hat pr�nexe Normalform g.d.w. \\[0.3cm]
\hspace*{1.3cm} $f = \mathcal{Q}_1 x_1 \cdots \mathcal{Q}_n x_n: g$ \\[0.3cm]
mit $\mathcal{Q} \in \{ \forall, \exists \}$ und $g$ quantorenfrei.
\vspace{0.5cm}

\textbf{Beispiel}: \\[0.3cm]
\hspace*{1.3cm} $\forall x: \exists y: x * y = 1$

\textbf{Negatives Beispiel}: Sei $f$ quantorenfrei \\[0.3cm]
\hspace*{1.3cm} $\big(\forall x: f\big) \rightarrow \big(\exists y: f \big)$ \\[0.3cm]
ist \underline{nicht} in pr�nexer Normalform
\vspace{0.5cm}

\textbf{Definition}: (Substitution) Sei
\begin{enumerate}
\item $x$ Variable 
\item $t$ Term 
\item $f$ Formel  
\end{enumerate}
\hspace*{1.3cm} $f[x/t]$ := $x$ wird �berall in $f$ durch $t$ ersetzt

\textbf{Beispiele}:

$\Bigg(\neg \textsl{empty}(\textsl{push}(x_1, s_1))\Bigg)[x_1/x_3] \;=\; \neg \textsl{empty}(\textsl{push}(x_3, s_1))$

$\Bigg(\textsl{empty}(s_1)\Bigg)[s_1/\textsl{push}(x, s_2)] \;=\;\textsl{empty}(\textsl{push}(x, s_2))$


\vspace*{\fill}
\tiny \addtocounter{mypage}{1}
\rule{17cm}{1mm}
Pr�dikatenlogik --- Normalformen  \hspace*{\fill} Seite \arabic{mypage}
\end{slide}

%%%%%%%%%%%%%%%%%%%%%%%%%%%%%%%%%%%%%%%%%%%%%%%%%%%%%%%%%%%%%%%%%%%%%%%%

\begin{slide}{}
\normalsize
\begin{center}
Umformung in pr�nexe Normalform
\end{center}
\vspace{0.5cm}

\footnotesize
Verfahren zur Umformung in pr�nexe Normalform
\begin{enumerate}
\item Beseitige ``$\leftrightarrow$'' und ``$\rightarrow$''
\item Schiebe Negationen nach innen: 
      \begin{enumerate}
      \item $\neg\Bigg(\forall x: f\Bigg) \quad\leadsto\quad \exists x: \neg f$
      \item $\neg\Bigg(\exists x: f\Bigg) \quad\leadsto\quad \forall x: \neg f$
      \end{enumerate}
\item Schiebe Konjunktionen nach innen: 
      \begin{enumerate}
      \item $\Bigg(\forall x: f\Bigg) \wedge \Bigg(\forall x: g\Bigg) \quad\leadsto\quad \Bigg(\forall x: f \wedge g\Bigg)$
      \item $\Bigg(\exists x: f\Bigg) \wedge g \quad\leadsto\quad \Bigg(\exists x: f \wedge g\Bigg)$

            falls $x \not\in \FV(g) \cup \BV(g)$
      \end{enumerate}
\item Schiebe Disjunktionen nach innen: 
      \begin{enumerate}
      \item $\Bigg(\exists x: f\Bigg) \vee \Bigg(\exists x: g\Bigg) \quad\leadsto\quad \Bigg(\exists x: f \vee g\Bigg)$
      \item $\Bigg(\forall x: f\Bigg) \vee g \quad\leadsto\quad \Bigg(\forall x: f \vee g\Bigg)$

            falls $x \not\in \FV(g) \cup \BV(g)$
      \end{enumerate}

\item Umbenennung: Sei $y \not\in \FV(f) \cup \BV(f)$
  \begin{enumerate}
  \item $\forall x: f \quad\leadsto\quad \forall y: f[x/y]$
  \item $\exists x: f \quad\leadsto\quad \exists y: f[x/y]$
  \end{enumerate}
\end{enumerate}

\vspace*{\fill}
\tiny \addtocounter{mypage}{1}
\rule{17cm}{1mm}
Pr�dikatenlogik --- Normalformen  \hspace*{\fill} Seite \arabic{mypage}
\end{slide}

%%%%%%%%%%%%%%%%%%%%%%%%%%%%%%%%%%%%%%%%%%%%%%%%%%%%%%%%%%%%%%%%%%%%%%%%

\begin{slide}{}
\normalsize
\begin{center}
Erf�llbarkeits-�quivalenz
\end{center}
\vspace{0.5cm}

\footnotesize
\textbf{Gegeben}: Formeln $f$ und $g$ 

\textbf{Definition}: $f$ und $g$ \emph{erf�llbarkeits-�quivalent} g.d.w.
\begin{enumerate}
\item  $f$, $g$ beide erf�llbar oder
\item  $f$, $g$ beide unerf�llbar.
\end{enumerate}
\textbf{Schreibweise}:  $f \approx_e g$.
\vspace{0.5cm}

\textbf{Beispiel:}  Es sei
\begin{enumerate}
\item  $p\in \mathcal{P}$ mit $\textsl{arity}(p) = 2$.
\item  $s\in \mathcal{F}$ mit $\textsl{arity}(s) = 1$.  
\end{enumerate}
Dann gilt: \\[0.3cm]
\hspace*{1.3cm} $\Bigg(\forall x: \exists y: p(x,y) \Bigg) \;\approx_e\; \Bigg(\forall x: p(x,s(x)) \Bigg)$

\textbf{Bemerkung}: Falls $f \approx_e g$ ist, so gilt h�ufig
\begin{enumerate}
\item $f$ ist $\Sigma_1$-Formel,
\item $g$ ist $\Sigma_2$-Formel und
\item $\Sigma_2$ ist \emph{Erweiterung} von $\Sigma_1$.

     ($\Sigma_2$ enth�lt mehr Funktions-Zeichen als $\Sigma_1$)
\end{enumerate}

\vspace*{\fill}
\tiny \addtocounter{mypage}{1}
\rule{17cm}{1mm}
Pr�dikatenlogik --- Normalformen  \hspace*{\fill} Seite \arabic{mypage}
\end{slide}

%%%%%%%%%%%%%%%%%%%%%%%%%%%%%%%%%%%%%%%%%%%%%%%%%%%%%%%%%%%%%%%%%%%%%%%%

\begin{slide}{}
\normalsize
\begin{center}
Skolemisierung
\end{center}
\vspace{0.5cm}

\footnotesize
\textbf{Satz}: Sei
\begin{enumerate}
\item $\Sigma = \langle \mathcal{V}, \mathcal{F}, \mathcal{P}, \textsl{arity} \rangle$ Signatur,
\item $s \not\in \mathcal{F}$  \underline{neues} Funktions-Zeichen,
\item $\Sigma' = \langle \mathcal{V}, \mathcal{F} \cup \{s\}, \mathcal{P}, \textsl{arity}\,' \rangle$ 

      mit $\textsl{arity}\,'(s) = n$ und $\textsl{arity}\,'(f) = \textsl{arity}(f)$ f�r $f\in \mathcal{F}$.
\item $f$ und $h$ seien $\Sigma$--Formeln.
\end{enumerate}
Dann gilt \\[0.3cm]
\hspace*{2.95cm} $\Bigg(\forall x_1, \cdots, x_n: \exists y: f\Bigg) \wedge h$ \\[0.1cm]
\hspace*{1.3cm} $\approx_e$ \quad $\Bigg(\forall x_1, \cdots, x_n: f[y/s(x_1,\cdots,x_n)]\Bigg) \wedge h$.

Das neue Funktions--Zeichen $s$ hei�t \\
\emph{Skolem--Funktions--Zeichen}.
\vspace{0.5cm}

\textbf{Beispiel}: Gruppentheorie \\[0.3cm]
\hspace*{1.3cm} $\Bigg(\forall x_1: \exists x_2: x_2 * x_1 = 1\Bigg) \;\approx_e\; \Bigg(\forall x_1: i(x_1) * x_1 = e\Bigg)$

Skolem--Funktions--Zeichen: $i$

\vspace*{\fill}
\tiny \addtocounter{mypage}{1}
\rule{17cm}{1mm}
Pr�dikatenlogik --- Normalformen  \hspace*{\fill} Seite \arabic{mypage}
\end{slide}

%%%%%%%%%%%%%%%%%%%%%%%%%%%%%%%%%%%%%%%%%%%%%%%%%%%%%%%%%%%%%%%%%%%%%%%%

\begin{slide}{}
\normalsize
\begin{center}
�berf�hrung in Skolem--Normalform
\end{center}
\vspace{0.5cm}

\footnotesize
\textbf{Gegeben}: $f \in \mathcal{F}$

\textbf{Ziel}:  �berf�hrung von $f$ in Skolem--Normalform

\textbf{Vorgehen}: 
\begin{enumerate}
\item �berf�hre $f$ in pr�nexe Normalform: \\[0.3cm]
      \hspace*{1.3cm} $f \leftrightarrow \mathcal{Q}_1 x_1: \cdots \mathcal{Q}_l x_l: g$  \\[0.3cm]
      mit $\mathcal{Q} \in \{ \forall, \exists \}$ und $g$ quantorenfrei. 
\item Falls $f$ die Form \\[0.3cm]
      \hspace*{1.3cm} $\Bigg(\forall x_1: \cdots\forall x_n: \exists y: h\Bigg)$ \\[0.3cm]
      hat, dann
      \begin{enumerate}
      \item w�hle neues Funktions--Zeichen $s$ mit \\[0.1cm]
            \hspace*{1.3cm}  $\textsl{arity}(s) = n$,
      \item ersetze $y$ in $h$ durch $s(x_1,\cdots,x_n)$.
      \end{enumerate}
      Resultat: \\[0.3cm]
      \hspace*{1.3cm} $f \approx_e \forall x_1: \cdots\forall x_n: h[y/s(x_1,\cdots,x_n)]$.
\item F�hre Schritt 2 solange durch, bis alle Auftreten von ``$\exists$'' eleminiert sind.
\end{enumerate}
\textbf{Ergebnis}: $f \approx_e \forall z_1: \cdots \forall z_m: k$ \\[0.3cm]
mit $k$ quantorenfrei.

\vspace*{\fill}
\tiny \addtocounter{mypage}{1}
\rule{17cm}{1mm}
Pr�dikatenlogik --- Normalformen  \hspace*{\fill} Seite \arabic{mypage}
\end{slide}

%%%%%%%%%%%%%%%%%%%%%%%%%%%%%%%%%%%%%%%%%%%%%%%%%%%%%%%%%%%%%%%%%%%%%%%%

\begin{slide}{}
\normalsize
\begin{center}
Klausel--Normalform
\end{center}
\vspace{0.5cm}

\footnotesize
\textbf{Gegeben}: $f = \forall z_1: \cdots \forall z_m: g$ \\[0.3cm]
mit $k$ quantorenfrei.

\textbf{Ziel}: �berf�hre $f$ in Klausel--Normalform

\textbf{Vorgehen}:
\begin{enumerate}
\item Bringe $g$ in konjunktive Normalform: \\[0.3cm]
      \hspace*{1.3cm} $g \leftrightarrow k_1 \wedge \cdots \wedge k_n$ \\[0.3cm]
      mit \quad  $k_i = l^{(i)}_1 \vee \cdots \vee l^{(i)}_{m(i)}$     \\[0.1cm]
      und \quad  $l^{(i)}_j$ Literal.

      Literal: $p(s_1,\cdots,s_l)$ oder $\neg p(s_1,\cdots,s_l)$.
\item Verteile Allquantoren auf $k_i$: \\[0.3cm]
      $\Bigg(\forall x:  k_1 \wedge k_2\Bigg) \quad \leftrightarrow\quad
      \Bigg(\forall x: k_1\Bigg) \wedge \Bigg(\forall x: k_2 \Bigg)$
\item Beseitige redundante Allquantoren: \\[0.3cm]
      \hspace*{1.3cm} 
      $\Bigg(\forall x: f\Bigg) \quad \leftrightarrow\quad f$ \quad falls $x\not\in\FV(f)$.
\end{enumerate}
\textbf{Ergebnis}: $f \leftrightarrow \forall(k_1) \wedge \cdots \wedge \forall(k_n)$

\textbf{Definition}: Sei $\{x_1,\cdots,x_n\} := \FV(f)$. \\[0.3cm]
\hspace*{1.3cm} $\forall(f) := \forall x_1: \forall x_2: \cdots \forall x_n: f$


\vspace*{\fill}
\tiny \addtocounter{mypage}{1}
\rule{17cm}{1mm}
Pr�dikatenlogik --- Normalformen  \hspace*{\fill} Seite \arabic{mypage}
\end{slide}

%%%%%%%%%%%%%%%%%%%%%%%%%%%%%%%%%%%%%%%%%%%%%%%%%%%%%%%%%%%%%%%%%%%%%%%%

\begin{slide}{}
\normalsize
\begin{center}

\end{center}
\vspace{0.5cm}

\footnotesize
\textbf{Gegeben}: $M \subseteq \mathbb{F}_\Sigma$ und $f \in \mathbb{F}_\Sigma$

\textbf{Frage}: Gilt $M \models f$?.

\textbf{Vorgehen}: 
\begin{enumerate}
\item Setze $N := M \cup \{ \neg f \}$.  \quad Dann: \\[0.3cm]
      \hspace*{1.3cm} $M \models f$ \quad g.d.w. \quad $N \models \falsum$.
\item Sei $N = \{ g_1, \cdots, g_n \}$.  
  \begin{enumerate}
  \item  �berf�hre $g_1 \wedge \cdots \wedge g_n$ in Skolem--Normalform $h$ \\[0.3cm]
         \hspace*{1.3cm} 
         $g_1 \wedge \cdots \wedge g_n \approx_e h$
  \item �berf�hre $h$ in       in Klausel--Normalform $k_1 \wedge \cdots \wedge k_m$. \\[0.1cm]
        \hspace*{1.3cm} $h \leftrightarrow k_1 \wedge \cdots \wedge k_m$
  \end{enumerate}
  Jetzt gilt:
  $N \models \falsum$ \quad g.d.w. \quad $\{k_1, \cdots, k_m\} \models \falsum$
\item �berpr�fe, ob $\{k_1, \cdots, k_m\} \vdash \falsum$ gilt.

      Dabei ist ``$\vdash$'' der \emph{Robinson--Kalk�l}.  Es gilt: \\[0.3cm]
$$
\begin{array}{ll}
                & \{k_1, \cdots, k_m\} \vdash \falsum   \\[0.3cm]
\mathrm{g.d.w.} & \{k_1, \cdots, k_m\} \models \falsum    \\[0.3cm]
\mathrm{g.d.w.} & M \models f. 
\end{array}
$$
\end{enumerate}

\textbf{Fehlt noch}: Definition des Robinson--Kalk�ls f�r Formeln in 
Klausel--Normalform

\vspace*{\fill}
\tiny \addtocounter{mypage}{1}
\rule{17cm}{1mm}
Pr�dikatenlogik --- Normalformen  \hspace*{\fill} Seite \arabic{mypage}
\end{slide}

%%%%%%%%%%%%%%%%%%%%%%%%%%%%%%%%%%%%%%%%%%%%%%%%%%%%%%%%%%%%%%%%%%%%%%%%

\begin{slide}{}
\normalsize
\begin{center}
Skolem-Normal-Form (Beispiel)
\end{center}
\vspace{0.5cm}

\footnotesize
Zeige, das gilt:
\\[0.3cm]
\hspace*{1.3cm} 
$\models \big(\exists x\colon \forall y\colon  p(x,y)\big) \rightarrow \big(\forall y\colon \exists x\colon p(x,y)\big)$ \\[0.3cm]
Dies ist �quivalent zu   \\[0.3cm]
\hspace*{1.3cm} 
$\Bigg\{ \neg \Bigg(\big(\exists x\colon \forall y\colon  p(x,y)\big) \rightarrow
\big(\forall y\colon \exists x\colon p(x,y)\big)\Bigg)\Bigg\} \models \falsum$ \\[0.3cm]
Berechnung der pr�nexen NF:
$$
\begin{array}{ll}
                  & \neg \Bigg(\big(\exists x\colon \forall y\colon  p(x,y)\big) \rightarrow \big(\forall y\colon \exists x\colon p(x,y)\big)\Bigg) \\[0.3cm]
  \leftrightarrow & \neg \Bigg(\neg \big(\exists x\colon \forall y\colon  p(x,y)\big) \vee \big(\forall y\colon \exists x\colon p(x,y)\big)\Bigg) \\[0.3cm]
  \leftrightarrow &                \big(\exists x\colon \forall y\colon  p(x,y)\big) \wedge \neg \big(\forall y\colon \exists x\colon p(x,y)\big) \\[0.3cm]
  \leftrightarrow &\big(\exists x\colon \forall y\colon  p(x,y)\big) \wedge  \big(\exists y\colon  \neg \exists x\colon p(x,y)\big) \\[0.3cm]
  \leftrightarrow &\big(\exists x\colon \forall y\colon  p(x,y)\big) \wedge  \big(\exists y\colon  \forall x\colon \neg p(x,y)\big) \\[0.3cm]
  \leftrightarrow &\exists v\colon  \Bigg( \big(\exists x\colon \forall y\colon  p(x,y)\big) \wedge  \big(\forall u\colon \neg p(u,v)\big) \Bigg)\\[0.3cm]
  \leftrightarrow &\exists v\colon  \exists x\colon  \Bigg( \big(\forall y\colon  p(x,y)\big) \wedge \big(\forall u\colon \neg p(u,v)\big) \Bigg)\\[0.3cm]
  \leftrightarrow &\exists v\colon  \exists x\colon \forall y\colon \Bigg( p(x,y) \wedge \big(\forall u\colon \neg p(u,v)\big) \Bigg)\\[0.3cm]
  \leftrightarrow &\exists v\colon  \exists x\colon \forall y\colon \forall u\colon \Bigg( p(x,y) \wedge \neg p(u,v) \Bigg)\\[0.3cm]
 \approx_e & \exists x\colon \forall y\colon \forall u\colon \Bigg( p(x,y) \wedge \neg p(u,s_1) \Bigg)\\[0.3cm]
 \approx_e & \forall y\colon \forall u\colon \Bigg( p(s_2,y) \wedge \neg p(u,s_1) \Bigg)\\[0.3cm]
  \leftrightarrow & p(s_2,y) \wedge \neg p(u,s_1) \\[0.3cm]
  \leftrightarrow & \Bigg\{ \big\{ p(s_2,y) \big\}, \big\{\neg p(u,s_1)\big\}\Bigg\}
\end{array}
$$


\vspace*{\fill}
\tiny \addtocounter{mypage}{1}
\rule{17cm}{1mm}
Pr�dikatenlogik --- Normalformen  \hspace*{\fill} Seite \arabic{mypage}
\end{slide}
 
\end{document}

%%% Local Variables: 
%%% mode: latex
%%% TeX-master: t
%%% End: 
