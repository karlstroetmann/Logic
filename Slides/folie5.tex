\documentclass{slides}
\usepackage[latin1]{inputenc}
\usepackage{ngerman}
\usepackage{epsfig}
\usepackage{amssymb}

\pagestyle{empty}
\setlength{\textwidth}{17cm}
\setlength{\textheight}{24cm}
\setlength{\topmargin}{0cm}
\setlength{\headheight}{0cm}
\setlength{\headsep}{0cm}
\setlength{\topskip}{0.2cm}
\setlength{\oddsidemargin}{0.5cm}
\setlength{\evensidemargin}{0.5cm}

\newcommand{\is}{\;|\;}
\newcommand{\schluss}[2]{\frac{\displaystyle\quad \rule[-8pt]{0pt}{18pt}#1 \quad}{\displaystyle\quad \rule{0pt}{14pt}#2 \quad}}
\newcommand{\vschlus}[1]{{\displaystyle\rule[-6pt]{0pt}{12pt} \atop \rule{0pt}{10pt}#1}}
\newcommand{\verum}{\top}
\newcommand{\falsum}{\bot}
\newcommand{\var}{\textsl{Var}}
\newcommand{\struct}{\mathcal{S}}
\newcommand{\el}{\!\in\!}
\newcommand{\FV}{\textsl{FV}}
\newcommand{\BV}{\textsl{BV}}
\newcommand{\gentzen}{\vdash_\mathcal{G}}
\newcommand{\komplement}[1]{\overline{#1}}

\newcounter{mypage}

\begin{document}
\begin{center}
Pr"adikatenlogik -- Semantik
\end{center}
\vspace*{0.5cm}

\footnotesize
\textbf{Definition}: \\
Sei Signatur $\Sigma = \langle \mathbb{T}, \textsl{Fz}, \textsl{Pz}, \mathtt{sign}_{F}, \mathtt{sign}_{P}, \mathcal{V}, \textsl{var} \rangle$ gegeben. 

Eine \emph{$\Sigma$-Struktur} $\struct$ ist Abbildung mit :
\begin{enumerate}
\item $\tau \mapsto \tau^{\struct}$ \quad f"ur alle  $\tau \el \mathbb{T}$.

       $\mathbb{B}^{\struct} = \{ \mathtt{true}, \mathtt{false} \}$..
\item Sei $f \el \textsl{Fz}$ mit  $f: \sigma_1 \times \cdots \times \sigma_n \rightarrow \tau$

      Dann: \quad $f^\struct: \sigma_1^{\struct} \times \cdots \times \sigma_n^{\struct} \rightarrow \tau^{\struct}$ 
\item Sei $p \el \textsl{Pz}$ mit $p: \sigma_1 \times \cdots \times \sigma_n \rightarrow \mathbb{B}$

      Dann: \quad $p^\struct: \sigma_1^{\struct} \times \cdots \times \sigma_n^{\struct} \rightarrow \mathbb{B}^{\struct}$ 
\end{enumerate}

\textbf{Beispiel}: Sei $\Sigma_G$ Signatur zur Gruppen-Theorie

Definiere $\Sigma_G$ Struktur $\mathcal{Z}$ durch:
\begin{enumerate}
\item $G^\mathcal{Z} := \mathbb{Z}$
\item $1^\mathcal{Z} := 0$ mit $0 \in \mathbb{Z}$
\item $*^\mathcal{Z} := +$ mit $+: \mathbb{Z} \times \mathbb{Z} \rightarrow \mathbb{Z}$
\item $=^\mathcal{Z} \;:=\; =$ mit $= : \mathbb{Z} \times \mathbb{Z} \rightarrow \mathbb{B}$
\end{enumerate}

\textbf{Aufgabe}: Definieren Sie eine Struktur $\mathcal{G}$ zur Signatur
$\Sigma_G$ der Gruppen-Theorie, so da\3 $G^\mathcal{G}$ endlich ist.


\vspace*{\fill}
\tiny \addtocounter{mypage}{1} 
\rule{17cm}{1mm}
Pr"adikatenlogik -- Semantik \hspace*{\fill} Seite \arabic{mypage}

%%%%%%%%%%%%%%%%%%%%%%%%%%%%%%%%%%%%%%%%%%%%%%%%%%%%%%%%%%%%%%%%%%%%%%%%

\begin{slide}{}
\normalsize
\begin{center}
Variablen--Interpretation
\end{center}
\vspace{0.5cm}

\footnotesize
\textbf{Definition}: Sei
\begin{enumerate}
\item $\Sigma = \langle \mathbb{T}, \textsl{Fz}, \textsl{Pz}, \mathtt{sign}_{F}, \mathtt{sign}_{P}, \mathcal{V}, \textsl{var} \rangle$: Signatur 
\item $\struct$: $\Sigma$--Struktur
\end{enumerate}
Dann ist \\[0.3cm]
\hspace*{1.3cm} $\mathcal{I}: \mathcal{V} \rightarrow \bigcup_{\tau \in \mathbb{T}} \tau^\struct$, \\[0.3cm]
$\struct$--\emph{Variablen--Interpretation} falls gilt \\[0.3cm]
\hspace*{1.3cm} $x \in \textsl{var}(\tau) \Rightarrow \mathcal{I}(x) \in \tau^\struct$.
\vspace{0.5cm}

\textbf{Notation}: $\mathcal{I} = \{ x_1 \mapsto t_1, \; x_2 \mapsto t_2, \;\cdots \}$

\vspace{0.5cm}
\textbf{Definition}: Sei
\begin{enumerate}
\item $\mathcal{I}$: $\struct$--Variablen--Interpretation,
\item $x \in \mathcal{V}_\tau$,
\item $c \in \tau^{\struct}$.
\end{enumerate}
Definiere $\struct$--Variablen--Interpretation $\mathcal{I}[x/c]$ durch \\[0.3cm]
\hspace*{1.3cm} 
    $\mathcal{I}[x/c](y) := \left\{
    \begin{array}{ll}
    c               & \mbox{falls}\; y = x;  \\
    \mathcal{I}(y)  & \mbox{sonst}.          \\
    \end{array}
    \right.$

\vspace*{\fill}
\tiny \addtocounter{mypage}{1}
\rule{17cm}{1mm}
Pr�dikatenlogik --- Semantik  \hspace*{\fill} Seite \arabic{mypage}
\end{slide}

%%%%%%%%%%%%%%%%%%%%%%%%%%%%%%%%%%%%%%%%%%%%%%%%%%%%%%%%%%%%%%%%%%%%%%%%

\begin{slide}{}
\normalsize
\begin{center}
Semantik der Terme
\end{center}
\vspace{0.5cm}

\footnotesize

\textbf{Gegeben}: 
\begin{enumerate}
\item Signatur $\Sigma$,
\item $\Sigma$--Struktur $\struct$,
\item $\struct$--Variablen--Interpretation $\mathcal{I}$.
\end{enumerate}

\textbf{Definition}: Wert eines Terms $t$:  $\struct(\mathcal{I}, t)$
\begin{enumerate}
\item Fall: $t$ Variable: \\[0.3cm]
      \hspace*{1.3cm} $\struct(\mathcal{I}, x) := \mathcal{I}(x)$
\item Fall: $t = f(t_1,\cdots,t_n)$  \\[0.3cm]
      \hspace*{1.3cm} $\struct\Bigg(\mathcal{I}, f(t_1,\cdots,t_n)\Bigg) := 
                       f^\struct\Bigg( \struct(\mathcal{I}, t_1), \cdots, \struct(\mathcal{I}, t_n) \Bigg)$.
\end{enumerate}

\textbf{Beispiel}: (Gruppen--Theorie)

Sei $\mathcal{I} = \{ x_1 \mapsto 7, \; x_2 \mapsto -2, \;, x_3 \mapsto 3, \cdots \}$ 
\begin{enumerate}
\item $\mathcal{Z}\Bigg(\mathcal{I}, x_1 * 1\Bigg) = *^\mathcal{Z}\Bigg(\mathcal{I}(x_1), 1^\mathcal{Z}\Bigg) = +(7, 0) = 7$
\item $\mathcal{Z}\Bigg(\mathcal{I}, x_1 * (x_2 * x_3)\Bigg) = 7 + (-2 + 3) = 8$
\end{enumerate}

\vspace*{\fill}
\tiny \addtocounter{mypage}{1}
\rule{17cm}{1mm}
Pr�dikatenlogik --- Semantik  \hspace*{\fill} Seite \arabic{mypage}
\end{slide}

%%%%%%%%%%%%%%%%%%%%%%%%%%%%%%%%%%%%%%%%%%%%%%%%%%%%%%%%%%%%%%%%%%%%%%%%

\begin{slide}{}
\normalsize
\begin{center}
Semantik der atomaren $\Sigma$--Formeln
\end{center}
\vspace{0.5cm}

\footnotesize
\textbf{Gegeben}:
\begin{enumerate}
\item $\Sigma$:      \quad Signatur 
\item $\struct$:     \quad $\Sigma$--Struktur
\item $\mathcal{I}$: \quad $\struct$--Variablen--Interpretation
\item $p(t_1, \cdots, t_n)$ \quad atomare Formel
\end{enumerate}
\hspace*{1.3cm} $\struct\Bigg(\mathcal{I}, p(t_1,\cdots,t_n)\Bigg) := 
                 p^\struct\Bigg( \struct(\mathcal{I}, t_1), \cdots, \struct(\mathcal{I}, t_n) \Bigg)$

\textbf{Beispiel}: (Gruppen--Theorie) 

Sei $\mathcal{I} = \{ x_1 \mapsto 7, \; x_2 \mapsto -2, \;, x_3 \mapsto 3, \cdots \}$ \\[0.3cm]
\hspace*{2.0cm} 
 $\mathcal{Z}\Bigg( \mathcal{I}, x_2 * x_1 = 1 \Bigg) \;=\quad =^\mathcal{Z}\Bigg( \mathcal{Z}(\mathcal{I}, x_2 * x_1),\;\mathcal{Z}(\mathcal{I},1)\Bigg)$\\[0.3cm]
\hspace*{0.6cm} $= \quad =^\mathcal{Z}( 5,\; 0) = \mathtt{false}$

\vspace*{\fill}
\tiny \addtocounter{mypage}{1}
\rule{17cm}{1mm}
Pr�dikatenlogik --- Semantik  \hspace*{\fill} Seite \arabic{mypage}
\end{slide}

%%%%%%%%%%%%%%%%%%%%%%%%%%%%%%%%%%%%%%%%%%%%%%%%%%%%%%%%%%%%%%%%%%%%%%%%

\begin{slide}{}
\normalsize
\begin{center}
Semantik der Junktoren
\end{center}
\vspace{0.5cm}

\footnotesize
\textbf{Gegeben}: Funktionen
\begin{enumerate}
\item $\neg: \mathbb{B} \rightarrow \mathbb{B}$,
\item $\vee: \mathbb{B} \times \mathbb{B} \rightarrow \mathbb{B}$,
\item $\wedge: \mathbb{B} \times \mathbb{B} \rightarrow \mathbb{B}$,
\item $\rightarrow: \mathbb{B} \times \mathbb{B} \rightarrow \mathbb{B}$,
\item $\leftrightarrow: \mathbb{B} \times \mathbb{B} \rightarrow \mathbb{B}$.
\end{enumerate}

\textbf{Definition}: Gegeben
\begin{enumerate}
\item $\Sigma$:      \quad Signatur 
\item $\struct$:     \quad $\Sigma$--Struktur
\item $\mathcal{I}$: \quad $\struct$--Variablen--Interpretation
\end{enumerate}
\begin{enumerate}
\item $\struct(\mathcal{I},\verum) := \mathtt{true}$ und $\struct(\mathcal{I},\falsum) := \mathtt{false}$.
\item $\struct(\mathcal{I}, \neg f) \;:=\; \neg\Bigg(\struct(\mathcal{I}, f)\Bigg)$.
\item $\struct(\mathcal{I}, f \wedge g) \;:=\; \wedge\Bigg(\struct(\mathcal{I}, f), \struct(\mathcal{I}, g)\Bigg)$.
\item $\struct(\mathcal{I}, f \vee g) \;:=\; \vee\Bigg(\struct(\mathcal{I}, f), \struct(\mathcal{I}, g)\Bigg)$.
\item $\struct(\mathcal{I}, f \rightarrow g) \;:=\; \rightarrow\!\Bigg(\struct(\mathcal{I}, f), \struct(\mathcal{I}, g)\Bigg)$.
\item $\struct(\mathcal{I}, f \leftrightarrow g) \;:=\; \leftrightarrow\Bigg(\struct(\mathcal{I}, f), \struct(\mathcal{I}, g)\Bigg)$.
\end{enumerate}


\vspace*{\fill}
\tiny \addtocounter{mypage}{1}
\rule{17cm}{1mm}
Pr�dikatenlogik --- Semantik  \hspace*{\fill} Seite \arabic{mypage}
\end{slide}

%%%%%%%%%%%%%%%%%%%%%%%%%%%%%%%%%%%%%%%%%%%%%%%%%%%%%%%%%%%%%%%%%%%%%%%%

\begin{slide}{}
\normalsize
\begin{center}
Semantik der Quantoren
\end{center}
\vspace{0.5cm}

\footnotesize
\hspace*{1.3cm} 
    $\mathcal{I}[x/c](y) := \left\{
    \begin{array}{ll}
    c               & \mbox{falls}\; y = x;  \\
    \mathcal{I}(y)  & \mbox{sonst}.          \\
    \end{array}
    \right.$

 $\struct\bigl(\mathcal{I}, \forall x \in \tau\!:\! f\bigr) \;:=\; $ \\[0.3cm]
 \hspace*{1.3cm} $\left\{
       \begin{array}{ll}
       \mathtt{true}  & \mbox{falls}\; \struct(\mathcal{I}[x/c], f) = \mathtt{true}\quad \mbox{f�r alle}\; c\in \tau^\struct\;\mbox{gilt}; \\
       \mathtt{false} % \mbox{sonst}.
       \end{array}
       \right.$
\vspace{0.5cm}

 $\struct\bigl(\mathcal{I}, \exists x \in \tau\!:\! f\bigr) \;:=\; $ \\[0.3cm]
 \hspace*{1.3cm} $\left\{
       \begin{array}{ll}
       \mathtt{true}  & \mbox{falls}\; \struct(\mathcal{I}[x/c], f) = \mathtt{true}\quad \mbox{f�r ein}\; c\in \tau^\struct\;\mbox{gilt}; \\
       \mathtt{false} % \mbox{sonst}.
       \end{array}
       \right.$ 
\vspace{0.5cm}

Falls $\FV(f) = \emptyset$, dann $\mathcal{S}(\mathcal{I}_1, f) = \mathcal{S}(\mathcal{I}_2, f)$ 
f�r alle $\mathcal{I}_1 = \mathcal{I}_2$.

Schreibweise: Falls $\FV(f) = \emptyset$ \\[0.1cm]
\hspace*{1.3cm} $\mathcal{S}(f) := \mathcal{S}(\mathcal{I}, f)$
\vspace{0.5cm}

\textbf{Beispiele}: 
\begin{enumerate}
\item $\mathcal{Z}( \forall x_1: \exists x_2: x_1 * x_2 = 1) = \mathtt{true}$
\item $\mathcal{Z}( \forall x_1, x_2: x_1 * x_2 = x_2 * x_1) = \mathtt{true}$
\item $\mathcal{Z}( \forall x_1: \exists x_2: x_2 * x_2 = x_1) = \mathtt{false}$
\end{enumerate}

\vspace*{\fill}
\tiny \addtocounter{mypage}{1}
\rule{17cm}{1mm}
Pr�dikatenlogik --- Semantik  \hspace*{\fill} Seite \arabic{mypage}
\end{slide}

%%%%%%%%%%%%%%%%%%%%%%%%%%%%%%%%%%%%%%%%%%%%%%%%%%%%%%%%%%%%%%%%%%%%%%%%

\begin{slide}{}
\normalsize

\begin{center}
Allgemeing�ltig
\end{center}
\vspace{0.5cm}

\footnotesize
\textbf{Gegeben}: $f$ sei $\Sigma$--Formel. Falls f�r jede
\begin{enumerate}
\item $\struct$: \quad  $\Sigma$--Struktur und jede
\item  $\mathcal{I}$: \quad $\struct$--Variablen--Interpretation
\end{enumerate}
\hspace*{1.3cm} $\struct(\mathcal{I}, f) = \mathtt{true}$ \\[0.3cm]
dann ist  $f$  \emph{allgemeing�ltig}.  

\textbf{Schreibweise}:
\hspace*{1.3cm} $\models f$. 

\textbf{Beispiele}: (Gruppen--Theorie) Seien
\begin{enumerate}
  \item $g_1 \quad := \quad (\forall x_1: 1 * x_1 = x_1)$
  \item $g_2 \quad := \quad (\forall x_2, x_3, x_4: (x_2 * x_3) * x_4 = x_2 * (x_3 * x_4))$
  \item $g_3 \quad := \quad (\forall x_5: \exists x_6: x_5 * x_6 = 1)$
  \item $g_4 \quad := $\\[0.1cm]
\hspace*{-1.3cm} $\quad (\forall x_7, x_8, x_9, x_{10}: x_7 = x_9 \wedge x_8 = x_{10} \rightarrow x_7 * x_8 = x_9 * x_{10})$
  \item $g \quad := \quad g_1 \wedge g_2 \wedge g_3 \wedge g_4$.
\end{enumerate}
Dann gilt:
\begin{enumerate}
\item $\models g \rightarrow \forall x_{11}: x_{11} * 1 = x_{11}$
\item $\models g \rightarrow \forall x_{11}: \exists x_{12}: x_{11} * x_{12} = 1$
\item $\not\models g \rightarrow \forall x_{11}, x_{12}: x_{11} * x_{12} = x_{12} * x_{11}$
\end{enumerate}

\vspace*{\fill}
\tiny \addtocounter{mypage}{1}
\rule{17cm}{1mm}
Pr�dikatenlogik --- Semantik  \hspace*{\fill} Seite \arabic{mypage}
\end{slide}

%%%%%%%%%%%%%%%%%%%%%%%%%%%%%%%%%%%%%%%%%%%%%%%%%%%%%%%%%%%%%%%%%%%%%%%%

\begin{slide}{}
\normalsize
\begin{center}
Folgerung
\end{center}
\vspace{0.5cm}

\footnotesize
\textbf{Definition}: Sei
 \begin{enumerate}
 \item $M\subseteq \mathbb{F}_\Sigma$
 \item $f\in \mathbb{F}_\Sigma$.  
 \end{enumerate}
$f$ \emph{folgt} aus $M$ g.d.w. 
\begin{enumerate}
\item f�r jede $\Sigma$--Struktur $\struct$ und
\item jede Variablen--Interpretation $\mathcal{I}$ 
\end{enumerate}
gilt: \\[0.1cm]
\hspace*{1.3cm} $\bigl(\forall m \in M: \struct(\mathcal{I},m) = \mathtt{true}\bigr)
                 \quad\Rightarrow\quad \struct(\mathcal{I},f) = \texttt{true}$ \\[0.3cm]
\textbf{Schreibweise}:
\hspace*{1.3cm} $M \models f$

\textbf{Beispiele}: (Gruppen--Theorie) 

Seien $g_1$, $g_2$, $g_3$, $g_4$ die  Axiome der Gruppen--Theorie.

Dann gilt:
\begin{enumerate}
\item $\{ g_1, g_2, g_3, g_4 \} \models \forall x_{11}: x_{11} * 1 = x_{11}$
\item $\{ g_1, g_2, g_3, g_4 \} \models \forall x_{11}: \exists x_{12}: x_{11} * x_{12} = 1$
\item $\{ g_1, g_2, g_3, g_4 \} \not\models \forall x_{11}, x_{12}: x_{11} * x_{12} = x_{12} * x_{11}$
\end{enumerate}

\vspace*{\fill}
\tiny \addtocounter{mypage}{1}
\rule{17cm}{1mm}
Pr�dikatenlogik --- Semantik  \hspace*{\fill} Seite \arabic{mypage}
\end{slide}

%%%%%%%%%%%%%%%%%%%%%%%%%%%%%%%%%%%%%%%%%%%%%%%%%%%%%%%%%%%%%%%%%%%%%%%%

\begin{slide}{}
\normalsize
\begin{center}
Erf�llbar
\end{center}
\vspace{0.5cm}

\footnotesize
\textbf{Definition}
$M \subseteq \mathbb{F}_\Sigma$ \emph{erf�llbar} g.d.w.
\begin{enumerate}
\item existiert $\Sigma$--Struktur $\struct$ und
\item existiert $\struct$--Variablen--Interpretation $\mathcal{I}$
\end{enumerate}
mit:
  \\[0.1cm]
\hspace*{1.3cm} $\forall m \el M: \struct(\mathcal{I},m) = \mathtt{true}$ 
\vspace{0.5cm}

\textbf{Beispiel}: \\
Seien $g_1$, $g_2$, $g_3$, $g_4$ die  Axiome der Gruppen--Theorie.

\begin{enumerate}
\item $M_1 := \{g_1, g_2, g_3, g_4, \neg x_1 * x_2 = x_2 * x_1 \}$

      $M_1$ erf�llbar
\item $M_2 := \{g_1, g_2, g_3, g_4, \neg x_1 * 1 = x_1 \}$

      $M_2$ unerf�llbar
\end{enumerate}

\textbf{Korollar}: \\[0.3cm]
\hspace*{1.3cm} $M$ unerf�llbar \quad g.d.w. \quad $M \models \falsum$
\vspace{0.5cm}

\textbf{Satz}: \\[0.3cm]
\hspace*{1.3cm} $M \models f$  \quad g.d.w. \quad $M \cup \{\neg f\} \models \falsum$
\vspace{0.5cm}


\vspace*{\fill}
\tiny \addtocounter{mypage}{1}
\rule{17cm}{1mm}
Pr�dikatenlogik --- Semantik  \hspace*{\fill} Seite \arabic{mypage}
\end{slide}

%%%%%%%%%%%%%%%%%%%%%%%%%%%%%%%%%%%%%%%%%%%%%%%%%%%%%%%%%%%%%%%%%%%%%%%%

\end{document}

%%% Local Variables: 
%%% mode: latex
%%% TeX-master: t
%%% End: 
