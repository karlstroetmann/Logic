%%%%%%%%%%%%%%%%%%%%%%%%%%%%%%%%%%%%%%%%%%%%%%%%%%%%%%%%%%%%%%%%%%%%%%%%

\begin{slide}{}
\normalsize
\begin{center}
\end{center}

\footnotesize
Beispiel:
\begin{verbatim}
  concat( [], L, L ).
  concat( [ X | L1 ], L2, [ X | L3 ] ) :- 
      concat( L1, L2, L3 ).
\end{verbatim}
Aufruf:    $\texttt{concat}(l_1, l_2, \mathtt{X})$  

Ergebnis: $\mathtt{X} = l_1 * l_2$ \\[0.1cm]
\hspace*{2.8cm} mit $*: \textsl{List} \times \textsl{List} \rightarrow \textsl{List}$ \\[0.1cm]
\hspace*{2.8cm} und $l_1 * l_2$ Verkettung von $l_1$ und $l_2$.

Also: \texttt{concat} implementiert Funktion $*$.

\textbf{Definition}:
\begin{enumerate}
\item Sei $t$ Term.  $t$ ist \emph{geschlossen} g.d.w. $\textsl{var}(t) = \emptyset$.   
\item Sei $G$ Ziel.  $G$ ist \emph{geschlossen} g.d.w. $\FV(G) = \emptyset$.
\end{enumerate}

\textbf{Beispiel}:
\begin{enumerate}
\item $[1,2,3]$: geschlossener Term 
\item $[1,\mathtt{X},7]$: nicht geschlossen wegen Variable $\mathtt{X}$
\end{enumerate}

\textbf{Sprechweise}:  Sei $\langle G, [] \rangle \leadsto_\mathcal{P}^* \langle \verum, \tau \rangle$.
\begin{enumerate}
\item \emph{Berechnung} von $G$ \emph{instantiert} Variablen in $G$ mit $\tau$.
\item $G\tau$ ist \emph{L�sung} von $G$.
\end{enumerate}

\vspace*{\fill}
\tiny \addtocounter{mypage}{1}
\rule{17cm}{1mm}
Prolog  \hspace*{\fill} Seite \arabic{mypage}
\end{slide}

%%%%%%%%%%%%%%%%%%%%%%%%%%%%%%%%%%%%%%%%%%%%%%%%%%%%%%%%%%%%%%%%%%%%%%%%

\begin{slide}{}
\normalsize
\begin{center}
Eingabe/Ausgabe--Spezifikation
\end{center}
\vspace{0.5cm}

\footnotesize
\textbf{Definition}: Sei $p$ $n$--stelliges Pr�dikats--Zeichen. 

Eine \emph{E/A--Spezifikation} f�r $p$ hat die Form: \\[0.3cm]
\hspace*{1.3cm} $\langle E\!A_1, \cdots, E\!A_n \rangle$  mit $E\!A_i \in \{ \mathquote{+}, \mathquote{-} \}$.  \\[0.3cm]
Schreibweise: \\[0.1cm]
\hspace*{1.3cm} $p( E\!A_1, \cdots, E\!A_n )$.  
\begin{itemize}
\item $E\!A_i = \mathquote{+}$: $i$--tes Argument ist Eingabe--Argument
\item $E\!A_i = \mathquote{-}$: $i$--tes Argument ist Ausgabe--Argument
\end{itemize}
Ziel $p(s_1,\cdots,s_n)$ \emph{gen�gt} E/A--Spezifikation g.d.w.
jedes Eingabe--Argument geschlossener Term ist.

\vspace{0.3cm}
\textbf{Beispiel}:  $\mathtt{concat}(+,+,-)$. 
\begin{enumerate}
\item $p([1,2],[3],\mathtt{L})$ \quad  gen�gt  E/A--Spezifikation
\item $p([\mathtt{X},2],[3],\mathtt{L})$ \quad  gen�gt  E/A--Spezifikation nicht
\item $p([\mathtt{X},2],[3],\mathtt{L})$ \quad  gen�gt  E/A--Spezifikation \\[0.1cm]
\hspace*{4.7cm} $\mathtt{concat}(-,+,-)$
\end{enumerate}
Aber:
\begin{enumerate}
\item $\mathtt{concat}(+,+,-)$ korrekt
\item $\mathtt{concat}(-,+,-)$ nicht korrekt
\end{enumerate}

\vspace*{\fill}
\tiny \addtocounter{mypage}{1}
\rule{17cm}{1mm}
Prolog  \hspace*{\fill} Seite \arabic{mypage}
\end{slide}

%%%%%%%%%%%%%%%%%%%%%%%%%%%%%%%%%%%%%%%%%%%%%%%%%%%%%%%%%%%%%%%%%%%%%%%%

\begin{slide}{}
\normalsize
\begin{center}
Korrektheit von E/A--Spezifikationen
\end{center}
\vspace{0.5cm}

\footnotesize
Es sei
\begin{enumerate}
\item  $\mathcal{P}$:  Prolog--Programm,
\item  $p$:  $n$--stelliges Pr�dikats--Zeichen und
\item  $\sigma$:  E/A--Spezifikation f�r $p$.
\end{enumerate}
 $\sigma$ \emph{korrekt bez�glich} $\mathcal{P}$ g.d.w. gilt: Falls
\begin{enumerate}
\item  $G$ Ziel, das $\sigma$ gen�gt, und
\item  $G\tau$ L�sung von $G$,
\end{enumerate}
dann ist $G\tau$ geschlossen.

\textbf{Beispiel}: $\mathtt{concat}(+,+,-)$ korrekt weil 
\begin{enumerate}
\item $\mathtt{concat}(l_1, l_2, \mathtt{L})$ gen�gt E/A--Spezifikation wenn \\
      $l_1$ und $l_2$ geschlossen sind.
\item L�sung hat die Form \\[0.3cm]
      \hspace*{1.3cm} $\mathtt{concat}(l_1, l_2, l_1 * l_2)$ \\[0.3cm]
      und $l_1 * l_2$ ist geschlossen.
\end{enumerate}

\vspace*{\fill}
\tiny \addtocounter{mypage}{1}
\rule{17cm}{1mm}
Prolog  \hspace*{\fill} Seite \arabic{mypage}
\end{slide}

%%%%%%%%%%%%%%%%%%%%%%%%%%%%%%%%%%%%%%%%%%%%%%%%%%%%%%%%%%%%%%%%%%%%%%%%

\begin{slide}{}
\normalsize
\begin{center}
\end{center}

\footnotesize
\textbf{Definition}: Es sei
\begin{enumerate}
\item  $G = p(s_1,\cdots,s_n)$: Ziel
\item  $p$: Pr�dikats--Zeichen mit E/A--Spezifikation \\[0.3cm]
\hspace*{1.3cm}  $p(E\!A_1,\cdots,E\!A_n)$.
\end{enumerate}
Definiere
\begin{enumerate}
\item \emph{Eingabe--Variablen} von $G$: \\[0.3cm]
      \hspace*{1.3cm} $\FV^+(G) = \bigcup \big\{ \textsl{var}(s_i) \;|\; E\!A_i = \mathquote{+} \}$. 
\item \emph{Ausgabe--Variablen} von $G$: \\[0.3cm]
      \hspace*{1.3cm} $\FV^-(G) = \bigcup \big\{ \textsl{var}(s_i) \;|\; E\!A_i = \mathquote{-} \}$. 
\end{enumerate}

\textbf{Mengenlehre}: Sei $M$ Menge von Mengen. \\[0.3cm]
\hspace*{1.3cm} $\bigcup M := \{ x \,|\; \exists m \!\in\! M : x \!\in\! m \}$ \\[0.3cm]
\textbf{Beispiel}:  Es gilt \\[0.1cm]
\hspace*{1.3cm} $\bigcup \Bigg\{ \{1,2,3\}, \{3,4,5\}, \{7\} \Bigg\} = \{ 1,2,3,4,5,7\}$. \\[0.1cm]
Allgemein: \\[0.3cm]
\hspace*{1.3cm} $\bigcup \{ m_1 , \cdots, m_n \} = m_1 \cup \cdots \cup m_n$

Sei $\mathtt{concat}(+,+,-)$ und $G = \mathtt{concat}( [ \mathtt{X} | \mathtt{Xs} ], [1], [ \mathtt{Y} | \mathtt{Ys} ] )$ 
\begin{enumerate}
\item $\FV^+(G) = \textsl{var}([ \mathtt{X} | \mathtt{Xs} ]) \cup \textsl{var}([1]) = \{ \mathtt{X}, \mathtt{Xs} \} \cup \{\} = \{ \mathtt{X}, \mathtt{Xs} \}$ 
\item  $\FV^-(G) = \textsl{var}([ \mathtt{Y} | \mathtt{Ys} ]) = \{ \mathtt{Y}, \mathtt{Ys} \}$. \\[0.1cm]
\end{enumerate}


\vspace*{\fill}
\tiny \addtocounter{mypage}{1}
\rule{17cm}{1mm}
Prolog  \hspace*{\fill} Seite \arabic{mypage}
\end{slide}

%%%%%%%%%%%%%%%%%%%%%%%%%%%%%%%%%%%%%%%%%%%%%%%%%%%%%%%%%%%%%%%%%%%%%%%%

\begin{slide}{}
\normalsize
\begin{center}
Vertr�glichkeit
\end{center}
\vspace{0.5cm}

\footnotesize
\textbf{Voraussetzung}: Jedes Pr�dikats--Zeichen hat \\
\hspace*{5.0cm} E/A--Spezifikation.

\textbf{Definition}: Programm--Klausel \\[0.3cm]
\hspace*{1.3cm} $H \;\mathtt{:-}\; B_1\mathtt{,}\, \cdots\mathtt{,}B_{i-1}\mathtt{,}\,
B_i\mathtt{,}\, \cdots\mathtt{,}\, B_n\mathtt{.}$ \\[0.3cm]
 mit gegeben E/A--Spezifikationen \emph{vertr�glich}, g.d.w.
\begin{enumerate}
\item F�r alle $i=1,\cdots,n$ gilt: \\[0.1cm]
      $\FV^+(B_i) \subseteq \FV^+(H) \cup \FV^-(B_1) \cup \cdots \cup \FV^-(B_{i-1})$.
\item F�r den Kopf der Klausel gilt: \\[0.1cm]
      $\FV^-(H) \subseteq \FV^+(H) \cup \FV^-(B_1) \cup \cdots \cup \FV^-(B_{n})$. 
\end{enumerate}

\textbf{Voraussetzung}:
\begin{enumerate}
\item $\mathcal{P}$: Prolog--Programm mit E/A--Spezifikationen.
\item Alle Klauseln aus $\mathcal{P}$ sind mit E/A--Spezifikationen \\
       vertr�glich.  
\item $G$: Ziel, das gegebener E/A--Spezifikation gen�gt.
\item $\langle G, [] \rangle \leadsto_\mathcal{P}^* \langle G_i, \tau_i \rangle$ \\[0.1cm]
\item $\langle G, [] \rangle \leadsto_\mathcal{P}^* \langle \verum, \tau\rangle$.
\end{enumerate}
\textbf{Behauptung}: 
\begin{enumerate}
\item $G_i$ gen�gt E/A--Spezifikation.
\item $\FV(G\tau) = \emptyset$.  
\item Alle durchgef�hrten Unifikationen sind korrekt.
\end{enumerate}


\vspace*{\fill}
\tiny \addtocounter{mypage}{1}
\rule{17cm}{1mm}
Prolog  \hspace*{\fill} Seite \arabic{mypage}
\end{slide}

%%%%%%%%%%%%%%%%%%%%%%%%%%%%%%%%%%%%%%%%%%%%%%%%%%%%%%%%%%%%%%%%%%%%%%%%

\begin{slide}{}
\normalsize
\begin{center}
Beispiel: \texttt{concat}
\end{center}
\vspace{0.5cm}

\footnotesize
\begin{verbatim}
    % concat( +, +, - ).
    concat( [], L, L ).
    concat( [ X | L1 ], L2, [ X | L3 ] ) :- 
        concat( L1, L2, L3 ).
\end{verbatim}
F�r die erste Klausel ist zu zeigen: \\[0.3cm]
\hspace*{1.3cm} $\FV^-\big(\mathtt{concat}(\mathtt{[]}, \mathtt{L}, \mathtt{L})\big) \subseteq \FV^+\big(\mathtt{concat}(\mathtt{[]}, \mathtt{L}, \mathtt{L})\big)$ \\[0.3cm]
Das folgt aus
\begin{enumerate}
\item $\FV^-\big(\mathtt{concat}(\mathtt{[]}, \mathtt{L}, \mathtt{L})\big) = \{ \mathtt{L} \}$.
\item $\FV^+\big(\mathtt{concat}(\mathtt{[]}, \mathtt{L}, \mathtt{L})\big)  = \{ \mathtt{L} \}$. 
\end{enumerate}

F�r die zweite Klausel ist zu zeigen:
\begin{enumerate}
\item $\FV^+\big(\mathtt{concat(L1,L2,L3)}\big) \subseteq \FV^+\big(\mathtt{concat([X|L1],L2,[X|L3])}\big)$
\item $\FV^-\big(\mathtt{concat([X|L1],L2,[X|L3])}\big) \subseteq$ \\
      $\FV^+\big(\mathtt{concat([X|L1],L2,[X|L3])}\big) \cup \FV^-\big(\mathtt{concat(L1,L2,L3)}\big)$
\end{enumerate}
Das folgt aus:
\begin{enumerate}
\item $\FV^+\big(\mathtt{concat(L1,L2,L3)}\big) = \{ \mathtt{L1}, \mathtt{L2} \}$,
\item $\FV^-\big(\mathtt{concat(L1,L2,L3)}\big) = \{ \mathtt{L3} \}$,
\item $\FV^+\big(\mathtt{concat([X|L1],L2,[X|L3])}\big) = \{ \mathtt{X}, \mathtt{L1}, \mathtt{L2} \}$,
\item $\FV^-\big(\mathtt{concat([X|L1],L2,[X|L3])}\big) = \{ \mathtt{X}, \mathtt{L3} \}$.
\end{enumerate}


\vspace*{\fill}
\tiny \addtocounter{mypage}{1}
\rule{17cm}{1mm}
Prolog  \hspace*{\fill} Seite \arabic{mypage}
\end{slide}

%%% Local Variables: 
%%% mode: latex
%%% TeX-master: "folie10"
%%% End: 
