\documentclass{slides}
\usepackage[latin1]{inputenc}
\usepackage{german}
\usepackage{epsfig}
\usepackage{amssymb}

\pagestyle{empty}
\setlength{\textwidth}{17cm}
\setlength{\textheight}{24cm}
\setlength{\topmargin}{0cm}
\setlength{\headheight}{0cm}
\setlength{\headsep}{0cm}
\setlength{\topskip}{0.2cm}
\setlength{\oddsidemargin}{0.5cm}
\setlength{\evensidemargin}{0.5cm}

\newfont{\chess}{chess20}
\newfont{\bigchess}{chess30}
\newcommand{\chf}{\baselineskip20pt\lineskip0pt\chess}

\newcommand{\is}{\;|\;}
\newcommand{\schluss}[2]{\frac{\displaystyle\quad \rule[-8pt]{0pt}{18pt}#1 \quad}{\displaystyle\quad \rule{0pt}{14pt}#2 \quad}}
\newcommand{\vschlus}[1]{{\displaystyle\rule[-6pt]{0pt}{12pt} \atop \rule{0pt}{10pt}#1}}
\newcommand{\verum}{\top}
\newcommand{\falsum}{\bot}
\newcommand{\gentzen}{\vdash}
\newcommand{\komplement}[1]{\overline{#1}}
\def\pair(#1,#2){\langle #1, #2 \rangle}

\newcounter{mypage}

\begin{document}

\begin{slide}{}
\normalsize
\begin{center}
   Algorithmus von Davis \& Putnam
\end{center}
\vspace*{1cm}


\footnotesize
\textbf{Geg}.: $K$ Klauselmenge

\textbf{Def}.: $K$ \emph{l�sbar} \quad g.d.w. \quad
$\exists \mathcal{I}: \forall k \in K: \mathtt{eval}(k,\mathcal{I}) = \mathtt{true}$ 

\textbf{Satz}: $K$ l�sbar \quad g.d.w. \quad $\neg K$ keine Tautologie

\textbf{Def}.: $K$ \emph{trivial} g.d.w
\begin{enumerate}
\item $\{\} \in K$ \quad oder 

      (Dann: $K$ unl�sbar.)

\item $\forall k \in K: \textsl{card}(k) = 1$ \qquad \qquad \quad und  \\[0.2cm]
      $\forall p \in \mathcal{P}: \neg\biggl( \{p\} \in K \wedge \{\neg p\} \in K\biggr)$.

      Dann: 
      \[ \mathcal{I} = \biggl\{ \pair(p, \mathtt{true}) \mid \{p\} \in K \biggr\} \,\cup\,
                       \biggl\{ \pair(p, \mathtt{false}) \mid \{\neg p\} \in K \biggr\} 
      \]
      L�sung von $K$.
\end{enumerate}


\vspace*{\fill}
\tiny \addtocounter{mypage}{1} 
\rule{17cm}{1mm}
Davis-Putnam-Algorithmus \hspace*{\fill} Seite \arabic{mypage}
\end{slide}

%%%%%%%%%%%%%%%%%%%%%%%%%%%%%%%%%%%%%%%%%%%%%%%%%%%%%%%%%%%%%%%%%%%%%%%%

\begin{slide}{}
\normalsize
\begin{center}
   Algorithmus von Davis \& Putnam
\end{center}
\vspace*{1cm}

\footnotesize
\begin{enumerate}
\item Vereinfachung mit der Schnitt-Regel \\[0.5cm]
      \hspace*{1.3cm}
      $\schluss{ k \cup \{ p \} \quad \{ \neg p \}}{k}$ \qquad
      $\schluss{ k \cup \{ \neg p \} \quad \{ p \}}{k}$ 
      \\[0.5cm]
      $\textsl{unitCut}: 2^\mathcal{K} \times \mathcal{L} \rightarrow 2^\mathcal{K}$

      $\textsl{unitCut}(K,l) = 
       \biggl\{ k \backslash \left\{ \komplement{\,l\,} \right\} \;\Big|\; k \in K \biggr\}$
\item Subsumption
      \\[0.5cm]
      \hspace*{1.3cm}
      $l \wedge (l \vee k) \leftrightarrow l$
      \\[0.5cm]
      $\textsl{subsume}: 2^\mathcal{K} \times \mathcal{L} \rightarrow 2^\mathcal{K}$

      $\textsl{subsume}(K, l) := 
       \biggl\{ k \in K \mid l \not\in k \biggr\} \cup \biggl\{\{l\}\biggr\}$.
\item
  $K$ erf�llbar \quad g.d.w. 

  $K \cup \biggl\{\{p\}\biggr\}$ erf�llbar \quad oder \quad
  $K \cup \biggl\{\{\neg p\}\biggr\}$ erf�llbar  

\end{enumerate}

\vspace*{\fill}
\tiny \addtocounter{mypage}{1} 
\rule{17cm}{1mm}
Davis-Putnam-Algorithmus \hspace*{\fill} Seite \arabic{mypage}
\end{slide}

%%%%%%%%%%%%%%%%%%%%%%%%%%%%%%%%%%%%%%%%%%%%%%%%%%%%%%%%%%%%%%%%%%%%%%%%

\begin{slide}{}
\normalsize
\begin{center}
   Algorithmus von Davis \& Putnam
\end{center}
\vspace*{1cm}

\footnotesize
\textbf{Algorithmus}
\begin{enumerate}
\item F�hre alle m�glichen Unit-Schnitte aus.
\item F�hre alle m�glichen Unit-Subsumptionen aus.
\item Falls $K$ trivial:  fertig!
\item Sonst: W�hle aussagenlogische Variable $p$ aus $K$.
      \begin{enumerate}
      \item Rekursiv: Versuche \\[0.2cm]
            \hspace*{1.3cm}  $K \cup \biggl\{\{p\}\biggr\}$ \\[0.2cm]
            zu l�sen. 
      \item Falls (a) erfolglos: Versuche \\[0.2cm]
            \hspace*{1.3cm} $K \cup \biggl\{\{\neg p\}\biggr\}$ \\[0.2cm]
            zu l�sen.  
      \end{enumerate}
\end{enumerate}


Zweckm��ig:  $\textsl{unitCut}()$ und $\textsl{subsume}()$ zusammen fassen
\\[0.4cm]
$\textsl{reduce}: 2^\mathcal{K} \times \mathcal{L} \rightarrow 2^\mathcal{K}$
\\[0.4cm]
$\textsl{reduce}(K,l)  = $ \\[0.4cm]
\hspace*{1.3cm}
$ \biggl\{\, k \backslash \left\{\komplement{l}\right\} \;|\; k \in K \wedge \komplement{l} \in k \,\biggr\} 
   \,\cup\, \biggl\{\, k \in K \mid \komplement{l} \not\in k \wedge l \not\in k \} \cup \biggl\{\{l\}\biggr\}.
$
\\[0.2cm]

\vspace*{\fill}
\tiny \addtocounter{mypage}{1} 
\rule{17cm}{1mm}
Davis-Putnam-Algorithmus \hspace*{\fill} Seite \arabic{mypage}
\end{slide}

%%%%%%%%%%%%%%%%%%%%%%%%%%%%%%%%%%%%%%%%%%%%%%%%%%%%%%%%%%%%%%%%%%%%%%%%

\begin{slide}
\begin{center}
8-Damen-Problem
\end{center}
\vspace*{0.5cm}

\footnotesize
K�nnen 8 Damen so auf einem Schach--Brett
aufgestellt werden, dass keine Dame eine andere Dame schlagen kann?

Eine Dame kann eine andere schlagen falls diese
\begin{itemize}
\item in derselben Reihe steht,
\item in derselben Spalte steht, oder
\item in derselben Diagonale steht.
\end{itemize}
\vspace*{-1.0cm}

\footnotesize
\setlength{\unitlength}{1.5cm}

\begin{picture}(10,10)

\thicklines
\put(1,1){\line(1,0){8}}
\put(1,1){\line(0,1){8}}
\put(1,9){\line(1,0){8}}
\put(9,1){\line(0,1){8}}
\put(0.9,0.9){\line(1,0){8.2}}
\put(0.9,9.1){\line(1,0){8.2}}
\put(0.9,0.9){\line(0,1){8.2}}
\put(9.1,0.9){\line(0,1){8.2}}

\thinlines
\multiput(1,2)(0,1){7}{\line(1,0){8}}
\multiput(2,1)(1,0){7}{\line(0,1){8}}
\put(4.25,6.25){{\chess Q}}
\multiput(5.25,6.5)(1,0){4}{\vector(1,0){0.5}}
\multiput(3.75,6.5)(-1,0){3}{\vector(-1,0){0.5}}
\multiput(5.25,7.25)(1,1){2}{\vector(1,1){0.5}}
\multiput(5.25,5.75)(1,-1){4}{\vector(1,-1){0.5}}
\multiput(3.75,5.75)(-1,-1){3}{\vector(-1,-1){0.5}}
\multiput(3.75,7.25)(-1,1){2}{\vector(-1,1){0.5}}
\multiput(4.5,7.25)(0,1){2}{\vector(0,1){0.5}}
\multiput(4.5,5.75)(0,-1){5}{\vector(0,-1){0.5}}
\end{picture}


\vspace*{\fill}
\tiny \addtocounter{mypage}{1} 
\rule{17cm}{1mm}
8-Damen-Problem  \hspace*{\fill} Seite \arabic{mypage}
\end{slide}

%%%%%%%%%%%%%%%%%%%%%%%%%%%%%%%%%%%%%%%%%%%%%%%%%%%%%%%%%%%%%%%%%%%%%%%%

\begin{slide}
\begin{center}
Repr�sentation des Problems
\end{center}
\vspace*{0.5cm}

\footnotesize
\setlength{\unitlength}{2.0cm}
\hspace*{-2cm}
\begin{picture}(10,9)
\thicklines
\put(0.9,0.9){\line(1,0){8.2}}
\put(0.9,9.1){\line(1,0){8.2}}
\put(0.9,0.9){\line(0,1){8.2}}
\put(9.1,0.9){\line(0,1){8.2}}
\put(1,1){\line(1,0){8}}
\put(1,1){\line(0,1){8}}
\put(1,9){\line(1,0){8}}
\put(9,1){\line(0,1){8}}
\thinlines
\multiput(1,2)(0,1){7}{\line(1,0){8}}
\multiput(2,1)(1,0){7}{\line(0,1){8}}

%%  for (i = 1; i <= 8; i = i + 1) {
%%for (j = 1; j <= 8; j = j + 1) \{
%%   \put(\$j.15,<9-$i>.35){{\Large p<$i>\$j}}
%%\}
%%  }

\put(1.15,8.40){{ p11}}
\put(2.15,8.40){{ p12}}
\put(3.15,8.40){{ p13}}
\put(4.15,8.40){{ p14}}
\put(5.15,8.40){{ p15}}
\put(6.15,8.40){{ p16}}
\put(7.15,8.40){{ p17}}
\put(8.15,8.40){{ p18}}
\put(1.15,7.40){{ p21}}
\put(2.15,7.40){{ p22}}
\put(3.15,7.40){{ p23}}
\put(4.15,7.40){{ p24}}
\put(5.15,7.40){{ p25}}
\put(6.15,7.40){{ p26}}
\put(7.15,7.40){{ p27}}
\put(8.15,7.40){{ p28}}
\put(1.15,6.40){{ p31}}
\put(2.15,6.40){{ p32}}
\put(3.15,6.40){{ p33}}
\put(4.15,6.40){{ p34}}
\put(5.15,6.40){{ p35}}
\put(6.15,6.40){{ p36}}
\put(7.15,6.40){{ p37}}
\put(8.15,6.40){{ p38}}
\put(1.15,5.40){{ p41}}
\put(2.15,5.40){{ p42}}
\put(3.15,5.40){{ p43}}
\put(4.15,5.40){{ p44}}
\put(5.15,5.40){{ p45}}
\put(6.15,5.40){{ p46}}
\put(7.15,5.40){{ p47}}
\put(8.15,5.40){{ p48}}
\put(1.15,4.40){{ p51}}
\put(2.15,4.40){{ p52}}
\put(3.15,4.40){{ p53}}
\put(4.15,4.40){{ p54}}
\put(5.15,4.40){{ p55}}
\put(6.15,4.40){{ p56}}
\put(7.15,4.40){{ p57}}
\put(8.15,4.40){{ p58}}
\put(1.15,3.40){{ p61}}
\put(2.15,3.40){{ p62}}
\put(3.15,3.40){{ p63}}
\put(4.15,3.40){{ p64}}
\put(5.15,3.40){{ p65}}
\put(6.15,3.40){{ p66}}
\put(7.15,3.40){{ p67}}
\put(8.15,3.40){{ p68}}
\put(1.15,2.40){{ p71}}
\put(2.15,2.40){{ p72}}
\put(3.15,2.40){{ p73}}
\put(4.15,2.40){{ p74}}
\put(5.15,2.40){{ p75}}
\put(6.15,2.40){{ p76}}
\put(7.15,2.40){{ p77}}
\put(8.15,2.40){{ p78}}
\put(1.15,1.40){{ p81}}
\put(2.15,1.40){{ p82}}
\put(3.15,1.40){{ p83}}
\put(4.15,1.40){{ p84}}
\put(5.15,1.40){{ p85}}
\put(6.15,1.40){{ p86}}
\put(7.15,1.40){{ p87}}
\put(8.15,1.40){{ p88}}
\end{picture}


\vspace*{\fill}
\tiny \addtocounter{mypage}{1} 
\rule{17cm}{1mm}
8-Damen-Problem  \hspace*{\fill} Seite \arabic{mypage}
\end{slide}


%%%%%%%%%%%%%%%%%%%%%%%%%%%%%%%%%%%%%%%%%%%%%%%%%%%%%%%%%%%%%%%%%%%%%%%%

\end{document}

%%% Local Variables: 
%%% mode: latex
%%% TeX-master: t
%%% End: 
